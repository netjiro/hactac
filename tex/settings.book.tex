%--------|---------|---------|---------|---------|---------|---------|---------|
%       10        20        30        40        50        60        70        80


% set the general page layout
\documentclass[11pt, twoside, titlepage, a4paper]{report}

% adjust the margins: 15mm inner binding, 25mm frame all around
\usepackage[a4paper,inner=40mm,outer=25mm,top=25mm,bottom=25mm,pdftex]{geometry}


%-------------------------------------------------------------------------------
% Want searchable ligatures, e.g: diff, stuff, fine           << mmap OR xelatex
% pdflatex: fontenc  T1  <<  not searchable in sumatrapdf, add mmap/cmap
% xelatex: fontspec  TU  <<  supposedly work, needs xelatex instead of pdflatex
%                    TU is default for fontspec xelatex

%for pdflatex:              % old pdflatex, need more stuff for modern works
\usepackage{mmap}           % add info to pdf, searchable ligatures, cmap is old
\usepackage[T1]{fontenc}    % set font encoding T1 to allow "|" ">" "<" etc
\usepackage[utf8]{inputenc} % set utf8 input file encoding

%% for xelatex:              % default utf8, need less stuff to work
%\usepackage{fontspec}       % searchable ligatures in pdf: https://tex.stackexchange.com/questions/67848/can-pdf-search-find-words-with-ligatures-in-xelatex-documents
%% setup for searchable ligatures in pdfs, need fontspec instead of fontenc
%% should be called before inputenc:  https://tex.stackexchange.com/questions/44694/fontenc-vs-inputenc/44699
%% set utf8 encoding, and set font encoding T1 to allow "|" ">" "<" etc
%\usepackage[utf8]{inputenc}    << actually ignored in xelatex


%-------------------------------------------------------------------------------
% set UK british hyphenation [english] sets US
\usepackage[british]{babel}


%-------------------------------------------------------------------------------
% set iso date format with ndash separator
\usepackage[yyyymmdd]{datetime}
\renewcommand{\dateseparator}{--}


%-------------------------------------------------------------------------------
% These page settings give images 1.0\linewidth around 135-140mm wide (ca 138mm)
% meaning a 300dpi image is around 1600 pixels wide

% For encapsulated postscript figures
% Use:
%
%   \includegraphics{width=10cm, height=10cm}{fig.eps}    or:
%   \includegraphics{width=10cm, height=10cm, keepaspectratio}{fig.eps}
%
% where width and height are optional
\usepackage{graphicx}   % For eps figures


% To get figures, tables, etc. where you want them.
% Use:
%
%   \begin{figure}[H]
%
%\usepackage{here}



%-------------------------------------------------------------------------------
% For nicer captions
%
% Valid options (between []) are:
%
% Indentation: hang, center, centerlast, nooneline
% Size: scriptsize, small, normalsize, large, Large
% Style: up, it, sl, sc, md, bf, rm, sf, tt
%
\usepackage[hang,small,bf]{caption}


%-------------------------------------------------------------------------------
% Want to change font?
%
% Uncomment your choice, if all uncommented, Computer Modern Roman is
% used. Note that some of these don't seem to work properly
%
%\usepackage{utopia}
%\usepackage{pspalatino}
%\usepackage{palatino}
%\usepackage{times}
%\usepackage{charter}
%\usepackage{pifont}
%\usepackage{chancery}
%\usepackage{bookman}
%\usepackage{avant}
%\usepackage{helvet}
%\usepackage{zapfchan}
%\usepackage{courier}
%\usepackage{newcent}


%-------------------------------------------------------------------------------
% Some optional packages:
%

%-------------------------------------------------------------------------------
% Index section
%
% Put a \index{keyword} at the word in the text, a \printindex where
% you want the index printed, and run "makeindex <reportname>.idx
% after LaTeX-compiling (and compile a second time)
%
%\usepackage{makeidx} %Index-section
%\makeindex

%-------------------------------------------------------------------------------
% Fancier enumeration
% You get a new \begin{enumerate}[XXX] where you can specify XXX to be
% text {i,I,a,A,1}, for example \begin{enumerate}[Uppg. a)] to get a
% Uppg. a)/b)/c) list.
%
%\usepackage{enumerate}



%-------------------------------------------------------------------------------
% add nicer headers and footers
%
\usepackage{fancyhdr}
\pagestyle{fancy}
% with this we ensure that the chapter and section
% headings are in lowercase.
\renewcommand{\chaptermark}[1]{\markboth{#1}{}}
\renewcommand{\sectionmark}[1]{\markright{\thesection\ #1}}
\fancyhf{} % delete current setting for header and footer
\fancyhead[LE,RO]{\bfseries\thepage}
\fancyhead[LO]{\bfseries\rightmark}
\fancyhead[RE]{\bfseries\leftmark}
\renewcommand{\headrulewidth}{0.5pt}
\renewcommand{\footrulewidth}{0pt}
\addtolength{\headheight}{0.5pt}   % make space for the rule
\fancypagestyle{plain}{%                  % for plain pages:
    \fancyhead{}%                         % get rid of headers
    \renewcommand{\headrulewidth}{0pt}%   % and the line
}


%-------------------------------------------------------------------------------
% Nicer headers and footers (like 94 F dd reports)
%
% \fancyhead[L/C/R]{} to change headers
% \fancyfoot[L/C/R]{} to change footers
%\usepackage{fancyhdr}
%\fancyhead[L]{Left page header}
%\fancyhead[R]{Right page header}
%\footrulewidth 0.5pt % Insert a line above the footer
%\pagestyle{fancy}


%-------------------------------------------------------------------------------
%\usepackage{pdftex}

%enable good pdf conversions with links and index, etc
%\usepackage[pdftex, colorlinks=true, pdfstartview=FitV, linkcolor=blue, citecolor=blue, urlcolor=blue]{hyperref}
%\usepackage[%pdftex,%
%bookmarks=true,%
%bookmarksopen=true,%
%pdfpagelabels,%
%pagebackref,%
%colorlinks=true,%
%linkcolor=blue,%
%linkbordercolor=white,%
%citecolor=blue,%
%citebordercolor=white,%
%urlcolor=blue,%
%urlbordercolor=white,%
%plainpages=false]{hyperref}
% provide some extra pdf information
%\pdfinfo{\Title{§title§}\Author{netjiro}\Keywords{game rpg role-playing tabletop tactical virtual}}

% this is the block used in the masters thesis instead of the one in the PhD thesis
%\usepackage{hyperref}
%%\hypersetup{pdftex=true}
%\hypersetup{pdftex}
%\hypersetup{bookmarks=true}
%\hypersetup{bookmarksopen=true}
%\hypersetup{colorlinks=true}
%\hypersetup{linkcolor=blue}
%\hypersetup{pagecolor=blue}
%\hypersetup{citecolor=blue}
%\hypersetup{urlcolor=blue}


%\usepackage[pdftex, colorlinks=true, pdfstartview=FitV, linkcolor=blue, citecolor=blue, urlcolor=blue]{hyperref}
%          \pdfinfo{
%            /Title      (Main title goes here)
%            /Author     (Author Authorson)
%            /Keywords   (list of keywords)
%          }











%===============================================================================
% local defines


% block listing, verbatim ------------------------------------------------------

% remove forced implicit vertical whitespace before and after verbatim environment
\makeatletter
\preto{\@verbatim}{\topsep=0pt \partopsep=0pt }
\makeatother

% old pattern for block listing:
%\goodbreak
%\raggedbottom
%\small \begin{samepage} \begin{verbatim}
%\end{verbatim} \goodbreak \vspace{\baselineskip} \begin{verbatim}
%\end{verbatim} \goodbreak \vspace{1.5\baselineskip} \begin{verbatim}
%\end{verbatim} \goodbreak \vspace{2\baselineskip} \begin{verbatim}
%\end{verbatim} \end{samepage} \normalsize
%\flushbottom
%\goodbreak

% simpler with blocklistgap:
\newcommand{\blocklistgap}{\goodbreak \vspace{\baselineskip}}
%\raggedbottom     % looks in raggedbottom flow, no stretching baselineskip
%\goodbreak \small \begin{samepage} \begin{verbatim}
%\end{verbatim} \blocklistgap \begin{verbatim}
%\end{verbatim} \end{samepage} \normalsize \goodbreak
%\flusbottom       % perhaps restore flushbottom at some point

%-------------------------------------------------------------------------------



%-------------------------------------------------------------------------------
% need a nice easily visible TODO marker
\newcommand{\todo}{\noindent\textbf{TODO:}~}
\newcommand{\TODO}{\noindent\LARGE\textbf{TODO:}\normalsize~}


%-------------------------------------------------------------------------------
% allow to force indentation of first line in section
% \indent is not working, so workaround \hspace{\parindent} works
\newcommand{\forceindent}{\hspace{\parindent}}
%\noindent  is a standard command


%-------------------------------------------------------------------------------
% couldn't find a degrees marker circle, so here's a quick'n'ugly version
% the trailing tilde adds an unbreakable, unshrinkable, unenlargable space
\newcommand{\degrees}{$^\circ$~}
\newcommand{\degree}{$^\circ$}
%\newcommand{\degs}{$^\circ$}    << command \deg already defined (also skip degs)
%\newcommand{\deg}{$^\circ$~}    << command \deg already defined (also skip degs)


%-------------------------------------------------------------------------------
% can't seem to get this working right for spacing, and when exporting to ebook
\newcommand{\vs}{$\backslash\ $}  % "versus" slash
\newcommand{\bs}{$\backslash\ $}  % just backslash
\newcommand{\ca}{$\approx$}
% without the last backslash space it would take two spaces to get a space in the text, but with it you will always get a space weather you want or not. This is weird.


%-------------------------------------------------------------------------------
% want clear dash insert commands
\newcommand{\dash}{-}     % just a normal hyphen dash  "-"
\newcommand{\ndash}{--}   % n-dash "--"
\newcommand{\mdash}{---}  % m-dash "---"


%-------------------------------------------------------------------------------
%link new command names to the original font sizes,
%for easier to remember smaller font size
\newcommand{\vsmall}{\footnotesize}  % simpler to remember
\newcommand{\vvsmall}{\scriptsize}   %
%\newcommand{\vvvsmall}{\tiny}


%-------------------------------------------------------------------------------
% don't want outlines and crap around links in the document.
% notoriously annoying package.
% info can be found here:
% http://theory.uwinnipeg.ca/localfiles/infofiles/teTeX/latex/hyperref/manual.pdf
\usepackage[colorlinks=true,linkcolor=black,urlcolor=blue]{hyperref}


%-------------------------------------------------------------------------------
% I want the possibility of switching between different styles for skills and actions. Either have them as list items or subsections/subsubsections, but without toc entries.
\usepackage{ifthen}


%-------------------------------------------------------------------------------
% Below I will now define \openXlist and \closeXlist list commands
% as well as \skill{} and \action{} itemizers.
% Then they can be used in the docs as
%
% \openskillslist
% \skill{name1}
% \skill{name2}
% \closeskillslist
%
% \openactionslist
% \action{name1}
% \action{name2}
% \closeactionslist
%
% handy if I want to switch easily between item listing for the pdf version
% and headings for the html doc version.
% Just change the \setboolean{skillsaslist} to true or false.




% With skillsaslist defined we can use \ifskillsaslist{\begin{description}}
% as well as \ifskillsaslist{\end{description}} around the skill listings
\newboolean{skillsaslist}
\setboolean{skillsaslist}{true}

% Define the \skill{name} to make either of
%    \item[name]
%    \subsubsection{name}
\ifthenelse{\boolean{skillsaslist}}{\newcommand{\skill}[1]{\item[#1]}}{\newcommand{\skill}[1]{\subsubsection*{#1}}}

% Define the \openskillslist to either of
%    \begin{description}
%    -- nothing --
\ifthenelse{\boolean{skillsaslist}}{\newcommand{\openskillslist}{\begin{description}}}{\newcommand{\openskillslist}{}}

% Define the \closeskillslist to either of
%    \end{description}
%    -- nothing --
\ifthenelse{\boolean{skillsaslist}}{\newcommand{\closeskillslist}{\end{description}}}{\newcommand{\closeskillslist}{}}




% With actionsaslist defined we can use \ifactionsaslist{\begin{description}}
% as well as \ifactionsaslist{\end{description}} around the skill listings
\newboolean{actionsaslist}
\setboolean{actionsaslist}{true}

% Define the \action{name} to make either of
%    \item[name]
%    \subsubsection{name}
\ifthenelse{\boolean{actionsaslist}}{\newcommand{\action}[1]{\item[#1]}}{\newcommand{\action}[1]{\subsubsection*{#1}}}

% Define the \openskillslist to either of
%    \begin{description}
%    -- nothing --
\ifthenelse{\boolean{actionsaslist}}{\newcommand{\openactionslist}{\begin{description}}}{\newcommand{\openactionslist}{}}

% Define the \closeskillslist to either of
%    \end{description}
%    -- nothing --
\ifthenelse{\boolean{actionsaslist}}{\newcommand{\closeactionslist}{\end{description}}}{\newcommand{\closeactionslist}{}}




% With itemsaslist defined we can use \ifactionsaslist{\begin{description}}
% as well as \ifactionsaslist{\end{description}} around the skill listings
\newboolean{itemsaslist}
\setboolean{itemsaslist}{true}

% Define the \eqitem{name} to make either of
%    \item[name]
%    \subsubsection{name}
\ifthenelse{\boolean{itemsaslist}}{\newcommand{\eqitem}[1]{\item[#1]}}{\newcommand{\action}[1]{\subsubsection*{#1}}}

% Define the \openitemslist to either of
%    \begin{description}
%    -- nothing --
\ifthenelse{\boolean{itemsaslist}}{\newcommand{\openitemslist}{\begin{description}}}{\newcommand{\openitemslist}{}}

% Define the \closeitemslist to either of
%    \end{description}
%    -- nothing --
\ifthenelse{\boolean{itemsaslist}}{\newcommand{\closeitemslist}{\end{description}}}{\newcommand{\closeitemslist}{}}




% With spellsaslist defined we can use \ifspellsaslist{\begin{description}}
% as well as \ifspellsaslist{\end{description}} around the spell listings
\newboolean{spellsaslist}
\setboolean{spellsaslist}{true}

% Define the \spell{name} to make either of
%    \item[name]
%    \subsubsection{name}
\ifthenelse{\boolean{spellsaslist}}{\newcommand{\spell}[1]{\item[#1]}}{\newcommand{\action}[1]{\subsubsection*{#1}}}

% Define the \openspellslist to either of
%    \begin{description}
%    -- nothing --
\ifthenelse{\boolean{spellsaslist}}{\newcommand{\openspellslist}{\begin{description}}}{\newcommand{\openspellslist}{}}

% Define the \closespellslist to either of
%    \end{description}
%    -- nothing --
\ifthenelse{\boolean{spellsaslist}}{\newcommand{\closespellslist}{\end{description}}}{\newcommand{\closespellslist}{}}



%-------------------------------------------------------------------------------
% comment notes escaping list segment, if spellsaslists is set
\ifthenelse{\boolean{spellsaslist}} {
    \newcommand{\spellslistnote}[1]{
        \closespellslist \textit{#1} \openspellslist
    }
} {
    \newcommand{\spellslistnote}[1]{
        \textit{#1}
    }
}




%-------------------------------------------------------------------------------
% readoutloud italic quote sections
\newenvironment{readoutloud}%
{\begin{quote}\begin{itshape}}%
{\end{itshape}\end{quote}}%


% can't find ttshape, like itshape in "readoutloud" environment
\newenvironment{ttshape}%
{\begin{ttfamily}}%
{\end{ttfamily}}%




%-------------------------------------------------------------------------------
% temporary separation line
%\newcommand{\tmpsepline}{\rule[0.25\baselineskip]{0.5\textwidth}{0.5pt}}
%\rule[0.25\baselineskip]{0.5\textwidth}{0.5pt} =xtl-  0.25
%\rule[0.8ex]{0.5\textwidth}{0.5pt} =xtl-  0.8ex
\newcommand{\tmpsepline}{

    \

    \rule[0.25\baselineskip]{0.5\textwidth}{0.5pt}

    \

}





%-------------------------------------------------------------------------------
% \cleartoleftpage
%     open to an empty left page, so to fill two opposed pages
%     cleardoublepage opens to a right page (usually odd page number)
% https://tex.stackexchange.com/questions/11707/how-to-force-output-to-a-left-or-right-page
\makeatletter
\newcommand*{\cleartoleftpage}{%
  \clearpage
    \if@twoside
    \ifodd\c@page
      \hbox{}\newpage
      \if@twocolumn
        \hbox{}\newpage
      \fi
    \fi
  \fi
}
\makeatother
%-------------------------------------------------------------------------------

