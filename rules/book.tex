%--------|---------|---------|---------|---------|---------|---------|---------|
%       10        20        30        40        50        60        70        80
%-------------------------------------------------------------------------------


%--------|---------|---------|---------|---------|---------|---------|---------|
%       10        20        30        40        50        60        70        80


% set the general page layout
\documentclass[11pt, twoside, titlepage, a4paper]{report}

% adjust the margins: 15mm inner binding, 25mm frame all around
\usepackage[a4paper,inner=40mm,outer=25mm,top=25mm,bottom=25mm,pdftex]{geometry}


%-------------------------------------------------------------------------------
% Want searchable ligatures, e.g: diff, stuff, fine           << mmap OR xelatex
% pdflatex: fontenc  T1  <<  not searchable in sumatrapdf, add mmap/cmap
% xelatex: fontspec  TU  <<  supposedly work, needs xelatex instead of pdflatex
%                    TU is default for fontspec xelatex

%for pdflatex:              % old pdflatex, need more stuff for modern works
\usepackage{mmap}           % add info to pdf, searchable ligatures, cmap is old
\usepackage[T1]{fontenc}    % set font encoding T1 to allow "|" ">" "<" etc
\usepackage[utf8]{inputenc} % set utf8 input file encoding

%% for xelatex:              % default utf8, need less stuff to work
%\usepackage{fontspec}       % searchable ligatures in pdf: https://tex.stackexchange.com/questions/67848/can-pdf-search-find-words-with-ligatures-in-xelatex-documents
%% setup for searchable ligatures in pdfs, need fontspec instead of fontenc
%% should be called before inputenc:  https://tex.stackexchange.com/questions/44694/fontenc-vs-inputenc/44699
%% set utf8 encoding, and set font encoding T1 to allow "|" ">" "<" etc
%\usepackage[utf8]{inputenc}    << actually ignored in xelatex


%-------------------------------------------------------------------------------
% set UK british hyphenation [english] sets US
\usepackage[british]{babel}


%-------------------------------------------------------------------------------
% set iso date format with ndash separator
\usepackage[yyyymmdd]{datetime}
\renewcommand{\dateseparator}{--}


%-------------------------------------------------------------------------------
% These page settings give images 1.0\linewidth around 135-140mm wide (ca 138mm)
% meaning a 300dpi image is around 1600 pixels wide

% For encapsulated postscript figures
% Use:
%
%   \includegraphics{width=10cm, height=10cm}{fig.eps}    or:
%   \includegraphics{width=10cm, height=10cm, keepaspectratio}{fig.eps}
%
% where width and height are optional
\usepackage{graphicx}   % For eps figures


% To get figures, tables, etc. where you want them.
% Use:
%
%   \begin{figure}[H]
%
%\usepackage{here}



%-------------------------------------------------------------------------------
% For nicer captions
%
% Valid options (between []) are:
%
% Indentation: hang, center, centerlast, nooneline
% Size: scriptsize, small, normalsize, large, Large
% Style: up, it, sl, sc, md, bf, rm, sf, tt
%
\usepackage[hang,small,bf]{caption}


%-------------------------------------------------------------------------------
% Want to change font?
%
% Uncomment your choice, if all uncommented, Computer Modern Roman is
% used. Note that some of these don't seem to work properly
%
%\usepackage{utopia}
%\usepackage{pspalatino}
%\usepackage{palatino}
%\usepackage{times}
%\usepackage{charter}
%\usepackage{pifont}
%\usepackage{chancery}
%\usepackage{bookman}
%\usepackage{avant}
%\usepackage{helvet}
%\usepackage{zapfchan}
%\usepackage{courier}
%\usepackage{newcent}


%-------------------------------------------------------------------------------
% Some optional packages:
%

%-------------------------------------------------------------------------------
% Index section
%
% Put a \index{keyword} at the word in the text, a \printindex where
% you want the index printed, and run "makeindex <reportname>.idx
% after LaTeX-compiling (and compile a second time)
%
%\usepackage{makeidx} %Index-section
%\makeindex

%-------------------------------------------------------------------------------
% Fancier enumeration
% You get a new \begin{enumerate}[XXX] where you can specify XXX to be
% text {i,I,a,A,1}, for example \begin{enumerate}[Uppg. a)] to get a
% Uppg. a)/b)/c) list.
%
%\usepackage{enumerate}



%-------------------------------------------------------------------------------
% add nicer headers and footers
%
\usepackage{fancyhdr}
\pagestyle{fancy}
% with this we ensure that the chapter and section
% headings are in lowercase.
\renewcommand{\chaptermark}[1]{\markboth{#1}{}}
\renewcommand{\sectionmark}[1]{\markright{\thesection\ #1}}
\fancyhf{} % delete current setting for header and footer
\fancyhead[LE,RO]{\bfseries\thepage}
\fancyhead[LO]{\bfseries\rightmark}
\fancyhead[RE]{\bfseries\leftmark}
\renewcommand{\headrulewidth}{0.5pt}
\renewcommand{\footrulewidth}{0pt}
\addtolength{\headheight}{0.5pt}   % make space for the rule
\fancypagestyle{plain}{%                  % for plain pages:
    \fancyhead{}%                         % get rid of headers
    \renewcommand{\headrulewidth}{0pt}%   % and the line
}


%-------------------------------------------------------------------------------
% Nicer headers and footers (like 94 F dd reports)
%
% \fancyhead[L/C/R]{} to change headers
% \fancyfoot[L/C/R]{} to change footers
%\usepackage{fancyhdr}
%\fancyhead[L]{Left page header}
%\fancyhead[R]{Right page header}
%\footrulewidth 0.5pt % Insert a line above the footer
%\pagestyle{fancy}


%-------------------------------------------------------------------------------
%\usepackage{pdftex}

%enable good pdf conversions with links and index, etc
%\usepackage[pdftex, colorlinks=true, pdfstartview=FitV, linkcolor=blue, citecolor=blue, urlcolor=blue]{hyperref}
%\usepackage[%pdftex,%
%bookmarks=true,%
%bookmarksopen=true,%
%pdfpagelabels,%
%pagebackref,%
%colorlinks=true,%
%linkcolor=blue,%
%linkbordercolor=white,%
%citecolor=blue,%
%citebordercolor=white,%
%urlcolor=blue,%
%urlbordercolor=white,%
%plainpages=false]{hyperref}
% provide some extra pdf information
%\pdfinfo{\Title{§title§}\Author{netjiro}\Keywords{game rpg role-playing tabletop tactical virtual}}

% this is the block used in the masters thesis instead of the one in the PhD thesis
%\usepackage{hyperref}
%%\hypersetup{pdftex=true}
%\hypersetup{pdftex}
%\hypersetup{bookmarks=true}
%\hypersetup{bookmarksopen=true}
%\hypersetup{colorlinks=true}
%\hypersetup{linkcolor=blue}
%\hypersetup{pagecolor=blue}
%\hypersetup{citecolor=blue}
%\hypersetup{urlcolor=blue}


%\usepackage[pdftex, colorlinks=true, pdfstartview=FitV, linkcolor=blue, citecolor=blue, urlcolor=blue]{hyperref}
%          \pdfinfo{
%            /Title      (Main title goes here)
%            /Author     (Author Authorson)
%            /Keywords   (list of keywords)
%          }











%===============================================================================
% local defines


% block listing, verbatim ------------------------------------------------------

% remove forced implicit vertical whitespace before and after verbatim environment
\makeatletter
\preto{\@verbatim}{\topsep=0pt \partopsep=0pt }
\makeatother

% old pattern for block listing:
%\goodbreak
%\raggedbottom
%\small \begin{samepage} \begin{verbatim}
%\end{verbatim} \goodbreak \vspace{\baselineskip} \begin{verbatim}
%\end{verbatim} \goodbreak \vspace{1.5\baselineskip} \begin{verbatim}
%\end{verbatim} \goodbreak \vspace{2\baselineskip} \begin{verbatim}
%\end{verbatim} \end{samepage} \normalsize
%\flushbottom
%\goodbreak

% simpler with blocklistgap:
\newcommand{\blocklistgap}{\goodbreak \vspace{\baselineskip}}
%\raggedbottom     % looks in raggedbottom flow, no stretching baselineskip
%\goodbreak \small \begin{samepage} \begin{verbatim}
%\end{verbatim} \blocklistgap \begin{verbatim}
%\end{verbatim} \end{samepage} \normalsize \goodbreak
%\flusbottom       % perhaps restore flushbottom at some point

%-------------------------------------------------------------------------------



%-------------------------------------------------------------------------------
% need a nice easily visible TODO marker
\newcommand{\todo}{\noindent\textbf{TODO:}~}
\newcommand{\TODO}{\noindent\LARGE\textbf{TODO:}\normalsize~}


%-------------------------------------------------------------------------------
% allow to force indentation of first line in section
% \indent is not working, so workaround \hspace{\parindent} works
\newcommand{\forceindent}{\hspace{\parindent}}
%\noindent  is a standard command


%-------------------------------------------------------------------------------
% couldn't find a degrees marker circle, so here's a quick'n'ugly version
% the trailing tilde adds an unbreakable, unshrinkable, unenlargable space
\newcommand{\degrees}{$^\circ$~}
\newcommand{\degree}{$^\circ$}
%\newcommand{\degs}{$^\circ$}    << command \deg already defined (also skip degs)
%\newcommand{\deg}{$^\circ$~}    << command \deg already defined (also skip degs)


%-------------------------------------------------------------------------------
% can't seem to get this working right for spacing, and when exporting to ebook
\newcommand{\vs}{$\backslash\ $}  % "versus" slash
\newcommand{\bs}{$\backslash\ $}  % just backslash
\newcommand{\ca}{$\approx$}
% without the last backslash space it would take two spaces to get a space in the text, but with it you will always get a space weather you want or not. This is weird.


%-------------------------------------------------------------------------------
% want clear dash insert commands
\newcommand{\dash}{-}     % just a normal hyphen dash  "-"
\newcommand{\ndash}{--}   % n-dash "--"
\newcommand{\mdash}{---}  % m-dash "---"


%-------------------------------------------------------------------------------
%link new command names to the original font sizes,
%for easier to remember smaller font size
\newcommand{\vsmall}{\footnotesize}  % simpler to remember
\newcommand{\vvsmall}{\scriptsize}   %
%\newcommand{\vvvsmall}{\tiny}


%-------------------------------------------------------------------------------
% don't want outlines and crap around links in the document.
% notoriously annoying package.
% info can be found here:
% http://theory.uwinnipeg.ca/localfiles/infofiles/teTeX/latex/hyperref/manual.pdf
\usepackage[colorlinks=true,linkcolor=black,urlcolor=blue]{hyperref}


%-------------------------------------------------------------------------------
% I want the possibility of switching between different styles for skills and actions. Either have them as list items or subsections/subsubsections, but without toc entries.
\usepackage{ifthen}


%-------------------------------------------------------------------------------
% Below I will now define \openXlist and \closeXlist list commands
% as well as \skill{} and \action{} itemizers.
% Then they can be used in the docs as
%
% \openskillslist
% \skill{name1}
% \skill{name2}
% \closeskillslist
%
% \openactionslist
% \action{name1}
% \action{name2}
% \closeactionslist
%
% handy if I want to switch easily between item listing for the pdf version
% and headings for the html doc version.
% Just change the \setboolean{skillsaslist} to true or false.




% With skillsaslist defined we can use \ifskillsaslist{\begin{description}}
% as well as \ifskillsaslist{\end{description}} around the skill listings
\newboolean{skillsaslist}
\setboolean{skillsaslist}{true}

% Define the \skill{name} to make either of
%    \item[name]
%    \subsubsection{name}
\ifthenelse{\boolean{skillsaslist}}{\newcommand{\skill}[1]{\item[#1]}}{\newcommand{\skill}[1]{\subsubsection*{#1}}}

% Define the \openskillslist to either of
%    \begin{description}
%    -- nothing --
\ifthenelse{\boolean{skillsaslist}}{\newcommand{\openskillslist}{\begin{description}}}{\newcommand{\openskillslist}{}}

% Define the \closeskillslist to either of
%    \end{description}
%    -- nothing --
\ifthenelse{\boolean{skillsaslist}}{\newcommand{\closeskillslist}{\end{description}}}{\newcommand{\closeskillslist}{}}




% With actionsaslist defined we can use \ifactionsaslist{\begin{description}}
% as well as \ifactionsaslist{\end{description}} around the skill listings
\newboolean{actionsaslist}
\setboolean{actionsaslist}{true}

% Define the \action{name} to make either of
%    \item[name]
%    \subsubsection{name}
\ifthenelse{\boolean{actionsaslist}}{\newcommand{\action}[1]{\item[#1]}}{\newcommand{\action}[1]{\subsubsection*{#1}}}

% Define the \openskillslist to either of
%    \begin{description}
%    -- nothing --
\ifthenelse{\boolean{actionsaslist}}{\newcommand{\openactionslist}{\begin{description}}}{\newcommand{\openactionslist}{}}

% Define the \closeskillslist to either of
%    \end{description}
%    -- nothing --
\ifthenelse{\boolean{actionsaslist}}{\newcommand{\closeactionslist}{\end{description}}}{\newcommand{\closeactionslist}{}}




% With itemsaslist defined we can use \ifactionsaslist{\begin{description}}
% as well as \ifactionsaslist{\end{description}} around the skill listings
\newboolean{itemsaslist}
\setboolean{itemsaslist}{true}

% Define the \eqitem{name} to make either of
%    \item[name]
%    \subsubsection{name}
\ifthenelse{\boolean{itemsaslist}}{\newcommand{\eqitem}[1]{\item[#1]}}{\newcommand{\action}[1]{\subsubsection*{#1}}}

% Define the \openitemslist to either of
%    \begin{description}
%    -- nothing --
\ifthenelse{\boolean{itemsaslist}}{\newcommand{\openitemslist}{\begin{description}}}{\newcommand{\openitemslist}{}}

% Define the \closeitemslist to either of
%    \end{description}
%    -- nothing --
\ifthenelse{\boolean{itemsaslist}}{\newcommand{\closeitemslist}{\end{description}}}{\newcommand{\closeitemslist}{}}




% With spellsaslist defined we can use \ifspellsaslist{\begin{description}}
% as well as \ifspellsaslist{\end{description}} around the spell listings
\newboolean{spellsaslist}
\setboolean{spellsaslist}{true}

% Define the \spell{name} to make either of
%    \item[name]
%    \subsubsection{name}
\ifthenelse{\boolean{spellsaslist}}{\newcommand{\spell}[1]{\item[#1]}}{\newcommand{\action}[1]{\subsubsection*{#1}}}

% Define the \openspellslist to either of
%    \begin{description}
%    -- nothing --
\ifthenelse{\boolean{spellsaslist}}{\newcommand{\openspellslist}{\begin{description}}}{\newcommand{\openspellslist}{}}

% Define the \closespellslist to either of
%    \end{description}
%    -- nothing --
\ifthenelse{\boolean{spellsaslist}}{\newcommand{\closespellslist}{\end{description}}}{\newcommand{\closespellslist}{}}



%-------------------------------------------------------------------------------
% comment notes escaping list segment, if spellsaslists is set
\ifthenelse{\boolean{spellsaslist}} {
    \newcommand{\spellslistnote}[1]{
        \closespellslist \textit{#1} \openspellslist
    }
} {
    \newcommand{\spellslistnote}[1]{
        \textit{#1}
    }
}




%-------------------------------------------------------------------------------
% readoutloud italic quote sections
\newenvironment{readoutloud}%
{\begin{quote}\begin{itshape}}%
{\end{itshape}\end{quote}}%


% can't find ttshape, like itshape in "readoutloud" environment
\newenvironment{ttshape}%
{\begin{ttfamily}}%
{\end{ttfamily}}%




%-------------------------------------------------------------------------------
% temporary separation line
%\newcommand{\tmpsepline}{\rule[0.25\baselineskip]{0.5\textwidth}{0.5pt}}
%\rule[0.25\baselineskip]{0.5\textwidth}{0.5pt} =xtl-  0.25
%\rule[0.8ex]{0.5\textwidth}{0.5pt} =xtl-  0.8ex
\newcommand{\tmpsepline}{

    \

    \rule[0.25\baselineskip]{0.5\textwidth}{0.5pt}

    \

}





%-------------------------------------------------------------------------------
% \cleartoleftpage
%     open to an empty left page, so to fill two opposed pages
%     cleardoublepage opens to a right page (usually odd page number)
% https://tex.stackexchange.com/questions/11707/how-to-force-output-to-a-left-or-right-page
\makeatletter
\newcommand*{\cleartoleftpage}{%
  \clearpage
    \if@twoside
    \ifodd\c@page
      \hbox{}\newpage
      \if@twocolumn
        \hbox{}\newpage
      \fi
    \fi
  \fi
}
\makeatother
%-------------------------------------------------------------------------------



\usepackage{svg}


%-------------------------------------------------------------------------------
\begin{document}


% too many words in common across docs nowadays
%--------|---------|---------|---------|---------|---------|---------|---------|
%       10        20        30        40        50        60        70        80
%-------------------------------------------------------------------------------


% Manually specify hyphenation for names.
% Remember: space separated word list: lead with space
% hyphens can only occur on specified "-" characters
% words without "-" will never be hyphenated, overrides rules
\hyphenation{
 cam-paign-ab-il-ity
 Thing-a-ma-jig
 milli-fort-night
 Kings-land
 Evil-nius Conq
 Massa Pawa
 Maj-san Go-san Mal-vi-na
 Ho-her
 Bea-ta Blo-dig
 Leg-io Legu-ano
 Gammel-Tant
 Gam-ling
 Hem-ske-lina
 Go-blan-da
 Stur-Skurk
 Hjal-mar Hjäl-te
 Bur-mak
 Lund-qvist
 Grim-Gnash
 Gros-Orc
 muta-mon-ster
 muta-meat
 Iffy-Griff
 Hoo-man Hoo-mans hoo-man hoo-mans
 Da-ta-ri-an Ma-ras No-stro-mo
 Uchly Namen
 Edwin Chro-mo-phobe
 Star-Craft Brood War Brood-War
 Space Hulk
 Hermann Hammer-hand
}



% disables chapter, section and subsection numbering
\setcounter{secnumdepth}{-1}





%-------------------------------------------------------------------------------
% make a simple title page
%-------------------------

\begin{titlepage}
\thispagestyle{empty}

\begin{center}

   \vspace{4 cm} % Vertical space

   \textbf{\Huge{Gold and Glory}}

   \ % empty paragraph

   \textbf{\Large{rpg-ish tabletop hacknslash\\
             for dead tree and digital tables alike}}\\


   \vspace{2 cm} % Vertical space
   %\textbf{\Huge{image\\ goes\\ here}}
   %\vspace{1 cm} % Vertical space
   \includegraphics[width=120mm]{./figs/sneaky-plan.png}

   % make it look better in html rendering.
   \vspace{2 cm} % Vertical space


   \ % empty paragraph

   \large{
       fast and flexible game rules and setting\\
       for interesting tactical monster hack battles\\
       wrapped in old school silly and cliché role playing
   }



   \vfill % Puts what's below at the bottom of the page

%   \textbf{\Large{netjiro}} %\hspace{5 cm} \large{author2}}

   \ % empty paragraph

   %\large{Csol}

   %\large{General Academy of Bloody Murder}

   \ % Empty paragraph

   \normalsize{\today}

\end{center}

%\thispagestyle{empty} % No page number on title page.

\end{titlepage}




%-------------------------------------------------------------------------------
% copyright etc on the back side of the title page
%-------------------------------------------------
\clearpage
\thispagestyle{empty}
\raggedbottom

\vsmall
\noindent
This work is licensed under a Creative Commons \\
Attribution-NonCommercial-ShareAlike 4.0 \\
International License. (CC BY-NC-SA 4.0).\\
\url{https://creativecommons.org/licenses/by-nc-sa/4.0/} \\
\url{https://creativecommons.org/licenses/by-nc-sa/4.0/legalcode} \\
If you want to use it in any other fashion please contact the author.

\

%\noindent
%All images are temporary placeholders, \\
%most are downloaded and unattributed.\\
%They need to be replaced with licensed art.

\normalsize





%-------------------------------------------------------------------------------
% introduction and contribution notes on one sheet, front & back %---------------------------------------------------------------
\cleardoublepage                   % force coming text on a right hand page
\thispagestyle{empty}
\raggedbottom

\chaptermark{hello}
\chapter*{hello :)}
%\addcontentsline{toc}{section}{introduction}

The \textbf{Gold and Glory} rules and world setting have evolved sporadically since early 2008. I was working on a postdoc contract in Utrecht and wanted to stay in contact with friends Up North in Scandinavia for some occasional gaming.
For a while it had been apparent that we needed new arrangements to keep gaming at a reasonable frequency. Family and work obligations made it difficult to schedule our traditional full day rpg campaign sessions when people had small kids and lived spread out across different countries.

So I wrote this stuff. Easy to schedule, quick online gaming, fun lighthearted rpg adventures, and deep interesting tactical fights. Designed primarily to run with voice chat, virtual tabletop, and perhaps shared write and sketch space. It has also turned out to work very well, simplified, for playing quick tabletop fights and adventures, even with young kids. Much of the tactical depth that makes the fights interesting is emergent from the simplest basic rules.
It is not meant to compete with the real gather-together-and-eat-homebrew-guck pen and paper role playing experience. It's for short, fun, and easy to schedule gaming.

We play this for a couple of hours in the evenings, after people have put their kids to sleep. It quickly became a once-a-week fixture. E.g: Wed 2000--2200. Simple.

\

Back in the days we briefly tried a Warhammer Quest "campaign". It was fun, but quickly died due to the poor design and horrible balance. We played some 1st ed Descent, but it was slow and required gathering for a full day. The best of them is the old Space Hulk. It's fast and with the right opponent it can be very tactically challenging and interesting. That's what I want to capture, but with campaign play.

%Many years ago, in my pre-teens, I experimented with writing a very simple rule set and world setting, to use for a comedy campaign when we couldn't gather the full crowd for my real rpg campaign. That turned out quite well. It was fun despite the extreme simplicity of both the rules and the world.

\

I've split the rules into the basic fundamentals, and the rest. Just pick and choose for a good fit for your group. Find your trade off between speed and complexity. The basic stuff is very fast and straight forward, while the full set requires noggin engagement, or virtual tabletop scripting.

Baseline is simple enough for clever young kids. Full complexity requires quite a bit more since I have tuned it for my regular players who are mostly engineers and scientists.

Fast gaming is crucial for us since we only have a couple of hours each session. The battles must be tactically interesting and the role playing fun. Coordination, timing, positioning, complementary skills and battle maneuvers are very important. Just charging in for dps dice rally doesn't work. This makes fights interesting. The role playing aspects are written for comedy. Fun, cliché, ridiculous.



%-------------------------------------------------------------------------------
\clearpage
\thispagestyle{empty}
\section*{note: versions}
%------------------------
This is my personal notes package, not a draft for publication. There are still lots of stuff in progress of being updated, with inconsistencies, notes, TODOs etc.
I'm updating whenever I have the time and inclination. Use a good diff tool to find what has changed, been removed or added. Easiest is of course to go directly to the \LaTeX{} source, but a decent pdf diff tool is perhaps sufficient? If you don't have something lying around I've tested
\vvsmall(late 2017)\normalsize~
\url{https://draftable.com/compare}
and find it ... \emph{passable} ... for a simple 2-way diff, but lacking block move, 3-way, etc.


\section*{note: shitty writing}
%------------------------------
Most errors in grammar, spelling, style, etc are probably actual errors. However, I will somewhat consistently break some proper style rules, by intent. Always feel free to point out or fix my errors.


\section*{note: comments and contributions}
%------------------------------------------
Anything you want to change or add? Contact me! My handle is \emph{netjiro} on github and gmail. Or even better: go to github \url{https://github.com/netjiro/hactac} and check out the source. Make your changes and send me a pull request. Even small stuff like spelling, grammar, clarifications, etc are appreciated. If enough people have interest in it I will \emph{try} to dig out some time to improve the quality to something closer to an early draft for publication review.


%\section*{note: pronouns}
%%------------------------
%In Swedish we have "hen" which essentially indicates someone where the gender is irrelevant in the context. In English is s/he or they, which I don't like. Here I use "he, him, his". Just read it as whatever offends you the least, or run substitution, easy peasy.
%
%%And just to be unnecessarily clear on this. For me, \emph{personally}: in my mind a "dude" is also of irrelevant gender, man/woman/other. But a dude can't be boring.






%-------------------------------------------------------------------------------
% table of contents page(s)
%--------------------------

% Add a table of contents
%\pagestyle{plain}
%\fancypagestyle{plain}
\cleardoublepage                % force next printed page a right hand page
\thispagestyle{empty}

%\setcounter{page}{0}
%\phantomsection\addcontentsline{toc}{chapter}{contents}
\tableofcontents                % always compile twice if you have changed much
%\newpage
%\clearpage





%-------------------------------------------------------------------------------
% rest of the chapters in separate files
%---------------------------------------
%\fancypagestyle{fancy}
%\pagestyle{fancy}
%\setcounter{page}{1}
% set flushbottom fill layout for the main chapters
\cleardoublepage
\flushbottom


%\mainmatter  -- nope, not in article or report or book ? hmm ?
%\setcounter{page}{1}


% Add all the chapters
%--------|---------|---------|---------|---------|---------|---------|---------|
%       10        20        30        40        50        60        70        80


%===============================================================================
%
% rules
%
%===============================================================================


% force start on right side page
\cleardoublepage


% set fancyhdr heading
\chaptermark{rules}

% manually fix the table of contents and no numbering or "chapter" heading
\phantomsection\addcontentsline{toc}{chapter}{Rules}
\chapter*{Rules}
%---------------
Never let the rules get in the way of fun. If something is more entertaining but contrary to the rules, go for fun. That said, a functional and consistent implementation of rules is important as it allows players to develop a better understanding of the world, a gut feeling for the risks and possibilities, take greater calculated chances, find that cool edge, discover a fun loophole, powerful combo, etc.

The rule set is here to \emph{help} create a fun game experience, not to kill it by a thousand cuts and struts. Strike a balance that suits your group. 


\phantomsection\addcontentsline{toc}{section}{introduction}
\section*{fundamental introduction: what, why, how}
%--------------------------------------------------
The idea is to create more tactical elements in tabletop/digital rpg-hacknslash. Inspired by some concepts from classic tabletop games like space hulk, descent, etc, but striving for more interesting battles, and tying it together with role playing game style adventures and campaigns.

This game is not for everyone, but it is designed to be easy to scale and tweak over a wide range of preferences: In it's simplest form it works well as a small and fast framework for quick gaming with young kids, on a table or across the living room. At full complexity it is a very challenging online tactical tabletop game wrapped in a light hearted role playing game environment.
For examples of scaling: At the very simple end see 
\hyperref[sec:youngkids]{\texttt{Bitter Candy}}, page~\pageref{sec:youngkids} and 
\hyperref[sec:basicenough]{\texttt{Ominous Crown}}, page~\pageref{sec:basicenough}, 
and at the really complex side see the adventure 
\texttt{Turn off the Lights} and 
the Lundqvist force in the \texttt{Goblin Destiny} 
campaign.

It's primarily designed for fast, online, tactically deep, tabletop and rpg campaign gaming. And specifically to make it easy to schedule game sessions for busy people and still have interesting gameplay that requires some thinking. 
%\todo My players are mostly engineers and scientists, so the difficulty level can get quite high. 

Cooperation and coordination, movement, positioning, and clever use of character skills is much more important than commonly found in monster hack games of today and yesteryear.

The \textbf{Gold and Glory} rule set, game setting, and adventure campaigns are not created with any form of \emph{realism} in mind. The idea is to make it \emph{fun} and \emph{interesting} to play, nothing else.

Please, when reading and playing, keep you eyes open and head in gear. If the rules don't make sense, seem unbalanced, or should be more fun in a different way, \emph{please tell me about it}. Ping me an email or make a change and send a pull request. My handle is \emph{netjiro} on gmail and github.


\

\textbf{Game targets:}
\begin{itemize}
    \item Schedulable: Interesting even in short online game sessions.
    \item Allow GM and players to select the level of complexity.
    \item Flexibility and options, with consequences and tradeoffs.
    \item Fast, dynamic, flowing fights.
    \item Keep the rounds short and quick.
    \item Tactical depth, movement and actions matter, no dice rally.
    \item Terrain and layout matters, positioning, movement, timing matters.
    \item Many different viable combat styles and character types.
    \item Require player coordination for combat success.
    \item Offensive and defensive combat styles.
    \item Initiatives that favour the fast through higher flexibility.
    \item Numerical advantage is synergistic, not just linear dps.
    \item Action coordination is highly advantageous.
    \item Specialities of characters to have significant influence.
    \item Different skills and equipment should play and \emph{feel} distinct.
    \item Success and failure should not be binary all or nothing, but gradual.
    \item Freedom in character building, no classes or similar limitations.
\end{itemize}

\

\textbf{How to get there:}
\begin{itemize}
    \item Simple core rules that create inherent depth, optional addons for variety.
    \item Effects and options stack together and can be combined.
    \item The degree of success or failure have real effects, not just success/fail.
    \item Trade realism for interesting options and effects.
    \item Many fast rounds give more alternatives than a few long rounds.
    \item Doing more things in a round makes it harder to succeed with actions.
    \item Faster movement makes it harder to succeed with actions.
    \item Characters with high initiative can decide when to act in a round.
    \item Relative positioning, facing, direction gives mods to actions.
    \item Attack, defence, movement, damage, etc are multidimensional, not one value.
    \item Actually care about how probabilities works.
\end{itemize}

\

Just charging into the opponents and hacking away will lead to disaster and gruesome death. Instead there are tactical considerations that enhance survival and make the battles more interesting. Ganging up on an opponent to overcome defences, coordinate actions and positioning to create openings, or distracting him to allow the others to get in better hits, are all very powerful tactics. Moving and positioning to get the upper hand in a battle will make a world of difference, especially when fighting many on many, or in complex terrain and map layout. Always analyse the terrain when planning your next segment of the fight. Be ready to reassess the combat plan as the circumstances change. Fights are fluid and variable, not static line dancing with sharp sticks. The rules set the scene for flexible shuffling of offensive or defensive focus and tactical initiative. Coordination of timing and ordering of actions between characters will have a large impact on any battle. Smart positioning, timing, and coordination of actions can be devastating to the opposition if done right.

\

Normal quick skirmishes with around 10 combatants take perhaps a millifortnight or two and play out similar to this: \\
Heroes 2-5 attack monsters 2-10: \\
round 1-5: Initial positional movement for tactical advantage. \\
round 3-8: Contact: a few attacks, parries, duels. \\
round 6-15: Repositioning, movement, attacks, parries and counter attacks. \\
round 10-20: heavy violence and tactical movement to exploit weaknesses. \\
round 15-25: flights, chases, mopping up the stragglers, or flee to fight another day.

Larger or more complex fights of course take longer to complete. In hot areas several small skirmishes will meld into long rolling fights with positioning for control and advantage, contact, rallies, restructuring, support or withdraw, defensive short rest periods, and so on, as the Heroes try to accomplish their goals.
Really large fights with 50+ combatants can run upwards of 10-15min per round. This should be avoided. It's just a lot of dead time for most of the participants.

\

There are more rounds per battle than in most common games, but each round is short and quick. Exploring and clearing a small dungeon can creep into 50+ rounds, but can be done in an hour or two even when explaining the rules to a first timer. This has been demonstrated many times in the dreaded \texttt{Dungeon of Testing} scenario that has been used to introduce new players to the game.
%demonstrated in the introductory screencast

Standard encounters for a few players and twice their number in regular opposition take perhaps 20-30 rounds and 20-40min. More complex fights take longer. In the advanced \texttt{Goblin Destiny} campaign playthrough we had six players control two goblins each, plus sidekicks and allies, and there were usually 20-40 fighters on the board in most battles. Then the rounds can sometimes creep upwards of 10min. This is not recommended. Keep battles small and fast, as that is more enjoyable for the players.
% I consider the time consumption a design failure on my part of that specific campaign. I should have done it differently. But the players loved the campaign and reported it as one of the best so I have not done any significant changes to it.

Several times we've had dungeon crawl adventures stretch to hundreds of rounds with skirmishes and lulls. As long as each round is fast this does not take much game time. In some cases the GM can call for things like "take 3r" since he knows nothing significant can happen for a little bit. Or fast forward even quicker. I've found that 3r movement increments are good when exploring through fog of war, movement, facing, lighting, and vision.


\section*{simplified or complex}
%-------------------------------
The fundamental design idea is to allow the GM and players to choose which level of complexity they want for their game. The simplest basic rule set is very fast and small enough to keep in the head or with a small paper. Easily assisted with some markers or counters, and simple tokens or minis. Good for quick physical tabletop gaming sessions, and quite friendly to small kids as long as they can count to 10, add and subtract.

As shown in the 
\hyperref[sec:basicenough]{\texttt{Ominous Crown} adventure example}, page~\pageref{sec:basicenough}, and the 
\hyperref[sec:youngkids]{\texttt{Bitter Candy} recounting}, page~\pageref{sec:youngkids}, 
a very small set of rules and skills, with very simple implementation, is enough to deliver a fun, fast, and varied game experience. A set of fewer than 10 skills and spells gives enough variability and agency for the players to feel they are in control, and their choices matter. \emph{Really don't need much.}

Then the group can pick and choose the rules, skills, etc which they feel add interesting gameplay, increasing variation and complexity. In total this book has over a hundred optional skills, spells, maneuvers, and abilities, many of which drastically change how the game plays.

At some point it's better to have automation assistance by simple scripting in digital/virtual tabletop software. I have a maptool implementation that covers the majority of the rule set for quick automation for our week night internet based slaughter sessions.


\section*{more freedom, more options, fewer musts, deeper tactics}
%-----------------------------------------------------------------
This is not a class based game and does not require a specific typical party makeup to function. Look for synergies in game style and tactics, but there are no "must have" classes or professions in a party composition. Consider: There are many more dimensions to the fighting here than in most regular tabletop and rpg games, allowing for more flexibility and larger option space. Positioning and map control is very important. Player coordination is essential. Healing is slow and very expensive, not generally a mid battle possibility. Just these basic traits mean it plays very different from for example D\&D, and more like tactical tabletops such as Space Hulk and Descent.


\section*{balance}
%-----------------
The fundamental design allows for a wide variety of characters to be "balanced" in the sense that a lot of different build types can have similar impact on the adventure at similar xp levels, depending on what the adventure is about, what situations they encounter, opposition composition, etc. They will play drastically different though and will not be equally potent against all types of enemies and encounters.

When balancing encounters to your party you are in effect balancing your own play style of the monsters against your players' play style of their characters. The tactical depth of the game means it's next to useless to say that X of monster A and Y of monster B will be a good balance against four heroes at around 300xp. It totally depends on how evil and clever tactically you want, \emph{and can}, play versus your players. Again, the old Space Hulk and Descent games were totally different games if your Genestealer opponent had IQ 85 or 145, or if your Descent overlord was nasty, clever, and really knew the game versus just playing for coffee, cake, and the social time.

Games with a fixed balance, threat rating, etc, essentially and fundamentally says: \emph{No matter what you do, this is an approximate balance}. For me that's a red flag and a symptom of shallow game design.

When you start playing with your player group, you'll learn how they play and what opposition they can manage with their current Heroes, and adapt. You'll need to think for yourself here. That said, all adventures have examples of opposition composition, gameplay and playthrough examples available.

\

\begin{readoutloud}
Sitting in a bar: Bow-Lars, Hammer-Mike, and Silk-Tongue-Taylor. They've had a few beers and discuss how they have handled the typical goblin bandits they've met recently. 

Hammer-Mike stands up on the wiggly chair and gestures wildly as he loudly proclaims the benefits of hammering away goblins so they can't get hits into him as easily as his less lucky friend Sword-Stephen. R.I.P. The two of them quickly made stinky minced meat of a wall of goblins.

Bow-Lars leans back and softly starts explaining that it took a lot longer for him to solo the last bandit hideout. He had to run around, take shots when he could, and generally keep distance and speed to remain safe. But in the end he got it done with barely a scratch. But had the bandits had any good archers, or brains, he would have been in trouble.

Silk-Tongue-Taylor smiles and nods his head over to the dark corner of the inn. There sits a trio of goblins feasting on some flamed bird. "I just had a chat with the ugly one to the right. Bribed him with a small sack of gold, and he gleefully slit the throat of most of his gang for me. So now he and his two buddies tag along as hirelings on most of my expeditions. Really useful to have a sneaky stabby goblin around, you know.
\end{readoutloud}

\section*{let the players help}
%------------------------------
Bring the players into the decisions on rules, skills, spells, equipment, mutations, gear, monsters, bosses, etc. My todo list is hundreds of items long, many of which are ideas from players, and many of the skills, spells, items, etc are at least to some part suggested, built, or thought up by players.

Be careful with the combinatorial complexity. Adding things easily creates combinations that become game breaking OP. Usually it's possible to tweak and twist the details so it works, but sometimes it's just better to skip it. The game is simply not fun if there are \emph{obvious} selections and combinations. Always strive to present nothing that are simply and clearly better than already existing components without having downsides and tradeoffs.






%-------------------------------------------------------------------------------
%T E R M I N O L O G Y
%---------------------

\phantomsection\addcontentsline{toc}{section}{terminology}
\section*{Terminology}


\begin{description}


\item[The Almighty 1D10.] 
All skill and trait checks are done by rolling a D10 (1D10) (outcome [1-10], not [0-9]).\\
\textbf{Success level = skill - 1D10} \\
E.g: skill 6, roll 4 = success +2 \\
E.g: skill 7, roll 7 = success +0 \\
E.g: skill 4, roll 7 = failure -3 \\
success =0 is just barely a success \\
success +3 is a good success \\
success +6 is a very good success \\
success +9 is an excellent success \\
failure -3 is a bad failure \\
failure -6 is a very bad failure \\
failure -9 is a critical failure or fumble

\textbf{Optional:} A roll of 10 is always a failure, regardless of skill level and modifications. Not a fumble! This is to get more chance into the game, so that you cannot guarantee a certain behaviour even if you plan it well and have mighty skills.


\item[Modification, penalty, bonus (mod):] 
all the same, although penalties tend to be bad for you while bonuses are usually good. Modifications are additive to an action's chance of success. \\
E.g: skill 5 mod+3 has a chance of 8 on a D10 to succeed, skill 6 mod-2 has 4.

The total Accumulated Modification Stack / Action Modification Stack, (modification stack / mod stack / mods / mod / ams / ms), is the sum of all mods added together.

E.g: walk (mod-3) and below 66\% hp (mod-1) and declaring 5ap (mod-2) gives ams:mod-6. So with mod-6 even a fighter with sword 8 has only a chance of 2 on a 1D10 to hit.


\item[Actions and Action Points (ap):]
Attack, jump, avoid, climb, take, sneak, give, drink. 
Actions are most of the stuff the character actively does, except for regular movement. Actions cost action points, and most actions cost three action points (3ap).

Action points are declared in the beginning of each round. The more actions points you want to use the more mods you will have on your rolls.

Don't think of action points as units of time. It is not designed to map to time and is thus a poor analogue and \emph{will} lead to broken reasoning, contradictions, and paradoxes.


\item[Movement and Movement Points (mp):]
Each square the character moves costs movement points. Movement speed is declared in the beginning of the round: maneuver, walk, run, or dash, giving a set amount of movement points. The faster you move the more mods you will have on your rolls in that round.

Typically you can \emph{maneuver} (M) 1 square without any difficulty to your actions, or \emph{walk} (W) 3 squares with mod-3 to all actions that round, or \emph{run} (R) 6 squares with mod-6 difficulty, or \emph{dash} (D) 9 squares with mod-9.\\
So, trying to hit something with your sword, or calculate how to divide the gold you just stole, all while dashing away as fast as you can, is not going to work.


\item[Skill, spell, ability, power, mutation, psionic, maneuver, trait, etc:] gives the character the option of performing actions that he otherwise would not have access to, or improves the chances of succeeding with rolls he already has available, or changes the rules or how the rules apply to the character under certain circumstances. The terms are treated as the same thing. They often add their own rules to the game, which start giving the characters a lot of interesting flexibility and combinations. Some skills change the rules and logic of the game significantly.

Maneuvers are actions or variations of actions. They add alternatives and flexibility. Think of them as mini skills. Examples are special attacks, movements, etc, or bonuses in certain situations.
\\   \todo This is also the name of the slowest combat movement mode, sorry about that. I'll think of something else for this and rewrite later on... \emph{Any good suggestions?}


\item[Round (r):] is a measure of time, one game round. It does not translate to a specific number of seconds. Each round all characters usually move a bit and take one or several actions. All characters have their own turn during the round, but they can often take actions and movement during other characters' turns as well.

A round is split into the following phases: declaration, all turns, end.
During declaration all characters declare movement and action points. Then all characters take their turns. Then the round ends and book keeping is done to prepare for the next round.

Declaration and end phases usually just takes a few seconds each if the players are awake. Most of the time is spent on the characters' taking their turns to move and act, i.e. actually doing something.

Optional: Set a 10 second timer to limit dilly-dallying on declaration if you have players who can't make up their mind.


\item[Square (sq):] is used as a measure of distance, i.e. the length of a square on the game map, or more correctly the distance a character travels when moving between two adjacent squares on the game map. The game is generally played on square tiles, but works just as well on hexes or gridless maps with some very minor adjustments.

The sq distance on diagonals is usually measured 1-2-1-2-..., meaning the first diagonal counts as distance 1sq, while the second counts as distance 2sq.\\
In a few cases the sq is also used as a surface area.

When playing on gridless maps just define a unit distance as 1sq. E.g: 10 mm is 1sq, or 1 inch, or the length of the eraser you happen to have 10x of lying around, or the loop around your finger when creating movement speed twine strings and crossbow range measurement tapes.


\item[Base contact:] is when two characters are standing in adjacent squares, connected by a side or corner. Or if playing on gridless map; when the characters are within their base radius, or contact radius, or simply if the figures or character representation minis or toys touch.


\item[Round Down:] (rd) is the norm. When a value is written as x/y it's assumed to be rounded down to nearest integer unless otherwise stated. E.g: 10/3 = 3, 11/3 = 3, -10/3 = -4.
Unless explicitly written as "round up" (ru) or "round nearest" (rn) assume it's intended to round down. In some cases "round towards zero" (rz) is used, e.g: 11/3 = 3, -11/3 = -3.


\end{description}









%-------------------------------------------------------------------------------
%B A S I C   R U L E S
%---------------------

\phantomsection\addcontentsline{toc}{section}{basic rules}
\section*{Basic Rules}
%---------------------
We start with the bare bones basic rule set, and then expand. Even the very simple rule set is enough to build deep tactical game play.

The more you try to do in each round the more difficult each action will be, and the faster you move the more difficult things get. Movement and actions can be taken in any order and split up over a round. Higher initiative characters can go first, last, or interrupt a character with lower initiative. And when being the target of an action you always get one action for immediate response, regardless of initiative, which cannot be interrupted.

In the beginning of each round you decide how many action points and movement points you want to have available for that round. Unused points disappear at the end of the round.

With some skills you can try to use more points than you have declared, but you get penalties that continue through future rounds.


\subsection*{All effects, skills, modifications and penalties stack together}
%----------------------------------------------------------------------------
The general goal is that all effects and skills stack together unless stated otherwise. This allows powerful combos and and multiple characters can work together to accomplish mighty results. Some of the combination results can be a bit strange, but should hopefully add interesting depth and options.
All penalties and mods also stack together and will in severe situations make success of even the simplest actions a remote possibility at best.

Some combination effects might be bugs that have not yet been seen in game nor foreseen during writing and testing. \emph{Let me know if you find weird or broken stuff.}


\subsection*{All rolls are affected by the mod stack}
%----------------------------------------------------
All rolls: skill rolls, str and dex rolls, balance rolls, climb rolls, swinging a sword, shooting a bow, casting a spell, etc, are affected by the action modification stack / mod stack / ams. Unless it specifically says the roll ignores the action modification stack / mod stack / ams.

Some actions take no mods, or some other kind of modifications. E.g: jump, possum, etc. But there are actually quite few exceptions, and they are clearly marked and explained.

\

\todo I know there are exceptions I've forgotten to clarify. \emph{Please let me know when you find them!}


\subsection*{Simultaneous turns during rounds}
%---------------------------------------------
All characters and monsters take turns during every game round. Generally each combatant makes a short series of movements and actions. Usually one or two attacks and/or parries, and moves a few squares. In any order. Movement can be split up, and occur before, during, or after actions.

In practice, characters will be performing actions during other character's turns. This will in effect often yield simultaneous turns during a round, just with different action timing flexibility depending on relative initiative. Characters with higher initiative can interrupt the movement and actions of characters with lower initiative.


\subsection*{Fuzzy time}
%-----------------------
This rule set is built around a flexible and fuzzy outlook on time. Before and After is usually preserved quite well, but can fray at the edges when it includes more than two people.

This is not realistic, but it does make the fights more interesting. It adds tactical depth. Actions and movements are often carefully arranged within the rounds to maximise effects of cooperation, to take advantage of temporary opportunities or limit exposed weaknesses.


\subsection*{Movement}
%---------------------
All movement cost movement points, usually 1mp per map square. In the beginning of each round all characters declare how many movement points they want for the the round by declaring their movement speed: maneuver, walk, run, dash, further explained below. The faster they want to move the more difficult things will be.

Movement across square sides cost one movement point. Diagonal movement across a square corner costs one or two movement points depending on how many diagonal moves you have made before in that round. Every second diagonal move costs two movement points. I.e: 1, 2, 1, 2. Clear the count for each new round.
This assumes a square grid. Hex boards usually allow only side moves. On gridless boards use a direct measurement scale and round up to nearest whole movement point if necessary.

Diagonal movement around corners and between objects are usually restricted.
Objects or terrain with corners that leave a lot of space to the corner of the square can be treated as "round corners" Friendly Heroes also have "round corners".
Object corners that go mostly or completely to the corner of the square are "sharp corners". Enemy units also have "sharp corners".
Moving diagonally around a "round corner" is acceptable, while you cannot move diagonally around a "sharp corner". In some rare cases it is possible to squeeze diagonally between two objects when both have "round corners", but not when either or both have "sharp corners".

Some terrain types or obstacles are difficult to pass, and may cost more movement points. It is always the square you move \emph{from} that decides the cost of the move. E.g: Moving into a double mp shrubbery from flat grass costs only one mp, while moving within a double mp shrubbery costs double mp and moving out of a double mp shrubbery to flat grass costs double mp.

\todo perhaps change this to : the square you move \emph{to} that decides cost. Simpler to see, even if it reduces planning requirements.

If a character tries, or is forced, to move more than he has movement points to do he will fall down into the square he's moving to, going prone, and get a persistent mod-3 to the rest of the round and the entire next round. Skills like off balance, defensive step, etc modifies this.


%\small \begin{verbatim}
% -------
%|       |
%|       |
%|       |
% -------
%\end{verbatim} \normalsize


\subsection*{Taking Action}
%--------------------------
Actions include attacking an enemy, parrying an incoming blow, kneeling down, drawing a flask from a belt, casting a spell, or most other interesting things the Heroes might want to do. Moving is generally not an action in itself, but can be done before, during, or after an action, as part of that action, or separately.

Actions cost action points (ap). A normal action costs 3ap, but some are faster, and others are slower, affecting their ap cost. In the beginning of each round all characters declare how many action points they want to have available for that round. The more action points they choose the more difficult all actions get.

The difficulty modification is the difference between declared ap and the character's baseline action points. The mod is always negative or 0, you don't get a positive mod from declaring fewer ap than your base. E.g: having base ap of 4 and declaring 6ap gives mod-2, and declaring 3ap gives mod=0. The maximum ap a character can declare is his base ap + dexterity.

Most characters start out with base ap of 3, meaning they can do about one action per round without extra difficulty modifications. The characters can increase their base ap by training skills like \emph{quick}.

The ap declaration modification stacks together with mods from movement, etc.


\subsection*{Movement gives action penalties}
%--------------------------------------------
Movement speed is declared at the start of each round. I.e: maneuver, walk, run, dash (m,w,r,d). The faster speed you select the longer distance you can cover, but actions will be more difficult to perform. Movement mod penalties stack with all other penalties and set a baseline difficulty on the mod stack for the round.

\

\small \begin{verbatim}
Movement base modifications:
maneuver mod-0
walk     mod-3
run      mod-6
dash     mod-9
\end{verbatim} \normalsize

\

The faster move speed you declare the more movement points (mp) you get to spend. It depends on the character traits how many mp the different movement speeds give. The speed is also modified by skills and equipment.

\

\small \begin{verbatim}
Typical base movement: maneuver walk run dash (m w r d):
human fighter     m1 w3 r6 d9
dwarven warrior   m1 w2 r4 d6
elven bladesman   m2 w4 r8 d12
\end{verbatim} \normalsize

\

E.g: Heroic Herman wants to move 2mp . He then declares movement speed Walk and must take a mod-3 penalty to all actions that round, even if the action is taken before HH moves.

Movements can be split around actions. However, other combatants can often react to actions before character can move again. E.g: Heroic Herman declares run, moves 4sq, attacks Monster Mike but misses. MM then attacks back before HH can move away. HH chooses avoid and yield and then moves two more steps, now out of harms way. Until next round... This is the \emph{right to react} rule, (\hyperref[righttoreact]{see below}).


\subsection*{High initiative selects order of actions}
%-----------------------------------------------------
The order of different characters' actions during each round is decided by their relative initiative. Characters with higher initiative can choose when they want to perform movement and actions relative to the timing of opponents with lower initiative. They can go before slower characters, or wait and then interrupt others' actions when it suits them. The faster can thus also split their different actions to fall between the actions of slower characters.

Initiative is a dex+1D4 roll, re-roll every now and then, e.g. each separate battle encounter, or might change in battle lulls, or if automated it's tactically interesting to have initiative re-rolled every round of combat.

Characters with lower initiative must also declare action points and movement speed before characters with higher initiative.


\subsection*{The right to react}
\label{righttoreact}
%-------------------------------
All characters can react to actions being performed to them. This means that a slower character can choose to perform one action in response to actions being done to them by faster characters with higher initiative.
The right to react gives the right to perform one action, including movement before and/or after that action. The reaction action does not have to target the attacking character, but can be a non target action or target any other character.

The reaction action can be interrupted by characters with higher initiative, just like any other action, \emph{but not by the original attacker}. 
E.g: Attacking Astrid moves in and attacks Defending Doris. AA has higher initiative and can move and attack before DD can do anything at all. DD can then for example parry the attack with one action. Or if AA misses his attack DD can make a return attack under the right to react before AA can move out of reach or make another attack.

Right to react continues in a chain if the two keep attacking each other.
E.g: Attacking Astrid attacks Defending Doris but misses. DD attacks AA under right to react, but also misses. Then AA can attack DD if she wants, still under right to react. And so on until they don't attack each other any more due to lack of action points, other interests, moved out of combat, attacked someone else, etc.\\
Note: If AA attacks DD, hits, and DD successfully defends, there is no continuation within right to react. AA took an action and DD reacted. Finished.

% is this correct, or should a right to react chain always proceed at fixed initiative of the initial attack? There are edge cases hiding here...
Optional: If using the rule where spending action points decreases in round initiative it is possible that the participants drop in initiative order to the point where other characters can interrupt and intervene in a right to react action chain. However, the right to react by the target will not disappear even if a third party injects an action in the chain. The target still has right to react to the first action even after a second attacker has interrupted the response action and the target has taken a reaction action to the interrupt action.

Note: defence actions are treated slightly differently from other reactive actions. A defence action, e.g. parry, does not allow for movement or turning \emph{before} the attack-defence sequence is resolved. The defender can still move or turn as part of, during or after, the defence action but only after the attack-defend is resolved and completed at present location and facing.

%E.g: Attacking Astrid strikes Defending Dave. DD choose to parry + yield + move three steps away to negate further attacks from AA. In sequence: 1) the attack is rolled, and is successful. 2) the defence is declared, the yield bonus is applied to the parry, but DD is not moved yet. 3) the parry defence action is rolled, and fails. 4) damage is applied. 5) DD is moved for the yield maneuver. 6) DD is moved for the further movement. Note: that DD could not have chosen to move 3sq then skip the defence on grounds that by then AA would be out of reach. If DD had higher initiative than AA she could have interrupted AA's attack by moving away, but then it would have happened before AA spent the attack ap and rolled for the attack action.


%TODO: decide which resistance roll version to use
\subsection*{Resistance rolls}
%-----------------------------
% original [-5,4] version, which only have one roll :
In some cases it' not relevant to separate the action into separate offensive and defensive rolls. Instead one single \emph{resistance roll} is made, and called for as: "attack versus defence" or "xxx vs yyy".
% or "xxx | yyy", "xxx \vs yyy".\\
The attacker rolls \verb|attack - defence + 5 -1D10|. On success $\ge$0 the attacker has succeeded with the action, with the diff indicating the level of success as with a normal roll. Note that a roll of 10 does not mean an automatic failure in the context of resistance rolls.
%-----------------------------
% the 2018 dual roll [-9,9] version, has issues
%Resistance rolls. "attack versus defence",  "xxx vs yyy" or "xxx | yyy", "xxx \vs yyy".\\
%Both attacker and defender roll \verb|value - d10| and compare their differences. Best diff wins. The difference between the attack diff and defence diff is how good the success is. I.e. success diff is:\\
%\verb|(attack - d10) - (defence - d10) == attack - defence + d10 - d10|\\ If both attacker and defender have the same value this gives an outcome range of [-9,9], where the usual success=0 is a regular success, success+3 is a good success, success+6 is a very good success, and success+9 is an excellent or perfect success. Same with fail-3, -6, -9. Also note that a natural 10 roll has no "automatic fail" effect in resistance rolls.
%
%TODO: skip this ?
%In some cases a resistance roll is requested where the defender must or can spend action points only if the attacker actually succeeds with his part of the roll, i.e. \verb|attack - d10| $\ge$ 0.














%-------------------------------------------------------------------------------
% B A S I C   A C T I O N S
%--------------------------

\phantomsection\addcontentsline{toc}{section}{basic actions}
\section*{Basic actions}


Below is a list of some basic common actions.

\openactionslist


\action{Action: "attack"} make an attack, roll for weapon skill. If you succeed and the target doesn't defend you get to roll for damage and hurt'em.


\action{Action: "defend"} try to not get hit by an incoming attack. Defend is done as one of several different forms of defence actions, such as "parry" or "avoid". Defence actions are sometimes done together with optional action maneuvers like "yield" or "dodge" to improve the odds of success.


\action{Action: "parry"} to block an incoming attack, roll for weapon skill.
Parry is a form of "defend" action. Some weapons or attack maneuvers have todefend or toparry modifiers which can make the parry roll more difficult. Some shields or weapons have bonuses to their parry or defend actions.


\action{Action: "avoid"} an attack instead of parrying it. Twist, bend, or move a bit. This requires the "avoid" skill. Avoid is a form of "defend" action. Some attacks have todefend or toavoid modifiers which can make the avoid roll more difficult.


\action{Actions: "give \& receive"} hand over an item between characters. It takes one action to give and one action to receive. Both actions requires a dex+9 roll. Normally this will not fail but when the action and movement mods are piling up the chances can drop.
% give, receive, hand over, handover, relay


\action{Action: "draw, sheath/holster/stow/put away"} take something from inventory or put it back again takes a full round as long as the item is easy to reach and not stuffed in a container. E.g: drawing or sheathing a sword, hanging a bow, etc. It's an easy dex+9 roll. 


\action{Actions: "pick up, put down, drop"} are normal actions (3ap) with a dex+9 roll. Just dropping things takes no time and requires no rolls, but carefully putting something down and grabbing an item always does.


\action{Action: "drink a potion"} is a full round action (1r) and requires a dex+9 roll. This does not include drawing the flask from equipment, or putting it back. The skills quickdraw and häfva can drastically reduce the time required to draw and drink a potion.


\action{Actions: "crouch, sit, lie, stand"} all takes a normal 3ap action that requires no roll. Rising after falling or when off balance is a normal 3ap action with a dex+9 roll.


\action{Action: "jump"} an obstacle or chasm requires a dex roll, modified by distance and speed. A mod-3 for every square the character tries to jump over beyond the first. E.g: jumping over a one square chasm is a regular dex roll mod-0, a two square burning pit is a dex-3 roll, and a 4sq creek is mod-9.

Jump action does not take any negative modifications from movement speed. It instead gets mod +speed/3, where speed is the speed of approach: the number of squares moved in a straight line \emph{before} the jump, (\hyperref[approach]{see below}).
E.g: Dashing 7sq before the jump gives Flying Fritz a jump mod+2..

The jump movement also costs movement points as normal. If the character moves further than his max movement including the jump he will continue the jump but fall on landing unless he can go off balance enough to cover the difference.

On a fail -1 to -2 the character came up short and missed the mark but managed to grab hold with his hands. He now hangs from the ledge, if that is possible on the material and square, and must climb up next round. On a fail-3 or worse jump roll, the distance short is fail/3. Then the character is probably on his way towards a spiky/fiery/wet meeting further down.

The jump skill will improve the odds when jumping. A roll of 10 is not an automatic failure for the jump. The result diff decides. Note that the jump action roll ignores the usual movement mods, but it does not ignore other mods to actions such as having declared high ap, low hp, off balance, or other effects.


\action{Action: "climb"} allows to climb one vertical square per 3ap action and requires a normal dex roll. Fail -1 to -2 means the character is stuck this round. Fail-3 means slip down one square, and fail-6 means falling. Each climb action also costs one movement point (1mp) even if no movement is made. A roll of 10 is not an automatic fail for the climb. The result diff decides.

Climbing is modified by the difficulty of the wall. A regular cliff face or uneven castle wall is mod-0. A smooth castle wall is mod-3 or mod-6. A simple tree is mod+3. Equipment such as rope is also helpful, e.g. mod+3.

The climb skill will improve the chances of climbing. Climbing assumes the character has nothing in his hands. One hand occupied is mod-6. Heroes with extra hands get a mod+3 per extra hand. Carrying a heavy load makes climbing difficult. Each encumbrance gives mod-3 to climb.


\action{Action: "rummage"} is used to take something out of a container, like a backpack, sack, etc. It takes 2+1dX rounds: \\
One round to take off / open the container. \\
1dX rounds to find and take out the item. X = (number of items in the pack) / (perception) round up. \\
And finally one round to close and put back the container.


\action{Action: "stow/pack"} is used to put something back in a container. Takes three rounds: 1r take off / open the container, 1r put the item back, 1r close / stow the container.


\closeactionslist















%--------|---------|---------|---------|---------|---------|---------|---------|
%       10        20        30        40        50        60        70        80
%-------------------------------------------------------------------------------


\phantomsection\addcontentsline{toc}{section}{character sheet}
\section*{The character statistics sheet}
\label{sec:charsheet}

The character's statistics sheet holds the basic information describing the character, and determines his chances for rolls, actions he can take, and so on. Just like most rpg games. It is stripped down to be small and simple.

The specifics of rolling characters is explained in the "campaign" chapter.

When playing, always keep track of the "original" rolled stats of the character. Write the current stat first, summed up from the base original stat and all modifications from skills, abilities, equipment, etc. Then write the original base stat behind, in parenthesis. E.g: "hp 14(11)". This way it is easy to track and recalculate the current stat values. With a lot of skills and modifications, as might happen with experienced characters, it's good to have the baseline written down.

\

\goodbreak
\begin{samepage} \begin{verbatim}
===================================
name                        (token)
-----------------------------------
str          hp abs
dex          m w r d
con          stamina
int          vision arc
psy          mana
per          ap
cha          xp
----------
skills
----------
spells
----------
equipment
money ( g s c)
===================================
\end{verbatim} \end{samepage}


\subsection*{First column character stats}

\begin{description}

\item[str] Strength is the physical strength of the character. It determines how large weapons he can use, how much he can carry, what his chances are to prevail or loose in a wrestling bout, etc.

\item[dex] Dexterity is the character's balance, agility, coordination, precision, etc. It determines if he remains standing after a tackle, if he falls when running over the rubble, how fast weapons he can use, etc.

\item[con] Constitution is the general physical sturdiness and resilience of the character. It determines his chances of keeping running and fighting when getting tired, his chances of staying awake and alive when he gets lethally wounded, as well as his resistance against poisons, and the base distance he can travel in a day.

\item[int] Intelligence is the reasoning capacity, cleverness, schooling and general knowledge of the character. It determines his chances of figuring things out, putting clues together to reach a conclusion, recalling some piece of knowledge, etc. It also determines how easily he can learn spells.

\item[psy] Psyche is the mental strength and resilience of the character. It determines his chances to avoid panicking when faced with scary stuff, of maintaining control over his faculties when opposing wizards try to manipulate him, etc.

\item[per] Perception is the character's general alertness to his surroundings. It determines if he manages to spot hidden monsters and items, as well as incoming attacks from behind, etc.

\item[cha] Charisma describes how pleasant or trustworthy the character seems to be. It determines if npcs will like him or not, if henchmen will stick around, what info an evening in the ale house will yield, as well as his chances of acting and bluffing, etc.

\end{description}


\subsection*{Second column character stats}

\begin{description}

\item[hp] Hitpoints is the amount of damage the character can take before running the risk of falling unconscious or dying.

\item[abs] Absorption is the amount of damage that the character's skin, clothing, armour, etc will subtract from incoming attacks.

\item[m w r d] Movement: maneuver, walk, run, dash. These are the four movement speeds the character can use. It determines how many squares the character can move in one round, as well as the negative modification he takes to rolls that round due to movement.

\item[sta] Stamina is the endurance under high physical activity. It determines how long the character can move at high speed and how many strenuous actions he can perform before he has to rest.

\item[vision] Vision is the distance the character can see clearly, as well as the kind of light sensitivity he has. E.g: "20 dusk" means that the character can see 20sq clearly, and that his eyes are light sensitive enough to retain full sight under dusk-like brightness conditions.

\item[arc] Arc is the total angle of vision, centred around the facing of the character. I.e: the character sees arc/2 degrees to each side of his facing.

\item[mana] Mana is the magical energy the character can spend on casting spells and empowering magical equipment and such.

\item[ap] base Action Points determine how many action points the character can declare in a round before taking action mod penalties.

\item[xp] Experience points are the current and total experience the character has accumulated through his life. The current xp is the "xp pool" that can be used to buy new skills, use special meta skills, etc. Keep the total XP in parenthesis behind the current XP pool, e.g: 14(195). Some effects in the game are calculated from the total XP.

\end{description}

For simplified tabletop gaming the vision arc should probably be ignored or simplified to the following possible values: 90, 180, 270, 360\degree. Since that is quick to trace visually on a square board. On a hex board it's easier with 120, 180, 240, 360\degree. With virtual tabletop tools like maptool the actual 1\degrees resolution is sometimes supported.

%\
%
%With the basic rule set it is possible to run fast and simple combat with good tactical depth.


% Base stats should cover a lot of actions and activities
% Base stats should be orthogonal. If you have e.g: dex and five other values which depend on dex, then you still only have one value, displayed from six different angles. Just cosmetics. Pointless. 





\phantomsection\addcontentsline{toc}{section}{simple enough}
\section*{Basic is enough}
%-------------------------
\label{sec:basicenough}  % Ominous Crown
With just this basic rule set it's enough to run very fast but tactically deep tabletop fights, as well as fun lightweight role playing of course. We've used the basic simplified rules when playing with kids as young as five years old with little or no previous tabletop experience. Smart kids, but still. They just need to be able to count to 10, and add and subtract a bit.

For most basic tabletop scenarios I suggest to skip vision range and arc, stamina, pain, etc. All depending on what is actually relevant for the players' experience of the scenario, adventure, or campaign you intend to play.

Then choose a set of skills to be available to the players, again, only those that are relevant for the game experience of the intended scenario. But enough to allow for some choice that have meaningful impact on what the characters can do to influence the story and how the adventure plays out. See how little is actually needed for the example below.

\

\noindent Let's pretend you want to present your players with an adventure, \texttt{Ominous Crown:}

\begin{enumerate}

\item They start in the village, at the inn. They get hired to go rescue the merchant's daughter. She was kidnapped when playing on the field behind the store. The Heroes can find strange prints in the sand where she played, and follow them through the forest to a dark cave.

\item In the cave the Heroes fight a bunch of goblins, and rescue the girl. But they also find a letter from Lord Ominous ordering the goblins to the Ruined Temple in the Old Forest to find the Crown of Power. He will reward them with lots of gold and small babies to eat.

\item Travelling through the forest they encounter some random animal to fight, evade, or befriend. At the Ruined Temple they must solve a riddle and get past some traps and avoid waking the Undead Priest.
Taking the Crown of Power they also find an old legend, saying the crown is very powerful, granting the king's warriors superior strength. But the crown only works if worn by someone of noble birth, seated on a throne at a very special location. Reading the legend and knowing the history of the region they figure out that the special place is in the middle of the village square. Which means Lord Ominous must be planning to invade the town if he wants to use the crown.

\item Now they must either go fight Lord Ominous in his castle, or waylay him and his warriors when they travel to the village, or get the villagers to prepare for war and defend their home, or find someone else of noble birth to take the crown and help defend the village.

\item Depending on what they choose they now must prepare for the final epic battle. This could mean scouting Lord Ominous' castle, sneaking about and finding a way when he could be vulnerable to attack. Or reconnoitring the roads and prepare to ambush the Lord and his warriors. Or build palisades and train the villagers to fight. Or dig through the village archives to figure out that the long lost great, great, great, granddaughter of the last lord of the land is still alive and must be the blacksmith's adopted daughter.

\item Finish with a serious fight, win or loose depending on preparations and tactics.

\end{enumerate}

Let's see what's needed. Assume first time players, young kids, playing on a dinner table. Keep hidden map sections and fog of war, simplified to exposing rooms and critters as the players enter the various map segments, or simply build the map on the table as they go. Hence, skip vision and arc. For young players with no experience: skip stamina, pain, and low hp mods. For older kids I suggest you keep mods for 66\% / 33\% / 0 hp, and death at negative constitution. This is a good dramatic signal when things start to go wrong and it requires minimal book keeping. Keep mana but skip arrow counter, to emphasise the power of magic, and it's fun to have a small heap of mana tokens on the character sheet for the player to fiddle with. Skip off balance but perhaps keep yield, another good dramatic signal. Skip encumbrance, skip food and drink other than for storytelling flair. Assume they have tent, bed rolls, and a cook pot for the evening camp, and torches when they enter the old ruins. Perhaps they have a donkey, old and friendly, named Goober? He'll kindly wait for them outside the temple when they go explore.

For a very first game, just create a set of archetype Heroes for the players to choose from. A few more than needed. Choose a small set of skills that give the players different options depending on which skills their Heroes have. For the simplest playthrough it's perhaps enough with just a few weapon skills: sword, shield, axe, spear, bow, avoid, and some simple magic spells: black bolt, fire wall, heal. This would be enough to give the players plenty of options for a group of Heroes to hack their way through the campaign. Just assume they can read, everything is written and spoken in Common, and so forth, the Castle has vines growing over the wall at some point if they look, etc. Choose a limited weapons set to showcase differences: 1h sword + shield, 2h axe, 2h spear, bow. So the players get a distinct difference between the relative strength of offence and defence, reach, and ranged fighting. And it will force them to fight better together and find tactical solutions to their respective weaknesses.

If the players figure out or remember to get a piece of equipment or a skill that could potentially be useful for the adventure, let them get it, and see if you can whip up a situation where it turns out to be useful. Makes the player feel good.

\

\noindent This very small set of rules and skills is enough for a great tabletop roleplaying experience with young kids or adults new to tabletop rpgs.


\subsection*{A little bit more?}
%-------------------------------
To give the players some more options and tradeoffs from the beginning, 
add: track, sneak, climb, literate, histography, dungeoneering, gossip, animal control and perhaps the spells speak to animals and illuminate history. This can allow, or limit, them to find alternative solutions and extra information, and force them to see that some solutions would be possible if they had chosen differently for their Heroes. For most styles of play that is important! The Heroes are not all powerful, the players must find ways to reach goals while \emph{limited} by their Heroes' abilities. 

Perhaps the animals of the forest can warn them about the undead priest in the Temple ruin, or tell which route Lord Ominous' warriors take when they travel. Dungeoneering can help them with temple traps and basic info on Lord Ominous' castle. Histography and literate explains the legend and the throne location to the middle of the town square. Gossip and literate find the long lost noblewoman. Sneak and climb give access to Lord Ominous' castle.

Here I would also suggest to keep stamina as part of the character management. It quickly adds a whole new dimension to the game and is not difficult to manage. It could also be a good idea to look into the difference between one handed and two handed weapons if that was skipped in the simpler version.

\

\

\todo add example summations of the two variations of simplifications.

\

\

With full complexity and some tweaking this little adventure can become a tricky mini-campaign. 
















\clearpage
%--------|---------|---------|---------|---------|---------|---------|---------|
%       10        20        30        40        50        60        70        80
%-------------------------------------------------------------------------------
% M O R E   R U L E S
%--------------------
\phantomsection\addcontentsline{toc}{section}{more rules}
\section*{More rules}

More, optional, rules. Ignore or use as you wish. Most extra rules can be picked up in isolation, but some depend on others to make sense. Pick and choose to find what suits your group and game style.


\subsection*{Fumble, fumbly, fumbles}
%------------------------------------
\emph{I actually don't use this rule in my games!}

If a rolled 10 on the Almighty D10 is a failure then roll again for a fumble check: if the new roll is higher than the chance to succeed it's a fumble! Think up some way the fumble plays out: It must be interesting \textit{\textbf{and fun!}} Don't just say it's a fumble and it's always something bad and boring. That just sucks and doesn't bring much to the game.

\textbf{Optional:} 
Perhaps a 10 + fumble roll is a regular fumble and a 10 + 10 roll is a critical insane fumble. A 1 in 100 chance is rare enough that you can go for excessive effect.

%jiro: I think it is fun. I've tried it. It can be hilarious. But it takes more time than I feel it gives back in fun.


\subsection*{Perfect, excellent, exceptional}
%--------------------------------------------
\emph{I don't use this one either!}

If a roll sequence of 10 + fail is a fumble, then perhaps a sequence of 1 + success is a perfect success? Prefer to find a \emph{fun} way of presenting it. Involve the player? Let him describe it. 

\textbf{Optional:}
A 1 + 1 is an exceptional rare success. Again, a 1 in 100 chance is rare enough to bring out all the stops and go excessive.

%jiro: same as fumbly, poor fun/time trade.


\subsection*{Declaring Extra Action Points raises initiative}
%------------------------------------------------------------
Each ap you declare above the character's base ap will give initiative+1. 
E.g: Fast Fabian has max ap 5 and declares 7ap for the round, raising his initiative by 2.


\subsection*{Taking action lowers initiative}
%--------------------------------------------
And when taking actions, each ap you spend will decrease initiative by 1. The initiative is decreased when the action is completed.
E.g: Making a 3ap attack decreases initiative by 3 after the attack is completed.


\subsection*{Fast Movement raises initiative}
%--------------------------------------------
The faster movement you declare the higher your initiative becomes. This is to make it easier to move and regroup which creates more flexibility in skirmishes.

\

\begin{verbatim}
Modifications to initiative from declared movement:
maneuver +0
walk     +3
run      +6
dash     +9
\end{verbatim}

\

E.g: Flightful Fred sees Mauling Morgan approaching. FF has initiative 7, MM has initiative 9. FF must declare his movement before MM since he has lower initiative. FF chooses to declare run and gets a +6 to his initiative resulting in 7+6=13. If MM wants to be able to move into strike position first, he must declare a run as well, since a walk (9+3=12) is not enough to move in before FF runs away.


\subsection*{Moving lowers initiative, just like actions}
%--------------------------------------------------------
\label{sec:movinglowersinitiative}
Each initiated movement, not each movement point, reduces initiative by 3. E.g: Running Runar has declared speed R, and gotten a initiative+6 bonus for it. He takes a small 2mp movement and his initiative drops -3. He takes another 3mp and his initiative drops another -3. Taking his last 1mp his initiative drops another -3. If he had taken only one movement, of 1-6mp, his initiative would only have dropped by -3.


\subsection*{Streamlining initiative list: waiting}
%--------------------------------------------------
When there are lots of Heroes and NPCs on the map things tend to get chaotic, especially when high initiative characters choose to wait until some with lower initiative have acted. This leads to having to loop through the upper parts of the initiative list again and again while only the lower initiative characters are acting for the most part.

To get away from this problem we add:
\begin{description}
\item[wait on other] will set the initiative of the character to one lower than the target he's waiting for. It does not cost any AP. The waiting character moves down the initiative list. The waiting character must be able to see the target he's waiting on.
\item[wait] will retain the initiative but costs one AP. This flags the character with a clearly visible "wait" marker in the initiative list and makes it clear he's ready to jump into action at any time. The player must clearly and quickly speak up when he wants to take action again.
\item[done] flags that the character is more or less done for the round, unless something happens to him. The character's initiative is set to 0 and he moves down the initiative list. Mark the character with a clear "DONE" marker.
\end{description}

With this it's possible to enforce a rule where the top initiative always have to take some action or choose a wait state. The Hero can wait on a target for free, but sacrifices initiative. Wait flexibly, without target, costing 1ap, or flagging as done but can only take reactive actions if the situation around him changes significantly.

For battles with 30+ characters involved this will speed up the process significantly.

The GM must decide what constitutes a \emph{significant change of situation} for a done character. A good rule of thumb is that the character can take reactive defensive actions and movement only, and no offensive actions, counter attacks, pursue targets, initiative interrupts, etc.


\subsection*{Actions that do not require rolls}
%------------------------------------------------
Some actions do not require any rolls. In that case they always succeed. However, in some situations there might be modifications stacked against them. In those cases the roll-less action is considered a "careful 10", meaning it has 10 of 10 to succeed as base chance, and does not fail on a rolled 10.

E.g: The revival powder can be used to bring a fallen comrade back to life. However, if the recently deceased character had the "Hand of Deity" ability he mods all resurrection attempts with mod-3. In this case the revival powder is considered to have a "careful 10" as base chance of success, and will thus have a chance of 7 to succeed in reviving the unfortunate dead.


\subsection*{Modified resistance rolls}
%--------------------------------------
Resistance rolls are sometimes done as a second step of an action roll. In these cases the resistance rolls are sometimes set to be modified by the success diff of the initiating action.

%TODO: rewrite domination/dominate to instead use skill check then psy check each round.
%TODO: rewrite this example, don't use domination:
E.g: Magus King rolls for "domination" against minion Weakie. Domination calls for a skill roll + mod psy vs psy. So, Magus King has domination 6, rolls a 4, which gives diff +2, i.e. he succeeds with +2. That +2 is now used as mod for the psy vs psy resistance roll. Magus King has psy 6, minion Weakie has psy 4. Thus the modified psy vs psy resistance roll is now 6+2 vs 4, i.e. 8 vs 4.\\
% vs version: attack - defence +5 -1d10
Magus King rolls mod psy vs psy : \verb|8 vs 4 : | \verb|(8 - 4 +5 -1d10)|, rolls 3, wins the resistance with +6. Minion Weakie is now under his control.
% vs version: (attack-1d10) - (defence-1d10)
%. So Magus King rolls \verb|8-d10| and minion Weakie rolls \verb|4-d10|: Result \verb|8-9 = -1| and \verb|4-7 = -3|. Diff \verb|-1 - -3| is success+2, and minion Weakie is now under his control.

%A mod resistance roll macro for maptool: \\
%\verb|/roll 5+mod+attack-defence-d10| \\
%A result >=0 is success, <0 is failure. \\


\subsection*{Declaring defence}
%-----------------------------
A defender can wait until he knows the success or fail roll result of an opponent's attack before declaring a parry or counter attack. This way the flow of the battle shifts every now and then, as offensive players "leave themselves open" to counter attacks.

The attacker who just missed and is now under counter attack can spend a new action parrying the incoming counter attack.

Defence action must be declared before damage is rolled, however.


\subsection*{Activating a friend}
%------------------------------------
A character with high initiative can spend an action to activate a character with slower initiative, e.g. shouting at a slow friend to get moving, or shoving a friend aside, out of harm's way. This triggers "The Right to React" just as if the fast character was attacking the slower one.


















%\phantomsection\addcontentsline{toc}{section}{movement \& facing}
%\section*{Movement \& Facing}




\subsection*{Moving}
%-------------------
Moving a character is not an action, but it follows the rules of initiative. Movement can be done by itself or as part of an action.

Optional: \emph{Stutter Step:} To avoid players making small incremental moves without actions the GM can decide that movement together with an action is free, but movement without an action costs 1ap. And if players cleverly start using free 0ap actions to trigger movement the GM can force a cost of 1ap there as well.

Optional: Moving lowers initiative, \hyperref[sec:movinglowersinitiative]{see section Moving lowers initiative}, page \pageref{sec:movinglowersinitiative}. Each initiated movement path lowers initiative by 3.


\subsection*{Facing}
%-------------------
On a square grid cardinal and diagonal facing is used: N,NW,W,SW,S,SE,E,NE.
On a hex grid only the six main faces are used: 0,60,120,180,240,300\degree.

The facing of a character is generally the same as the direction of the last leg of the previous movement per default, but can be changed by turning. E.g: Moving Morgan is starting his movement path with 2 steps to the north, then places a waypoint, and then heads three more steps east. His facing at the end of the movement is then east by default.

Maneuver movement is in any direction, and the character does not have to be facing in the direction he is moving. For walk, run, dash the character will be facing in the direction of movement.

Note that turning to any facing is generally free when done together with movement. E.g: Finishing the walk facing east, then just turn the character facing north without spending any extra movement or action points as part of the move, before doing anything else.

It's not free to move then take action then turn for free. In that case the turn is separate or together with the action and may cost ap or mp.


\subsection*{Turning}
%--------------------
Turning can be done by either taking actions, or by spending movement points. Turning follows general initiative just like other movement or actions.

Turning to a specific facing directly before, during, or after a movement is free and considered part of the movement. \\
Spending one movement point separately also allows to turn to any facing, and can be done apart from any other movement or action. \\
One can also turn by spending action points. The cost is dependent on the amount of facing change: \\
Turning 45deg is free with most actions, before or after the action \\
Turning 90deg costs 1ap \\
Turning 135-180deg costs 2ap \\
Certain armour and equipment will make it more costly to turn.\\
On a hex grid 60deg is free, 120deg is 1ap, 180deg is 2ap.

\emph{Note:} Defensive actions such as parry and avoid do not allow a free 45deg turn \emph{before} the action. Offensive action generally do allow a free 45deg turn as part of the action done before \emph{or} after.


\subsection*{Flanking attacks}
%-----------------------------
Attacks from the side or back are more difficult to defend against. Attacks incoming at 90deg from the character's facing are mod-3 to parry. At 145deg it's mod-6 and at 180deg it's mod-9.\\
For hex boards: 60\degrees is mod=1, 90\degrees (reach) is mod-3, 120\degrees is mod-6, and 180\degrees is mod-9.

The target can not change facing before taking a defensive action. Not even if that action is avoid + yield. He can generally take a free 45\degrees turn together with the action after the defensive roll, or pay extra ap or mp for a larger turn, or together with any movement \emph{after} the defence roll.

The target is also not allowed to spend ap or mp before a defensive action unless he has a higher initiative, since the right to react only allows for one activation, where the defend (parry, avoid, etc) must come first.

A target with higher initiative than the attacker can choose to interrupt the attack and turn to face the attacker any way he chooses to negate the mod, but that must be done before the attack is rolled as usual.


\subsection*{Attacking from outside the arc of vision}
%-----------------------------------------------------
When attacking from outside the target's vision cone, but within the target's perception range, the target must roll for perception:

\noindent Success: The target is aware of the incoming attack. If he has higher initiative he can interrupt the attack with his own movement and actions.

\noindent Fail: The target is not aware of the attack, he cannot interrupt, and he has mod-3 to any defence against it.

\

If the attacker is successfully sneaking, his sneak success modifies the perception range and perception roll of the target. If the target fails his perception roll against a sneak attack he is unaware of the attack and can not make a defence under right to react or interrupt with higher initiative. Even if unaware the target still has the right to react after the attack, and can move and take actions as usual.


\subsection*{Attacking to the side}
%---------------------------------
Attacking forward or diagonally forward, within +/- 45deg from facing, is normal and carries no modification. \\
Attacking an enemy to the side ($\ge$90deg) is mod-3. \\
Attacking an enemy diagonally back ($\ge$135deg) is mod-6. \\
Attacking an enemy directly behind (180deg) is mod-9. \\
The enemy must be within the cone of vision for the attack to be possible at all. \\
These are the same mods as for parrying incoming strikes from the sides or behind.

On a hex grid it's: 60deg mod=1, 90deg mod-3 (reach attack), 120deg mod-6, 180deg mod-9.


\subsection*{Partial squares \& difficult terrain}
%------------------------------------------------
A square that is only partially clear is still possible to stand on, but will give a mod-X to actions performed while standing there. The typical mod for a partially blocked square is mod-3, while severely blocked squares can be mod-6 to mod-9 depending on surroundings and situation.

Some squares contain difficult terrain, such as rubble, debris, etc. These squares also give mods to all actions performed while standing or moving through the area. E.g: Light rubble and undergrowth mod-3, heavy debris or marshland mod-6, etc.

Moving through difficult terrain or partially blocked squares require a dex+9 roll, modified by ams as usual and the terrain modification of the square.

Attacking around a blocked corner gives mod-3 to the attack. Enemies are not considered to have blocked "sharp" corners for this situation.


\subsection*{Crawl spaces and tight tunnels}
%------------------------------------------
Severely blocked squares may not allow for normal actions and movement, but will only allow for crawling through. Crawl speed is dex/3 sq/r, round down but minimum 1sq/r.

Small character such as goblins and halflings will be able to run or dash in areas where larger characters can only maneuver or walk. 
Common narrow tunnels will allow movement as follows: \\
Humans and orcs can only maneuver. \\
Elves can walk. \\
Dwarves can run. \\
Halflings and Goblins can dash.


\subsection*{Small characters can pass friendly diagonals}
%----------------------------------------------------------
Goblins and Halflings are considered small characters and can pass through a "friendly diagonal", i.e: a diagonal between two friendly characters which are considered to have "round corners". 
Normal size characters such as humans, orcs, dwarves, elves, etc cannot.


\subsection*{Forced movement, falling, off balance}
%--------------------------------------------------
Forced extra movement, beyond declared, causes the character to fall down and go prone, and forces a mod-3 for the rest of the current round and the full next round. Rising from a prone position takes a normal 3ap action. Falling down can be avoided by purposefully going \emph{off balance}, a maneuver skill which most races have from the start. Controlled falling is done by \emph{face plant / dive} and avoids the persistent mod-3.

When a character goes off balance he immediately suffers a mod-3 for the rest of the round. The off balance mod also persists for the next round. E.g: Hero Albin has spent all his movement to get into attack position on Monster Boo for next round. But then, \emph{supplies}, Monster Boo attacks and Hero Albin wants to "avoid + yield". To do that he must spend one extra movement point which he does not have for the final move of the yield action. This will cause him to go off balance, incurring the mod-3 which will also persist for the next round.

Note that the off balance penalty kicks in after the parry + yield roll in the example above. Otherwise most situations would not gain any actual benefit from going off balance when defending and yielding since the off balance penalty would cancel the yield bonus.

It's possible to have multiple off balance if the character purchases the skills.
When the character tries, or is forced, to move beyond his off balance level he will fall down, just as if he didn't have off balance.


\subsection*{Speed \& distance of approach}
\label{sec:approach}
%-----------------------------------------
Some skills/actions like jump, tackle, charge have bonuses based on the speed/distance moved in a straight line just before the action.

If the character makes a turn in the move it's only the final straight line that counts. It's also the distance covered, not the movement points spent, that determines the speed/distance of approach. E.g: moving 3sq on rough terrain costing 2mp/sq is still approach speed of 3 not 6.

Normally the bonus from speed/distance of approach is the distance/3.











%\phantomsection\addcontentsline{toc}{section}{action duration}
%\section*{Action durations}




\subsection*{Fast and slow actions}
%----------------------------------
A normal action takes "one action (1a)" time to perform and costs 3ap. Some actions cost more, some cost less.

\noindent
A very fast action costs 1ap (vfast / fast+2) \\
A fast action costs 2ap (fast / fast+1) \\
A normal action costs 3ap \\
A slow action costs 4ap (slow / slow-1) \\
A very slow action costs 5ap (vslow / slow-2)

\noindent
Under special circumstances some actions are 0ap instant actions and take no time at all.


\subsection*{Actions that take a full round}
\label{sec:fullroundactions}
%-------------------------------------------
Throwing a javelin takes a full round, that means the character can take no other actions that round. He will fire in order of initiative. A high initiative character can choose to fire before or after a low initiative character.

A full round action is considered to take all the ap the character can declare without incurring extra action mods. E.g: The Hero had max action points 3 when rolled up, then bought quick 2, giving him a base ap of 5. For him all full round actions will take 5ap. This is mostly irrelevant but important for some skills and situations.

In some cases it's relevant to calculate divisions of rounds, in which case a round is generally considered to be three actions (3a), and nine ap (9ap). 
%Hence a super fast character ($>$9ap) can in some cases be considered to be able to take more than a full round action per round.

%TODO: change example
%E.g: fast magic 3 allows to cast spells in half the time. A one round spell would then take 5ap.

\

Unless otherwise specified, any action taking 0-9ap can be expanded into a 1r action, regardless of what base ap the character can declare. E.g: Slow Sune has base ap of 4, and want to perform a strange ritualistic wavy-arms-move requiring 6ap. Instead of declaring 6ap and taking mod-2 for the action, and round, he simply expands 6ap action into a full round action, declaring his normal max 4ap, and taking no mods.

For actions taking over 9ap it's possible to expand into multiple round actions:\\
\verb|rounds = round up (ap / 9)|


\subsection*{Actions that take more than one round}
\label{sec:multiroundactions}
%--------------------------------------------------
Some actions take one or more rounds. Spellcasting is a typical example. Those actions retain the maximum mod penalty that they have taken through all the rounds during the performance. The roll is done at the end though, just before the action comes into effect.

E.g: Aiming the arrow. Assassin OilySnake is reloading and aiming his crossbow for three rounds before firing. During these rounds he better be sitting very still, since otherwise he will be taking movement penalties. He also should not take any other actions during this time since that will require him to declare more ap than he can do without modification, and thus incur an action mod. So, in the end of round three he fires, rolls a 5. He has crossbow 4 (shitty assassin), aim mod+1. He succeeds =0. If he had moved much or taken any other actions he would have missed.

E.g: Wizard Willful is casting StaffLight which takes him 5r. He maneuvers slowly the first three rounds, busy building the spell. But in round four he walks and gets a mod-3. In the last casting round, round five, he is again maneuvering slowly. So at the end of round five, he rolls his casting StaffLight casting skill roll, but with a mod-3 for the walking in round four. He has StaffLight 7, rolls a 6, fails -2, and as the spell fizzles out he curses the damned goblin that forced him to move too fast.












%\phantomsection\addcontentsline{toc}{section}{stamina}
%\section*{Stamina}




\subsection*{Stamina}
%--------------------
Stamina is a measure of how long the character can keep up with high power activity, such as hard battling, running, etc. When the stamina runs out the character will have problems performing taxing actions and movements.

Each attack generally costs one stamina point. \\
Defence actions cost no stamina.\\ 
Maneuver movement speed costs no stamina.\\ 
Every round you walk costs one stamina point. \\
Every round you run costs two stamina point. \\
Every round you dash costs three stamina points. \\
Some skills and maneuvers also require drawing stamina.

The characters generally regain one stamina at the start of each round. If stamina reaches 0, you must start rolling for con for every action that requires drawing stamina. If the con roll fails the character can just stand there catching his breath, still loosing the action ap anyway. If the con roll succeeds the character can perform the action, but will go into negative stamina.

When you roll con for stamina draw check the con roll is penalty modified by the negative stamina, ignoring the mod stack.

With negative stamina, all actions are also modified by the negative stamina, above the normal mod stack.
E.g: Wheezy the Dwarf has been battling heavily and is now at stamina -2. He wants to makes an attack, costing 1 stamina. His has con 7, and must roll con 7 mod-2 =5. He succeeds and can draw 1 stamina. The action is performed at stamina=-2 though, it costs 1 stamina to perform, drawn after the action is completed. He must thus roll for his attack as usual but with an extra mod-2 for the low stamina. Luckily he succeeds. Now he has stamina -3. He decides to do a second attack the same round, and must now roll a con 7 mod-3 =4 to be able to attack or he is too tired. He rolls a 6, a fail-2, and just stands there panting for 3ap instead of smiting the giggling goblin.

A character regains 1 stamina at the beginning of each round regardless of declared movement and activity. Resting characters regains 1 extra stamina each round if they pass a con roll with all normal mods. Since rest means 0ap 0mp the main mods will be pain, low hp, and such. 

Some attacks and actions cost no stamina for every second use, or every third, or every second and third in sequence or round. This can be written as \verb|stamina(-1,0,0)| meaning that the first use in a sequence costs one stamina, while the second and third costs no stamina. E.g: if a \verb|stamina(-1,0)| action is done three times it costs -1, 0, -1 stamina.

%TODO: remove the "con/3 extra stamina" regain when resting. Rests too fast.
%A resting character regains con/3 stamina each round. 

%DONE: the low con regain is currently in the maptool code 191209
%Characters with low con regain extra stamina when resting slower than normal:\\
%con=2 regains 1stam/2r\\
%con=1 regains 1stam/3r\\
%con=0 regains 1stam/4r\\
%con<0 does not regain extra stamina when resting.











%\phantomsection\addcontentsline{toc}{section}{carrying stuff}
%\section*{Carrying stuff}




\subsection*{Encumbrance}
%------------------------
If a character carries too much stuff he gets encumbered and will have modifications to movement and actions.

\noindent 
Encumbrance is calculated as: \verb|encumbrance = total weight carried / str.|

E.g: Trekking Tom has str 6, and carries a total load of 4.0 weight. His encumbrance is then 4.0/6 = 0.66 which is rounded down to 0. He thus has no encumbrance modifiers. When he then packs on another 5.0 weight he has encumbrance 9.0/6 = 1.5, rounded down to encumbrance 1. With another 4.0 weight he has encumbrance 13.0/6 rd to 2. He will then have a mod-2 encumbrance penalty which will affect actions and movement.

\

\noindent
Each full encumbrance gives modifications: \\
All actions are mod-1 per encumbrance \\
Dash speed is -1 sq/r and +1 stamina per encumbrance \\
Run speed is -1 sq/r and +1 stamina per encumbrance/2 \\
Walk speed is -1 sq/r and +1 stamina per encumbrance/3 \\
Maneuver speed is -1 sq/r and +1 stamina per encumbrance/4 \\
Long distance travel is also reduced by 1 per encumbrance. \\

Also, stuff has to be carried somewhere. Item slots are hands, one each, shoulder and back, three slots: slung left, right and centred. Some belts have carry slots. Some sheaths and quivers can be fastened along arms or legs, etc. Shields can be carried on an arm or outside a backpack.

Containers are useful in that they only take one slot, while providing several slots for items. Containers are also good in that they can reduce the enc of items. A good backpack halves the enc from all items in it.

Some actions like climb, swim, etc take heavier mods from encumbrance.


\subsection*{Carrying heavy objects}
%-----------------------------------
Some objects require a certain str to be carried. If the character has 0 to 2 more strength he can only move maneuver while carrying the object. If he has +3 strength he can walk, and with +6 he can run with the object, +9 for dash.
Carrying heavy objects costs 1 + required strength / 3 stamina each round.

If several characters cooperate to carry a heavy object they must each have the required diff to be able to move faster, not the diff in total.

E.g: Rock and Cliff are carrying a stretcher with loot. The stretcher requires str 6 to carry. Rock has str 8 and Cliff has str 9. They each carry 3 of the 6 points of weight of the stretcher. Rock has diff=+5 and Cliff has diff=+6. Since Rock only has +5 he cannot run. Even if Cliff had taken a greater part of the load they would not be able to run with the loot, since he would then be below +6 instead of Rock.

%NOPE:  Alternative: \\
%Heavy objects count to encumbrance just like other gear. That stretcher, heavily loaded with loot, might have enc 100.0 or 200.0, making it very slow and stamina sucking to carry around.









\subsection*{Awareness, spotting, finding}
%-----------------------------------------
The field of view is the area which is within the view arc, in view range, and sufficiently lit. The perception range is the character's perception value. Perception range and rolls for spotting are modified by the usual action modification stack of the spotter, such as movement, extra ap, pain, low hp, etc. Spotting is best done when standing still, attentive, doing nothing else.

A sneaking/hidden target modifies perception range and perception rolls of the spotter by the success diff of the sneak/hide roll. This is on top of the spotter's own ams mods.

\

A target character or object can either be obviously visible, sneaking/hidden, or in cover by some other object such as hiding in a bush etc. When obviously visible the target is out in the open and not trying to hide. When sneaking/hidden the target is possibly in the open or in some limited obscuring object but having passed a sneak roll. When hidden in cover the target is hiding in some object that obscures vision, e.g. a bush or some debris. A target can also be completely hidden behind a vision blocking object, e.g. behind a wall, but then the target is not possible to spot at all.

\begin{itemize}

\item An obviously visible target will be automatically spotted in the field of view. 

\item A partially obscured but otherwise not sneaking/hidden target within field of view can be spotted with a perception roll, further modified by how obscured the target is. E.g: crouching in a bush, mod-3.

\item A fully obscured but otherwise not sneaking/hidden target can be detected (heard, sniffed, felt) when inside the perception range with a successful perception roll.

\item A sneaking/hidden target can be spotted when inside the field of view but outside perception range if the spotter passes a perception roll.

\item A sneaking/hidden target will be automatically spotted when inside the field of view and within perception range.

\item A sneaking/hidden target can be detected when inside detection range but outside field of view if the spotter passes a perception roll with a further mod-3 difficulty.

\item A sneaking/hidden target cannot be spotted when outside the field of view and outside modified perception range.

\item For sneaking/hiding targets, obscuring items and terrain modifies the sneak roll instead of the perception roll.

\end{itemize}

Some targets are hidden very well. In such situations a normal perception roll is not enough, and they must be found by taking the time to search using the "find" skill. Example of this are characters and objects camouflaged or hidden by a multi-round sneak roll.

Calling for perception rolls of course notifies the players of hidden stuff in the vicinity.
%Roll only once for all hidden stuff, and see if the perception success is better than the hidden success of each respective hidden.
The GM can of course roll the perception rolls hidden from the players, but that is a bit more boring. Alternatively the GM can call for per rolls every now and then when there is nothing to find as well...


\subsection*{Sneaking and Spotting}
%----------------------------------
Roll for sneak: All environmental modifiers apply to the sneaking roll, then the sneak roll diff is applied as a mod to the perception and find spotting rolls.

Sneaking and hiding, both self and other objects, are done with the sneak skill. Conceptually equivalent to some extent in that the target is not really hidden from view, just partially obstructed, in shadow, or placed/moving so that it is less likely to be noticed.

Some basic sneak modifiers:\\
area is clean/open -3\\
area has some stuff around =0\\
area is cluttered +3\\
daylight or otherwise well lit -3\\
lighting by torch and otherwise dark =0\\
night or unlit dark area +3\\
weather is calm, still and quiet -3\\
weather is as usually, some wind and moving vegetation =0\\
weather is loud and everything moves, e.g. stormy  +3

Spending rounds, not actions, to seriously camouflage a character or hide an object will require a find roll to detect them. The environment modifiers are still applied to the sneak roll and then the diff is applied as modifier to the find roll.

For quick and easy sneaking around when scouting outside of combat one can simply roll for sneak occasionally when the environment changes to determine the spotting mods. Then only roll for detection by those opponents who have high enough perception, as long as the sneaky bastard doesn't run straight past the nose of the tired guard.

Spotting by characters that are inattentive, sleepy, or fiddling with something else is generally a mod-3 to the perception roll even if the spotter has mod-0 ams. Spotting when actively engaged in moving or actions are modified by ams as normal.


\subsection*{How often to roll for sneaking and spotting?}
%---------------------------------------------------------
Under normal action conditions roll for sneak once per round when the character is moving in an area where he might get noticed, in view or detection range. And the same for perception for the spotting opponents.

Outside of general action, just roll now and then when environment changes for a general success diff to apply to any guards that get close.


\subsection*{Surprise}
%---------------------
\label{sec:surprise}
Unaware and relaxed characters can be surprised. It then takes them one round to gather their wits. Roll int to see if they gather their wits fast enough.

A successful roll means they are not surprised and cannot be considered unaware. They have full normal ap available and movement speed M.

A failed roll means they take the fail diff as mods to all actions for the rest of the round, have only half ap (ru), and can move at most 1sq. They can still expand one action to a full round action if they don't have enough ap.


\subsection*{Unaware opponents}
%------------------------------
\label{sec:unaware}
Unaware characters, such as city guards, and goblins too focused on their stamp collections, are easy targets. They have perception mod-3, and will take some time to react. When attacked, startled, or succeeding a perception roll to notice the advancing hero, they will still require an int roll to get in gear. Otherwise they will be surprised and not do much until the next turn.


\subsection*{Long range vision}
%-------------------------------
Past his vision radius the character cannot discern details and must ask the GM what is further away. It is not enough to designate targets properly. Firing at a target beyond vision range is mod-3. Casting spells is mod-3.













%\phantomsection\addcontentsline{toc}{section}{Special damage effects}
%\section*{Special damage effects}


\subsection*{Some special damage effects}
%----------------------------------------
Some attacks have special effects that take place when the attacks succeeds. Sometimes it is required that the effect penetrates armour and does at least one hp damage, sometimes not.


\begin{description}


\item[Poison]
Poisons do damage or other effect over time. Most poisons has a few characteristics: strength, duration, and damage per round. Each round roll poison strength \vs constitution to see if the poison does damage or not. This will persist until the poison duration expires. Common poisons have str 5, duration 5 rounds, and does 1 hp damage per round. Strong poisons probably have str 10 and does 2 hp damage per round. Other poisons can be slow, and only does damage every other, or third round, etc...

Poisons do not have to do damage, they may just paralyse the victim, confuse it, or have other strange effect. Perhaps the target must pass con rolls or go into frenzy attacking the nearest person.

Antidotes, first aid, and medicine can be used to counter or limit the effects of poisons.


\item[Stun]
A stunned target will get a stun mod that affects his actions and movements. Each stun point removes one movement point, one action point, and pushes a mod-1 to the mod stack. At the end of each round the character will remove stun points equal to his constitution value.

E.g: Stunned Steve was bitten by an Elequito and got a stun 5. He immediately looses 5mp, 5ap, and has mod-5 to all actions. He has con 4, so at the end of the round he removes 4 stun points, giving him a stun 1 for the next round. For the next round he starts with 1mp less than declared, 1ap less than declared, and mod-1 to all actions. Before the third round his con removes the remaining stun 1.


\item[Web, ensnare, etc]
Web effects usually comes from a spider's spin attack, some magical hold field, or similar. Web effects reduce mobility and give action mods.
A typical spider web effect is dex-2, move-2, mod-3 to all actions until the effect is broken. Multiple web effects stack up to completely immobilise a target.

Some webs have residual effects such as requiring time to clean off. E.g: a persistent mod-1 to all actions and mp-1 to movement until someone spends an action, or several, cleaning it off.

Breaking a spider web is str vs web str action, note that it's modified by the web, or multiple webs in some cases.

Friends can help break loose by adding their str, and can help clean by spending cleaning actions.

Items accumulate mod-1 when used to parry spin web attacks, until cleaned off. Cleaning off requires a regular 3ap mod+9 action.


\item[Snag]
The attacker's weapon and target or target's blocking weapon are entangled until the end of the round, unless some cool maneuver or action can untangle them quicker.

It is possible to make a tug of war to rip weapons out of the opponents hands when they are snagged. Roll str vs str (unmodified). If any side wins with +3 or more he has managed to rip the opponent's weapon loose. It's also possible to tug or push a snagged opponent off balance.

Weapons with the \emph{snag} property can be used to intentionally entangle the parrying weapon or shield. Snagging with the snag maneuver is tricky and carries a mod-x. Snag can also happens on a fail-6 or worse with a roll of 10 if you do not try to snag, even with weapons that don't have the snag attack property.


\item[Knock back]
The target is moved one space away for each point of knock back. 
Large targets, with size more than one square, ignore one point of knock back for each occupied square after the first one. E.g: a 2x2sq troll ignores 3 knock back points.
Some creatures have knock back reductions even if they are smaller than multiple squares.


\item[Knock down]
The target will be knocked down and go prone unless they pass a dex roll modified by ams and the knock down effect.


\end{description}






%% TODO remove? this is already covered by the weapons. ? good to reiterate?
%\subsection*{Fast and slow weapons}
%%----------------------------------
%Attacking or parrying with a fast weapon cost fewer action points then with a normal weapon. An action with fast+1 weapon costs 2ap, and an action with a fast+2 weapon costs 1ap.
%
%Attacking or parrying with a slow weapon consequently costs more ap. An action with a slow-1 weapon costs 4ap, and slow-2 costs 5ap.


\subsection*{Breaking weapons}
%-----------------------------
When parrying a damage higher than the abs of the weapon, it might break.
Each excessive damage point gives a 10\% chance of the weapon breaking.
If the weapon does not break, it still takes a permanent -1 abs per every 3 excessive damage (round up).

Excess damage above the weapon abs both continues into the target and damages the weapon.


%\subsection*{Parrying strength limits}
%%-------------------------------------
%Parries are limited in how much incoming damage they can block from the attack. Excess damage continues into the target.
%A character can block incoming damage equal to three times the parrying weapon's strength requirements plus excess strength points. E.g: A sword with strength requirement 4 wielded by a character with strength 6 can block at most 14 damage from an incoming attack.
%
%This does not affect the "breaking weapons" rule above. Any damage above the parrying weapon abs will damage both the target and the weapon. Even if the parrying strength limit is lower than the parrying weapon abs.


\subsection*{Hacking through objects}
%------------------------------------
Hacking away at large objects such as doors, walls, statues, etc are at mod+6. If you have skills that improve damage for good strikes it should be quick work.
Some weapons are not meant to be used against objects. Using your sword or axe to hack away at a stone statue or stone wall will have a 10\% chance each strike to reduce the abs of the weapon by 1.

Pick axes and hammers are meant to be used against objects, and suffer no damage. They usually also do extra damage to objects of certain type or material.


\subsection*{Insufficient strength or dexterity penalties}
%---------------------------------------------------------
All weapons require a certain strength. If that is not met the user takes a dam-1 and mod-1 penalty for each missing str. For weapons with a minimum dexterity requirement each lacking dex point will cause a mod-1 and cost (lacking dex)/3 round up extra action points.

E.g: Willow the Weak has str 4 and tries wielding a broad sword which requires str 5, he then has a dam-1 modifier and mod-1 to all actions with the sword.
And Fumbly Fabian has dex 5 but his new found exotic rapier of pointy murder requires dex 7. Fabian thus takes mod-1 and +1ap, which means his fast rapier is now just 3ap normal speed. So unfortunate...


\vspace{10mm}
\TODO meh, rewrite excessive strength bonus points allocation structure
\subsection*{Excessive str bonus}
%--------------------------------
Heroes with higher strength than the weapon/armour/action requires can use certain skills and maneuvers to get bonus effects to damage, reduced stamina or action point cost, etc. Each point of excessive strength can only be used for one effect at a time, but enough points allow for combining effects.

Always use the highest base strength requirement of all equipment, maneuvers, skills, etc when calculating the free excessive strength points.
E.g: A character with strength 8 carrying a heavy shield (str6 at normal speed) and a spear (str4 in 1h grip) has 2 points of excessive strength available.

%TODO: rewrite example, after changes to stamina
E.g: str9 with a weapons that requires str3 gives +6 excessive strength points. With the skills "strength bonus" and "easy grip" the extra strength can be used for dam+2 \emph{or} for two attacks that do not require stamina, or for dam+1 and one attack without stamina cost.

% new allocation example, including stronk tank
E.g: Giant Gerda has str 13, 
plate armour (str5)
tower shield (str8 or slow-1 str5)
great axe (str9 or slow-1 str6)


She has slugger 3, stronk tank 2, fast strength.
She can spend 6 excessive strength to reduce her armour penalty class by two levels, to have the plate feel like a leather armour, and 

Reallocating extra strength points takes one round per point, unless the Hero has flexible muscles.

\emph{Note:} that high dexterity requirements do not work the same way. High dex is generally just a threshold cutoff. If the character has high enough dex the bonus is available, regardless if there are other bonuses that have dex requirements that are active at the same time.


\subsection*{Reach}
%------------------
Weapons with the reach property can hit targets one extra square away per level.
E.g: a spear with reach 1 can hit an opponent who is separated from the attacker by one empty square. However, attacking with reach is usually more difficult and most reach weapons have a mod-x when using reach.
A few weapons are made for fighting with the opponent a bit away and can have modifications when using them in base contact, e.g: reach 0 mod-3 and reach 1 mod=0.


\subsection*{Initiative when attacking against weapons with superior reach}
%--------------------------------------------------------------------------
If moving closer to attack an opponent wielding a weapon with longer reach than yourself the opponent can, if the attack is noticed, interrupt your attack with an attack against you instead, even if you have higher initiative. But then the interrupt attack must be taken at longer reach then you were going to attack at. This does not apply if you are not moving closer to attack.

E.g: Attacking Albin and Defending David stand 3sq apart. AA has initiative 10 and a sword and wants to attack DD who has initiative 6 and a spear. AA must first move in to base contact before he can attack with his sword. However, that means DD can choose to make an interrupt attack at reach 1 mod-3 before AA has closed to base contact. DD cannot choose to make a reach 0 mod-0 interrupt attack.

Ranged weapons, e.g. bows and crossbows, don't have this benefit. There the order of initiative will decide if the attacker can close before the shooter can let the arrow fly.


\subsection*{Pain makes life more difficult}
%-------------------------------------------
Major wounds give pain modifiers to the action modification stack until healed enough to be minor wounds.
For most people a wound gives dam/3 (round down) pain penalty mod that go to the action mod stack and stays there until the wound is treated, pain killers are administered, or some such action is taken to reduce the pain. This is why it is important to track each wound hp on the stat sheet instead of just the total remaining hp.
Veteran is a great skill to have.

Optional: Some creatures have different pain threshold and the wounds then give pain equal to damage/threshold instead of dam/3.


\subsection*{Wounds make life more difficult}
%--------------------------------------------
When a character gets severely wounded he becomes weaker and this makes it harder to perform actions. These modifications are not reduced by the veteran skill. As hitpoints drop it gets worse:\\
At 66\% hp he has a mod-1 to all actions. \\
At 33\% hp he has a mod-2 to all actions and cannot dash. \\
At 0 hp he has mod-3 to all actions and cannot run or dash.\\
Black Knight can ignore some mods like this.


\subsection*{Come on and die already}
%------------------------------------
Characters die when they reach -con hp. When a character reaches 0 hp or below he is in bad shape and in risk of dying. For every round he is not fully resting he needs to roll con modified with the negative hp, ignoring ams. If he fails he loses consciousness. Regain consciousness is also a con roll modified by negative hp, ignoring ams. Passed out characters can roll at the beginning of each round. The veteran skill helps to reduce the modification of the con rolls.

E.g: AlmostDeadDave has 11 hp when whole and con 7. At -3 hp he must pass a con-3 = 4 roll or pass out each round he tries do do anything more than maneuver with no actions. Since he has con 7 he dies at -7 hp.

Optional: Some large creatures with very high hp will have a multiplier when referencing con for this effect, e.g: dies at -con*3 or some such. This multiplier also affects the chance of fainting from negative hp, etc.


\subsection*{Regaining hit points}
%---------------------------------
A character generally regains con/3 hp per day. If the character has con = 2 he recovers 1 hp every two days, and with con = 1 he gets one hp every three days. At con = 0 he's in such poor shape that he requires medical or magical treatment to heal.

A daily successful medical treatment roll also heals success diff hp even if all wounds have been treated, see \emph{medicine}.

Optional: if tracking wounds list for pain purposes, wounds heal one point at a time in order of acquisition. E.g: if the wounds list is: 5,2,4, then after healing 5hp it would be: 3,0,3. Still giving two pain.


\subsection*{Travelling the region and area maps}
%-----------------------------------------------
When moving around between encounters the heroes probably traverse a regional map. The scale is in leagues, generally at 1 sq = 1 league. A character can walk a number of leagues each day equal to his level of constitution without getting tired enough to have extra mods. The party's movement is therefore limited by the character with the lowest constitution. The travel skill improves long distance travelling speed.

The party's vision on the regional map is equal to the highest value of the track skill of any member. Parties without track only have regional vision of the square they are currently occupying.

Riding is faster. A characters can ride a distance equal to his constitution plus ride skill (con+ride) but also limited by the quality of the horse. A horse has a cruise speed rated in leagues per day. A decent horse is around 10-15 steps/day.

When riding in a wagon or cart the distance is not based on the character's con and ride skill, but only determined by the draft animals travel distance and the wagon's mods. Someone needs the skill ride to drive the cart though.

Travelling by horse and cart is fast and convenient in many cases as long as there are roads or easily navigable terrain such as grassland, moors, steppe, or some such. It can be very slow or impossible in forests, hills, and mountains. A donkey with a small cart can perhaps travel 10 leagues on roads, 5 leagues on grassland, 2 leagues in hills, 1 league in forest, and cannot enter mountains. A larger wagon cannot travel in forest either and is severely limited even in hilly terrain.


%Region maps generally have the scale of 1 sq = 1 league. And a league is about how far a character with con=1 can travel per day. A con=5 character can travel 5 leagues per day.
%
%One square on the region maps is about one league of distance. How long that league is, is another matter altogether. However, when describing distances to the players, the distance is in leagues, i.e. in squares on the region maps.

%In historical reality a league was often the flexible distance that someone walks in about an hour on easy terrain. A Hero with con X will travel X leagues per day without being overly tired afterwards. If you \emph{really must} have a real world reference, think of it as about 5km.


\subsection*{Escaping combat, fleeing, giving chase}
%---------------------------------------------------
A character has successfully escaped from combat when he has moved off the edge of the battle map. It's a simple and fun rule, creating chase scenes and affecting positioning and tactics for increased tactical depth. It's not meant to be realistic.

Allow the track, travel cruise speed, and exhaustion to determine if the pursuers catch up, and how fast. Then make a new improvised battle map to describe where they catch up, based on the terrain in that location on the regional map.\\
If the fleeing party is run down it's the chasing party that has most influence on the positioning on the new battle map.\\
It the fleeing party stops to face their pursuers they will decide most of the positioning.

Note that it's easy to get into a position where the fleeing party is always faster in short bursts on the battle maps, but the pursuing party is always faster and more resilient travelling the regional map to pursue them. In that case the travel speed takes precedent and decides that the fleeing party cannot get away indefinitely. They may flee a couple of battle maps, but sooner or later they will be run into the ground. E.g: subtract 1d6 max stamina from the fleeing party members for every map they flee after the first one, until they can rest for a day.

A league on the regional map is a huge amount of battle maps, so the fleeing party probably doesn't get far before ending up on a new battle map, blood thirsty pursuers breathing down their neck. It's probably more interesting if the first battle escape means they are run down 1sq away on the region map, that way they may sometimes be able to force a battle in a different terrain than they fled from. But probably a good idea to limit this to once: 1sq from first escape, then the pursuers decide.

After having escaped from combat a character cannot immediately re-enter the map if he changes his mind. E.g: force a minimum amount of rounds away dependent on the character's speed and some random roll. Away 1d10 rounds +0 for maneuver, +3 for walk, +6 for run, +9 for dash.


\subsection*{Hunger and thirst}
%------------------------------
The character must eat and drink. Each day he is without food his max stamina drops by -1. Each day he is without water his max stam drops with -3. Note: when max stam goes below zero it's difficult to draw stamina for actions, and all actions take the negative stamina as mods, above ams, meaning it's going to get very difficult to do anything.\\
The character dies from thirst or starvation when max stam reaches -con.

When food and water is available again the character recovers max stam at con/3 per day.








%\phantomsection\addcontentsline{toc}{section}{ranged attacks}
%\section*{Ranged attacks}


\subsection*{Ranged weapons}
%---------------------------
Ranged weapons have the benefit of attacking distant targets and cannot be parried. But they have slow rate of fire and do not do as much damage as melee weapons.


\subsection*{Short and long range}
%---------------------------------
Each ranged weapon has a base range where the attack is mod=0 and do normal damage. Short range normally gives a small positive mod. Longer ranges have negative mods and damage reduction.

\

\small \begin{verbatim}
Short is base range / 2                  mod+1
Normal range is base range               mod-0
Long range is base range * 1.5           mod-3, dam-1
Very long range is base range * 2        mod-6, dam-2
Extreme range is base range * 3          mod-9, dam-3
\end{verbatim} \normalsize

\

\noindent Some weapons behave differently than these standard numbers.


\subsection*{Rate of fire}
%-------------------------
The faster the character fires the more difficult the attack will be. In some cases the fastest attacks require that the weapon or ammunition is readied as quickdraw items, and that they can be drawn in one or zero action


\subsection*{Too short range, base contact}
%------------------------------------------
Ranged weapons suffer mod-3 when used against a target in base contact with the attacker. There should be at least one empty square between the attacker and the target. The short mod+1 is also not in effect in base contact so the attack has a total mod-3.

Note that the attacker is not affected by base contact with other characters than the target, or if the target is in base contact with any other characters. If the target is engaged in melee however, then there is a penalty, see below under \emph{firing into melee}.


\subsection*{Firing into melee}
%-----------------------------
Firing into melee is tricky, mod-3, and you risk hitting a friendly in base contact with the target.

If you fail-3 or worse you hit a friendly melee participant in base contact with the target instead. The miss will never hit a "better" target for the attacker. A fail-1 or fail-2 is a normal miss without risk of hitting a friendly target.

This penalty does not apply if the target is engaged in melee against the firing character. In this case the \emph{too short range} penalty above will usually be in effect. Therefore it is preferable to attack ranged opponents with melee weapons without reach.

The maneuvers \emph{fire support} and \emph{target pointer} change this behaviour.

\

\noindent Optional:

If you fail-3 to -5 you hit a friendly melee fighter in base contact with the target, in missile flight path if possible. The "alternate" target is probably the previous or next amongst the melee-ers in the missile flight path, in that order.

If you fail-6 or worse you might also hit melee-ers in base contact with target but perpendicular to missile flight path (i.e. to the side of the target, viewed from the missile). Which target gets hit is random, but must be friendly or it's a miss as usual.



\subsection*{Arrow recovery}
%---------------------------
To make life simple one can ignore missile ammo for most dungeon situations. A quiver of 30 arrows will be seen as enough for a normal dungeon with shopping access before and after. Special arrows, smaller quivers, large complexes or series of fights will require keeping track of ammunition. Assume half the arrows need to be replaced, or one arrow per 2r of combat.

Optional: The fights can get more interesting if it's necessary to keep track of ammunition count.
Arrows and bolts have a 50\% chance of being recoverable after hitting a "soft target", and 50\% of breaking. For a quick measure just assume you can recover half the arrows you shot at a target when you reach the corpse.
Missed arrows cannot be recovered unless they strike a soft surface, and most dungeons are made of stone...

The skill \emph{arrow recovery} is another fun option to keeping track of the survivability of arrows and which corpse is kindly holding them for you.


\subsection*{Fast moving targets}
%---------------------------------
It is more difficult to hit fast moving targets. Speed is the declared movement of the target, not the distance in squares it has moved in the round. This only applies to missile attacks. Melee attacks don't get the mods. The skill \emph{lead target} reduces these mods. \\
To hit mod is speed / 3

E.g: Fast Fabian dashes at 14 and is mod-4 to hit even if he has only moved 5 squares since the beginning of the turn.


\subsection*{Stationary targets}
%-------------------------------
\textbf{Melee} attacks against a target that is unaware of you, lying on the ground, sitting still, etc, and is not in combat mode and moving around will give mod+3. Stationary targets in combat mode will not give mods. I.e. targets that declare move 0 will still not be "still" enough since they are assumed to move around a bit in combat.

Prone targets are also considered as mod+3 stationary for melee attacks even if they are awake and on the way to rise and run away later in the round. 

\

\noindent \textbf{Ranged} attacks against a totally stationary target gets a mod+1 but this does not apply to prone targets. Missile attacks against prone targets instead have mod-1.


\subsection*{Targets behind cover}
%---------------------------------
Attacks against targets that are behind significant cover is mod-3. E.g: half person cover, or an archer shooting from behind a rock. \\
Attacks against targets that are mostly behind cover is mod-6. E.g: someone peeking out behind a large boulder, or a crossbowman shooting from behind a battlement or through a crenel.\\
Peeking out from behind a corner could be mod-9, and so on.


\subsection*{Shields are in the way of ranged attacks}
%-----------------------------------------------------
Attacking someone standing behind a shield, i.e. the shield is in the way as seen along the flight path of the projectile, pushes a mod on the attack depending on how large the shield is and how well the person is using it.

Some very skilled people can sometimes attempt to parry incoming projectiles but it requires the maneuver \emph{missile parry} and very high skill.


\subsection*{Cover from shields, hiding behind shields}
%------------------------------------------------------
Most shields have a modifier against ranged attacks that is a form of cover, when attacked from the shield side. Simplified to $\pm$45\degrees from facing.

%Shield cover is the 90deg region centred on (+/-45deg) the facing of the character.
%Shield side is the 90deg region centred on (+/-45deg) the diagonal forward of the shield arm. I.e: left shield arm covers 0-90deg counter clockwise from the heroes facing, while right shield arm covers 0-90deg clockwise from the heroes facing.

Heroes hiding behind their shields can choose which 90\degrees arc they are protecting with the shield, announced when they start the action. Hiding behind a shield is a normal action but requires no roll and remains active indefinitely without spending further ap until the character does something else. Some actions can be taken while still hiding behind the shield, GM discretion.


\subsection*{Initiative of area effects}
%---------------------------------------
Persistent area effects, e.g. fire spells, poison clouds, etc either have an initiative of their own or inherit the initiative from the action that created them. They act and apply effect in order of initiative just like everything else.

E.g: At initiative 9 Wizzy Wizard casts a Fire Ball with duration 3r. It does damage immediately when it's created, then again at initiative 9 in each of the following 2 rounds. Anyone stuck in the fire at the end of the first round better make sure to have initiative above 9 next round and enough mp to be able to escape, or take damage again at initiative 9 when the fire effect activates again.

Also, anyone running into an effect area has that effect applied immediately upon entry, but only once per round even if leaving and entering multiple times in one round or if the effect activates after the target entered and was affected.


\subsection*{Piling corpses}
%---------------------------
A normal size corpse gives a square a mod-3 poor terrain modifier. Each additional corpse is another -3, to a maximum of -9. Quickly leading to very tricky terrain when standing your ground fighting enemies. At some point it's time to just move and fight somewhere else. This also means that having a heap of fallen enemies covering the squares just in front of you is an excellent defensive strategy. Perhaps you can sacrifice your less useful allies for the same benefits?

Shoving or dragging a corpse 1sq is usually a 1r action, with corpse kick it's faster.


\subsection*{Assisting with actions}
%-----------------------------------
Some actions and skill checks for the character can potentially be assisted by other characters. The assistants roll and add their success diff/3 as positive mod to the primary Hero's roll. GM discretion regarding who many can assist, how long time they have, etc.
% think D&D have something similar, calling it skill challenge or some such










%-------------------------------------------------------------------------------
%S P E L L S   A N D   M A G I C
%-------------------------------


\phantomsection\addcontentsline{toc}{section}{magic}
\section*{Magic}


Wave'em hands and mumble something mysterious. Sure to transmogrify your internal mana to sparkling effects of wonder, death, and mayhem, to affect or afflict friends and foe alike.

Sorcerers, wizards, mages, witches, and warlocks bend reality to their will by force of mind, concentration, intricate and arcane knowledge, calling upon ancient magical compacts and constructs that echo through the fundament of the world, and so on. Or perhaps they just stroke the ego of mad gods.


\subsection*{When and how spells can be cast}
%--------------------------------------------
Casting spells require concentration, voice, and gestures. Spellcasting often takes several rounds and the spell takes effect in the last round of casting following order of initiative.

The cast roll is subject to the highest action stack modification that the character has had during the casting period. \hyperref[multiroundactions]{See \emph{actions that take more than one round} above.}

%TODO: rewrite all Xa spells to Xap instead.
Some spells have casting time counted in actions, 1a, 2a, etc, and should be treated just like normal actions at 3ap per action. Some spells have their casting time specified directly in action points. Some spells have 0a casting time and thus cost no action points. But even 0a/0ap spells follow initiative unless stated otherwise.

Spells that take one round or less to cast can be cast whenever in the round, following order of initiative, just like performing regular actions. That means they can also be cast as reactionary defence actions, like a parry, or retaliations after incoming attacks.
E.g: it's possible to summon a 1r ward flash defence to reduce the damage of an incoming attack, just like parrying with a regular shield.

A spellcaster can always choose to cast a spell that takes $\le$9ap in one full round instead, declaring only his base ap for the round and potentially skipping some heavy action mods. 
%This does not apply when spellcasting have been shortened by \emph{fast magic}.
%\todo ? keep this ?


\subsection*{No power draw unless successful}
%--------------------------------------------
Casting spells that fail and fizzle causes no mana loss, unless it is a serious failure.\\
Fail-1,-2 costs no mana. \\
Fail-3 or worse is 1 mana lost. \\
Fail-6 or worse will cause the loss of the full cost of the spell.

This is to make spellcasters more enduring in game. It also fits well with the image of the wizard building the spell, then empowering it.


\subsection*{Extra mana for effects}
%-----------------------------------
Many spells have optional power effects, such as increasing damage, range, or duration by spending more mana. Mana must be spent for these effects individually.

E.g: Warlock Warner casts Shock Bolt, with normal cost 1m, but adds two mana for extra damage and one mana for extra range. The total cost is now four mana.
Just make sure you have enough levels in the magic skill to be able to spend that much mana in one go.


\subsection*{Range of spells}
%----------------------------
The range of the spell applies only to the instant when it's cast on the target, unless stated otherwise. Once the spell has gone into effect the range no longer matters. 

There are spells which require the caster to remain within range for the duration, or which has another range during the duration, after casting. This kind of deviation is then stated in the spell description.


\subsection*{Duration of spells}
%-------------------------------
Spellcasting completes in the last round it is being cast, in order of initiative. The spellcaster can now choose when the spell should begin to take effect: \\
The spell's effect can start immediately after casting and then that round counts as the first round of it's duration. Or: \\
The effect can start at the very beginning of the next turn, ignoring initiative, and have that turn count as the first round of the duration.

This can be important. E.g: The wizard casts a spell which takes one round to cast and has one round of duration. If the duration starts and ends with the same round he casts it then it will perhaps benefit his friends more if he casts it before his friend's perform their actions. However, it he wants to take advantage of it himself he would better have the effect start with the next round since he probably cannot take any more actions in the round of the casting.

This option is available also for spells that take less than 1r to cast.


\subsection*{Keeping spells active}
%----------------------------------
Most spells with duration must be kept active by force of will, concentration. Unless otherwise specified, each active spell requires one point of psy dedicated to keeping it from unravelling and dissipating. That point of psy is "in use" and deducted from the caster's psy while the spell is active.

Some spells with duration do not require any psy to keep active, while others can require several psy, or int, str, con, dex. Some burn stamina or mana, etc.

Deviation from the norm is noted as the maintenance requirement of skill.

%spell maintenance, maintain spells


\subsection*{Regaining mana}
%---------------------------
All mana is regained when the caster has rested fully from a good night's sleep. Some dungeons allow for casters to regain some mana when resting fully a number of turns, or when spending time at a certain location.
Wise wizards and skilled sorcerers alike sometimes carry mana restoring potions, crystal ball batteries, power staffs, etc, when they go hunting for gold and glory.


\subsection*{Casting when dry}
%-----------------------------
Trying to cast with no mana left? Try rolling against psy. At each attempt; the character must roll against psy modified with the total cumulative negative mana level the caster is trying to draw down to. If he fails the mana is not drawn and the spell cannot be cast. If he fails with -3 or worse he passes out. To regain consciousness he must attempt to roll the psy modified by actual negative mana level each round until successful.

E.g: Crimson Caster is at mana -1, tries to cast a cost 3 spell and must roll psy-4 to be able to cast. He has psy=7 and rolls an 9, which means failure=-5. He falls unconscious. Following rounds he rolls psy-1 once per round until he wakes up again.


\subsection*{Casting is exhausting}
%-----------------------------------
Drawing mana costs stamina. Each point of mana drawn also draws one point of stamina. If the Hero can't draw stamina the spellcasting will fail.

If the Hero tries to draw stamina when starting from 0 or negative stamina he must roll con modified by negative stamina once for the whole spell. The Hero can roll con before starting casting the spell instead of when he finishes casting it if he so wishes. The roll is valid as long as he doesn't loose stamina while casting.


\subsection*{Smooth casting}
%---------------------------
The caster can spend more xp to train a spell so that it will not draw stamina together with mana. Learning to cast a spell smoothly takes 50\% more xp than the regular cost of learning the spell. Any xp spent learning the spell in the usual fashion can be counted towards learning to cast the spell smoothly.

E.g: Tired Ture knows how to cast unground. He has spent 11xp to train unground 6. But he's noticed he's often out of breath when he need to cast, so he wants to learn how to cast it smoothly. Since unground is scf 0.33 the smooth cast unground is scf 0.50. He can take the 11xp he's already spent on unground, but that only gets him to lvl 4, for 8xp. So he spends another 1xp to afford smooth cast unground to lvl 5, for 12xp. Tadaaa. No more burning stamina when ungrounding his enemies.


\subsection*{Lacking psy}
%------------------------
Most spells have a minimum psy requirement to cast. If the character does not meet them the casting of that skill will take a mod equal to the missing psy.
E.g: Wizard Woosey wants to cast a psy 7 spell, he has psy 5, thus all casting of that spell will be at mod-2.


\subsection*{Lacking int}
%------------------------
Most spells have a minimum int requirement. For every lacking int the spell becomes twice as expensive to learn. $\mathrm{Cost} = \mathrm{scf} \cdot \mathrm{lvl}^2 \cdot 2^{\mathrm{diff}}$. This is a lot, and it will often be cheaper to just raise the intelligence.


\subsection*{Sacrificing permanent psy}
%--------------------------------------
A caster can sacrifice a permanent psy to get 10 mana points (can increase the total mana past the character's maximum) and a mod+3 to all spells cast until the "sacrificial" mana points are gone. And for all the munchkins: yes, he casts using the "extra mana" first.

Sacrificed psy ends up as a permanent mod to the original psy stat. Keep track of it separately if there is any way in the campaign where they might be recovered. Divine intervention, or similar...

This is a very taxing action and can only be done once for each adventure. The sacrificial action takes a full round.


\subsection*{Hands and mouth free}
%---------------------------------
Casting magic is hindered if the hands are occupied or bound and if the wizard is gagged or has an enclosed helmet.

Each occupied hand gives mod-1. Both hands bound give mod-3. A gag gives mod-3.
E.g: a bound and gagged wizard has mod-6 to cast spells.

Heavier armour types also make casting more difficult.

Use "trusty old X", see below, to be able to cast while holding objects in hand.


\subsection*{The wizard's staff}
%-------------------------------
The wizard can choose to have a "trusty old staff" in his hands without incurring penalties for not having his hands free. Some wizard has a sword instead of a staff. Fidgety Fredrick had a tower shield he loved so much he cast better with it than without.

The "trusty old x" is a specific weapon or other large item, which the wizard has invested time and xp in. Thus it takes time and xp to replace if lost, or change. This way a wizard can cast most spells without penalties even if he is carrying weapons. Making an item a "trusty old x" with mod=0 costs 5xp. 
In some cases the wizard can invest more time and xp to have the item act as a security blanket, giving a mod+1 when casting. Making an item a "trusty old x, mod+1" costs another +10xp.

A wizard can have more than one item at the same time, but the max bonus is +1. Just don't loose them, because then the xp are gone.

\

Optional: It can be fun to limit the creation or replacement speed of trusty old items. Let's say the wizard can't just cough up 15xp directly, but has to spend 5xp first, then go on an adventure with the staff, then the other 10xp for the mod+1.

Perhaps even force him to play one adventure without the "trusty old" status and have mod-1 before he can spend the first 5xp to mod=0.


\subsection*{The paladin's armour}
%---------------------------------
Similar to the wizards staff, the paladin must be able to cast his magic while dressed in combat armour. Heavy armour generally give spellcasting penalties. The paladin can spend xp to cancel the mods. Each mod point costs 5xp to cancel. The armour is then "divine armour", just like a "trusty old" something for the wizard. By spending the extra +10xp set the paladin can push his armour to mod+1, just like the wizard's staff. 
Or the paladin can get the tank skill to reduce or cancel the armour penalties that way.
Don't forget to also pay "trusty old" xp to cancel the mods from sword and shield...

E.g: A paladin in full plate with tank 4 has a spellcasting mod from his armour at -2. So he spends 2*5=10xp to get it to mod=0 and another 10xp for the mod+1. Then he also have mods for his sword and shield, so he spends another 5xp for the sword and 5xp for the shield. He has now spent a total of 30xp on his equipment to have a mod+1 when casting spells.











%-------------------------------------------------------------------------------
% P L A Y I N G   T H E   M O N S T E R S
%----------------------------------------

\phantomsection\addcontentsline{toc}{section}{howto monster}
\section*{Playing the monsters}
%------------------------------
It is necessary to simplify the workload for the GM who has to manage all the monsters. Monsters are not generally long lived. They usually die within a few rounds of appearing. Most monsters can be very simplified and their handling made quicker by ignoring a lot of effects.


\subsection*{Some simplifications}
%----------------------------------
Ignore stamina, but don't let monsters use a lot of stamina all the time. They can use some, but not several stamina continuously round after round. Some monsters, like orcs, tend to have massive stamina and can gladly spend a lot, but keep it reasonable.

Ignore some pain, a lot of monsters don't react the same way to pain as heroes. A lot of monsters will instead influence their decision making by how much damage or pain they have taken. For single monsters or humanoid monsters it might be interesting to apply pain when appropriate.

Ignore mods from significant damage. If it makes life easier and there are lots of monsters, especially short lived ones.

Let the monsters die at hp=0, instead of having them linger around rolling for con each round.

Don't let monsters have too many advanced skills. It is better to keep the skills that add a lot of complexity to the more special monsters. The bosses, the leaders, the sneaky bastards, etc. Most should just have basic attacks and movements, perhaps a special maneuver or so.

These simplifications are great for "cannon fodder" critters, and that saves time enough so that you can play the bosses and interesting antagonists with more depth.


\subsection*{Full complexity for boss monsters}
%----------------------------------------------
It's more interesting to have special NPCs and opposition use full complexity, with all the intricate details, specialities, and loop holes. Simply adding things like counter attack, combat group, Oy!, leader, synchronise, etc drastically changes the way the opposition feels. Use it to give spice and make some mobs extra interesting.


\subsection*{Things to think about}
%-----------------------------------
Most monsters are not suicidal. They will run away when they meet overpowering enemies, if they can. Some monsters might run headlong into certain death but it is not the norm.

Intelligent monsters will instead flee, re-group, sound the alarm, make ambushes and traps instead of direct open battle.














%-------------------------------------------------------------------------------
% M I S C E L L A N E O U S   G M   T I P S
%------------------------------------------

\phantomsection\addcontentsline{toc}{section}{misc tips}
\section*{Miscellaneous GM tips}
%-------------------------------

As you get used to the game and try different things you'll figure out what works well for you and your group, and what doesn't. Don't be afraid to experiment. Log what you're doing so you don't forget, a few years down the line it's all very fuzzy in the old noggin.


\subsection*{fun --- long term vs short term}
%--------------------------------------------
\todo avoid short term fun at the expense of long term fun


\subsection*{Nerfing}
%--------------------
Nerfing rules, skills, abilities, and equipment is no problem unless they are in play on a character. My suggestion is to just allow that character to keep the original version, but force all future acquisitions to use the new version. Player characters will have a reasonable turn over rate anyway, and it's a bit fun to have \emph{legacy} ancient weird stuff in play sometimes.

During campaign play I strive to not forcibly nerf, and always allow the character to re-trade his XP and gold for something else if I \emph{really} have to nerf something already in play. Again, consider the long term \emph{player enjoyment} vs the combat balance. Perhaps there are other ways to contextually limit something that has turned out to be overpowered, too complex, or broken.


\subsection*{Hero manual}
%------------------------
I recommend that the players copy the skills they choose for their character directly to a cheat sheet. Especially if you or your group is tweaking and fiddling with the skills, etc, a lot during the game, it's important to note down what the original version of the skill actually said when the player decided to train it.

This also means that the players will effectively compile a small rule book specifically for their own character. The basic rules, notes, ideas, etc. Very useful. It also offloads the GM, since the player will have a very good understanding of the rules affecting his character.

\

% 200218:
So far, for over a decade, I have always let the player choose to keep the old version of a skill/equipment/whatnot when we tweak and update, or update if they want to.
It's a bit fun to have the weird legacy stuff circulating even after tweaks.


\subsection*{Finding stuff}
%---------------------------
Either roll find/per rolls hidden when needed so that the players won't notice, or ask the players to roll them, and then ask for rolls every now and then even when there is nothing to find. If asking for rolls you probably need to have at least a 2:1 ratio of asked vs real roll situations. This will keep the players anxious on a roll request, but not sure that there will be something to find.

I prefer asking for rolls in most situations, then roll hidden every now and then anyway.


\subsection*{Role Playing vs Tactical Optimisation}
%--------------------------------------------------
Generally: role playing limits tactical optimal planning and behaviour. Make sure you've discussed this with your group so you all know what level of rp vs tactics you're all comfortable with. This also involves what level of metagaming and information leakage you have across the table, battle map, or voice chat.

\

\noindent \emph{Know your players. Discuss in advance. Find common ground.}

\

In the most severe option on one side you can use individual fog of war and vision for each Hero, and insist that all players share information of the battle map \emph{in character} and time what they say into actions and rounds for their Hero. Remember that it's possible to shout and murder at the same time. This is definitely doable with some tools and it can be a whole lot of fun, but in my experience not what most people prefer or have the most fun with.

The other extreme is the shared battle map and open collaborative reasoning between all players without time limits for every round and individual action. This can be a lot of fun and very social, but some players will zone out and participate less. It tends to lead to where one or a small clique of players mastermind the whole party, essentially just giving suggestions or orders across the table or chat, and the rest of the players just executing.

Most groups quickly find common ground naturally, but be aware of the issue and discuss.


%\subsection*{Have the players narrate assisted actions}
%%------------------------------------------------------
%Fun if the players argue why and how they use their skills can roll for assist







%-------------------------------------------------------------------------------
% Y O  U N G   K I D S
%---------------------

\phantomsection\addcontentsline{toc}{section}{young kids}
\section*{Playing with young kids}
%---------------------------------
\label{sec:youngkids}  % Bitter Candy
We've successfully played with kids as young as five years old, and without any previous tabletop or role playing experience. It's enough that they can count to 10, add and subtract a bit. So, clever kids, but still.

\

\noindent \emph{Keep it simple. Keep it fun.}

\

% Bitter Candy
Use physical objects and markers instead of numbers on a paper. Improvise the battle map with everyday objects and use their toys as Hero minis. Any toy up to the size of a hand works great.


\subsection*{Bitter Candy}

I had an insane adventure when I playtested with a trio of young girls. I had them build a small play farm on a coffee table, and choose one small toy each. Then told them that an evil bitter taste was spreading throughout the land, making all the candy taste like coffee. They must find the source of the horror and put a stop to it immediately.

Thus, Sparkly Pony and two old Rubber Trolls had to venture from their coffee table farmstead, across the Living Room Expanse and the Valley of Hallway, to reach the Forbidden Kitchen. There they had to climb the Chair Mountains to reach the Oakwood Plateau, whence the miasma originated. Finally, on the high flatland they faced Lord Grapefruit and all his purple and green grape minions. Along the way they had encountered a friendly sock monster who had taught them a useful spell of Stink Cloud, and fought a couple of evil car demons to get a set of SpeedWheel boots. They also had to solve the riddle of the BrassHandle before the DoorMaster allowed them to enter the Forbidden Kitchen.

Bonus that they got to eat all the grapes they defeated, though they were kind enough to leave the carcass of Lord Grapefruit for their mother.

\

We used strips of paper for movement speed and a ruler for bow range. Smartphone for dice since the household were not gamers and didn't have anything except some old tiny yahtzee d6 which were too small and boring.
Simplified rule, skill, and equipment sets similar to the \hyperref[sec:basicenough]{\texttt{Ominous Crown}} adventure example, page~\pageref{sec:basicenough}.

I drew small paper cards of the weapons and equipment, with corresponding skill values written in the corner. We used coins as markers for hit points, small balls of green and red paper for action points and difficulty modification points.








%-------------------------------------------------------------------------------
% T A B L E T O P
%----------------

\phantomsection\addcontentsline{toc}{section}{tabletop}
\section*{TableTop}
%------------------
Not originally designed for actual physical tabletop gaming, we've run both simple and fairly complex versions of the game on physical tabletop, especially when playing with young children.
A few sheets of paper, pencil, eraser, and a small objects are all that's needed. Though much more impressive with cool terrain, props, and minis.


\subsection*{Markers and Minis}
%------------------------------
With markers it's quick and easy to keep track of the character's current stats. Suggest using small markers like coins, plastic bits, paper balls, whatever of various colour, size, shape to represent: mods, mp, ap, hp, stamina, mana, pain, stun, base mods, duration counters, etc.

Keep three separate areas:\\
per round accounting: mods, mp, ap\\
persistent status: hp, stamina, mana, pain, stun, etc\\
a marker bank, separate per character to have it close at hand.

Each round the players will fiddle with mp, ap, and mods, so that should be quick and clear without risk of touching and disrupting the heaps of hp, stamina, mana, etc. So, \emph{keep them separate}.

I suggest front loading hp, stamina, mana, so that the pools are filled with "nice markers" when they are whole and rested, and getting smaller and smaller as time goes by, while the piles of "nasty markers" like pain, grow. That way it's easy to see when someone is getting into risky situations: His nice heaps are getting uncomfortably small.


\subsection*{Simplified tabletop light and vision}
%-------------------------------------------------
In daylight people have their normal vision ranges and usually just see whatever is on the board. For large boards it might be relevant to measure approximate vision distances before placing minis, and using "blips" on the board to represent stuff in the distance that has not been clearly resolved. E.g: this blip is a group of a half dozen or so small humanoids, that one looks like a large monster, etc.

In low light conditions it's easier to approximate light and vision range based on light sources and the Heroes' low light vision. E.g: a candle lights a small room, a torch lights a large room. Use blips for things lurking in the dark.








%-------------------------------------------------------------------------------
% O N L I N E   G A M I N G
%--------------------------

\phantomsection\addcontentsline{toc}{section}{online}
\section*{Online gaming}
%-----------------------
Face to face, coffee and cake, friends and banter. Gathering around a table is more fun. But for many it's a rare treat. Life and logistics gets in the way. Children and work take priority and when your gaming friends all of a sudden live a continent away the Sunday game commute is perhaps not environmentally defensible any more.

For us, online play, once a week on a fixed weekday evening 2000--2200 has been shown to work well enough to actually be schedulable.


\subsection*{Tools: virtual tabletop and voice chat}
%---------------------------------------------------
The game is primarily designed for short sessions with fast gaming, and we play mainly online over voice and vtt. As of late 2019 we're still using maptool and mumble. They work well, even for the sometimes enormous battles of the Goblin Destiny campaign. I'm sure other VTTs work fine but maptool is the only one I've found that is self hosted, has vision blocking, fog of war, and individual token vision. It's also quick and easy to work with as a GM. 

For voice chat I recommend mumble + murmur and make sure everyone has a decent headset. \vvsmall(early 2020)\normalsize.~ Don't use webex, skype, teams, etc as those are way too unstable and have horrible sound quality, lag, and jitter. Hangouts and Discord are better than skype but have much longer latency than mumble. Teamspeak and ventrilo were ok, haven't tested for a while, but mumble beats them as well. And since mumble can be self hosted you can have exceptionally good latency and jitter, especially between the GM and every single player. 

\

Since most of my projects have longevity exceeding a decade I've come to rely primarily on open source tools. Proprietary stuff simply doesn't have the survival statistics. Economy, product strategy, and acquisitions kill tools too frequently for me to invest time in them.

% I'd like to find one that also works well on tablets and smartphones.

% some 
% VTT: maptool, vassal, roll20, fantasy grounds, foundry, ...
% ? table top simulator on steam?





%-------------------------------------------------------------------------------
% M A P T O O L 
%--------------

\phantomsection\addcontentsline{toc}{section}{maptool}
\section*{Maptool}
%-----------------
Maptool is the only Virtual TableTop system I've used for this game since the inception back in 2008. Not totally accurate, first few experiments we just used jarnal and doodled our way through a couple of mini test encounters.

Open source. Self hosted. You as GM or player have control and access to all the work you create, it's not locked away on a server somewhere, dependent on whether the hosting company survives / changes strategy / gets acquired in the coming years. The code is mainly java so it's fixable if the external community dies. I saw the likelihood of this little hack'n'slash game being long lived, so open source and self hosting were \textit{\textbf{must have}} when I scouted for tools.

Maptool had rudimentary support for fog of war and vision / vision blocking. This adds such a great feeling of exploration and immersion when playing, and lots of tactical depth. The experience of entering a dark dungeon, lighting a torch, and carefully sneaking around corners is wonderful, even in a simplified top down view. Just as having enemies hide in the long shadows cast by trees and bushes in the light from your camp fire. Great, and something most other tools were lacking. And many still lack today, \vvsmall(early 2020)\normalsize

Maptool scripting is passable, but horrible to work in. Despite this, I've implemented automation for most everything in the game over the years. Almost all actions as GM is one or two clicks by now, and very little to keep in the noggin between rounds.

It's also reasonably quick to import maps and set vision blocking, as well as make improvised maps.


\subsection*{Maptool light and vision}
%-------------------------------------
Set vision range and arc for each character. I've also built standard versions for the various typical npcs.

Light condition multipliers should be set as follows:
\small\begin{verbatim}
normal    all light sources have the specified range: humans, halflings
dusk      1.5x range of light sources: dwarfs, orcs, goblins
night     2.0x range of light sources: elves
dark      3.0x range of light sources:
black     self light up to infra range regardless of light sources
\end{verbatim}\normalsize

This means an elf standing close to a camp fire has full daylight conditions, and a human with a torch is at most 1r dash away from a monster when he sees it. Hehehe.

Magical dark vision should have different multipliers, or set a personal self light to suitable radius.

Dwarven infra vision, goblin smellyvision, etc, can be set with a self light radius of a few squares.








%--------|---------|---------|---------|---------|---------|---------|---------|
%       10        20        30        40        50        60        70        80





%-------------------------------------------------------------------------------
%S K I L L S ,   A B I L I T I E S ,   E T C
%-------------------------------------------

\cleardoublepage

\phantomsection\addcontentsline{toc}{chapter}{Skills}
\chapter*{Skills}
\chaptermark{skills}


We'll be using the term "skill" as a compound term for all skills, traits, powers, and abilities that are not character based or expanded traits. This includes regular skills like literacy and counting, weapon skills, and maneuvers such as avoid, along with character trait enhancers such as resilient and strong. Magical spells, mental and mystical powers, etc, all baked into one.

All skills carry a "skill cost factor" (scf), which determines how much the skill costs to train to a certain level. The cost is always rounded down to the nearest whole experience point.\\
\\
The cost of training a skill to a certain level is generally $scf \cdot lvl^2$. \\
E.g: training bow to 4 is (1.2 * 4*4) = 19.2 \ca 19xp. \\
E.g: increasing bow from 4 to 6 is (1.2 * (36-16)) = 24xp. \\
\\
Skill costs vary, but the general idea is something like this: \\
Weapon and main battle skills cost $1.0 \cdot lvl^2$. \\
Special high power tactical skills cost $1.0 \cdot lvl^2$ or more. \\
Low power tactical skills and useful miscellanea costs $0.7 \cdot lvl^2$. \\
General rpg support skills cost $0.5 \cdot lvl^2$. \\
Basic combat maneuvers cost 10xp. \\
Low power abilities cost 20xp. \\
High power abilities cost 30xp. \\
Massive abilities cost 50xp or more.

\

% NOPE: childhoods and professions are removed for now.
%Some basic skills are pre-packaged in various starter packs, such as "normal %childhood", "spoiled brat", "slum scum", and such things.
%
%\
%
Some skills are difficult to come by. They might only be trainable from a specific master, or learned from a musty old book somewhere, or gained by eating or drinking something strange which was bought or found and then paying the XP. Exotic skills and abilities can therefore be difficult to come by, while basic skills can be learned almost anywhere.

The skills pertaining to magic can be found in the "magic" section. %magic.tex
E.g: magic, power casting, fast magic, and spells

\

Costs written as "cost 20xp" means that the total cost up to that level of the sill is 20xp. If it's written as "cost +20xp" then the cost to increase the skill from the previous level to to the target level is 20xp. See "tank" for a good example.







%-------------------------------------------------------------------------------
% combat skills
%--------------

\phantomsection\addcontentsline{toc}{section}{combat}
\section*{Combat skills}

The combat skills are generally weapon skills or skills that allow for doing more damage, combining attacks, etc. Further combat oriented support skills, attack maneuvers, etc are found in the maneuvers - attack and abilities sections.

\

\openskillslist

\begin{samepage}
\skill{Skill: \emph{weapon class}} weapon skills in a weapon class. The chosen level of simplification is that all weapons in any given "class" can be used with the same skill.

\begin{verbatim}
sword          scf 1.0
knife          scf 0.7
axe            scf 0.9
hammer         scf 0.8
club           scf 0.6
staff          scf 0.8
spear          scf 0.8
flail          scf 1.0
shield         scf 0.7
throw          scf 0.8 (e.g: knife, axe, javelin, improvised)
bow            scf 1.2
crossbow       scf 0.8
\end{verbatim}
%sling          scf 1.0
\end{samepage}

Knife is a limited variant of the sword skill. Blade weapons dam $\leq$3 are the primary use, with no mods. Blade weapons with damage 4 have mod-1, while damage 5 and 6 have mod-2. All heavier have mod-3.
Training in the knife skill can be converted to training in the sword skill over a weekend course, where $sword = \sqrt{(knife)^2 \cdot 0.7}$ round down, loosing the excess xp. Such courses are available in most larger cities and cost a couple of gold.

Some weapons are similar:\\
The sword skill allows the use of knives and rapiers at no modification.\\
Axe skill allows the use of: hammers mod-1, clubs mod-1.\\
Hammer skill allows use of: clubs mod-1, axes mod-2.\\
Club skill allows use of: hammers mod-2, axes mod-3.
The staff and spear skills are mutually mod-1.\\
All other melee weapons are mutually mod-3.\\
All ranged weapons are mutually mod-3.\\
\emph{But:} similarity modifies never allows for the "other" weapons or weapon groups at levels greater than 9. So, even if someone has axe 14 he will be limited to hammer 9 due to this cap.

Starting to train a new weapon skill while knowing a similar one should also start from the shared modified base.

Melee weapons gives the user the ability to make "attack" and "parry" actions, while ranged weapons only allows for "attack" actions, and cannot be used for parrying.

Most weapons have one or more optional attack/defend action alternatives, but they usually require further XP expenditures to learn. Most cost 5xp and simply gives minor alteration to the weapon stats, e.g. dam-1 pen+1.


\skill{Skill cost special "speciality weapon X":} allows to use speciality weapons without the mod-1 that they otherwise impose. Each speciality weapon requires it's own speciality skill. They usually cost 5xp.


\skill{Skill scf 0.5 "brawl":} allows for unarmed combat. A character generally has two fists and one kick to use. This can be used with the skills "double" and "triple". Orcs and goblins also have claw and bite attacks.\\
Fists do str/3 round down damage, fast+1, first two attacks are free \\
Kicks do 2+str/3 round down damage \\
Unarmed parry with hands/arms/legs are parry-6 against weapons, or parry-3 if opponent is unarmed. Unarmed parry is always deflect but without the additional deflect mod-3. All parried weapon hits will still do 50\% of the rolled damage, round down, and all parried unarmed hits will still do 33\% of the rolled damage, round down. A success+3 will reduce the end damage with 1, a success+6 will reduce it by 2, and a success+9 will reduce it by 3. \\
First two attacks require no stamina.


\skill{Skill scf 1.3 "martial arts":} allows for more effective unarmed combat. A character generally has two fists and two kicks to use. This can be combined with the skills "quadruple", "triple", "double". \\
Fists do (str+dex)/3 round down damage, toparry-3, toavoid-3, fast+1. \\
Kicks do 2+(str+dex)/3 round down damage.\\
First two fists or first kick do  not require stamina. \\
Martial arts parrying is parry-3 against weapons, or parry-0 if opponent is unarmed. Martial arts parry is always deflect but without the extra deflect mod-3. Parried weapon attacks (not unarmed attacks) will still do 33\% of rolled damage, round down. A success+3 will reduce the end damage with 1, a success+6 will reduce it by 2, and a success+9 will reduce it by 3.

\skill{Skill scf 0.5 "lightning strike" (martial arts upgrade):} allows for very fast fist and kick attacks but at the cost of a dam-1 penalty. Fist attacks are fast+2 (1ap) instead of fast+1 (2ap), and kick attacks are fast+1 (2ap) instead of normal (3ap)..

A character cannot have a higher skill level in lightning strike than he has in martial arts.

\skill{Skill scf 0.5 "thunder strike" (martial arts upgrade):} allows for very powerful fist and kick attacks.  Thunderous powerful strikes gain a dam+2 bonus but are slower. Fist attacks are normal speed (3ap) instead of fast+1 (2ap), and kicks are slow-1 (4ap) instead of normal speed (3ap).

A character cannot have a higher skill level in thunder strike than he has in martial arts.


\skill{Skill scf 4.0 "weapon master":} is the dude who knows how to handle just about any weapon imaginable, including unarmed combat and ranged weapons as well as melee weapons.


\skill{Skill scf 0.8 "avoid":} is used to try to avoid the incoming melee attack. The user must of course be aware of the incoming attack. Avoid can be used with the "yield" maneuver option, for (usually) a mod+3. To avoid is a defence action, just like parrying. \\
To avoid falling object, sprung traps etc is generally mod-3 to mod-6, and that usually includes a required yield step to get away from the trapped/target square. \\
To avoid fast missiles such as arrows, the special maneuver "missile parry" is required, and is generally mod-9. \\


\skill{Skill scf 1.0 "disengage":} allows for the character to make a safer movement out of opponent's weapon reach modifying the (possibly) incoming "right to react" attacks triggered by the movement if the opponent has \emph{opportunity} or \emph{intercept}. The level of disengage directly modifies the opponent's attack against the disengaging target.
Disengaging is a full action, 3ap, per opponent that the character wants to safely disengage from, but requires no rolls.

E.g: Disengaging Dave wants to get away from melee against Mauling Morgan and Striking Stanislav, both of which have \emph{opportunity}. DD has disengage=4 and spends one action (3ap) to move away, out of range from both opponents. However, DD only spent one disengage action, and can thus only mod down one of the opponents. He chooses to mod MM. MM and SS now has the right to react to the movement. MM has a mod-4 when attacking DD due to the DD's disengage action, but SS still has mod-0 for his attack. If DD had paid two actions (6ap) he could have given both MM and SS mod-4 on their "right to react" attacks against him.


\skill{Skill scf 0.5 "fancy attacks":} Melee attacks can be made more difficult to parry. For every level of "fancy attacks" the attacker can choose to take a mod-1 and at the same time make the attack mod-1 to defend. Limited by the maximum finesse of the weapon.

E.g: Sneaky Sebastian has "fancy attacks"=4 and sword=11. He makes a primary attack with a fancy attack mod-4 and rolls a 6, which is a success+1. His target, Unfortunate Urban, now has a mod-4 to parry the incoming attack. Since he has sword=5, he will probably not succeed.


\skill{Skill scf 1.0 "double":} allows for combining two weapons actions into one single action. The user must have two weapons (or shield), one in each hand. He can then make two attacks, two defends, or one of each, in one single action. The skill must be rolled before the combined action and costs one stamina extra. If the roll fails the combined action falls back to one normal action for the primary weapon, and the user has the option to spend another action for the secondary weapon. A weapon skill roll must still be taken for each weapon, and the stamina for extra attacks must be paid as usual. A "double" action combo costs one extra stamina.

The double action requires action points equal to the slowest of the actions combined within the double action. E.g: Double with a rapier and a dagger costs 2ap. Double with a longsword and a tower shield costs 4ap since the tower shield is slow-1.

Double attack actions can be parried like normal actions, with two separate successive parry actions, as long as the defender has two weapons, one for each of the incoming attacks. E.g: if the defender has a shield and a sword, and defends against a double axe attack, he can parry one axe with the shield and one with the sword. If he does not have two weapons to parry with he can only parry against one of the incoming attacks. E.g: a defender with a two handed sword can only parry one of the double attacks.
If the defender succeeds with a "double" roll before parrying, and has two weapons/shields, he can parry the two attacks with one "double" action.

It is possible to parry a double attack with one parry and one avoid action, but then both suffer mod-3.

Combining a double defence with yield or dodge bonus applies the bonus to both defence actions if the "double" roll is successful, but only to one of the defence actions if they are performed as separate successive actions.

Synchronised double attacks cannot be parried with two different actions, but must be parried with one double action, or by different people. Roll for "synchronised" before double to make a synchronised double attack.

Synchronised Double parries can attempt to parry twice for the same incoming attack, once for each weapon. Alternatively use the two weapons as one combined parry, and combine the abs of both weapons to block the attacking damage if both parries are successful.

%TODO: This part made sense somehow, but I can't see it right now, check later
%The "double" actions can be pre-declared and saved for later in the same round. The pre-declared action must be fully declared: e.g: Defend with the shield against an attack from monster X. If the next action the character makes is not the pre-declared action, then the "double" status of that action is wasted and lost.


\skill{Skill scf 1.5, 2.0, 2.5, ... "triple, quadruple, and so on":} some critters have more arms than others. Works like double, except that a fail collapses it to the next lower: e.g: triple to double, quadruple to triple. Each fail point collapses the skill one step. A fail-1 collapses a quadruple to a triple, and a fail-2 to a double. In the end it collapses to just normal actions. Any of the "higher" double variant combinations still only cost one extra stamina.


\skill{Skill scf 1.0 "whirlwind":} The character can make several fast attacks within an extended whirlwind action. The first attack is done at normal speed for the weapon but each following attack is fast+2 (min 1ap).

Before each attack in a whirlwind sequence the character must pay one stamina and roll a successful whirlwind check.

Each attack can target different enemies. Each attack requires successful whirlwind rolls before the attack roll can be made. When the whirlwind roll fails no more attacks can be done in that whirlwind sequence. The first attack can always be performed even if the first whirlwind roll fails though.

The damage of each successful attack is limited (capped) at the lowest damage rolled through the whirlwind sequence. Roll for damage as normal, but the damage dealt is limited by the lowest so far in the whirlwind sequence.

E.g: Wispy the Waif attacks Monsters m1 m2 m3. She has whirlwind=5, sword=7.
She declares whirlwind, pays one stamina and normal ap, rolls 3 $\Rightarrow$ success, rolls 5 for the sword attack on monster 1, does 3 in damage. \\
She continues whirlwind, pays 1 more stamina and 1ap, rolls 4 $\Rightarrow$ success, rolls 6 for sword success against monster 2, then rolls damage 4, but the damage is now limited capped at 3 since that is the lowest damage that has been rolled in the whirlwind sequence. The second sword attack also costs her one more stamina, as usual, beside the cost of the whirlwind roll. \\
She then tries for a third attack, pays one stamina and 1ap, but rolls 7 and fails the whirlwind. She cannot make another attack in this sequence. She can now start another action with more whirlwind attacks with the damage cap reset.

Whirlwind becomes even more dangerous when combined with for example double... Then each weapon has its own the damage cap track. It can also be combined with different attack maneuvers.


\skill{Skill scf 1.5 "quick shot":} reduces snap and quick shot penalties by one for each skill step.


\skill{Skill scf 1.0 "sniper":} reduces range penalties by one for each skill level. It also reduces mods from small and covered targets by one for each level. Covered targets includes coverage from shields.


\skill{Skill scf 1.0 "lead target":} reduces penalties for shooting fast moving targets by one for each skill step.


\skill{Skill cost special "accurate":} allows for the user to generally improve damage when making good, very good or perfect successes with attacks. \\
lvl 1, cost 10xp: A good (+3) success: roll two damage rolls and choose which to use. (will average \ca 66\%) \\
lvl 2, cost 20xp: A very good (+6) success: roll three damage rolls and choose which to use. (will average \ca 75\%) \\
lvl 3, cost 30xp: A perfect (+9) success: roll four damage rolls and choose which to use. (will average \ca 80\%)


\skill{Skill cost special "consistent":} allows for the user to generally improve damage when making good, very good, or perfect successes with attacks. \\
lvl 1, cost 15xp: A good (+3) success always makes at least 33\% (round up) of the maximum damage. (will average \ca 66\%) \\
lvl 2, cost 30xp: A very good (+6) success always makes at least 66\% (round up) of the maximum damage. (will average \ca 83\%) \\
lvl 3, cost 45xp: A perfect (+9) success always makes 100\% of the maximum damage.


\skill{Skill cost special "precise":} gives extra penetration points to damage for good strikes. \\
lvl 1, cost 10xp: A good (+3) success gives penetrating+1 to the attack. \\
lvl 2, cost 20xp: A very good (+6) success gives penetrating+2. \\
lvl 3, cost 30xp: A perfect (+9) success gives penetrating+3.


\skill{Skill cost special "staggering":} makes a good hit more difficult to parry: \\
lvl 1, cost 10xp: A good (+3) success gives mod-1 to defend against the attack.  \\
lvl 1, cost 20xp: A good (+6) success gives mod-2 to defend against the attack.  \\
lvl 1, cost 30xp: A good (+9) success gives mod-3 to defend against the attack.  \\


\skill{Skill cost special "slugger":} gives extra damage with weapons that have extra strength bonuses. Each level of slugger gives excessive strength damage bonus counted as (extra str)/1 instead of (extra strength)/3 as per "strength bonus". Slugger also gives damage bonuses to brawl and martial arts equal to the level of slugger. \\
lvl  1  costs  10 \\
lvl  2  costs  20 \\
lvl  3  costs  30 \\
lvl  4  costs  50 \\
lvl  5  costs  75 \\
lvl  6  costs 100 \\
lvl +1  costs +30

E.g: Samuel Strong, has a sword with str 5 requirement, and he has str 7. With his slugger=2 he gets one point of damage bonus for each of his two extra strength points. Thus he has dam+2 from his combination of extra strength and slugger.


\skill{Skill scf 0.5 "charge":} reduces movement penalties by one for each skill level when the user moves to attack an opponent. The initiative for the attack is also increased equal to the level of "charge". However, the maximum initiative bonus and movement penalty reduction is limited by the number of steps taken to reach the opponent. Also, the attack's maximum damage is increased by one for each tree steps taken to reach the opponent. A charge costs one stamina. A charge must proceed along the shortest possible straight line to the target. If a character turns in a charge then it's only the final straight distance that counts as the distance charged and limits the bonuses.

The charge effects are only applied to the first attack the character makes on the charged target. The charge bonus effects is removed if the Hero takes any other actions on the way to the target.

Both the charge movement and attack is done at elevated initiative. This is meant to decrease the chances of the target, or bystander, interrupting the charge attack in progress.


\skill{Skill scf 0.5 "wrestle":} Better grabbing, holding, and breaking free. Characters with wrestling can add their level of wrestling to their str and dex values when grabbing, holding, and breaking loose.


\skill{Skill scf 0.5 "tackle":} Better tackles.
When tackling: add your tackling level to the resistance rolls.

Optional defending: take an action and add your "tackle" or "block" level to the resist roll.

Thus tackle also makes you more likely to remain on your feet since the diff of the resistance roll is applied to the following dex roll.


\skill{Skill scf 0.5 "block":} makes it easier to block opponents moving in base contact. Attempts at blocking can ignore initiative if the character passes a block roll. Passing the roll also means the blocking action requires only 1ap instead of 3ap. If the Hero is being tackled he can take a normal 3ap action and add the level of "block" to his resistance roll.

\emph{Note:} that if the attacker has declared "tackle" before the blocker declares "block" it's a reactive block and takes 3ap. If the blocker declares "block" before the passing opponent decides to tackle it only takes 1ap.

If the blocking character declares movement 0, and does not move, he gets a mod+1 to block actions.

\emph{Note:} block also makes you more likely to remain on your feet after a tackle since the diff of the resistance roll is applied to the following dex roll.


\skill{Skill scf 0.5 "back stab":} is used to attack an unaware opponent with a melee weapon. Usually it's preceded by a successful sneak, but it's enough that the target are unaware of the attacker and fails the perception roll.

The attacker is probably "behind" (out of vision field) of the target, unless he is exceptionally good at sneaking. First roll for the back stab, then for the weapon attack. A back stab costs 1ap extra above the cost of the weapon attack action.\\
A failed back stab attack is treated as a normal attack.\\
A successful +0 back stab attack gives double damage and penetrating +1 \\
A good +3 back stab attack gives triple damage and penetrating +2 \\
A very good +6 back stab attack gives quadruple damage and penetrating +3 \\
A perfect +9 back stab attack gives quintuple damage and penetrating +4 \\
And so on...

When combined with accurate, consistent, or precise, this becomes brutally lethal.


\skill{Skill cost special "tank":} Can reduce some penalties of heavy armour, while keeping the good stuff. Penalties to e.g: dexterity, encumbrance, movement, vision, strength requirements are reduced a bit according to the level of the tank skill.\\
lvl 1, cost 10: chain mail has the modifications of leather, leather has no mods. \\
lvl 2, cost 20: plate mail has the mods of chain mail \\
lvl 3, cost 30: full plate has the modifications of plate mail \\
lvl 4, cost 40: heavy plate has the modifications of full plate \\
lvl 5-8 cost+15 each: chain, plate, full, heavy: as two classes lighter \\
lvl 9-11 cost+20 each: plate, full, heavy: as three classes lighter \\
lvl 12-13 cost+40 each: full, heavy: as four classes lighter \\
lvl 14 cost+80: heavy plate as five classes lighter, i.e. as no armour.\\
%I.e: keeping the penalties constant while upgrading to a one class heavier armour costs +50xp. << for old costs @ +10 for all steps
E.g:\\
Chain mail as leather: 10xp\\
Plate as leather: 70xp\\ % diff 60
Full plate as leather: 140xp\\ % diff 70
Heavy plate as leather: 240xp\\ % diff 100
Heavy plate as night gown: 320xp

%  5 +15  55 chain -2
%  6 +15  70 plate -2
%  7 +15  85 full  -2
%  8 +15 100 hvy   -2
%
%  9 +20 120 plate -3
% 10 +20 140 full  -3
% 11 +20 160 hvy   -3
%
% 12 +40 200 full  -4
% 13 +40 240 hvy   -4
%
% 14 +80 320 hvy   -5


\skill{Skill scf 0.5 "synchronise":} Characters that pass their "synchronise" rolls can make simultaneous attacks. If they are hacking at the same target, that target cannot take different actions to defend against the attacks. The target can roll for double, triple, etc to defend against several simultaneous attacks in one action however.

Roll for synchronise before making the attacks for each of the attackers to see if they are synchronised. The synchronised actions happen in the lowest initiative of the characters involved.

It is possible to synchronise other actions as well. The characters must be in vision or communication range to be able to synchronise their actions.
Synchronised "double, triple, etc" attacks must also be defended against in one action.


%TODO: ? remove this since it makes positioning less relevant ?
\skill{Skill cost special "angle of attack":} Characters with angle of attack can make attacks, but not defence, at greater angle from their facing than normal. \\
lvl 1, cost 10xp: Attacks to the side are at mod-0, attacks diagonally behind is mod-3, while attacks directly behind is mod-6. \\
lvl 2, cost 20xp: Attacks to the side and diagonally behind is mod-0, attacks directly behind is mod-3. \\
lvl 3, cost 30xp: Attacks in any direction is mod-0.


%TODO: ? should we have a angled defence ?  makes surrounding and most positioning irrelevant, which is kind of shit.


\closeskillslist

















%-------------------------------------------------------------------------------
% support skills
%---------------

\phantomsection\addcontentsline{toc}{section}{support}
\section*{Support skills}

\openskillslist


\skill{Skill scf 1.0 "scary":} The character has spent time, money and effort to adapt a scary demeanour and behaviour. This can be used to strike fear into the opposition, making them easier to vanquish. Scary has a range equal to the level.

To advance into or in the scary range of an opponent one must succeed with a psy+3 vs scary roll.

To initiate melee combat with a scary target one must succeed with a psy vs scary roll.

To make a ranged attack against a scary target when within the scary range one must succeed with a psy+3/scary roll.

A target fail-1 means a wasted action, the character is too afraid to do anything. A target fail-3 means retreat to safe distance. A fail-6 means flee.

If an opponent succeeds overcoming the scary effect once he is immune to it for the rest of the encounter, for that type of action. E.g: if Gergish the goblin manages to advance into scare range of Halfvard the Hero, succeeding with the psy+3 \vs scary roll he doesn't need to roll for future advancing. However he must still roll psy \vs scary for attacking until he has succeeded with that once.

Use aura to denote scary in new maptool.


\skill{Skill scf 1.0 "roar":} allows the user to scream great loud intimidating battle roars which might intimidate and scare off the opposition. The bellower will roll roar+str/3 \vs psy+str/3 for each target. On a success the target cannot attack for 1d3 rounds, and will have an initiative-5 mod against the roar:er. If it is a good (3+) success then the targets must flee away from the noise maker, and hopefully towards safety... until he succeeds with a psy roll (one per round), at least one round.

This skill can be used once per new encounter, when reinforcements arrive, etc, but not several times on the same targets.

However, some monsters react in the opposite way to shouting characters, and some monsters just don't care.


\skill{Skill scf 1.0 "rousing roar":} The character can give rousing battle screams that energise his companions to do more bloodshed.
Cost 1 stamina per attempt. Takes 1a (primary). The user and every companion in range 2 regains 1 stamina. I.e. the user spends then regains, ending up with net 0, if he succeeds, otherwise he loses one stamina.
A group of bellowing barbarians get pumped very quickly...


\skill{Skill scf 1.0 "battle cry":} The character can give rousing battle cries that spur his fellow murderers to greater achievements.
Costs 1 stamina per attempt, Takes 1a. Every companion in range 2 gets mod+1 dam+1 to all attacks in the round.


\skill{Skill scf 1.0 "taunt":} The character can taunt the opposition to attack him and not his fellow grave robbers. The taunted gets the taunter's success diff as mod to his int rolls:

The taunted must pass an int roll if he wants to attack anyone other than the taunter. If the target fails-3 he must attack even if it means limited movement. If the target fails-6 or worse he must try to fight his way to the taunter to attack him if there is no clear way to get to him. But no, the grinning goblin cannot force the hair brained hero to go through the minotaur if there is a clear way around it.

Taunting takes one round at normal difficulty, or one action with mod-3. Taunt can target a crowd instead of one monster, but with mod-3. If the target does not understand the language, it is mod-3.

The taunt is in effect success diff rounds, or until the target succeeds with the modified int roll, or the taunter is dead of course.

Each successive taunt attempt against the same target suffers cumulative mod-3, unless successful. E.g: third attempt against the same target is at mod-6.

Leader opponents may still assign minions to do the attacking, unless he fails -3 or worse.


\skill{Skill scf 0.5 "Oy!":} The character can activate others in one fast+2 action, costing 1ap, instead of a full action. The target is activated as in the initiative of the activating character as usual.
Roll for "Oy!" with the usual mods. Range is equal to level of Oy! plus the perception of the target.

If the roll fails the activation action takes a regular full action.


\skill{Skill scf 1.0 "rally":} The character can give a short talk to lift the spirits of all his fellows nearby, giving a mod+1 to all actions for a while. Duration charisma/3 (round down) rounds. The range is limited to the perception of affected character for the boost to go into effect, but after being boosted the Hero can move further away. Roll for success subject to normal mod penalties. Rallying takes one action.

Instead of boosting all nearby allies he can choose to boost only one target and get a mod+3 to the rally roll and range. Spending a full round instead of one action on the rallying attempt also gives a mod+3 to the roll and the range.

A target can only get rallying bonuses from one "rallyer" at a time.


\skill{Skill scf 1.0 "leader":} The character is a natural leader. He gives a mod+1 to all actions of his fellows within range (level of "leader"), but only to fellows with psy less than or equal to the level of "leader". The leader effect requires no rolls or actions per default.

The leader can spend a normal (3ap) action and get a temporary cha/3 (round down) bonus to his leader skill level for that round.

A target can only get leadership bonuses from one leader at a time.

E.g: Putte Paladin has leader 5. Jolly James stands within 5 range from PP, and has psy 4, he then gets a mod+1. Faraway Fred stands 7 squares away and does not get a mod+1. Strongminded Sam stands next to PP but has psy 8, and does not get a mod+1.

A leader can also give orders to leader/3 (round down) extra characters under his leadership for each action giving orders, activating by Oy!, and similar. Great for handling henchmen.


\skill{Skill scf 1.0 "tactician":} The tactician can shout advice to fellow compatriots giving mod+1 to all combat actions. The advice range is limited to the perception of the affected allies. Tactician requires no rolls per default but each "advised" target hero requires one fast+2 (1ap) action. The advice mod+1 bonus persists until end of the round for all combat actions against the original opponent. Tacticians advice does not activate the ally.

The tactician can "cover" a number of characters and monsters combined equal to his level of tactician. The mod+1 only applies to actions done by a "covered" hero against a "covered" opponent. Covering range is limited to line of sight. Characters and monsters of normal size or smaller do not count as blocking line of sight.

The tactician can spend a normal (3ap) action to get an int/3 (round down) bonus to his tactician skill for that round.

A target can only get tactical bonuses from one tactician at a time.


%\skill{Skill scf 2.0 "brilliant tactician":} As tactician, but mod+2 bonus.


\closeskillslist














%-------------------------------------------------------------------------------
% other skills
%-------------


\phantomsection\addcontentsline{toc}{section}{other}
\section*{Other skills}

Miscellaneous skills that are not necessarily combat related.



\openskillslist


\skill{Skill scf 0.5 "jump":} mod+1 to jump actions the for each skill level of jump.


\skill{Skill scf 0.5 "climb":} mod+1 to climb actions for each skill level of climb.


\skill{Skill scf 1.0 "balance":} A well balanced character is less likely to fall down or loose his footing. A limited skill, but a great way to keep on your feet when tackling, moving about in rough terrain, etc.
The character can ignore movement and environment modifiers equal to his level of "balance" when rolling dex for balance related events such as running, falling down, etc. % But not for other rolls, actions and activities.

"Balance" also removes modifiers for crappy terrain, half blocked squares, undergrowth, debris, rocks, tunnel walls, etc, equal to the level of balance.

At the start of each round the "balance"d character can also attempt a balance roll to remove success diff modification points from off balance penalties from the previous round. This costs 1 stamina to attempt and is a 1ap action but follows regular initiative order.


\skill{Skill scf 0.5 "ride":} is useful when sitting on a horse. Getting the horse to go where you want, take actions from horseback, or special horse maneuvers require rolling for ride. Moving around on horseback doesn't require spending any extra action points, but taking special horse maneuvers do. Don't declare movement for the rider. Instead the Hero declares movement for the horse. The ride skill roll replaces movement mods.

Roll for ride once per round after having declared movement and action points. Any fail diff is applied as mod to all actions taken from horseback for that round. Riding at horse speed maneuver is mod-0, at walk is mod-1, run mod-2, dash mod-3.\\
Fail-3: the rider has to spend one action to keep control of the horse.\\
Fail-6: almost falls off, and must spend the round getting back in order.\\
Fail-9: falls off.

%TODO: ? remove the horse riding destination deviation ?
%In addition, a fail also means you might not get the horse to the exact square you want when you move it. The final movement destination deviates in direction 1d8 and distance equals to the fail. Distance is capped by movement speed: walk is max 1sq, run is max 3sq, dash is max 6sq.

Also, if you can ride you can drive a horse and cart. Maneuvering the cart in high speed or difficult terrain requires ride rolls.

Ride also increases the travel distance on the world map. The Hero can travel up to his con+ride each day, provided the horse is not the limiting factor. Horses and other riding animals have a cruise rating that sets the cap on how far they can travel each day.


\skill{Skill scf 0.5 "travel":} will allow the character to travel the distances on region maps faster. He will add his level of travel to his con when calculating distance per day. Also applies when riding, and can exceed the distance rating of the horse. Just add the level of travel to the base distance.


\skill{Skill scf 0.5 "quick draw":} allows the user to draw/ready equipment faster, in a normal 3ap action instead of a full round. Roll for quickdraw to see if it takes one 3ap action or one round to draw/ready. Quickdraw also reduces the rummage time by one round for each skill step. Rummage will always take at least one round though.

Each level of quick draw also allows the character to equip, readily accessibly, one item which he can always draw and ready in one 3ap action instead of a round.
E.g: Rapid Rudolf has quick draw = 4, he has an axe, a shield and two throwing knives on his body, readily available at all times. Those items he can always draw and ready in one 3ap action.

Optional: A successful quick draw roll on an item in a quick draw slot means the draw action is an instant 0ap action instead of a normal 3ap action. Very useful for quick changing weapons or drawing new arrows and throwing knives.

Optional: When drawing items not in quick draw slots the success level improves the speed:\\
success+3 means 2ap draw action\\
success+6 means 1ap draw action\\
success+9 means an instant 0ap draw action

Quick draw also allows for holstering and stowing away items. Same rules apply as for drawing.


\skill{Skill scf 0.25 "häfva":} is the subtle art of emptying a flask straight into your stomach without spilling a drop or shooting it out of your nose. It allows the master practitioner to drink a potion or other liquid dose in one action instead of one round.

Better result means faster drinking:\\
success+3 takes only 2ap\\
success+6 takes only 1ap\\
success+9 is a 0ap instant action


\skill{Skill scf 0.5 "dead drunk":} is the reflex of the in-bred reptilian brain to drink when passed out. If the character has a useful potion flask in an easy to reach place he will automatically pull it out and try to drink it in one round. Roll to succeed. The "dead drunk" roll ignores all modifications.

He will only drink one bottle each time he is passed out, but will try several times until successful. If there are different possible bottles he will instinctively choose the right one, the one which best helps the reason why he is passed out.


\skill{Skill scf 0.20 "possum":} is for those who want to lie down and take a breather during a fight. The Hero can act dead for a bit, to fool most opponents to pass by and attack something else. A successful roll will fool anyone who does not stop to look (take an action and roll per). An enemy which does stop to take a look has the feign death success diff as negative mod for the per roll. Anyone who stops for a round and tries a "find" on the acting corpse will not be fooled even if they fail their find roll.

Modified by damage:\\
half hp: mod+1\\
quarter hp: mod+2\\
zero hp: mod+3


\skill{Skill scf 0.5 "arrow recovery":} allows the happy hunter to recover arrows from downed targets. Any targets, even if they have not been fired upon (GM discretion so it doesn't get out of hand). All arrows recovered are regular  base damage +0 type arrows suitable for the weapon at hand.\\
Success +0 means the body has one recoverable arrow. \\
Success +3 gives two arrows, and so on...

Modify the roll with the armour absorption of the target (mod=-abs). Also limit the maximum amount of arrows to one per 5 max hp of the target.
Searching a target takes one round, but quick looting can be used to reduce the time.


\skill{Skill scf 0.5 "gossip":} is a good way to learn what is going on in the region. Are the woods safe to travel? Any clue where that bandit camp might be? Is the merchant trustworthy? What are the general thoughts on the local mayor?

Roll for gossip once per week mod+6, per day mod+3, per evening mod-0, and if visiting multiple sites in an evening it is mod-3 at each site.

Liberal with beer mod+3, stick out mod-3, strange information mod-3, dangerous information mod-3, etc.

Call for gossip rolls when you want information from a general evening out, or special site. Also if you are spending some time with a specific person (mod-3).

E.g: spending an evening in a busy tavern is the typical mod-0 roll.

Also use gossip post-fact to see if the character has knowledge of any "common" facts or information. E.g: Have you heard anything about the BlackStreak raiders?


\skill{Skill scf 0.5 "find":} is used for searching specific items/details/etc. Find is great for long term intent searching. The secret lock for the door, the movable book shelf, the small key in the pile of corpses, the hidden trap you know must be there. Find is heavily modified by the environment.

Searching an area with find takes 1r/sqr at normal speed.

Adding 10x time to the search gives mod+3. Adding 100x time to the search gives mod+6. Speedup to half normal time is mod-3, quarter of normal time is mod-6. Subtract 10\% from the time for every success step.

Perception rolls take care of all normal detection, find is useless there.


\skill{Skill scf 0.5 "fast find":} is used to quickly search for hidden items/details/etc. The character can make find rolls on a number of squares equal to his fast find level each round. It does not affect the time modifiers for careful finding.


%TODO: fix sneak, spotting, hiding, finding
\skill{Skill scf 0.5 "sneak":} is used to move around unseen and hide from casual inspection. Sneak hides both the Hero and objects the Hero wants to keep unseen. Spending more time, rounds instead of actions, the Hero can seriously camouflage himself, someone else, or hide objects so they are more difficult to find.

Regular combat sneaking: Sneaking costs 1ap for the whole round. Each round will require at least one roll. Sneaking through different terrains and lighting conditions can require several rolls per round. Quickly stuffing away an object out of sight takes one action (3ap).

The sneak success diff is then used as a mod against a spotting opponent's perception roll. See the "Sneaking and Spotting" rules for further info.
Generally it's important to stay outside the field of vision of any opponents, and very important to stay outside the perception range if within the field of vision.

Action and environment modifiers:\\
Movement has modifiers as per usual mod stack: movement type and mobile\\
Standing still and not moving is mod+3\\
A deep shadow, bush, or partially obscuring object is mod+3\\

When moving through an area:\\
area is clean/open -3\\
area has some stuff =0\\
area is cluttered +3\\
daylight or otherwise well lit -3\\
lighting by torch and otherwise dark =0\\
night or unlit dark area +3\\
weather is calm, still and quiet -3\\
weather is as usual, some wind and moving vegetation =0\\
weather is loud and everything moves, e.g. stormy  +3

E.g: Sneaky Susan wants to get past Drowsy Dory who is on guard duty. She has sneak 7. It's daytime but relatively dense overgrown garden. She has mod-3 for the lighting and mod+3 for the dense vegetation. SS traces a route that should take her past DD's in just 3r at walk speed (giving another mod-2 since she has mobile 1). First round SS is outside DD's field of vision and is safe as long as she passes her sneak roll. Second round she rolls 4, success+1. But since she is now in DD's field of view she can get spotted. DD rolls for perception (mod-1 for SS success diff, and mod-3 for inattentiveness), and fails to spot SS. But third round SS comes too close to DD, within DD's perception radius. Who knew DD had per 8! That's not normal for a sleepy npc guard! Thus DD automatically spots SS when she passes inside the field of view and inside the perception range. Tough luck!

Sneak can also be used to cover trails when travelling. Successful sneak applies the diff as mod to the track roll of potential followers.

One cannot start sneaking while being detected and in the opponent's vision arc. He has to get out of line of sight, either behind walls or objects, then start sneaking. After he successfully sneaks he can move back into line of sight in the same round.

When sneaking around in a dungeon to scout and map out it's easier to just roll every 3, 5, 10r or so, or simply when the environment changes significantly, such as moving into a lit area or when entering a new room which may be occupied. Sneaking past guards and possible opponents should always require rolls.

Hiding things:\\
Camouflaging oneself, others, and hiding objects takes more time. The sneak roll diff will be the modifier to the find roll for the opponent. The mod is capped by how many rounds was spent hiding or camouflaging, and is also limited by the environment.

E.g: Spending four rounds hiding a loot stash for later pickup and rolling a success+5 still caps the find mod-4 since only four rounds were spent hiding it.
Hiding something small in a dark pile of rubbish is much easier, perhaps mod+3, then camouflaging an orc in the corner of a somewhat clean dungeon


\skill{Skill scf 0.5 "pickpocket":} allows the character to steal things from another character or enemy's person without detection. Requires not being detected, e.g. sneaking. Pick pocket success is applied as a diff against the target's perception roll to see if he notices the theft.

Stealing held item mod-9 and str vs str-3 \\
Stealing item in front container mod-6 and normal rummage \\
Stealing item in back container mod-3 and normal rummage \\
Stealing item in pocket/belt mod=0 \\
Stealing item hanging free on clothing/belt mod+3


\skill{Skill scf 0.5 "locks \& traps":} allows the character to pick locks, disarm traps, or set them, and handle other interesting mechanics which the character might have knowledge of.

Picking a lock usually takes 1d4 rounds or so, depending on the lock. Subtract one round for a good success, and so on. Some locks take longer or have mods on them depending on difficulty. It is also possible to lock a lock.
Most serious locks require the use of lock picks and special tools. Simple locks might work with improvised or other equipment.

Traps must first be found and identified with "locks \& traps", "find", or "dungeoneering". After that roll for disarming the trap, or perhaps simpler to trigger it in a safe manner? Disarm time and difficulty depends on the type of trap and what tools and equipment is available. But a good start is perhaps 1d6 rounds.
A regular fail means you didn't manage to disarm the trap, but didn't trigger it either. A fail-3 means you triggered the trap but have one action (and declared movement) to get out of the way before you get hurt. Fail-6 is crap and means you have sprung the trap in a bad way, prepare for pain.

Setting traps takes time. Each kind of trap has its own time requirement. A good success+3 reduces the time by 25\%, A very good success+6 reduces the time by 50\%, and an excellent success+9 reduces the time by 75\%.


\skill{Skill scf 1.0 "McGyverism":} is the ability to turn miscellaneous everyday objects and debris into improvised useful gadgets and contraptions. The use of this skill is very much dependent on GM discretion.

All contraptions are built from "stuff", using "tools". A suggested contraption is given "stuff" and "tools" requirements by the GM, along with time and difficulty. This is the minimum requirements to be able to complete the task on regular time and with the expected result. Any lacking points "stuff" is mod-1 and any lacking points of "tools" is mod-3.
All the required "stuff" is consumed during construction but none of the tools.

Gathering material from the surroundings is common. Use the find skill to see how much "stuff" can be scrounged from the surrounding 3x3sq area. This is heavily influenced by the type of surroundings. Gathering time is 1r per enc stuff or one action per enc stuff if the character can quickloot.

It is a good idea to pack "stuff" to be able to use for the contraptions. Each superfluous enc of "stuff" gives a mod+1 to the roll. Stuff is a special type of equipment that is not really useful for much else. Some regular items can qualify as stuff, but only at half enc. E.g: rope, twine, hide/skin, staffs/spears/knives, cooking utensils, spices, etc.

Tools help. Each superfluous enc of "tools" give mod+1 to the McGyverism roll. Some regular items qualify as "tools" but at half enc. E.g: knives, etc. However, no items can qualify as both stuff and tools at the same time.

After completion and possible use it is possible to recover some of the "stuff" used in construction. Roll again for McGyverism: \\
Success+0: recover 25\% of the materials in enc rounds. \\
Success+3: recover 50\% in enc/2 rounds. \\
Success+6: recover 75\% in enc/3 rounds. \\
Success+9: recover 100\% in enc/4 rounds. \\
The character can choose which items of "stuff" to recover first if any specific items were used in the construction.

Example stuff and tools requirements: \\
small trap: 1d4 enc stuff, 1 enc tools \\
temporary weapon: 5/success enc stuff \\
small bridge: sq$^2$ * 10 enc stuff, 2 + sq enc tools.\\
trap disarm gadget: 1d3 enc stuff, 1d3 enc tools


\skill{Skill scf 0.5 "track":} is finding stuff in the wilderness. It is used for following a trail for example, and to forage for food when the provisions run out.

Track can also be used to hide trails when travelling. Successful track applies the diff as mod to the track roll of potential followers.

Track also sets the vision radius of the party when travelling on regional maps.

A track mod-6 can detect followers or other groups within the track radius, mod further if the target is sneaking.

Foraging or hunting takes half a day. Roll against track. If it succeeds the character has found food for the day. Foraged food does not last long and will spoil after three days. \\
Success +0 found food for this day for one person. \\
Success +3 found enough for two people. \\
Success +6 found enough for three people. \\
Success +9 found enough for five people.

If the tracker also succeeds with a "traps" roll he doubles his food production. If he succeeds with a bow or crossbow he also doubles his production.

Track can be used to find locations in the wilderness. Each day they can move to and search a number of squares equal to their track level, provided they have the con to move far enough. For actively hidden or camouflaged locations they should also pass find rolls.


\skill{Skill scf 0.5 "literate":} reading and writing, reading maps, road signs, here be dragons, beware of the beholder, things like that. A quick sign takes an action to read. A note perhaps a round or two. A letter can take ten to a hundred rounds perhaps. \\
Adding 10x longer time to a small note gives mod+3. This means tens of rounds. \\
Adding 100x longer time to a small note gives mod+6. This means that it probably requires longer resting periods.


\skill{Skill scf 0.5 "counting":} dividing plunder, more for me, less for you. \\
Adding 10x time to a simple calculation gives a mod+3. Generally while resting. \\
Adding 100x time to a simple calculation gives a mod+6. Generally when camping.

Counting is modified when under stress unless passing a psy roll. Modify chance with failed psy diff. Counting something simple is a full 1r action. More complex series and expressions normally take 10-int rounds, full round actions.

The problems of miscounting, e.g. calculating division of plunder when one or both of the parties fail their counting rolls: First, roll for counting with a good mod, they can take their time in most situations. Allow mod+3 if they are dividing on the spot after an adventure, mod+6 if they have set up camp etc.
If one party fails his count then adjust with -10\% of proper value per failure point, unless the other party is honest enough to correct him. If both parties fail then adjust in either direction and no-one will be the wiser.


\skill{Skill scf 0.66 "haggle":} is useful when buying and selling items. Make a resistance roll between the seller and buyer and adjust the price by 10\% * diff.

You don't have to haggle, you can just pay the man and be gone. If you do haggle, it might be cheaper, or it might be more expensive...

When dealing with basic uninteresting purchases, just roll for haggle: \\
success +0 : 10\% cheaper \\
success +3 : 20\% cheaper \\
success +6 : 33\% cheaper \\
success +9 : 50\% cheaper


\skill{Skill scf 0.5 "appraise":} is used to estimate the monetary value of weapons, utility items, jewellery, artefacts, etc, and take into account the current market situation, location, etc.
Success means a good estimate within 25\% of actual value. Success+3 or better means a spot on estimate. Failure gives a wrong estimate, above or below. fail=1: factor 2 off. fail=2: factor 5 off. Fail 3 or worse, no idea of the value.


\skill{Skill scf 0.5 "size up":} is the skill of estimating the approximate XP level and characteristics of an opponent or other character close by. It takes an action to make an assessment. Taking full round gives mod+1 and three full rounds give mod+3. \\
fail -3: within factor two error estimate. \\
success +0: within $\pm$25\% estimate. \\
success +3: within $\pm$10\% estimate. \\
success +6: exact XP value.


\skill{Skill scf 1.0 "famous":} are those who get recognised. Good for getting random people to hopefully be impressed and treat you well, bad for when your nemesis sends hired henchmen to track you down for a late night killing.


\skill{Skill scf 0.5 "monsterology":} is the study of monsters and their traits and habits. How fast does a troll regenerate? How many bugs can usually be found around a cave troll? What is the maneuver mobility of an orc? How maneuverable are the biterunners and how large are their packs? Things that can be very useful to know for the careful adventurer.

This can also be used to figure out which monsters are up ahead from clues and early warning information. E.g: this particular horrible stench probably means that there are iffygriffs up ahead, or that track is done by a grey stalker demon.

For very simple common monsters such as goblins, orcs, etc one can roll a modified int roll instead of monsterology, but the information gathered that way is less reliable and detailed. E.g: int-3 or -6.


\skill{Skill scf 0.5 "dungeoneering":} is there another exit perhaps? Where might the treasure chamber be? Any serious trap construction around? Does this castle have a secret escape tunnel?

For very simple things a modified int roll can substitute a dungeoneering roll, but the information is less detailed and reliable. E.g: int-3 or -6.


\skill{Skill scf 0.5 \emph{language}:} might be good to know different languages, like the one that strange ghost spoke, when it warned you about the trap you just triggered.

Adding 3x time to a short text gives mod+1 \\
Adding 6x time to a short text gives mod+2 \\
Adding 10x time to a short text gives mod+3

The languages commonly encountered: Common, Ancient, Dwarvish, Elvish, Svartlingo, Lizardzpeak, ...


\skill{Skill scf 0.5 "histography":} is the knowing of when and where and what and who of all the far away and long lost places and people of the world.


\skill{Skill scf 0.5 "first aid":} heals 1 hp and 1 pain. Max 1 success per wound. "First aid" can be attempted on one self with a mod-3. A healing attempt takes 2+1d4 rounds. The practitioner needs a "first aid kit" to use the skill, or suffer a mod-3. First aid kits usually have limited amount of uses before they need to be restocked.

First aid can also be used to eliminate a poison from doing more damage, and heal one wound if already taken from the poison.


\skill{Skill scf 1.0 "medicine":} heals 2+1d4 hp, limited by the damage of the wound. Max one success per wound. Also remove all pain from the wound. Medicine can be attempted on one self with a mod-3. A healing attempt takes 4+1d6 rounds. The practitioner needs a "medicine kit" to use this skill or suffer a mod-3. Medkits usually have limited amount of uses before they need to be restocked.

Medicine can also be used to eliminate a poison from doing more damage, and heal 2+1d4hp of the poison damage.

Medicine can also be used once per day, during relaxed long term treatment, to heal success diff hp to the patient without being applied to specific wounds.


\skill{Skill scf 1.0 "acrobatics":} Looks cool and can be of real use. The character can attempt acrobatics to move above, around, through obstacles in the way of getting to the destination. Modified by obstacle difficulty and mod-3 per adjacent obstacle in the path beyond the first, or if there are free squares in between obstacles it is possible to separate into separate acrobatics actions. Must also have enough movement points left to complete the movement or go off balance. Takes one action per obstacle. \\
fail-[1-2] stops moving. \\
fail-3 falls down. \\
fail-6 falls and takes (dex roll: 0/1) damage penetrating.

Acrobatics is modified heavily by encumbrance. Each encumbrance gives mod-3.
Acrobatics movement ignores the "block" skill.


\skill{Skill scf 0.5 "pack mule":} Adds +1 per lvl to the character's strength when calculating encumbrance.


\skill{Skill scf 0.5 "animal command":} Can command trained animals. The command is modified by how well the animal is trained. Each animal has a set of commands it knows. Each attempt usually takes 1a (3ap).

A character with animal command can control at most his lvl/2 (round up) animals.


\skill{Skill scf 1.0 "companion command":} Can command companions such as henchmen, demons, monsters, critters, etc. The command roll is modified by the ties, training and intelligence of the companion, as well as the nature of the order. Position, attack, and defend orders are simpler when they don't put the companion in mortal danger. Telling even your trusted goblin henchman to attack the charging minotaur is going to be at least a mod-3.

Issuing a command takes 1a and the companion must be within earshot. Commanding companions beyond line of sight is mod-3. Spending a full round gives mod+3 when issuing order.

A character with companion command can control as many companions as his skill level/2 (round up).


\skill{Skill scf 2.0 "unfazed":} The character can make as many 45 or 90deg free facing changes per round as his level of unfazed, ignoring initiative. Changing facing is a free instant interrupt action.


\skill{Skill scf 0.5 "meditate":} A character who successfully meditates regains two stamina per round instead of 1. It takes one round to sit down and attempt to start meditating, this round is treated as normal rest round. Once the character has successfully started meditating he will regain two stamina per round still meditating instead of one.
Meditating in a combat zone is mod-3. Need "calm" in range 10-psy.


\skill{Skill scf 1.0 "tunnelist":} This Heroic digger can carve out tunnels and hack through objects faster than anyone else. Each level of "tunnelist" increases digging/hacking speed by 50\% of base speed. E.g: A tunnelist 4 is hacking away at 3x the normal speed.

The tunnelist is also less affected by cramped spaces and limited squares. Each level of tunnelist cancels one point of negative mods from limited cramped squares such as from tunnel outcrops and close walls. This effect is the same as for balance, but limited in that it only applies to crap terrain from encroaching walls, outcrops, tunnels, etc.


\closeskillslist
















%-------------------------------------------------------------------------------
% meta skills
%------------


\phantomsection\addcontentsline{toc}{section}{meta}
\section*{Meta skills}

Meta skills don't generally give new actions, boost attacks or rolls, etc. Instead they tend to change the rules of the game.



\openskillslist


\skill{Skill scf 2.0 "focus":} The character can choose to focus more on some actions than others. E.g: focusing more on the 2nd action (attack) than the 1st action (parry).

A character can shift around mods between actions, take heavier mods on an earlier action to relieve mods on a later one, or force heavier mods on future actions to make the current action easier to succeed with. The total amount of mod points that can be focus shifted between actions in a round is equal to the level of focus.

Focus points cannot raise the chance to succeed higher than the original unmodified value. Focus can only cancel out \emph{action point} modifications, not mods from movement, terrain, tohit, toparry, etc.

Saving focus points: Take heavier mods on the current action than necessary.
E.g: Focused Fred is currently attempting a shield parry with 9 to succeed. He can then take a focus mod-2, making the parry have a chance of 7 to succeed, and save two focus points. Those two focus points can be spent to cancel mods on future actions. E.g: The next action is an attack, and FF has mod-4 total from his declared action points and movement speed. Focused Fred can now spend the two saved focus points to cancel two of the four mod penalties, turning the mod-4 into a mod-2 for the attack.

Saved focus points are lost at the end of the round if they have not been spent before that.

Borrowing focus points: Force heavier mods on future actions to cancel mods on the current action. E.g: Focused Fred wants to attack but suffers a mod-4. He choose to focus +3 points to improve his chances, giving a total mod-1. His next action is another attack, but at a simpler target with only mod-2. He choose to pay back his 3 focused points, giving a total mod-5 for the attack.

A Hero cannot borrow more focus points from future actions than the amount of ap he has left for that round.

Unpaid negative focus points carried over to next round costs one stamina, gives a baseline mod equal to the unpaid points for the round, and deduct one ap per point from the declared action points.

A Hero can spend action points to pay for borrowed focus points. At the end of the round any unused ap will be used to cover borrowed focus points.

Note that even if Focus does stack with Overextend, it doesn't help, since overextend applies the penalty mod immediately.


\skill{Skill scf 2.0 "overextend":} to use more action points than was declared for the round. Overextended ap come at a high price and overextend can only be used once per round.
\begin{itemize}
    \item When using overextend to get more ap for a round each ap adds mod-1 to the mod stack \emph{before} the action is taken.
    \item The overextended ap are subtracted from the declared ap in the next round.
    \item The overextended ap mod is carried over to the next round.
    \item Overextend costs one stamina to use.
\end{itemize}

E.g: Overambitious Oscar wants to make two more attacks with his fast daggers (fast+1 actions, 2ap each) but he only has 1 ap left. Luckily, he has overextend 5. Since overextend can only be used once per round he pulls out 3 extra ap, giving him a total of 4 ap. Immediately he takes a mod-3 to his mod stack for this round, one for each extra ap. He pays one stamina. He then makes the two attacks. Then the round ends. In the new round OO now has a mod-3 from the 3 ap overextended, as well as deducting 3 from his declared ap. This round is not looking goood ...


\skill{Skill scf 2.0 "multitasking":} allows to declare extra ap to do something else at the same time as performing full round actions. Multitasking costs one stamina to use. Worth noting that most multi-round actions will take the highest mod incurred during any round of the action.

Actions that take a full round or multiple rounds to perform always require the character's base max ap each round. With multitasking the Hero can declare extra ap which can be used for other actions. He can declare one extra ap per level of multitasking.

E.g: Concurrent Claire is shooting a bow, which takes her two rounds. But she also wants to be able to avoid incoming attacks from the goblin close to her. She has 5 base max ap, and multitasking 4. For the first round she declares 8ap, spends a stamina, and takes the usual mod-3 for the 3 extra ap. When the goblin attacks she avoids, spending 3ap, then the 2-round bow fire action eats her 5 base max ap, leaving her with 0ap left. Her ally Heroic Herman kills the goblin. In the next round she declares only 5ap since the goblin is already dead. She completes the bow attack and gets mod-3 to the roll since in round 1 she took a mod-3 from declaring 8ap.

Note: that in round two she could just as well have declared 8ap just like in the first round and still get only mod-3 to the attack, but she didn't want to spend the extra stamina.


% kolla kostnaden för quick till olika nivåer:

% 3 -> 4 : 16-9=  7    @2 = 14    @3 = 21    @4 = 28
% 3 -> 5 : 25-9= 16    @2 = 32    @3 = 48    @4 = 64
% 3 -> 6 : 36-9= 27    @2 = 54    @3 = 81    @4 =108

% 3 -> 7 : 49-9= 40    @2 = 80    @3 =120    @4 =160
% 3 -> 8 : 64-9= 55    @2 =110    @3 =165    @4 =220
% 3 -> 9 : 81-9= 72    @2 =144    @3 =216    @4 =288

%     10         91        182
%     11        112        224
%     12        135        270

% 50xp är att höja svärd från 7->10  (100-49 = 51), så redan där är det värt
% att höja till dubbla actions @2 !   lite tidigt kanske?
% Å andra sidan är 50xp en bra kostnad för en major ability som reach etc

\skill{Skill scf 4.0 "quick":} gives one extra action point per level. \\
E.g: base max action points 4 and quick 3 means the Hero can declare 8ap and only take mod-1 action penalty for that round.


\skill{Skill scf 3.0 "mobile":} reduces the movement penalties from declaring faster movement by one for each skill level. \\
E.g: mobile=2 means maneuver mod-0, walk mod-1, run mod-4, dash mod-7.


\skill{Skill scf 1.5 "rapid":} Improves initiative by one for each skill level.


\skill{Skill scf 1.0 "veteran":} reduces fear, morale, and pain penalties by one for each skill level. Also gives a bonus to psy for rolls against fear and morale equal to level.


\skill{Skill scf 1.0 "luck":} The character can re-roll results he is not happy with. The character has a number of re-rolls equal to his luck skill level each dungeon/session/day. However, each re-roll costs XP! The first attempt to re-roll an original roll costs 1xp. Each successive attempt to re-roll the same original roll is at double the xp cost (1, 2, 4, 8, ...).

E.g: three re-rolls on the same original roll costs a total of 7xp. Three re-rolls on different original rolls cost just 3xp. This is kind of expensive stuff. Take care.

Luck is an instantaneous 0ap interrupt effect.


\skill{Skill scf 1.0 "black cat":} The character can force re-rolls on the opponent (usually the GM, unless PvP). The amount and cost for the re-rolls are the same as for "luck". Black cat is an instantaneous 0ap interrupt effect.


\skill{Skill scf 1.0 "rabbit's foot":} The character can give re-rolls to his fellow compatriots. The amount and cost are the same as the "luck".
Rabbit's foot is an instantaneous 0ap interrupt effect.


\skill{Skill scf 4.0 "gut feeling":} Before every new session/day/dungeon the character can roll and store a number of rolls equal to his level of gut feeling. These can then be used instead of making a new roll when one is called for. Each stored roll can of course only be used once. \emph{Storing} a roll costs no xp, but \emph{using} a gut feeling stored roll costs 1xp. Gut feeling is an instantaneous 0ap interrupt effect.

The character can store any kind of die rolls. E.g: d10, d7, d13, but can of course only use them instead of a called for roll of the same kind or "better". E.g: a saved d7 gut feeling can replace a d10 damage roll, but not a d10 skill check. A saved d12 gut feeling cannot replace a d10 damage roll, but can replace a d10 skill check.


\skill{Skill scf 1.0 "prescient packing":} The character can pack his luggage with uncanny knowledge of the future. Whenever he needs a special item he can roll for prescient packing. If he succeeds, that item can now be found in his backpack. If he fails, he cannot try again for that item until he re-packs. If he succeeds he can roll for another unit of the same item later on as well.
The character must already own the items produced, or he can use "prescient shopping" to have already bought them.

The character must also pre-pack his luggage with "useful bundles" that take up space and weight until they are unwrapped and produced as useful items. Prescient packing cannot be used when there are no more useful bundles left in the luggage.

Prescient packing is negatively modified by the weight (encumbrance) of the item, rounded to nearest whole value.

Prescient packing is modded beneficially by the amount of "useful bundles" left in the luggage. Each bundle gives mod+1.

Each "useful bundle" is enc 1 and takes one slot in the pack. When successful, the appropriate amount of bundles are replaced with the item, to cover the item's encumbrance value.

Prescient packing takes a whole rummage action session to attempt. The item is found on a successful roll, otherwise not.


\skill{Skill scf 1.0 "prescient shopping":} The character can spend money to purchase items that turn out to be useful for future enterprises. He can spend money to buy "useful bundles". These bundles can then be unwrapped to produce useful items. This is similar to "prescient packing" and works the same way.
The gold must be spent and the bundles available when rolling. Items cannot be rolled for again, if previously failed, until another shopping spree has refilled the stock of bundles.

Bundles are divided into basic, special, exotic. Basic bundles can produce basic items, special bundles can produce special and basic items, and exotic bundles can produce most kind of stuff. Access to special and exotic bundles is difficult and requires large cities or well stocked speciality stores.

One cannot unwrap items that are worth more than the total value of the bundles of that or a more special type, and each item draws its cost from the value pool of the bundles.

This skill cannot be used to produce custom magical equipment, etc, unless it is clearly available at a previous shopping location, and the massive amounts of gold have been pre-paid.

Prescient shopping goes great with prescient packing.


\skill{Skill scf 1.0 "prescient training":} The character can save XP, and spend them on skills and abilities whenever he chooses, not just in the lulls between dungeons and adventures. The character can "insta-train" at most one level of skills and abilities per level of "prescient training" per session/dungeon/adventure, i.e. between two normal "lull" training times.


\skill{Skill scf 1.0 "forgiving forgetfulness":} The character can forget skill levels in a skill or ability and get back 50\% of the spent XP. Those regained xp goes back to the pool, and the other 50\% are lost.
A character can "forget" one level of a skill for each level of "forgiving forgetfulness", or one level of an ability for each three levels of "forgiving forgetfulness".

Skills that are acquired through "similarity modifiers" cannot be forgotten by themselves. The original skill is forgotten with them, or the other way around.



\closeskillslist















%-------------------------------------------------------------------------------
% boost skills
%-------------


\phantomsection\addcontentsline{toc}{section}{boost}
\section*{Character boost skills}

Not satisfied with that low strength or myopic vision? No problems, just pay some XP and get bulging biceps or crystal 20/20.

The following skills modify the basic character traits, as additive bonuses.
E.g: Simon Strong has str 7, and strong 3, which gives him an effective str 10.
The cost for buying strong 3 is \verb|2*3^2=18| points, regardless of the initial str character trait value. Remember to keep track of the character's original values! A good way is to write str 10(7) on the character sheet.


\openskillslist

\skill{Skill scf 2.0 "strong"} increases str by one for each skill level. \\
\skill{Skill scf 2.0 "agile"} increases dex by one for each skill level. \\
\skill{Skill scf 2.0 "tough"} increases con by one for each skill level. \\
\skill{Skill scf 2.0 "smart"} increases int by one for each skill level. \\
\skill{Skill scf 2.0 "determined"} increases psy by one for each skill level. \\
\skill{Skill scf 2.0 "perceptive"} increases per by one for each skill level. \\
\skill{Skill scf 2.0 "charming"} increases cha by one for each skill level. \\
 \\
\skill{Skill scf 1.0 "hawk eyed"} increases vision by one for each skill level. \\
\skill{Skill scf 1.0 "fish eyed"} increases vision arc by 10 for each skill level. \\
\skill{Skill scf 1.0 "resilient"} increases max hitpoints by one for each skill level. \\
\skill{Skill scf 1.0 "enduring"} increases max stamina by one for each skill level. \\
\skill{Skill scf 1.0 "powerful"} increases max mana by one for each skill level. \\
\skill{Skill scf 2.0 "fast"} increases movement. \\
dash bonus is +fast/1. \\
run bonus is +fast/2 round up. \\
walk bonus is +fast/3 round down. \\
maneuver bonus is +fast/4 round down.


\closeskillslist




















%-------------------------------------------------------------------------------
% unsorted skills
%----------------



%\phantomsection\addcontentsline{toc}{section}{unsorted}
%\section*{Unsorted skills}

%\openskillslist

%\closeskillslist






%-------------------------------------------------------------------------------
%S P E C I A L   M A N E U V E R S
%---------------------------------


\phantomsection\addcontentsline{toc}{section}{maneuvers}
\section*{Maneuvers}


Maneuvers are special actions or optional "modifiers" that can be made as part of, or in conjunction with, an action.

\openskillslist


\skill{Maneuver (cost 5) "opportunity":}
Triggers the right to react when an opponent is trying to move out of melee weapons range after he's already engaged in melee. For weapons with reach it also triggers the right to react if the target is moving \emph{away} into a range with a worse reach mod but still within max reach of the weapon. But it only triggers once per attempted movement.

Note: Opportunity does not cover the same situation as "intercept" does, since intercept only applies to targets not engaged in melee with the Hero, while opportunity only applies to targets already engaged in melee with the Hero.


\skill{Maneuver (cost 15) "intercept":}
Triggers the right to react when someone not engaged in melee is moving into melee range (incl reach if relevant). The Hero can choose where in the movement to intercept if more than one movement squares are within melee range.

Note: Intercept does not cover the same situation as "opportunity" does, since opportunity only applies to targets already engaged in melee with the Hero, while intercept only covers targets passing by without being engaged in melee.

If an attacking opponent without \emph{intercept} moves into melee range to attack, the Hero can strike first. But if both the attacker and the Hero have intercept the initiative decides who strikes first. But the Hero can still force the attack to come from an earlier square in the movement even if the attacking opponent strikes first.


\skill{Maneuver (cost 5) "yield":}
Can be used together with any defensive action like avoid, parry, deflect, where the user moves to end up at least one square further away from the attacker, and (usually) gains a +3 to the avoid or parry action. Some characters have different bonuses to their yield maneuver. The character must end up at least one square distance further away from the attacker than he was before, or he cannot perform the yield maneuver and get the bonus. Just moving to another square at the same distance from the attacker is not a yield.

Upgrade the yield bonus: \\
cost  10xp: gives +1 extra yield bonus. \\
cost  20xp: gives +2 extra yield bonus. \\
cost  30xp: gives +3 extra yield bonus. \\
cost  50xp: gives +4 extra yield bonus. \\
cost  75xp: gives +5 extra yield bonus. \\
cost 100xp: gives +6 extra yield bonus.

E.g: Slow Sebastian has yield+2 bonus to start with. He curses his genes, spends 20xp, and gets a total of yield+4.


\skill{Maneuver (cost 10) "dodge":}
Gives a bonus to to a defend action if the Hero moves one step with the action. In contrast with yield, dodge does not require that the Hero moves away from the target. Any movement enables dodge, even towards the enemy.

Upgrade dodge:\\
base dodge: gives +1 extra dodge bonus.\\
dodge 20xp: gives +2 extra dodge bonus.\\
dodge 30xp: gives +3 extra dodge bonus.\\
dodge 50xp: gives +4 extra dodge bonus.\\
dodge 75xp: gives +5 extra dodge bonus.\\
dodge 100xp: gives +6 extra dodge bonus.


\skill{Maneuver (cost 5) "off balance":}
Is used to take one extra step beyond declared movement. It can only be done once per turn, and only after all movement has been spent. The off balance move puts a mod-3 base modification to all actions for the rest of the turn, excluding the action taken together with the move, if any. The mod-3 base modification stays in effect for the next round as well.

%TODO add extra off balance step upgrades, need to fix the scripts
%     perhaps add a new variable: offbalancemod which is incremented
%     and then reset next round, or use ExtraStep counter...
%upgrades: give extra off balance steps: \\
%cost 10xp max two off balance steps per round, total mod-6 \\
%cost 30xp max three off balance steps per round, total mod-9


\skill{Maneuver (cost 20) "defensive step":} allows for the character to take one extra movement point per round if yielding, after his original movement has already been used up. This does not force the character to go off balance, but it costs one stamina.


\skill{Maneuver (cost 20) "side step":} allows for the character to take one extra movement point per round if dodging, after his original movement has already been used up. This does not force the character to go off balance, but it costs one stamina.


\skill{Maneuver (cost 15xp) "cornering":} allows to make diagonal movement across corners which are otherwise blocked by walls, objects, or an enemy. And together with "diagonal squeeze" the Hero can move through a diagonal between a blocking corner and an enemy.


\skill{Maneuver (cost 15xp) "diagonal squeeze":} allows to move between two enemies across a diagonal which would otherwise be blocked by them. And together with "cornering" the Hero can move through a diagonal between a blocking corner and an enemy.


\skill{Maneuver (cost 5) "going the distance":}
Allows the Hero to march on up to double the con each day when travelling the world map. Max stamina is reduced by one for each extra step taken above con until the Hero has rested a full day.

Pushing successive days will go into negative stamina, provided the Hero passes a con roll modified by current negative stamina. One roll for each extra step attempted. Failed roll means either break for the day, or have the entire party loose one step for a resting pause before trying again.


\skill{Maneuver (cost 10) "switch":} allows for two friendly willing characters in base contact who both have switch to each take a 3ap action action and switch places. They must pay the movement cost for the move as well. The switch happens in the order of initiative of the character with the lowest initiative. However, the faster character can take one extra action to "activate" the slower character and thus initiate the switch.

Small characters like goblins and halflings can use switch to move past friendly willing characters that do not have switch under normal conditions. This costs one action and double movement. They cannot do so in narrow corridors and confined spaces.

It also makes it easier to pass by a diagonal where both side squares are occupied by objects with "round corners", e.g. friendly characters, etc.

upgrades: faster switching: \\
cost 20xp switching is a fast (2ap) action \\
cost 30xp switching is a very fast (1ap) action


\skill{Maneuver (cost 10) "careful":} allows the character to make an action with a mod+1 to succeed, and where the "10" roll is not an automatic failure.
This doubles the ap cost of normal (3ap*2=6ap) or slower actions, and adds 3ap to fast (2ap+3ap=5ap) or very fast (1ap+3ap=4ap) actions.
Full round actions are also doubled, e.g. shooting a bow normally takes 2r, and with careful it costs 4r.


\skill{Maneuver (cost 10) "push":} attack can be used to force opponent back during melee combat. The attack has mod-3 to succeed, and costs 1 stamina extra.

If the attack succeeds, and is not successfully avoided or deflected (not parried), the attacker rolls a str vs str mod with attack diff. If this succeeds the attacker can push the target one square away from him (diagonal ok). Note that this may cause the target to go off balance (or fall down) if he does not have enough movement points left to cover the move. The attack does normal damage if it hits, even if the push does not succeed.

It is also possible to use "push" with a shield parry (parry bonuses applies). Same mod-3 and one stamina cost there as well. The original attacker can choose to take another action to parry or avoid the push parry.
% Apply the shield tackle bonuses to the shield push?
% NOPE, already have the parry skill bonuses for the shield applied.


\skill{Maneuver (cost 10) "quick parry":} is a faster than normal parry/deflect that is more difficult. It is fast+1 mod-3. It costs 1 stamina extra.


\skill{Maneuver (cost 10) "deflect":} is a way of parrying that turns the incoming weapon away from doing a lot of damage instead of taking the damage on the parrying weapon or shield. Deflecting instead of a normal parry incurs a mod-3 difficulty. The benefit is that it is possible to protect against more powerful attacks using deflect than parry. On a successful deflect the weapons used for the deflective parry only has to absorb half (round up) the rolled damage of the incoming strike, instead of all of it. This can mean the difference between breaking the weapon or saving it.

If the deflect succeeds and the weapon can absorb half (round up) the rolled damage, then the attack is successfully deflected. If the deflect succeeds but the weapon cannot absorb half the incoming damage (round up), then the remaining part up to half the rolled damage is passed on to the target character. At least half the damage will be gone from the attack in any case as long as the deflect succeeds. If the deflect does not succeed the target will take full damage as usual of course.


\skill{Maneuver (cost 10) "guard":} gives the character the ability to parry attacks that are targeted at adjacent figures, ignoring initiative, as long as he has access to the incoming attack.

The guard defence action has mod penalty depending on where the attack comes from relative to the target and the guarding Hero.

Access to attacks: The guard must have access, i.e. adjacent square or weapon with enough reach, to the base square side of the target the attack is incoming through, and where no object or characters are in the way. Usually this means sides adjacent to own base square. Further away means more difficult.

The guard parry mods are as follows:
\begin{samepage} \small \begin{verbatim}
       -0    -3    -6
         \    |    /
   ------- -------
  |       |       |
  |  own  |  trg  | -- -9
  |       |       |
   ------- -------
         /    |    \
       -0    -3    -6
\end{verbatim} \normalsize \end{samepage}

Guard stacks with reach and phalanx as follows:\\
With reach, e.g. reach 1 mod-3:
\begin{samepage} \small \begin{verbatim}
               -3    -6    -9
                 \    |    /
   ------- ------- -------
  |       |       |       |
  |  own  | empty |  trg  | -- -12
  |       |       |       |
   ------- ------- -------
                 /    |    \
               -3    -6    -9
\end{verbatim} \normalsize \end{samepage}

With reach and phalanx the "reached" square can be assigned as the square occupied by the guarded target, thus removing most or all guard angle mods. E.g for a weapon with reach 1 mod-2:
\begin{samepage} \small \begin{verbatim}
       -2    -2    -2
         \    |    /
   ------- -------
  |       |       |
  |  own  |  trg  | -- -2
  |       |       |
   ------- -------
         /    |    \
       -2    -2    -2
\end{verbatim} \normalsize \end{samepage}

Keep in mind that facing still matters, and the usual facing mods apply if the guardian is not facing the right way.


\skill{Maneuver (cost 10) "sweep":} means that if the weapon kills one opponent and has damage left, the attack can continue into next opponent in base contact with both first opponent and attacker (or within reach of attacker). Another to hit roll must succeed at mod-3 for the left over damage to continue into next attacker. This does not count as multiple actions and thus does not suffer additional action penalties. The next victim can try parrying with an extra mod-6 from the unexpected incoming danger.

Sweep also works the same way if the target avoids, teleports away, or some similar thing that makes the attack able to continue on to the next unfortunate target


\skill{Maneuver (cost 10) "missile parry":} means it is possible to parry an incoming missile if the attacker is seen when firing. Usually very difficult action at mod-9. \\
Parry large incoming object (swords, sticks, stools, runts): mod-3 \\
Parry thrown projectile (javelin, knife, axe, sword): mod-6 \\
Parry bow arrows: mod-9 \\
Parry crossbow bolts: mod-12

Shields gives a ranged attacker penalty modifiers (hitting the shield) instead of being used as missile parry modifiers. This only applies if the shield carrier is behind the shield, as seen from the direction of the incoming missile. Shields can also be used actively to parry missiles, with the usual parry modifications of the shield.

The skills avoid and dodge can also be used with "missile parry" even if it isn't technically a parry any more.

It is recommended to stack missile parry and deflect for the larger and heavier flying items, like rocks. For very large items, like boulders, GM discretion is advised, and then you really have to be good to succeed.


\skill{Maneuver (cost 10) "phalanx":} can be used as improved defence of several allies, protecting each others' sides. The phalanx must form a straight line (diagonal is ok).
Parries and deflection defence actions get a mod+1 for each of the (two possible) sides that the bearer has covered by friends that also has phalanx. Phalanx does not provide a bonus for "avoid" defence, or if the Cowardly Hero yields in a defensive action.

E.g: a three abreast shield line where all users know and use the phalanx maneuver will give a parry mod+1 bonus to the edge fighters, and a parry mod+2 bonus to the centre fighter. If the edge fighters have walls on their "open sides", then they also get the full mod+2 defence bonus from phalanx.

Phalanx also grants the same bonus if a side is "blocked" by a wall or similar large object. A single phalanx character fighting in a narrow tunnel with walls on both sides get mod+2. With a wall or large statue on one side he gets mod+1 even if his other side is open to attack.

Phalanx does not require shields, but the phalanx bonus is only available to parries and deflect defence actions, not to avoids.
It is not possible to yield while maintaining a phalanx bonus. A phalanx must move together or loose its bonus. Therefore the phalanx usually moves in the lowest initiative of all participants, and cannot yield. A commander or tactician can use the leader or tactics skills to give the phalanx orders to move or act in their initiative.

For weapons with reach 1 or more and poking attacks, such as spears, it allows for the wielders to attack above or through an allied front rank to reach a target. Even if the rank is only one ally wide. The ally in front must also have phalanx for this to work.

For ranged attacks it allows the use of double rank firing lines, where the front line shoots from a kneeling position, and the back line fires above them. It is also possible to shoot over a shield line, but with mod-3.


\skill{Maneuver (cost 10) "combat unit":} allows for all people with the same combat unit maneuver to count as one character for the sake of activation, ordering, strategy bonus (when fighting the same strategy covered target), etc. Bonused as one only applies when they are fighting tightly together, and is at GM discretion.

Each combat unit is a unique and named "combat unit" skill. E.g: combat unit dank daredevils, combat unit evil snails, combat unit red peacock, etc.


\skill{Maneuver (cost 10) "up close":} cancels the contact mod-3 for ranged attacks when in base contact with target. The short mod+3 is not in effect though.


\skill{Maneuver (cost 10) "personal":} the short mod+3 is not cancelled for ranged attacks in base contact.


\skill{Maneuver (cost 10) "fire support":} allows to shoot into melee without the mod-3 when the target is engaged by someone who has the \emph{target pointer} skill.

The level of "fire support" determines how large the "target pointer" partner can be:\\
lvl 1, cost 10xp: tiny target pointer (runt, bird, rat, bug, spider, etc) \\
lvl 2, cost 20xp: small target pointer (halfling, goblin, dog, etc) \\
lvl 3, cost 30xp: normal target pointer (human, orc, dwarf, elf, etc)

A fail-3 or worse still hit the target pointer instead of the target.

It is not possible to shoot through an ally to reach the target, a clear path must exist. Skills like \emph{lean} helps here.

If the target is engaged in melee with anyone who does not have the "target pointer" skill the mod-3 still applies. All allies in melee with the target must have "target pointer" for a mod-0 shot.

Upgrades: \\
"unfriendly fire" cost 10xp: a fail will not hit the "target pointer". \\


\skill{Maneuver (cost special) "target pointer":} allows to get support fire from a ranged ally when fighting in melee. The ranged ally must have the "fire support" skill.

The cost of the "target pointer" skill depends on the size of the character: \\
Size "tiny" cost 5xp: runt, bird, rat, bug, spider, etc. \\
Size "small" cost 10xp: halfling, goblin, dog, etc. \\
Size "normal" cost 20xp: human, orc, dwarf, elf, wolf, etc.


\skill{Maneuver (cost 10) "anticipate":} allows the character to be waiting for a specific event, and when it occurs gain a mod+3 to the first action against it, ignoring initiative. If it does not, but something else happens the character gets a mod-3 to the first action to react to it. Anticipation status persists over rounds until dropped without further action point cost.

A character can "anticipate" an attack from a specific area, etc. Anticipating the correct event then gives a mod+3 to the first action taken against that occurrence, and can ignore initiative. He also has a mod+6 to all perception rolls against events in that area.

On the other hand, if an event occurs that was not covered by the anticipation, then the anticipating character has mod-3 to the first action taken against that event and the opponent can ignore initiative. It also gives a mod-3 to perception rolls for events outside the area.

E.g: Anticipating Arnold is waiting behind a table, aiming his crossbow into the left corridor a couple of squares away, anticipating monsters to storm out of it. Three rounds later a couple sneaky goblins come rushing out of the right corridor. Anticipating Arnold had been expecting attack from the wrong area. He can not interrupt the goblins attack charge with his ranged weapon since he was anticipating attack from another direction, and he has a mod-3 to the first action.

E.g: A character can be anticipating: \\
Incoming enemies from a small area. \\
Movement from an existing enemy. \\
Specific attacks from an existing enemy. \\
In general: GM discretion.

It takes a full 3ap action to start anticipating a future event, and no movement faster than maneuver while anticipating. It costs nothing to stop anticipating something, but does not ignore initiative. And since incoming unanticipated enemies can ignore initiative against the anticipating character, he will not generally be able to drop anticipation fast enough, and will thus incur the mod-3 penalty for actions against most unanticipated events.


\skill{Maneuver (cost 10) "quick looting":} allows the character to loot a corpse or loot spot in 1a instead of 1r. Loot spots that takes longer to loot now takes half the time (round down).

Upgrades:\\
lvl 2 cost +10xp: loot in 1ap instead of 3ap, long loot at 33\% of time\\
lvl 3 cost +10xp: loot in 0ap instead of 1ap, long loot at 20\% of time


\skill{Maneuver (cost 10) "ninja looting":} allows the character to loot a corpse or loot spot without his compatriots noticing. This is not possible if the others have declared that they are watching the corpse, and spend 1a (primary) each round actually watching it. This maneuver requires that the character already has "quick looting".


\skill{Maneuver (cost 10) "corpse kick":} is useful for moving corpses. Spending just 1a (3ap) the Hero can kick one corpse one sq in any direction.


\skill{Maneuver (cost 10) "bashdoor":} allows a character to move through a closed door, bashing it open without spending an extra action. Roll for str with mod+1 for walk, mod+2 for run, mod+3 for dash.
Sturdy doors will have a negative mod!

If the attempt fails the character stops and the door remains closed. The Halted Hero must pass a dex roll to not fall down.


% TODO: fix the aim crap
%       it doesn't work well
%\skill{Maneuver (cost 10) "half aim":} allows the character to aim for a half figure area or large body parts such as torso, legs, arms, etc, but not the head.
%The aimed attack is slow-1. , and it requires a success+3 or better to hit the aimed at area, otherwise it is a regular hit or miss.
%
%
%\skill{Maneuver (cost 10) "small aim":} allows the character to aim for a small area or body parts, such as the head, the stomach, the left upper arm, the right thigh, etc.
%The aimed attack is slow-3, gives mod+1, and it requires a success+6 or better to hit the aimed at area, otherwise it is a regular hit or miss.
%
%The small aim maneuver requires the character to also have the half aim maneuver. Optional: if it is a success [+3,+5] then it can be treated as a successful half aim instead of just a regular hit.
%
%
%\skill{Maneuver (cost 10) "tiny aim":} allows the character to aim for a tiny area, such as the neck, the heart, the knee, the left hand, etc.
%The aimed attack is a long action and takes one full extra round just for the aiming, giving mod+2, and it requires a success+9 or better to hit the aimed at area, otherwise it is a regular hit or miss.
%
%The tiny aim maneuver requires the character to also have the small aim maneuver. Optional: if it is a success [+6,+8] then it can be treated as a successful small aim instead of just a regular hit, and so on to half aim.
%
%
%\skill{Maneuver (cost 10) "micro aim":} allows the character to aim for a microscopic area such as the right eye, the third vertebrae, the left pinky toe, etc.
%The aimed attack is a long action and takes three extra full rounds just for the aiming, giving a mod+3, and it requires a success+12 or better to hit the aimed at area, otherwise it is a regular hit or miss.
%
%The micro aim maneuver requires the character to also have the tiny aim maneuver. Optional: if it is a success [+9,+11] then it can be treated as a successful tiny aim instead of just a regular hit, and so on to small or half aim.


\closeskillslist




%-------------------------------------------------------------------------------
\phantomsection\addcontentsline{toc}{section}{attack}
\subsection*{Attack maneuvers}

Maneuver (cost varies) \emph{special attack}: are different attack maneuvers that some weapons or wielders have. Special attacks are generally difficult to perform and to defend against, or faster or slower with more or less damage. They mostly carry mod-X for the action, and toparry-X or toavoid-X. The "fancy attacks" skill can be considered the general form of many of these special attack maneuvers, but it has a poor 1:1 ratio of mods to the action and mods to the target. These special maneuvers can have better ratios or other useful effects. Most special maneuvers require one stamina extra, above what the attack would otherwise take.

Note: that the mods toparry, toavoid, todefend, etc require a weapon with enough finesse to cover this difficulty modifier. Otherwise it's capped by the weapon finesse.


\openskillslist


\skill{Maneuver (cost 5) "poke":} allows for a poke attack which is available with some weapons. It usually has a mod-1 dam-1 pen+1 effect, but some weapons have different effects.


\skill{Maneuver (cost 5) "swing":} allows for a great slow swing attack which is available with some weapons. It usually has a slow-1 mod-1 dam+2 toavoid+2 effect, but some weapons have different effects.


\skill{Maneuver (cost 10) "fjutt":} is a very weak and flimsy but fast attack. It can be performed with any melee weapon unless specified. It costs one extra stamina. The standard stats are:\\
fast+1 mod-3 with damage 50\% of normal (round down).\\


% TODO: fix feint, make it more interesting
\skill{Maneuver (cost 10) "feint":} is a regular attack but is fast+1, costing one less ap then usual, and does 0 damage. Very good to weaken out the opponent with lots of annoying attacks. The first feint does not cost stamina unless upgraded, and does not count against the number of stamina free attacks for that weapon.

If the target chooses to ignore a feint it can be upgraded to either a weak or full attack:\\
With a \emph{weak attack} upgrade the whole maneuver has max 50\% damage and costs the usual ap for the attack with that weapon, and one extra stamina.\\
With a \emph{full attack} upgrade the whole maneuver has full damage but costs one ap more than a normal attack with that weapon, and one extra stamina.

E.g: Feinting Fabio attacks with his rapier, feinting. Target Tom parries. FF's feint costs only 1ap (rapier is fast+1 to begin with), and TT's parry costs 3ap (normal shield). FF continues to attack feint. This time TT ignores the feint, and FF chooses to upgrade to a regular attack at normal damage. FF pays 2ap extra above the 1ap he has already paid for the feint, along with an extra stamina. If this had been done with a regular sword he would have had to pay two stamina, since he has already done one (feint) attack earlier in the round.

%E.g: Feinting Fabio attacks with his longsword, feinting. Target Tom parries. FF's feint costs 2ap, and TT's parry costs 3ap. FF continues with another attack feint, another 2ap and one stamina. TT parries. FF makes a third attack feint, another 2ap and one stamina. TT ignores it. FF upgrades it to a full attack, pays another 2ap since the "full attack" upgrade is total 4ap for the maneuver, and since it's the first real attack in the round it does not cost any more stamina than FF has already paid for it even if it's the third feint.\\
%FF has thus paid 8ap and 2 stamina for three attacks, but with the certainty that the final attack had already hit since it was an upgraded feint.


\skill{Maneuver (cost 10) "wayward strike":} normal ap, mod-1, todefend-3. Takes one extra stamina. Can be used with all weapons that have enough finesse.


\skill{Maneuver (cost 10) "roundabout smack"} takes double ap for normal 3ap and slower weapons, and +2ap for fast (2ap+2ap=4ap) and vfast (1ap+2ap=3ap) weapons, mod-3, todefend-6, 75\% dam (round down). Takes one extra stamina. Can be used with any weapon that has enough finesse.


\skill{Maneuver (cost 10) "double tap":} slow+1, gives two fast strikes at the target, each with mod-3, damage is 75\% (round down). Takes one extra stamina, plus the normal stamina for the two attacks. Cannot be used with slow weapons.

Two hits can be parried in two actions or one double action. Roll both to hit at the same time, then the target selects his parrying strategy.


\skill{Maneuver (cost 10) "massive strike":} costs double ap for normal or slower weapons, +2ap for fast or vfast weapons, mod-1, toavoid+1, 150\% dam (round down). Takes one extra stamina.


\skill{Maneuver (cost 20) "followup attack":} When a target yields during defence \emph{followup attack} allows the Hero to take an extra "free" movement step and a followup attack, ignoring initiative. The Hero moves one sq towards the target, and makes the followup attack immediately before the target can move further away.

The followup attack costs two extra stamina, one for the extra step and one for the next attack, but costs no movement points, even if it's a second diagonal and would normally cost 2mp.

Weapons with reach has an extra mod+1 when used for a followup attack.


\skill{Maneuver (cost 10) "counter attack":} The user can parry and make a counter attack in the same action using the "counter attack" maneuver. Both the parry and attack rolls are at mod-3. The attack is dam 75\% (round down). Requires two weapons, one in each hand, e.g. sword and shield. Can be used with slow weapons, the entire action cost ap equal to the slowest of the weapons. It costs one extra stamina above the normal cost of the attack.


\skill{Maneuver (cost 5) "strength bonus":} The Strong Hero can use extra strength above the weapon requirement to do extra damage. For each +3 excessive strength he gets damage +1.


\skill{Maneuver (cost 10) "easy grip":} The character can trade a +3 excessive strength damage bonus to get one extra attack per round that does not cost stamina.


\skill{Maneuver (cost 10) "mighty grip":} The character can trade a +6 excessive strength damage bonus (regularly dam+2) to hold a 2h weapon in one hand instead. The weapon gets a slow-1 mod when held in one hand, as well as a mod-1. \\
The character can also trade a +9 excessive strength to hold it in one hand without the slow-1 penalty, but still with the mod-1 penalty.


\skill{Maneuver (cost 10) "fast strength":} A character with enough strength and dexterity can choose to trade excessive strength damage bonus to a fast+1 modifier for the actions with the weapon instead. Also comes with a mod-1 difficulty though. \\
A slow-2 (5ap) weapon requires str+6 and dex$\geq$3 to be slow-1 (4ap).\\
A slow-1 (4ap) weapon requires str+6 and dex$\geq$6 to be normal (3ap).\\
A normal (3ap) weapon requires str+9 and dex$\geq$9 to be fast+1 (2ap).\\
A fast+1 (2ap) weapon requires str+12 and dex$\geq$12 to be fast+2 (1ap).


%TODO fix critical damage once the aim stuff has been fixed
%\skill{Maneuver (cost 10) "critical damage":} The character can aim for and hit vital areas and gain higher damage. A 2x damage requires a successful "small aim" maneuver. A 3x damage requires a successful "tiny aim" maneuver. A 4x damage requires a successful "micro aim" maneuver.


\skill{Maneuver (cost 10) "double arrow":} Like all true action hero archers the character can load two arrows at the same time and shoot at two adjacent targets. The action is slow-2. Both arrows are mod-6 to hit and damage is 75\% (round down), but counts as one attack for the sake of stamina. It is another mod-3 to hit both targets for each empty square between them, starting from the base mod-6 when they are in base contact. The targets cannot be "in line" as seen from the
Oddly enough this maneuver works just as well with crossbows, throwing daggers, rocks, and other projectiles, as it does with regular bows and arrows.


\skill{Maneuver (cost 10) "arrow stab":} Instead of shooting you opponent the Hero can use the projectile to stab the target instead. Damage is 50\% (round down) and penetrating 0. The attack is mod-3 to hit, and a regular 3ap action. It counts as one regular attack for sake of stamina. For the attack skill roll the character can use the highest of either: bow,knife,rapier,sword,spear skills.
This move was made popular by the heroic dancer Leggy-Lars who used it in the movie "Lard of the Sphinx" to stab an orc in the throat before loading the arrow and shooting the next orc. Since the damage is low it is suggested that it is combined with a "critical damage" maneuver. \\
It works just as well with rocks, chairs, bottles, and other projectiles as it does with arrows.


\skill{Maneuver (cost 10) "shoota-inna-back":} Easier to hit and does extra damage when attacking fleeing targets. Even when using melee weapons. The attack is mod+1 dam+1 if the target is running away at run or dash speed.


\skill{Maneuver (scf 0.5) "breaker":} Adds extra "damage" when calculating if a parrying weapon or shield breaks or takes damage when used to parry an attack. It costs 1 stamina extra to make a "breaker" attack. Add "damage" equal to the level of \emph{breaker} when comparing damage vs absorption after a successful parry.

E.g: Breaker Bob attacks Parrying Pete with a long sword, choosing to make a "breaker" attack. BB spends one extra stamina for the attack, he hits and rolls damage 8. PP parries successfully with his staff. BB has breaker 4 so he has an effective "damage" of 12 vs PP's staff with abs 8. The staff now has a 40\% risk of breaking and PP rolls 5. The staff holds but looses 2 abs (excess damage / 3 round up). If PP hadn't parried, or missed his parry, BB's strike would have landed, doing the normal rolled 8 damage, of which PP's leather armour would have eaten 1 damage, and PP would have taken 7 hp damage in the end.


\skill{Maneuver (cost 10) "knock down" / "trip up":} Two separate attack maneuvers. Both means hit someone so they fall down. "Knockdown" is str based while "trip up" is dex based.

It's a regular attack which costs 1 extra stamina and has mod-3. If successful and not avoided the attacker and target will take a it's a str or dex based resistance roll to determine if the target falls down or not. The attacker takes the knockdown success diff as mod to his side of the resistance roll, and the target takes his defence diff (if successful) to his side of the resistance roll.


\closeskillslist













%-------------------------------------------------------------------------------
%S P E C I A L   A B I L I T I E S
%---------------------------------


\phantomsection\addcontentsline{toc}{section}{abilities}
\section*{Abilities}

Special abilities are very rare skills, and a bit expensive, but the mark of Real Heroes. They open interesting possibilities in the fight for gold, glory, and more gold.

Some ability skills have a "combination cost factor" (ccf), which affects the cost of training future ability skills. Once a ccf ability has been trained each subsequent ability skill will have it's cost multiplied by the ccf of all previously trained abilities. The combination cost factor does not affect the upgrade training of its own ability. The ccf does not affect regular skills or maneuvers, just the "ability" skills.

E.g: Dastardly Dennis has trained Incorporeal Double which has ccf=1.5. This means that every ability he trains in the future will cost 1.5 * original cost. If he also trains Dominating Stare, with another ccf=1.5, then Dominating stare will cost him 1.5*50=75xp, and all future abilities will cost 1.5*1.5=2.25 times the original cost.

Upgrading an ability does not affect its ccf. E.g: purchasing regeneration twice still gives regeneration ccf=2.0.

Special "abilities" cannot be purchased and trained by XP without first getting the "special component". Special components can be a blessing from a deity, a piece of mutagenic plutonite, bites from radioactive spiders, strange magic gone wrong, secret lore from a legendary book, tutoring by a far away hermit, training at the dojo of awesomeness, the egg of a phoenix, etc. GM discretion.

Such special component items or events can be bought by lots of gold, as part of adventure loot or rewards, end goal for mini quests and adventures, etc. This makes it more tricky and random to get the special abilities. If you, as a GM, want to make an ability available for gold, a good guideline might be to set the price at the XP cost in gold. Special components are always exotic in rarity, and should not always be available even in the largest and richest of trading centres.

But it really is more fun to build a small adventure around the heroes hunting for that elusive Greenulescent Mushroom, rumoured to give anyone who eats it a "Reactive Skin", only to have it give the Hero who eats it a "Damaging Aura" instead. Muahaha.


\openskillslist

\skill{Ability cost 30xp ccf 1.0 "Combat Advantage":} When the character makes a successful attack which the opponent parries or avoids he will get the "combat advantage". The advantage kicks in after the completion of the actions taken by the character and the target, and only matters between those two combatants.

From then until the advantage is broken the character gets mod+1 to all his attacks against the target, and the opponent gets mod-1 to all his attacks against him.

The combat advantage is broken when: \\
The character misses an attack against the opponent. \\
The character defends against an attack from the opponent. \\
The opponent chooses to not defend against an attack from the character. \\
The character attacks someone else, or takes a non combat action.

The combat advantage does not break when: \\
The character defends from attacks from other enemies. \\
The round is over. Combat Advantage persists between rounds.

Upgrades: \\
cost +10: multiple advantage. Maintain mod+1 against more than one simultaneous opponent, following the rules above. Note that the character will loose advantage over any opponent he defends against, but not if he defends against other enemies against which he does not have advantage. \\
cost +30 ccf is set to 1.3: max +2 advantage. Increases one step at a time, requires two successful attacks in succession to reach mod+2. \\
cost +50 ccf is set to 1.5: max +3 advantage. Increases one step at a time, requires three successful attacks in succession to reach mod+3.


\skill{Ability cost 30xp ccf 1.2 "Captain Cardio":} At the end of each round when the character has not spent any stamina he instead recovers one stamina. When resting he recovers two stamina per round.


\skill{Ability cost 30xp ccf 1.2 "Energizer Bunny":} Always spends one stamina less than required each round. Spending zero does not mean gaining one stamina, unless combined with for example "Captain Cardio".


\skill{Ability cost 30xp ccf 1.2 "Jack Hammer":} For every attack that the character does pay full stamina for, he is allowed to pay one less stamina for the next attack, within the same round.


\skill{Ability cost 10xp ccf 1.0 "Hardened":} Increases pain threshold by one for each level. Each level costs 10xp.


\skill{Ability cost 30xp ccf 1.0 "Painless":} Character is immune to pain modifications.


\skill{Ability cost 20xp ccf 1.0 "Tasty":} The character is the obvious target, among other equally accessible targets, for monsters lacking int. GM discretion.


\skill{Ability cost 30xp ccf 1.0 "Foul":} The character is the last target, among other equally accessible targets, for monsters lacking int. GM discretion.


\skill{Ability cost 20xp ccf 1.0 "Annoying":} The character is the obvious target, among equally accessible targets, for monsters who fail their int roll when seeing the character for the first time. GM discretion.

For each extra 5xp spent on upgrading "annoying" the int roll is reduced by 1. The level of annoying can be increased later on as the character progresses.


\skill{Ability cost 30xp ccf 1.0 "Bland":}  The character is the last target, among equally accessible targets, for monsters who fail their int roll when seeing the character for the first time. GM discretion.

For each extra 5xp spent on upgrading "bland" the int roll is reduced by 1. The level of bland can be increased later on as the character progresses.


\skill{Ability cost 20xp ccf 1.0 "Gapfinder":} The character has an uncanny ability to find gaps and time attacks. He can make reach or ranged attacks through one obstacle in line of sight to hit a target behind it. GM discretion when this applies. A large boulder or wall might not have any gaps, while an inter-spaced opponent probably has. Attacks through gaps are made at mod-3 generally. \\
Note that this can be used to ignore shooting into melee rules.

Upgrade: \\
cost +10xp will remove the mod-3 penalty.


\skill{Ability cost 20xp ccf 1.0 "Angry":} The character gets angry when wounded. Each active pain point mod gives him dam+1 in melee attacks and str+1 to strength tests (not damage bonus). Pain that is countered by pain killers, painless, etc is not active pain. Pain that is countered by veteran is still active pain.


\skill{Ability cost 20xp ccf 1.0 "Divine Retribution":} The character has a vengeful god on his side. When he dies his dying corpus will channel the vengeance of his angry god in form of a divine blast that gives penetrating damage to everyone within range 3. Damage equals his total XP/50 round down. The corpse is mod+3 to all kinds of revive after the divine blast, thanks to the remaining empowerment of the god's energy.

Upgrades: \\
cost +10xp gives damage+1. \\
cost +10xp gives range+1.


\skill{Ability cost 50xp ccf 1.5 "Divine Favour":} A God, or something, is surely watching over this guy. All opponents suffer tohit-1 when attacking him. This can be purchased multiple times with each level step doubling in cost: lvl 2 +100xp, lvl 3 +200xp, lvl4 +400xp, ... I.e: total cost to lvl 3 is 350xp. Extra levels don't change the ccf cost of the divine favour ability.


\skill{Ability cost 50xp ccf 1.5 "Reactive Skin":} The character's skin is strangely alive by itself, and gets pissed when damaged. Each incoming melee attack that fully penetrates the armour, even if it does zero damage to the character, will be answered by a magical dam 1 penetrating chock bolt to the offending opponent.

Upgrades: \\
Aggressive Skin, cost 20 + scf 5.0: more damage\\
Stunning Skin, cost 10 + scf 3.0: stun effect

\skill{Ability cost 20 + scf 5.0 "Aggressive Skin":} The Reactive Skin ability does +1 damage penetrating for every level of Aggressive Skin. E.g: at lvl 4 (cost 100xp) the Reactive Skin + Aggressive Skin 4 does a total of 5 damage penetrating.
% cost levels: 25, 40, 65, 100  ::  100xp for total dam 5

\skill{Ability cost 10 + scf 3.0 "Stunning Skin":} When the Reactive Skin ability does damage it also delivers stun effect equal to the level of Stunning Skin.
% cost levels: 13, 22, 37, 58, 85, 118  ::  37xp for stun 3, 118xp for stun 6


\skill{Ability cost 50xp ccf 2.0 "Regeneration":} The character regenerates 1hp for each 5 rounds regardless of actions. This skill can be purchased several times, with regeneration = 1hp every 5r/lvl (round up). Upgrading does not affect the ccf.


%TODO rewrite Hand of Deity to have nr of rounds praying increasing chance and effect, but minimum one round! Make effect capped at roll success, but also random: mod+(max(success/3,1d3)) dam+(max(success,1d6)) duration rounds (max(success,1d10))... Spending XP also gives bonus to chance and effect, and short prayers (<1r) should require XP spending.
\skill{Ability cost 30xp ccf 1.2 "Hand of \emph{Deity}":} The character can pray to be granted combat bonuses from his god. The chance of the prayers to be answered is equal to his total XP/50 (round down). The chance of having the prayers answered are improved if spending extra XP. For each extra XP spent the chance of response is mod+3.

The bonus effect is mod+1 and dam+1 to all actions, and the duration is 3 rounds for every round of prayer, up to max 9r duration.

The character can also spend 1xp on a quick 3ap prayer and have a chance of getting a mod+3 dam+3 for the rest of the round. Or he can spend 2xp to have a chance to get mod+6 dam+6 for two actions in the current round, or 3xp to have a chance to get mod+9 dam+9 for one action in the current. %However, the chance for quick prayer is total XP/100 (round down) instead of XP/50.


\skill{Ability cost 50xp ccf 1.5 "Damaging aura":} The character will will deal 1 damage penetrating to enemies ending their round in the damaging aura (willingly or not). The aura has radius 1.

Upgrades:\\
cost +20 radius+1 \\
cost +20 dam+1


\skill{Ability cost 20xp ccf 1.0 "Breathless":} The character does not have to breath. He is thus immune to most gas attacks, drowning, etc.


\skill{Ability cost 30xp ccf 1.2 "Engaging Opponent":} An opponent must succeed with a psy roll to be able to disengage from melee combat with an "engaging opponent" character. Disengaging is switching to another target or moving outside of weapon range after having made a previous attack at the engaging opponent. It does not mean that the attacker must continue to make attacks, just that he cannot move away and cannot attack anyone else without first rolling a psy roll.

upgrade: \\
cost +10 gives target mod-3 to the psy roll. \\
This upgrade can be purchased multiple times.


\skill{Ability cost 50xp ccf 1.5 "Grapple":} The character can force all targets in base contact to be unable to move. They can still take actions, but can not attempt any movement until the "grapple" character has moved out of base contact, been knocked unconscious, etc. \\
Costs 1 mana to use per round, and requires a normal 3ap action to maintain each round. Note: active affects all characters in base contact.

Upgrades: \\
cost +50xp gives radius+1.


\skill{Ability cost 30xp ccf 1.2 "Decimating Stare":} The hero with the frosty cold gaze can stare down an opponent to make it stop and reconsider it's current activities. The target will cease all offensive actions, stand and stare, or sometimes just wander off for a while. It will still defend itself, but at mod-3.

The initial stare is a normal 3ap action and costs one mana. The hero must succeed with a psy-3 \vs psy resistance roll to initiate the effect. The target must stand in front of the hero, diagonal is ok, and in base contact.

To maintain the effect the hero must spend a normal 3ap action each round, one mana, and succeed with a psy \vs psy roll. The monster must also stay in the field of vision of the hero and throw him a glance at least once per round.

When a target successfully defends against a decimating stare it has mod+3 against future decimation attempts from the aggressor.


\skill{Ability cost 50xp ccf 1.5 "Dominating Stare":} The hero with the flaming eyes can stare down an opponent and gain temporary control over it.

The initial stare is a one round action and costs two mana. The hero must succeed with a psy-6 \vs psy resistance roll to gain control over the opponent. The dominated target is then under the control of the hero, but dazed until the beginning of next turn. The target must stand in front of the hero (diagonal ok), in base contact, and be facing him for the attack to be possible.

To maintain control each round, the hero must spend a primary action, one mana, and succeed with a psy-3 \vs psy roll. The opponent must also stay in the field of vision of the hero, and throw him a glance at least once per round.

The monster cannot be commanded to perform actions that would be immediately detrimental to its health. E.g: Cannot commit harakiri, but can gladly attack a large monster.

When a target successfully defends against a dominating stare it has mod+3 against future domination attempts from the aggressor.


\skill{Ability cost 50xp ccf 1.5 "Incorporeal Double":} An incorporeal double image, pale and ghostly, can disengage from the hero and move away on it's own. The ghost can pass through other characters and monsters, minor obstacles and doors, but not solid walls. The ghost is immune to normal physical attacks, but can take damage from magical attacks and weapons. If the ghost is damaged it does not loose hp, but half the damage (round down), penetrating, is applied to the hero. The ghost is visible as a misty apparition or ghost, but is not scary.

Forming the ghost takes a full round, with no movement or actions, and it disengages from the hero's body to take up an empty adjacent space.

Dissolving the ghost takes a full round, but requires no actions, and it must be in an empty adjacent space.

The ghost moves at walking speed. Spending one mana+stamina/rnd can move it at running speed. Spending two mana+stamina/rnd can move it at dashing speed. It always flees back to the hero at full dash for free. Spending one mana and one hp and succeeding with a psy roll can dissolve it immediately without having to have it move back.
It has a 360deg vision with range equal to the hero's perception, and of the same vision type as the hero. Everything the ghost sees the hero sees as if he was there as well.

The hero can spend one action and one mana to immediately switch places with the ghost double. There is no acclimatisation period since the hero is aware of the ghost's environment at all times.


%TODO: ? change teleport to base 20 + scf 0.2 regular roll skill ?
%        and if so, then remove the movement limitations and running teleport
%        and perhaps also set 3ap and remove fast teleport, or set to 1r
%TODO: ? cost ?  at 30+0.2 it costs 50xp for teleport+lvl10
\skill{Ability cost 30xp + scf 0.2 ccf 1.5 "Teleport":} For thieves and fighters alike.\\
\emph{"Real Heroes go to the toilet without moving!"}\\
-- Famous quote by Formidable Fabian.\\
Range, target, speed of activation, acclimatisation, and other factors depend on the Hero and can be independently trained. By default it takes one round to teleport to an unblocked square within line of sight. Each attempt costs one mana or stamina, decide when buying the ability and it cannot be changed later. The range is limited to 3x psyche of the Hero. Accuracy is dependent on an int roll. Success means hitting the designated target, and any fail means deviating diff squares from the target in direction 1D8 (1D6 for hex, and 1D360 for gridless). After teleportation the Hero must roll for perception, and any fail diff will be applied as a negative mod for the rest of the round. For multiple teleportations in the same round the acclimatisation mod is cumulative.

It is possible that the Hero materialises on an already occupied square, or within a solid object. In this case the Hero will be shot to the closest empty square suffering (distance)$^2$ penetrating damage as a result. If multiple closest options exist the Hero can choose if he passes a psyche roll, otherwise enumerate and roll for a random final destination. The (distance)$^2$ is also used as a mod for the acclimatisation perception roll, and for an extra dex roll to see if he can remain on his feet.

Teleporting normally takes a full round, but can be done faster. Attempting to teleport in a normal 3ap action is mod-3, and trying for a very fast 1ap action teleport is mod-6. These mods also affect the max range psy calculation and the accuracy int and acclimatisation per rolls. %TODO ? should it really affect the int and per rolls? more fun if not?

Upgrade: Additional Accuracy\\
cost +5xp per level: each level gives mod+1 to the intelligence targeting roll.

Upgrade: Accelerated Acclimation\\
cost +5xp per level: each level gives mod+1 to the acclimatisation perception roll.

Upgrade: Range Extension\\
cost +5xp per level: each level gives range +3.

Upgrade: Long Range\\
cost 10xp: Max range can be extended to 3x normal. Landing deviation is (diff)$^2$ instead of diff for the intelligence accuracy roll. Costs one extra mana/stamina.

Upgrade: Sight Range\\
cost +10xp: Allows for unlimited range when teleporting within line of sight, but costs one extra mana/stamina.

Upgrade: Sight Precision\\
cost +10xp: Doesn't require an accuracy int roll when teleporting within line of sight, but costs one extra mana/stamina.

Upgrade: WannaGoHome\\
cost 10xp: The Hero can memorise teleportation target locations which he can teleport to from anywhere with no range limitations. He can keep int/3 (round up) locations in memory, and it takes 10-int+1D3 days to learn a new location.

Upgrade: Recall\\
cost 10xp: The Hero can memorise an object or location and teleport to it from anywhere within range even if it's out of sight. He can keep int locations or objects in memory, and it takes 10-int+1D3 rounds to memorise a new target.

Upgrade: Backtrack\\
cost 10xp: The Hero can teleport back to an already explored square within range even if it's no longer in line of sight. A scouted square is any point which is exposed on the map, either by previous knowledge, personal or allied scouting. I.e. it's not necessary that the Hero has personally scouted the region. Allied shared vision/knowledge is sufficient.

Upgrade: Reckless Exploration\\
cost 10xp: The Hero can designate destination to any square on the map, within range, even if that square has not been scouted yet. \emph{Here, hold me beer...}

Upgrade: Group Hug\\
cost 20xp per level: The Hero can bring with him any target in touch contact, up to his level of Group Hug. Each target costs an extra mana/stamina. An unwilling target can be brought if the Hero wins a psyche vs psyche resistance roll. Attempting to bring unwilling targets also gives a mod-1 to the destination accuracy int roll per unwilling target.\\
Touch contact means either the hero is touching the target, or the target is touching the hero. It doesn't have to be skin contact, but it can't be just stringing out a rope that both hold on to. Touching standard equipment close to the body such as clothes or armour is ok, as is a well used primary weapon, e.g. an old staff. A new and unfamiliar weapon might not work...


\skill{Ability cost 50xp ccf 1.2 "Reach +1":} The character has an innate reach+1 mod-0 to all attacks regardless of weapon. This is in addition to the reach of the weapon.

E.g: Lars LongArm has Reach+1. He is wielding a spear with reach 1 mod-3. This means that Lars has reach 0 at mod-0, reach 1 at mod-0 and reach 2 at mod-3.

Reach also works fine with shields, and reach + shield + guard is an excellent combination to defend you friends.


\skill{Ability cost 30xp ccf 1.3 "Wall Bug":} The character has found a bug in the mesh of reality. He can insert his arm, including weapon, into a solid, vertical, and reasonably smooth surface, and have it emerge from another similar surface. It costs one mana and one action to push through a wall, and the other surface must be in line of sight, with undisturbed vision path. It takes one action to retract the arm, but costs no mana. Fighting through walls is mod-3.

Level 2: two arms, costs +10xp, but still costs only 1 mana and one action to use. Fighting through walls is mod-3.

Level 3: two arms and a head, costs +20xp, but then the character can peek, throw, and shoot via walls. Very useful. Still costs 1 mana and one action. Fighting through walls is mod-0.

Level 4: walk through walls, costs +40xp, but then the character can walk through the wall. Costs 1 mana and 1 stamina 1 extra movement point and 1 action.

Works well with Gapfinder and Lean to improve line of sight.


\skill{Ability cost 30xp ccf 1.0 "Lean":} Character can trace vision and line of sight from any non occupied surrounding square. Also applies when making magic or ranged attacks, but not regular melee attacks, not even with reach. Facing direction is maintained.

The leaning character can be attacked by response actions as though he is occupying the leaned into square. He can also be attacked by interrupt actions from enemies with higher initiative during the time he is leaning out.

Maptool: use a small "eye" marker that indicates the lean vision origin.


\skill{Ability cost 20xp ccf 1.0 "Tingly magic":} The character will get a tingling feeling when he is subjected to magic or targeted by the buildup of a magic spell. He will not detect clean dormant or inactive magic. He will detect leaking dormant/inactive magic, but only if he is in the spell radius.
Power of the ability is 0, when resistance vs hidden spells.

The add on skill "extra sensitive", scf 2.0, increases power of "tingly magic" to lvl of "extra sensitivity"


\skill{Ability cost 30xp ccf 1.5 "Detonate":} When killing an opponent, the character can cause the target to detonate, doing 1hp damage to all adjacent figures, including friendly fire, but excluding the Hero.

Upgrades:\\
cost +20xp: +1 damage, can be purchased max 3 times. \\
cost +20xp: +1 penetrating, can be purchased max 3 times. \\
cost +20xp: +1 area radius, can be purchased max 2 times. \\
cost +20xp: no friendly fire \\


\skill{Ability cost 30xp ccf 1.0 "Crowdfighter":} When fighting in a crowd, the character gets upset and gets dam+1 per opponent in base contact, beyond the first one.


%TODO: instant blink should perhaps be in order of initiative instead of ignoring initiative? as this make it very easy to always escape when the big bad boss is trying to hack you to pieces. And always escaping by blink and spending time off board is kind of boring game play
\skill{Ability cost 20xp ccf 1.2 "Blink":} The character can blink out of existence for a few rounds, then re-appear in the same place, or closest unoccupied space. It takes one round to activate, costing 3ap, and the character blinks out at the end of the round.

Blinking out costs one mana or one stamina, decide which when buying the skill and it cannot be changed later. The cost is the same regardless of how long the character is away.

When the character blinks back into existence he re-appear at the beginning of the round ignoring initiative. He also has a "stun" mod for one round, during which he re-acclimatises to reality. The mod is equal to -3 per round he was "away", to a max of mod-9, and min of mod-3.

The duration of the blink must be decided when declaring the blink out.
The character must be away for at least one round. The maximum duration of a blink is the character's psyche rounds.

With upgrades the blinking can become faster and more adaptive. If the character can blink in during a round, and starts that round out of existence, he must still declare movement as usual in reverse order of initiative.

Upgrades: \\
cost +10xp: adaptive blink, allows the character to choose when he wants to blink back into existence instead of declaring it when he blinks out. \\
cost +10xp: fast blink, blinks out in one 3ap action, in order of initiative.
and blinks back into existence in order of initiative. The stun mod is reduced by 3, to a min of mod-3. \\
cost +10xp: instant blink, blinks out in an instant 0ap action, ignoring initiative. The stun mod is reduced by 6 to a min of mod-3. Requires the "fast blink" upgrade first. \\
cost +10xp: short blink, can blink in and out of existence in one round. The stun mod is set to mod-3 for less than a round. \\
cost +10xp: warped blink, if the character is out of existence in the beginning of the round he can choose to declare movement when he blinks in, ignoring initiative.


\skill{Ability cost 20xp ccf 1.0 "NullSkull":} When rolling psy vs psy rolls against incoming magic influences the character can add +5 to his psy to resist.

Upgrades:\\
cost +10xp: +5 extra resistance, can be purchased max 3 times.


\skill{Ability cost 100xp ccf 1.0 "Uber":} The character is sooo cool that a roll of 10 is no longer an automatic fail. A natural 10 roll is just like any other roll of the die; if his chance to succeed is 10 or greater, he will succeed.


\skill{Ability cost 50xp ccf 1.3 "Extra Extremity":} The character grows another arm. Buy twice for two extra arms. People will probably look at you really strangely from now on...


\skill{Ability cost 20xp + scf 0.5 ccf 1.0 "Switcheroo":} The character can change target of an attack after the attack and damage rolls have been made, but before the target spends his reaction action. Both the declared target and the target  switched to have the right to react to the attack.

The new target cannot be further away from the declared target than the level of switcheroo. The new target cannot be harder to hit than the original target.

When the character has reached level 5 he can switch to targets that are more difficult to hit than the original. Adjust the success/failure, and damage accordingly. At level 10 he can declare target after rolling the dice.

For high levels of switcheroo the character can also move after rolling to be able to see other targets. Max movement is lvl-10. This of course costs movement points as usual.


\skill{Ability cost 30xp ccf 1.0 "Reactive movement":} The character can always declare movement last, ignoring initiative. When two characters both have reactive movement it is highest total xp that decides who declares last.


\skill{Ability cost 30xp ccf 1.0 "Brütal Slaughter":} The character kills his enemies in such horrible ways that the remaining foes must roll fear modified by the overkill damage diff each time he makes a kill in their ranks. Only targets with int. For each kill the opponents get more used to it and get mod+1 to their fear rolls.


\skill{Ability cost 30xp ccf 1.0 "Munchkin":} The munchkin can choose one extra occupation for every level of munchkin, with all the bonuses and penalties that come with it.


\skill{Ability cost 120xp ccf 2.0 "Ambidextrous":} The character can handle weapons in both arms with equal ease and at the same time. He has an automatic "double" which cannot fail, although it still costs stamina to use.

Upgrades:\\
cost 50xp: the automatic double does not cost stamina. \\
cost 50xp: allows triple with one extra arm for people with more extremities.


\skill{Ability cost 50xp ccf 1.0 "Cockroach":} The character has a knack of never quite dying. If he gets totally annihilated, dead, and destroyed he still has a 3+((1000 - total XP)/100) (round up) roll of wandering into town in time for the next adventure (not session). He has probably lost most or all of his stuff (GM discretion), but he is still alive.

E.g: After the successful rescue of princess CreamCup, the two remaining Heroes rest at the GoldenFlagon inn, wallowing in their riches. They lost two of their friends in the Dragon's cave; Wandering Wally was spiked to death by a javelin trap and Fat Fabian was eaten by the Dragon. They glance up from their drunken stupor just in time to see Fat Fabian limping up to the bar. He is wearing nothing but rags and his stupid feathered cap, and decorated with some very distinctive bite marks the size of a dragon's maw. How the Hell could he have survived being eaten by the Dragon?


%\skill{Ability cost 30xp ccf 1.2 "'tis but a scratch"}
%\skill{Ability cost 30xp ccf 1.2 "just a flesh wound"}
\skill{Ability cost 30xp ccf 1.2 "Black Knight"} The Hero doesn't take mod-1 when wounded down to half hp. The \emph{Black Knight} ability can be bought three times, each costing 30xp (but does not increase ccf), to also eliminate the mod at quarter hp and zero hp. \emph{"'tis but a scratch."}


\skill{Ability cost 20 + scf 1.0 ccf 1.2 "Sixth Sense":} The Hero has the ability to sense things that cannot otherwise be detected, and intuit information that cannot otherwise be known. GM discretion. Some examples:

Spotting sneaky villains: The Hero has failed the per check when Sneaky Smith is getting ready to gut him from behind. But passing a Sixth Sense roll will let him know something's up in time to act.

Knowing where to look: The key is hidden somewhere in the room. Sixth Sense has a radius equal to the level. Standing close enough and passing a roll will indicate which squares to concentrate on for the search. GM should probably only allow one roll per "item" or situation. Alternatively, if not in range but still passed the roll, then indicate a larger region as possible location.

Which way: Standing in front of a tunnel crossing, which is the right way to go? Note that "The Right Way" may mean different things to different characters.


\skill{Ability cost 20xp ccf 1.2 "Stacking":} Heroes can stack together and occupy the same space. This means they can both stand in, and attack from, the same square. All stacking heroes must have the "stacking" ability to be able to co-exist in the same space.

Attacking a stacked target is mod-3, and fail-1 to fail-3 means hitting one of the other stacked targets.

%Stacking heroes gives all a mod-1 per extra hero occupying the same space.
% NOPE: adding another 20xp upgrade to remove the mod-1 means it's too expensive for what it's worth, and reducing the initial cost to 20 means the base line is too cheap.

% Stacking works perfectly fine with "guard", giving mod=0 to all angles.


\skill{Ability cost 30xp ccf 1.2 "TronVol":} Heroes can meld together to form one large super critter. All Heroes in the TronVol merge must have the TronVol ability, and each TronVol merge target is a separate TronVol ability (does not incur extra ccf beyond the first.
E.g: TronVol:\emph{PurplePanickler} is formed by the Heroes Purple Pete and Pinky Panickler, and is a malformed large hodgepodge monster with three legs, two arms, two heads, a tail, and some tentacles.

Forming and splitting a TronVol monster takes 3r. Forming costs one mana from each Hero, but splitting is free. The Heroes can upgrade the TronVol monster to form faster by collectively spend xp: +10xp form/split in 2r, +20xp form/split in 1r.

When forming the TronVol monster, calculate current hp, stamina, and mana as the average percentage of the Heroes' current values. The monster forms with no pain mods. When splitting again, push half of the monster's taken damage onto the heroes, collectively, and let them argue on how to split it. None of the monster's pain mods are transferred to the Heroes. The Heroes of course keep all the damage and pain they had before forming the TronVol monster. If the monster's stamina and mana is lower when splitting than when it was formed, then push half the diff collectively onto the Heroes.

The TronVol monster has stats something like the following, and roll to add/deduct points here and there to give it some variation.
\small\begin{verbatim}
str:      sum(heroes)
dex:      min(heroes) -1
con:      max(heroes) +3
int:      min(heroes) +1
psy:      max(heroes) +1
per:      max(heroes) -1
cha:      min(heroes) -3

hp:       sum(heroes) *1.5, round down
abs:      sqrt(number of heroes) +1, round down
m:        min(heroes) +1
w:        min(heroes) +2
r:        min(heroes) +3
d:        min(heroes) +4
sta:      max(heroes) * sqrt(number of heroes), round down
vis,arc:  avg(heroes)
mana:     sum(heroes)
ap:       max(heroes) -1

pain threshold: 3 + (number of heroes)
avoid,dodge:  min(heroes) -3, and yield/dodge bonus 0
double bonus: sqrt(number of heroes) *3, round down
other skills: max(heroes)
\end{verbatim}\normalsize

TronVol monsters can learn abilities and skills that does not come from the constituent Heroes. The Heroes pay collectively. When splitting xp, the TronVol constituent heroes still count separately.

For the general shape, either find something fun and suitable, or roll up some random body: humanoid, lizard, bear, demon, cryptomythologial mixture, etc. Roll for extra/fewer arms, tails, horns, claws, tentacles, fur, scales, etc.

Adding heroes to an existing TronVol monster constellation is possible, and modifies stats, skills, and body configuration as suitable.

Use larger tokens for merged Heroes. E.g: Maptool: with 2-5 merged heroes set large size: 2x2, with 6-11 heroes set 3x3, and for 12+ set 4x4.


\skill{Ability cost 30xp ccf 1.5 "Ghoul":} Mmm, Braiiiiin... Sooo Tasty!
The Hero(?) can regenerate hp by eating the corpses of his enemies. For each 3r spent eating the Ghoul will heal 1hp every 10r after completing the meal. A meal cannot be larger (in hp healing points) than the minimum of the Ghoul's con and one quarter the original hp of the victim (round down). A new meal cannot be started before the previous has been depleted by healing the prepared hp.

The corpses must be fresh. The meal value of the corpse is decreased by one for every round it's been dead before the Ghoul starts eating.

And, as they say, \emph{You Are What You Eat}, so the stats of the Ghoul will change based on the source of food. For every meal each stat of the Ghoul is reduced or increased by one in the direction towards the original stats of the victim. Positive effects last (meal size)/2 days (round down), while negative effects last (meal size) + 1d3 days. For multiple meals the effects stack to a maximum of +/-3.

Upgrades:\\
Fast Food (cost 10xp): eating each "hp prep" in 2r instead of 3r.\\
Fast Metabolism (scf 5): each level reduces the healing interval by one, to a minimum of 3r/hp.\\
Live Food (cost 20xp): eating live food also restores one stamina and one mana in addition to the hp regeneration, at the same speed.\\
Old Food (cost 10xp): can get full meal value from corpses which are up to a day old.

\closeskillslist












%-------------------------------------------------------------------------------
%Possible disabilities and curses
%--------------------------------


\phantomsection\addcontentsline{toc}{section}{curses}
\section*{Disabilities and Curses}


Basic stuff:
Reduced stats, skills or abilities.
Equipment failures, or losses.


Fun stuff:
Strange equipment failures.
Causing problems for companions rolls, skills, etc.


Curse "Heavy Mana": The character also loses one stamina for each mana he spends or loses.


Curse "Slow": The character is slow-1, and all actions that have $\leq$0ap cost now always cost +1ap extra.


Curse "unlikely": The character must always roll twice for a required roll, and choose the worst result.


Curse "forgetful packer": The character has a certain risk of not having remembered to pack an item, or lost it along the way, when he tries to retrieve it from the back pack.


Curse "squeamish": The character has reduced pain threshold and even small cuts give pain.


Curse "undiagonal": The character can never move diagonally


Curse "prediagonal": The character always starts each round with the FirstDiagonalTaken flagged as 1, giving the movement cost as: 2-1-2-1...







%-------------------------------------------------------------------------------
%FUTURE SPECIAL ABILITIES, once they can be supported by maptools
%----------------------------------------------------------------

\phantomsection\addcontentsline{toc}{section}{future}
\section*{Future stuff, requires MapTool support}



%--------|---------|---------|---------|---------|---------|---------|---------|
%       10        20        30        40        50        60        70        80




%-------------------------------------------------------------------------------
%M A G I C   S K I L L S
%-----------------------

\cleardoublepage

\phantomsection\addcontentsline{toc}{chapter}{Magic}
\chapter*{Magic}
\chaptermark{magic}
\label{cpt:magic}


\phantomsection\addcontentsline{toc}{section}{skills}
\section*{Magic skills}

All magic skills and spells must be learned from a teacher, book, or similar. They cannot be trained from nothing without guidance like normal skills. This includes the base skills \emph{magic} and \emph{spell caster}, as well as the support skills like \emph{power casting}, and all spells.

Teachers can be the village witch, a professor at the City Magic College, an ally in the party who happen to know that spell, or the enemy wizard you captured a few weeks ago and been torturing ever since (\emph{but would you trust him?}). When creating a new character with magic skills and spells it's assumed that he had access to some magic training when growing up, perhaps 1d3 spells.

Some magic items can be used without any magic skills, others require that the user has some knowledge of the \emph{magic} skill. To learn and cast spells it is necessary to have trained \emph{spell caster}, and that is least expensive directly when the character is created.


\begin{description}


\item[Skill scf 2.0: "magic":] The caster can not spend more mana on each spell casting than his level in magic. This also limits the amount of mana he can pump into an already active spell each round, or how much mana he can use to activate or power a magical item.

The magic skill is also used when trying to understand magical texts and foreign spells and devices, etc. Or when activating certain magical items. It also provides some protection against certain types of spells.

Note: Not all magical devices, gadgets, trinkets, and weapons require the magic skill to use. Some just require being drawn, calling an activation word, or other circumstance. Some gadgets will also draw mana from the user by themselves without the user needing the magic skill to infuse them with power, and some items are powered by themselves or other sources.


\item[Skill cost special "spell caster":] is requried to be able to learn and cast spells. It is not required for the use of magical items nor to train the magic skill.

The xp cost to train spell caster varies depending on the total xp of the character, increasing as the character advances:\\
If trained at character creation it costs 20xp.\\
After that it costs 20 + (total xp / 10).\\
E.g: For a 123xp Hero it costs 33xp, and for a 358xp Hero it costs 56xp to train spell caster.


\item[Skill scf 1.0: "power casting":] is when the spellcaster pays extra mana power to increase his chances for the casting roll. The caster gains mod+1 for each mana spent. The caster can not spend more mana than his skill in power casting. This cannot be used to raise the end total chance of success above 9.

Mana spent for power casting is not counted to the maximum total limit set by the caster's magic skill level.


\item[Maneuver cost 10: "held spell":] delays the final touch of the spell so that the effect can appear on very short notice. Can with good effect be used together with anticipate to inject the effects of a prepared multi-round-casting spell at a precise point in the action sequence of a battle round. It costs 3ap to cast a held spell. Holding a spell is free in the first few rounds but costs 1 mana per every exceeded period of 1 + int/3 rounds.

\

\todo fix "pinpoint on map" issues, miss allowance distance = lvl squares.

\item[Maneuver cost 5 + scf 0.5: "indirected strike":] allows the casting of \emph{strike, chock, flash} spells without direct line of sight to the target. By taking a mod-3 or 1r extra casting time the wizard can set a target for the spell by an approximate location, such as a description from a spotter communicated in 1 action (3ap), guesstimated location from recently seeing the target move out of sight, and similar, GM discretion.


\end{description}






%-------------------------------------------------------------------------------
%S P E L L S
%-----------

\phantomsection\addcontentsline{toc}{section}{spells}
\section*{Magic spells}

Magic spells cannot be trained from nothing without the user learning the spell from a master, scroll, grimoire, etc. Purchasing a spell text costs 1-10 gold depending on the rarity, strangeness, power of the spell, etc. GM discretion.
Spells can also be deduced from magical objects and other sources, with a lot of study and some modified magic rolls.

Consider the spells listed below to be suggestions only. Build individual spells to make the use of magic more varied and unreliable. Perhaps the Hero wants to purchase a Black Bolt spell, but the grey old master only teaches him the Thread of Dread Darkness which is very similar to the Black Bolt but with some minor changes in range, effect, etc.

Magic spells are generally slow to cast. A normal time spell takes 1-2r to cast, a slow spell can take 3r or longer. A fast spell takes a couple of actions or one round.

When casting a spell, each mana pumped into the spell, including casting cost, reduces the chance of success by one. I.e. a black bolt with two extra mana for damage and one for range will cost 4 mana to cast and carry a mod-4 to cast.

Spells that have range don't allow for short,long,vlong,extreme range modifications of any sort unless explicitly stated. E.g: a black bolt r10 will not reach a target 11sq away from caster, but a shock bolt r20 can reach 30sq at dam-3.


\begin{description}

\item[Skill scf x.x: \emph{spell}] Most spells have a default skill cost factor of 0.33. Some spells are cheaper, some more expensive. Some spells have minimum requirements on psy or int. 

If the character does not meet the int requirements, then the cost of the spell is doubled for each missing int point: e.g: Wizard Woolhead has int 4 and wants to learn an int 7 spell. He is missing 3 int points, thus the spell costs $2^3=8$ times as much to learn. He then spends his xp on a simpler spell.

If the character does not meet the psy requirements he will have mod-1 per missing psy when casting the spell.

\end{description}

These spells should be seen as examples, different masters teach their students different variations, some better, some worse, some just different. It is likely that most wizards' spells will be different from each other.

Spells with more complexity or targets have higher int requirements, and spells with higher power or control influence has higher psy requirements.

\subsection*{Custom spells, player created spells}
%-------------------------------------------------
More fun is to let the player write new spells for his character, and you as GM put requirements on them. That way the spell casters become more unique, and it's more personal for the player. This could also be a way around the limitation that the spells have to be tought by someone. Perhaps the Hero can just figure it out by himself?

When the spells are new they are in flux. Prototypes. And they have to be tested under a wide range of conditions. The first few adventures and tens of usages of the spells in varied situations should give you as GM, the player, and the rest of the group a decent insight into the functions and aspects of the spell. During this period the details, stats, and requirements of the spell should be flud and change over time, until you've found the sweet spot. Remember to consider how the spell might work if it's used by a different caster, or in combination with a different group, tactics, opposition, equipment, combat styles, etc.






% typical: relative base, 1h/2h  dam , extras
% sword    +2   +2/+4                             finesse 6
% axe      +3   +3/+6                             parry-3 toparry-1 toavoid+1
% spear    +1   +1/+2 pen 0/1 parry -1/+1         reach
% staff    -1   -1/+1 parry +1/+2                 fast, reach
% braw     -3                                     (at str 6)
%   fist   -4                                     fast, deflect
%   kick   -2                                     deflect
%
% bow      =0   --/=0                             20sq, pen, 1/r
% crossbow +3   --/++ pen 1-3                     15sq, pen, 1/2r
%
%
% typical psy int =6, mana 15
%             assume armour2 for ppp
% blkblt1  =0 +2p                                 10sq, ppp, 1/r
% blkblt2  +3 +5p                                 10sq, ppp, 1/2r
% shkblt1  =0 +2p                                 15sq, ppp, 1/r
% shkblt2  -1 +1p                                 20sq, ppp, 1/2r
% frcblt3  -1 =0p                                 10sq, p1,  3ap
% frcblt3  -3 -1p                                 10sq, p2,  3ap


% guide lines ?
% psy - power, damage, effect
% int - speed, complexity, range, multi-target, control
%
% assume baseline psy+3 dam+pen, count full penetration as pen2 on average
% ok to have psy+3 dam+pen + scale (psy+3)/4 per mana for 1-2r casting
% range 10 as baseline, trade 1dam/pen for 5 range
% usually no extended ranges, all based on one max range, some with long etc
% 








% baseline for spells:   int 3, psy 3
%                        cast 1r 1m          spear/strike, +1r casting for bolts
%                        dam 6 (5-7) pen 0
%                        range 10
%
%  int +1 for range extension flexibility
%  int +1 for damage extenstion flexibility
%  int +3 for range+5          or trade dam + cast time, see below
%  int +3 for quicker casting  3r > 2r,   2r > 1r,   1r > 3ap
%
%  psy +1 for dam+1 / pen+1 / more power / effect
%  trade 3 dam+pen against 5 range






% damage psy based, parity with sword, alpha boost (dam/3) / mana
% medium range as base range for bolts
% no parry or avoid defense against bolts
% no shield defense or cover for bolts
% no shooting into melee mods
% no movement speed mods
% 
% base dam  psy+2
% base range 10
% base cast 2r for spear and strike, 3r for bolt
% 
% dam class time    base int   dam/psy
% dam 1-3    1a/3ap  int6       dam = psy    4ap : dam=psy+1 or extra effect ?
% dam 4-6    1r      int5       dam = psy+1  or psy=dam +pen1
% dam 7-9    2r      int4       dam = psy+2  or psy=dam +pen2
% dam 10-12  3r      int3       dam = psy+3  or psy=dam +pen99
%
% ? but can't generally scale base damage higher than (psy dam) / mana
% ? even at long times
% 
% treat penetrating as dam+3 on average,   trade dam <> pen equally
%
%
%
%
% scale extra mana for dam / range / stun / radius / pen / etc
%
% damage:                    int base    dam base
% 3ap  dam 1-3    +1/m         int6      dam psy
% 1r   dam 4-6    +2/m         int5      dam psy+1
% 2r   dam 7-9    +3/m         int4      dam psy+2
% 3r   dam 10-12  +4/m         int3      dam psy+3
%
% range:                     int mod AND dam mod
%      touch                   int-3     dam+3
%      r4sq       +2/m         int-2     dam+2
%      r7sq       +3/m         int-1     dam+1
%      r10sq      +5/m         int       dam
%      r15sq      +7/m         int+1     dam-2
%      r20sq      +10/m        int+2     dam-3
%      r25sq      +12/m        int+3     dam-4

%
% treat bolt as one class higher than spear or strike :  more useful, no target mods
% treat ball as two classes higher than spear and strike  :  are effect, no mods
% 
%
% Trade cast time and range against damage and psy int requirements
% DONT trade more than one time class !!!  NOT 3r > 2r > 1r for int+6 psy+6
%
% cast 3r > 2r        dam-3 /  psy+3 & int+3
% cast 2r > 1r        dam-3 /  psy+3 & int+3
% cast 1r > 3ap       dam-3 /  psy+3 & int+3
% 
% range 10 > 15       dam-2 / cast+1r
% range 10 > 20       dam-3 / cast+1r
% allow long @dam-2   dam-1
% area effect         dam-2 per radius+1  /alt  cast+1r per radius
% 
% 
% int 3 as base for normal complexity
% int+2 for multi target
% int+2 for area effect, int+1 per extra radius+1
% int+1 for extra range / mana
% int+1 for extra damage / mana
% int+1 for pen1-2, int+2 for pen3-ppp
% int+1 for buffer / hold effect
% trade int-1 for psy+1 or mana+1 cost
% trade int-1 for +1r cast time
% trade range-5 for int-1
% trade range touch for int-3 ?
%
% 
% Examples below are BOLTS and thus +1r casting
% comparable spear/strike should be   3r >> 2r,   2r >> 1r,   1r >> 3ap
%
% black bolts  cast 3r  !! dam 9            >> int=3, psy=7
%              dam psy+0 dam+/m ppp         base psy+2, dam-2 ppp, int+1
%              range 10 (med)               int,psy +0
%              range +5/m                   int+1
%              >>                           int 6, psy 9
%
% shock bolt   cast 3r  !! dam 9 fundament  range 20  dam-3 
%              dam psy+0 dam+/m ppp         base psy+2, dam-2 ppp, int+1
%              range 20                     range 20  dam-3
%              range long @dam-3            dam-1, but take int+2 instead ?
%              range+/m                     int+1
%              >>                           int 8, psy 9
% 
% force bolt   cast 1r      (2r>>3ap : dam-3, psy+3 & int+3)  psy+2,-3=psy-1
%              range 10                     --
%              
%
%
%
%
% stun effects ?   
%
% add stun:   int+1,  stun3 = dam-1





%-------------------------------------------------------------------------------
% test spellslist formatting

%\
%
%\
%
%\todo testing spellslist formatting
%%-------------------------------------------------------------------------------
%%\TODO remove or switch to this:   check how it looks with spells as list ------
%\openspellslist
%
%\spell{Ur Bolt} int 3, psy 3, scf 0.33, \\
%cast 2r 1m, 
%dam 6, 
%range 10 \\
%\textit{Of Ebedan Ur, oldest of bolts, archetypical of the ancient spellcasting arts of killing. A quick flash of clear blue light, sharp crack, smell of dry parchment. But when cast less eloquently it's well known to produce the soft quack of a drunk duck and a faint smell of old wine.}
%
%\spell{Onsager Spear} int 6, psy 7, scf 0.33, \\
%cast 3r 1m, 
%dam 10 +3/m, 
%range 10 +5/m \\
%\textit{The Onsager battle monks developed a simple martial spellcasting creed, and the Spear is their most classic, simple, and well known infantry battle spell. It's unknown how they managed to bring their massed infantry to the required mental capacity.}
%
%\
%
%\spellslistnote{Jeremiah Black, with no imagination and well known to never drink anything stronger than the milk of his two headed goats. Survived only by his most traditional, efficient, and horribly boring Black Bolts. They tend to have high requirements but have been shown to be highly effective in a wide range of situations.}
%
%\spell{Black Bolt} cast 2r 1m, 
%dam 9 +3/m penetrating,
%range 10 +5/m,
%int 6 psy 7 \\
%\textit{A faster cast, medium power version, of Black's bolt library.}
%
%\spell{Black Bolt} int 6, psy 9, scf 0.33, \\
%    cast 1r 1m, 
%    dam 6 +2/m penetrating,
%    range 10 +5/m,
%    int 6 psy 9\\
%    \textit{A weak but very fast cast version of the archetypical Black Bolt.}
%
%
%\closespellslist
%%-------------------------------------------------------------------------------
%
%\
%
%\






\subsection*{Bolts, lances, missiles, strikes, sparks, etc}
%----------------------------------------------------------
Many offensive spells have similar general mechanics. 

\begin{description}

\item[bolt, zap, spark, lightning:]
Bolts don't take target \emph{to hit} modifications on the spell casting roll. They ignore things like target partial cover, shields partially in the way, target movement speed, partially obstructed view by fire, etc.
Generally these magic energies don't travel in a straight line from caster to target but might curve, form a branched lightning tree, split and reform, or snake forward erratically.

These spells are more powerful than the spear or strike categories and often require more time or mana to cast for similar effect.


\item[spear, lance, dart, arrow, quarrel, missile:]
These work similar to projectiles. The spells take target \emph{to hit} mods directly as casting mods. Target size, target movement speed, partial cover, shields in the way, shooting into melee, fire, etc all mod the spell casting roll directly. These are the same to-hit target mods as a firing arrows from a bow would take. 

%TODO: think there are edge cases where this will not work right

Often the spell formation shoots in a rapid line to hit the target, or whatever gets in the way. A failed casting, which would have succeeded if not for the \emph{to hit} mods from cover, movement, target size, etc can be interpreted as a successfully cast spell which strikes cover or near the target instead.

%TODO: should this then still draw full mana, or reduced mana for minor fail? Can draw reduced mana and still make a nice flashy effect.

Since the spell behaves like a projectile, taking mods like a bow or crossbow, skills like sniper, lead target, etc work to alleviate those mods, just like for bows.


\item[strike, chock, flash:] 
These are mental target lock spells that take casting mods based on target psy, actions, and magic training. Thus they tend to be less useful against other spellcasters. The casting roll takes the following mods, stacking, based on target:\\
mod -(target psy / 3)\\
mod -3 if target takes a 3ap action to resist the spell\\
mod -(target magic lvl / 3) (rd) E.g: magic 4 : mod-2
% mental lock spells ignore cover etc, but mods by psy etc
%              ignores target move mods, shooting into melee mods
%              BUT: they take cast mods for target psy, if target resists,
%                   and if the target knows magic skill.
%                   mod -(target psy / 3), 
%                   mod -3 if target takes a 3ap action to resist 
%                   mod -(target magic lvl) 
%              Hence, almost useless against wizards


\item[blast, ball, cloud, storm:]
These are area effect spells. They can be targeted on characters, objects, or locations, and take no target mods.


\end{description}

\

\small \begin{verbatim}
Ur Bolt        cast 3r, 1m, dam 5,
               range 10
               int 3, psy 3

Onsager Spear  cast 2r, 1m, dam 10 +3/m,
               range 10 +5/m
               int 5, psy 8

Keregt Strike  cast 2r, 1m, dam 8,
               range 10 +5/m
               int 4, psy 6

---------------

black bolt     cast 3r 1m, dam 9 +3/m, penetrating,
               range 10, 
               int 5, psy 9

black bolt     cast 2r 1m, dam 6 +2/m, penetrating,
               range 10,
               int 5, psy 9

shock bolt     cast 3r 1m, dam 5 +2/m, penetrating,
               range 20 +10/m, long dam-3,
               int 8, psy 8

shock bolt     cast 2r 1m, dam 4, penetrating,
               range 15 +7/m, long dam-2,
               int 6, psy 6

force bolt     cast 1r 1m, dam 3
               range 10
               int 7, psy 4

force bolt     cast 1r 1m, dam 5 +1/m
               range 10 +5/m, long -2dam,
               int 10, psy 6

---------------

bright arrow   cast 2r, 1m, dam 6
               range 10 +5/m
               int 4, psy 4

bright arrow   cast 1r, 1m, dam 5
               range 10 +5/m
               int 7, psy 7

snuppel dart   cast 2r, 1m, dam 4, stun 5 +3/m
               range 15 +7/m
               int 6, psy 5


---===### current above ###===---
\end{verbatim} \normalsize

\

\noindent 
The spells listed below are not re-balanced yet.\\
See balance guidelines in comment segment above.

\

\small \begin{verbatim}
---===### legacy  below ###===---


fire ball      cast 3r 2m, dam 5 +1/m, range 10 +5/m, radius 2 +1/m,
               duration 3r +2/m, damage each round
               int 8, psy 8

fire ball      cast 2r 2m, dam 4 +1/m, range 10 +5/m, radius 1 +1/m,
               duration 1r +1/m, damage each round
               int 7, psy 6

fire blast     cast 1r 2m, dam 3 +1/m, range 5 +5/m, radius 1 +1/m,
               int 6, psy 6

fire storm     cast 1r 1m, dam 5 +2/m, range self, radius 1 +1/m,
               duration 3r +3/m, damage each round,
               caster is immune to fire for the duration
               The storm follows the caster at most up to walking speed,
               if the caster moves faster the storm will disappear.
               The max damage of the storm each round is reduced by one for
               each movement speed declared for the round.
               int 6, psy 8

minion missile cast 1r 1m, dam 4 +2/m, speed 20, seeking 10 -2/r,
               penetrating 1, lifetime 20r +10/m,
               Hover by caster until target assigned,
               then seeks target until missed seeking or max lifetime.
               Seeking does not fail on roll of 10 as normal.
               They are smart enough to curve around objects and avoid obstacles
               unless they miss their seek roll (one per rnd). The target does
               not have to be in line of sight from the caster once assigned.
               Seeking penalty only after target assignment, not total time
               since summon.
               Assigning one or more bolts to one target is a
               very fast action (1ap).
               int 6, psy 7

ward shield    cast 2r 2m, range touch, duration 5r +3r/m
               charge 5 +3/m (pay on cast)
               Absorbs incoming damage before it strikes the character
               or his equipment. Each absorbed damage point costs
               one charge from the spell.
               Ignores penetrating, i.e. penX just passes through the spell.

ward shield    cast 1r 2m, range touch, duration 5r +3r/m
               charge 4 +2/m (pay on cast)
               Absorbs incoming damage before it strikes the character
               or his equipment. Each absorbed damage point costs
               one charge from the spell.
               Ignores penetrating, i.e. penX just passes through the spell.

ward flash     cast 3ap 1m, range personal, duration 1r first attack only,
               absorbs 5dam +3/m on the first attack that hits the caster.
               Absorbs incoming damage before it strikes the character
               or his equipment.
               Ignores penetrating, i.e. penX just passes through the spell.
               int 5, psy 5

ward skin      cast 2r 2m, range touch, duration 5r +3r/m
               charge 4 +2/m (pay on cast)
               Absorbs incoming damage before it does hp damage, but does not
               spend charges on damage that is absorbed by equipment or armour.
               Each absorbed damage point costs one charge from the spell.
               Ignores penetrating, i.e. penX just passes through the spell.
               int 7, psy 5

force shield   cast 3ap 1m, range personal, duration 1r,
               absorbs 3dam +1dam/m on every attack during the round.
               Absorbs incoming damage before it strikes the character
               or his equipment.
               Ignores penetrating, i.e. penX just passes through the spell.
               This ward absorbs damage for the active round. It is not limited
               in how much damage it can absorb.
               int 5, psy 7

force sphere   cast 2r 1m, range personal, duration 5r +3/m,
               radius self +1/2m.
               Creates a force wall sphere around the caster and anyone inside
               the encompassing radius.The sphere will absorb damage from all
               incoming physical attacks up to a maximum of caster psy per
               attack. The damage absorption costs mana, which is paid per
               attack, and not on casting. Each int/3 damage costs 1 mana.
               Attacks doing more damage than the caster's psy will penetrate
               the shield and strike the target with reduced damage.
               Ignores penetrating, i.e. penX just passes through the spell.
               The caster can cancel the shield at any time ignoring initiative.
               Cancelling the shield is a 0a instant interrupt action.
               Anyone inside the shield can cast spells and attack as usual.
               All actions are mod-1 when inside the shield. Movement costs
               double.
               int 5, psy 7

strike blast   cast 2r 1m, range touch, duration 5r +3/m
               damage 4 +3/m (pay on cast) add to max weapon damage.
               Enchant a weapon for extra damage for the next strike it hits.
               Only active for one strike. Also works on projectiles.
               The weapon takes equal magical damage from the blast,
               check against the abs of the weapon as usual for incoming damage.
               Add +1m to allow a trigger word instead of first strike.
               int 3, psy 5

strike blast   cast 3ap 1m, range touch, duration 1r
               damage 4 +3/m (pay on cast) add to max weapon damage.
               Enchant a weapon for extra damage for the next strike it hits.
               Only active for one strike. Also works on projectiles.
               The weapon takes equal magical damage from the blast, check
               against the abs of the weapon as usual for incoming damage.
               int 7, psy 5

encharge       cast 2r 2m, range touch, duration 5r +3/m
               damage +1 +1/m (pay on cast) add to max weapon damage.
               to hit +1 +1/m (pay on cast) add to chance to hit when attacking.

encharge       cast 2r 2m, range touch, duration 5r +3/m
               abs +3 +3/m (pay on cast) add to weapon abs.
               parry +1 +1/m (pay on cast) add to chance to parry.
               Only works on weapons and shields, not armour.

slow           cast 1r 1m, +1target/2m, range 15 +5/m, duration 5r +2/m
               Movement costs double movement points.
               All actions are slow-2 (takes +2 extra ap)
               Full/multi-round actions take +1r
               Continue slow on psy+2/m vs psy each round (mana paid once
               on cast) up to the max duration. First round auto success.
               Once slow is lost on a target that target is free from
               the spell.
               int 4, psy 3

hasten         cast 2r 2m, +1target/3m, range touch, duration 3r +1/m
               Adds 3ap +1/m to target each round for the duration.
               Adds movement: M+1mp, W+2mp, R+3mp, D+4mp
               Initiative is raised by 3+3/m.
               int 8, psy 8

hold           cast 1r 2m, +1target/2m, range 10 +5/m, duration 3r +1/m
               Target cannot do movement at all, but stationary actions can
               can be performed without penalty.
               Continue hold on psy vs psy +2/m each round (mana paid
               once on cast)
               int 6, psy 7

paralyse       cast 2r 2m, +1target/2m, range 10 +5/m, duration 3r +2/m
               Target cannot move or take actions that require any movement,
               even if it is only with the hands.
               Continue paralyse on psy-3 vs psy +2/m each round
               (mana paid once on cast).
               int 7, psy 9

paralyse       cast 2r 2m, psy vs psy +1/m, range 5 +2/m,
               Target cannot move or take actions that require any movement,
               even if it is only with the hands.
               duration diff+1r +2r/m (pay on cast, diff from psy vs psy roll)
               int 7, psy 8

strengthen     cast 2r 2m, +1target/2m, range touch, duration 5 +2/m
               The target gets strength+3 +2/m for the duration of the spell.
               int 5, psy 5

fortify        cast 2r 2m, +1target/2m, range touch, duration 5r+2/m
               The target gets constitution+3 +2/m for the duration of the spell
               int 5, psy 5

fear           cast 1r 1m, psy vs psy +2/m, range 5 +2/m, area aura,
               duration 2*diff +3r/m
               int 4, psy 6

blind          cast 1r 1m, range 15 +5/m, duration 5r +2/m
               target gets vision=1 for the duration.
               int 4, psy 6

open door      cast 3ap 1m, range 15 +10/m, targets +1/m
               strength 5 +3/m
               Open a door. If the door is locked or barred the strength of the
               spell works as either physical strength to break down the door
               or as "pick lock" to try to open the lock.
               scf 0.2

close door     cast 3ap 1m, range 15 +10/m, targets +1/m
               Close a door.
               If caster has seen the door before he doesn't need line of 
               sight if he can pass an int-3 roll.
               scf 0.2

seal door      cast 3ap 1m, range 15 +10/m, targets +1/m, duration 10r +5/m
               strength 5 +3/m
               Seal a door with the strength of the spell. Attempts to break
               down the door will have to overcome the strength with str vs str.
               scf 0.2

dislocate      cast 2r 1m, range 20 +5/m, target 1 +2/m, psy vs psy +3/m
               target forgets the local location information.
               (Press ctrl+shift+o in maptools to reset map info to what
               the characters can see at the moment, as long as all heroes
               have been successfully dislocated).
               int 5, psy 5

pilfer         cast 2r 1m, range 5 +5/m,
               Can steal one item from the target.
               If the item is unseen (e.g: in a container) than it is a random
               item from among the potential items.
               A held item can be stolen with a successful psy-6 vs psy +1/m
               roll.
               The item vanishes from the old location and materialises in the
               caster's hand, with a nice tingling pop sound.
               int 5, psy 6

heroism        cast 2r 1m, range 10 +5/m, duration 10r +5/m
               select up to 5 +3/m targets which get that heroic feeling
               they receive +6 psy to all fear tests
               they receive +1 to all offensive actions
               The effect vanishes if the targets get outside the range of the
               spell, which is centred on the caster for the duration.
               A target looses the effect if he does not attack in a round
               when he has the opportunity to do so, unless attacking is clearly
               detrimental to his goal.
               int 5, psy 5

heal           cast 2r 1m, heal 3 +2/m, time 2hp/r, range contact
               int 5, psy 4

heal           cast 2r 1m, heal 2 +2/m, time 3hp/r, range contact
               int 7, psy 6

heal           cast 3r 1m, heal 5 +3/m, time 1hp/r, range contact
               int 7, psy 7

heal           cast 1r 1m, heal 1 +1/m, time 3hp/r, range contact
               int 6, psy 6

heal           cast 2r 1m, heal 1 +1/m, time 5hp/r, range contact
               int 9, psy 7

heal all       cast 3r 1m, heal 1 +1/m, time 2r/hp, range 3 +1/m
               heals all creatures within range.
               int 7, psy 7

force wall     cast 1r 1m, size 3 +1/m, duration 10r +5/m
               psy vs str +3/m to stop anyone passing the blocked passage
               attacks and thrown or shot objects are always blocked
               int 3, psy 6

force wall     cast 2r 2m, size 3 +1/m, duration 10r +5/m
               psy vs str +3/m to stop enemies passing through the blocked
               passage, friendlies can pass at maneuver speed
               attacks and thrown or shot objects are always blocked
               int 8, psy 7

staff light    cast 2r 1m, duration 50r +25r/m
               can be other objects than staff, but a trusty object is mod+3.
               basic light intensity is candle (5)
               +1m gives light intensity as lamp (10)
               +2m gives light intensity torch (15)
               +3m gives light intensity fire (20)
               +4m gives light intensity strong (30)
               int 3, psy 2, scf 0.2

marsh light    cast 2r 1m, duration 20 +20r/m
               Create a candle intensity orb of light that the caster can
               move around at speed 3 +3/m. Moving the light any amount in one
               round is a fast action pushing mod-2 to ams.
               +1m gives light intensity torch (10)
               +2m gives light intensity fire (20)
               +3m gives light intensity strong (30)
               int 5, psy 3, scf 0.2

mana transfer  cast 2r 1m, transfer 3m/r +3/m, range 5,
               willing target or object only, in/out
               Transfer is full round action, can only maneuver
               and do no actions for the duration.
               Max psy mana per cast.
               int 3, psy 3

raise dead     cast 5r 2m, duration 10r + 3*diff +10r/m, range touch
               Immediately activates control dead, with same diff, if the
               caster knows the control dead spell. If not ...
               int 4, psy 4

control dead   cast 2r 1m, psy vs psy +1/m +diff
               duration 3*diff +10r/m, range 20 +10/m
               Activates automatically after successfully raising a dead.
               Must roll psy vs psy +1/m +diff each round to keep control,
               mana spent once when casting or activating.
               When control is lost it can be regained by casting again.
               The caster is aware of his minion's surroundings as though
               being there. Vision is min(psy_caster|psy_target).
               Giving a control directive takes 1a, and must be very simple,
               e.g: attack target/group, move to location, guard area, etc.
               int 8, psy 8

lower dead     cast 3r 1m, psy vs psy +1/m, range 5 +5/m
               int 5, psy 7

mass raise     cast 10r 3m, duration 10r + 3*diff +20/m, range 5 +5/m
               target diff +1/m
               int 8, psy 8, scf 0.66

mass control   cast 3r 2m and psy-3 vs psy +1/2m +diff
               duration 3*diff +10r/m, range 30 +10/m
               Activates automatically after successfully raising the dead.
               Must roll psy vs psy +1/2m +diff each round to keep control,
               mana spent once when casting or activating.
               Simplify if necessary to: duration is reduced by the chance of
               breaking free squared.
               When control is lost it can be regained by casting again.
               The caster is aware of his minion's surroundings as though
               being there. Vision is min(psy-caster|psy-target).
               duration 3*diff +10r/m, range 50 +20/m
               int 10, psy 10, scf 0.66

mass lower     cast 5r 2m, psy-3 vs psy +1/2m, range 5 +5/m
               int 8, psy 8, scf 0.66

drain          cast 3r 1m, psy vs psy +3/m, range contact,
               drains diff mana points from target
               gives caster diff/2 (round up) mana, i.e. half is lost
               Must be in contact during whole casting.
               Cannot drain the target below 0 mana. If draining the target to
               0 mana the target must roll psy or falls unconscious. Roll psy 
               once per round to wake up again.
               int 5, psy 8

darkvision     cast 2r 1m, vision 15+5/m arc 180+90/m, duration 20r+20r/m
               int 3, psy 2

rot            cast 2r 1m, dam 4 +1/m 1hp/r penetrating, range 8 +2/m,
               magical rotting disease is doing the damage.
               one pain per damage dealt.
               int 6, psy 5

poison gas     cast 3r 2m, , area radius 4 +1/m, duration 10r +5r/m
               dam 1/r penetrating. Damage only if psy vs con +diff +3/m,
               roll once for susceptibility.
               int 4, psy 4

revive         cast 5r 5m, psy vs psy +1/m, the dead wanna stay dead.
               the body must first be healed to at least 1 hp before or during
               the revival ritual.
               The character tends to come back altered from the ordeal in the
               world of the dead.
               int 8, psy 10, scf 1.0

fumbly         cast 2r 1m, duration 5r +2/m, range 5 +2/m, target 1 +1/m
               Gives a base mod-3 -1/m to all actions during the duration.
               int 4, psy 4

fumble         cast 3ap 1m, duration 1r +2/m, range 10 +5/m, target 1 +1/m
               Gives base mod-3 -1/m to all actions during the duration.
               int 4, psy 4

pain           cast 1r 1m, duration 3r +2/m, range 10 +5/m, target 1 +1/2m
               pain = diff(cast) or diff(psy vs psy) +2/m
               Determine pain diff version when buying the spell.
               Inflicts pain mods on the target during the duration.
               int 4, psy 6

stun           cast 1r 1m, range 10 +3/m, target 1 +1/m
               Stun 9 +6/m
               int 5, psy 5

frenzy         cast 3r 2m, range 10 +2/m, power +1/m pay at casting
               duration until fail psy vs psy +power each round mod-1/r active
               duration minimum 1 round.
               primary action to roll for keep duration.
               Target is frenzied and will use all actions to attack nearest
               target, all out, regardless of friend or foe.
               int 8, psy 8

earthling      cast 3r 2m, activation range contact, duration 10r +5/m,
               power +3/m
               Gives the target the ability to walk through solid objects
               at a maximum movement speed of maneuver.
               The target must pass a psy+power roll each round or bounce
               back to where he started, taking 1d4 damage penetrating.
               int 5, psy 6

flying         cast 2r 2m, activation range contact, duration 2r +2/m,
               power +3/m
               Gives the target the ability to fly at a minimum speed of run.
               The target must pass a psy+power roll each round or fall down.
               int 6 psy 6

levitate       cast 2r 1m, target 1 +1/m, duration 5r +5/m,
               range group base contact, speed 2 +1/m
               int 4, psy 5

levitate       cast 2r 2m, range 5sq +5/m, duration 5r +5/m, target 1 +1/m
               Gives the targets the ability to levitate slowly at the maximum
               movement of walk. Targets must stay within the range of the
               caster.
               Unwilling targets must be forced by psy vs psy.
               int 6, psy 6

boiler plate   cast 1r 1m, range vision, duration 1r +1/m, target 1 +1/3m
               If the target has a metal armour he takes damage per round equal
               to the abs of the armour, penetrating.
               "Let's boil them in their expensive metal plate..."
               int 9, psy 6

teleport       cast 2r 1m, target 1 +1/2m, range contact.
               Teleport willing targets to any square in line of sight of the
               caster, or to any site the caster has memorised for
               teleportation and/or marked with a teleport marker.
               All transported must roll per, and takes fail diff as mod
               for the rest of the turn.
               Contact with other targets is only necessary for activation,
               not for entire casting.
               A caster can keep int teleportation sites in memory.
               Memorising a site takes 10-per rounds (minimum 1r) and then
               placing a teleport marker on the site.
               A fail-3 will deviate from destination 1d8 direction and
               distance fail^2 sq.
               A fail-6 will teleport the target somewhere else. Way else...
               Materialisation conflicts are resolved in some interesting way.
               A teleport marker costs ~1g and weighs 0.5enc.
               It is also possible to teleport unwilling targets, but it
               requires a psy vs 3psy +1/3m roll and takes 2r instead of 1.
               int 5, psy 5

sentinel       cast 3r 3m, duration 100r +30/m, charge 10 +3/m, range 3
               Creates a forbidding ghost warrior that will attack all non
               friendly targets that approach with a magic ghost spear.
               The sentinel warrior will never move from the square it was
               created. It will make one attack per round at skill 8, and up
               to two parries at skill 8, ignoring ams and initiative.
               The ghost warrior cannot be damaged or parried with non-magical
               weapons.
               It is created with a certain charge. Each attack takes one
               charge, and each hp damage it takes from magical means and
               weapons also costs one charge. It will dissolve when the
               charge is depleted.
               spear 8 dam 4, abs inf, reach 1, initiative 12
               vision 10 360deg dark (3x)
               is light source range 3, soft blue white
               int 8, psy 8, scf 0.66

detect magic   cast 1r 1m, radius 5r +3/m, power per/3 +1/m
               The caster will be aware of all unhidden spells within the
               radius. Hidden spells that are hidden with lower power than
               the detect spell are also detected. Hidden spells that are
               hidden with higher power are not detected. Spells which are
               hidden at equal power are detected 50\% of the time.
               int 5 psy 5

big boss' eye  cast 2r 1m, duration 10r +5/m, move 5sq/r +3/m,
               vision 20 +10/m 360deg +dusk=1m +night=2m
               Creates an almost invisible eye per-6 to notice, which floats
               in the air. The eye cannot pass through solid (1+sq) walls,
               but can pass through doors, thin walls, etc.
               The caster will see everything the eye sees as if he was there.
               Great for getting sight to other spell targets...
               int 9, psy 6

identify       cast 10r 1m, duration 10-int (min 3),
               Identifies up to 1mana spell +1/m, not hidden spells
               Hidden spells at power 0 +1/m
               int 9, psy 5, scf 0.5

disrupt        cast 1r 2m, range 5 +5/m,
               disrupt a spell being cast, requires resistance roll.
               disruptor: psy-3 + 1/2m
               target caster: psy + spell mana
               int 8, psy 8, scf 0.66

dispell        cast 2r 3m, range 3 +3/m,
               dispell an existing spell effect, requires resistance roll:
               dispeller: psy-3 + 1/2m
               target spell: psy(caster) + spell mana
               int 8, psy 3, scf 1.0

distribute     cast 1r 1m, range 5 +5/m,
               move one item +1/m from one willing character to another
               int 5, psy 3

teamspeak      cast 2r 1m, range 10 +10/m, duration 10 +10/m,
               targets int +3/m
               Select targets which will be able to hear each other without
               as long as they are within range of each other (not caster)
               even if they whisper.

boost          cast 1r 1m, range psy +5/m, duration 5 +5/m,
               targets 1 +1/2m
               Increase one of the target's base stats
               (str,dex,con,int,psy,per,cha) by 2 +1/m for the duration.
               Casting is mod+1 if touching target.
               Once cast the target can move out of range



summon demon   ! don't buy this yet, not finished with the background work
               cast 3r 3m, power 0 +3/m, duration 5r +5/m
               Summons a demon for the duration of the spell
               roll list dX: (common demon 5 slots, rare 1 slot)
                1 -  5  grey stalker
                6 - 10  green mouth
               11 - 15
               xx - +1  large grey stalker



perhaps a magic trap can come in handy sometimes
------------------------------------------------

embed magic    cast 3r 1m, duration 100r +100r/m, capacity 5 +5/m
               Can embed another magic formula in an inanimate object to be
               triggered by touch, destruction, or other "mechanical"
               disturbance. The capacity limits the power of the embedded spell.
               int 7, psy 5, scf 0.25

trigger rune   cast 5r 1m, duration 100r +100r/m, range 10 +5/m
               Can trigger one or more embedded formula within range.
               Triggers can be complex combinations of mechanical disturbance,
               sound, light, etc, with complex logic, but very limited
               "intelligence".
               It can react to triggering conditions within its range, and can
               target guide the triggered embedded spells.
               int 7, psy 5, scf 0.25

hide casting   cast 1r 1m, duration 5r +5r/m, capacity 5 +3/m, power 0 +1/m.
               Hides a formula being cast so that it cannot be detected by
               detect magic spells, tingling magic ability, etc.
               The power is set against the sensitivity of the detector.
               The capacity must be equal or larger than the total mana of the
               spell that is to be hidden.
               int 8, psy 8, scf 0.25


\end{verbatim} \normalsize
%\pagebreak[1]
%\small \begin{verbatim}
%TODO embedding stuff etc, NOT COMPLETE %------------------------------------------
%
%hide embedding cast 2r 1m, duration 100r +100r/m, capacity 5 +5/m
%               int 8, psy 8, scf 0.25
%
%permanence     cast 50r 5m, power 0 +1/3m, range 5 +1/m
%               The power must be equal to the total mana cost of all embedded
%               spells to make permanent.
%               Makes spells permanent. Used to embed magic in items.
%               It does not make the power permanent, the embedded permanent
%               spells still require mana supply to function more than once.
%               int 10, psy 10, scf 1.0
%
%storage        cast 25r 3m, power 0 +1/1m, range 5 +1/m
%               The spell creates a magical storage for mana energy. It can be
%               charged and discharged by transfer spells, etc.
%               Used to build and power magical devices.
%               int 10, psy 10, scf 1.0
%
%generate       cast 100r 5m, power 0 + 1/3m, range 5 +1/2m
%               The regen spell will function as a mana supply, generating
%               mana point equal to its power each day, just like a character.
%               Used to build and power magical devices
%               int 12, psy 12, scf 1.5
%
%nexus          cast 1day 10m, power 0 +1/10m, range 5 +1/2m
%               The power is the amount of mana that will be supplied per
%               activation discharge. This is the ultimate mana power supply,
%               since it can activate again and again in rapid succession when
%               combined with permanence.
%               It is susceptible to weak mana flow and void fields though,
%               by supplying lower and perhaps insufficient amounts of mana.
%               The mana coming from a nexus spell fleeting and strange. No one
%               has yet managed to figure out a way to store the mana for later
%               use. It has to be used immediately by channelling it into a
%               spell.
%               int 12, psy 15, scf 2.0
%
%
%
%
%\end{verbatim} \normalsize



\

More spells on the way, not balanced 

\

\small \begin{verbatim}
destructive dispell
    cast 1r, 1m, range 20 +10/m, 
    psy+3/m vs psy+mana
    blast with range of target spell mana
    damage is target spell mana
    i.e. target spell disintegrates in a blast with damage and radius equal to mana

destructive feedback
    cast 1r, 1m, range 20
    psy+3/m vs psy
    damage 6 penetrating and draws equal mana from target, 1mana/damage
    i.e. target's mana is damaging
    Cannot draw/damage below target mana 0.

invisibility
    cast 1r, 1m, personal +1 target/2m touch
    duration 3r +2/m

elsewhere
    cast 3ap, 1m, personal
    teleport 2D4-5 sq X/Y (+/-3sq)
    jump teleport anywhere else, away from incoming blow
    (alternatives: 2D6-7 +/-5sq , 2D3-4 +/-2sq)
    If the target square is occupied the caster bounces back to original square
    but an incoming blow has perhaps missed, treat as avoid success+9 roll
    unmodified.

goaway - knockwave
    cast 1r, 1m, self, radius 1sq+1/m
    knockback 1+1/m, affects all within radius
    all affected are knocked back, away from the caster

fiery eye        int 5 psy 5
    cast 3r 1m, touch, duration 20+1d10r +10/m       (duration based on object?)
    creates fire, light source r10sq +5/m
    the caster gets vision 360deg from the fire source.  (wish for tool to limit
    default day vision, +1m dusk, +1m night, +1m dark     rate of turning of arc
    range to target object is unlimited                   eg 90deg, 45deg/r rate
                                                          of rotation)

one hand
    cast 3r 1m, range 10 +5/m, duration 5r +3/m
    One of the target's limbs is paralysed and useless.
    psy+3/m vs psy each round to maintain spell after first round.
    
other's hand
    cast 3r 1m, range 10 +5/m, duration 5r +3/m
    The caster takes control of one of the target's limbs.
    Any actions taken with the limb are paid from the caster's ap, and are
    slow+1 ap from usual cost. Any action requires a roll to succeed limited by
    the minimum of any skill the action requires and the success diff of the 
    psy vs psy control roll for the round:
    Control: psy+3/m vs psy each round to maintain spell after first round.
    Maintenance: int-1, psy-1

mimic
    cast 2r 1m, range 10r, duration 10r +5/m
    range after casting: 30r +20/m
    Make an object look and feel like something else.
    A perception roll penalty modified by casting success will see that there is
    something strange with the object. Mod by the complexity of the object.
    A find roll will see through the illusion.
    Maintenance: any movement or significant complexity requires int-1 psy-1
    Just stationary object is regular psy-1.

-

\end{verbatim} \normalsize


%--------|---------|---------|---------|---------|---------|---------|---------|
%       10        20        30        40        50        60        70        80
%-------------------------------------------------------------------------------
% removed spells
%
% hasten ?





%-------------------------------------------------------------------------------
%M A G I C A L   E Q U I P M E N T
%---------------------------------

\phantomsection\addcontentsline{toc}{section}{equipment}
\section*{Magical equipment}

Wizards and Warriors alike can gain great power by wielding the arcane artefacts imbued with various magical energies. Below are some suggestions of magical items and artefacts. Make up your own to suit your adventure.

\

\small \begin{verbatim}
Quaff! the     Metal tin with a golden glowing liquid. Generally costs ~1gold.
soft drink     Restores 4 hitpoints and 4 stamina in one round.
of heroes      Has a tendency to be addictive.
               Advertisement: "Yngolf Storpec says it even tastes good."
               The fine print: All side-effects are purely co-incidental
               and have never been scientifically or thaumaturgically proven
               to be caused by the use of Quaff! Quaff corp is not responsible
               for any mental or physical disturbances, changes, manifestations,
               or other interesting or harmful events.

Blast Off      Beautiful crystal vial of clear liquid. Costs ~5gold.
               restores 5+1d5 hitpoints and all stamina in 5 rounds.
               If the vial is broken before empty it detonates doing
               1d5 damage to a radius 1 area blast.
               Everyone caught in the blast get 2 stamina restored.

The Seeking    mod+3, dam 5, range 16, str 3 (no str bonus)
Bow            range mods as regular bow, rate of fire as regular bow,
               The arrows will always curve and re-target to the largest 
               creature within range 10 regardless where it is aimed 
               (except user). Outside range 10 it gives a mod+1 to hitting the
               selected target. Largest is decided by base size, or str.
               Fuck-off Fred swears it seems to like friendly fire.

Slight hammer  dam 5, abs 14, str 5, finesse-2
of solitude    After a regular hit the user can choose to spend 1 mana to:
               roll psy+str vs opponent str. Effect = diff/3
               For every effect point the opponent is dealt 1 damage
               penetrating and knocked back 1 square.
               Charisma -2 while owned (grumpy).

Mostly         dam 7(12), abs 15, str 7, finesse-2
Magical        On each hit, roll for 3 on a d10, if successful the MMM activates
Morningstar    and forcibly draws 1 mana from the user,
(heavy)        same rules for unconsciousness as "casting when dry".
               Activation adds 5 extra to max damage of the weapon (12) for the
               strike that just hit.
               All damage is treated as both physical and magical damage for
               purposes of damaging magical and non magical living creatures.

Terribly       A magical torch with a perpetual soft bioluminescence (range 8).
Tasty          It grows magical glowing mushrooms from the top.
Torch          They are very tasty. Any monsters nearby (distance 3) must 
               succeed a psy roll or get the torch and eat the top mushrooms.
               They grow back in a day, but the torch is dark without them.
               Eating the mushrooms will cause "frenzy" (see spell)
               for 5+1d10 rounds.
               They cost a lot, difficult to find, and illegal in most areas.

The DMT of     The last person who found out what DMT stood for died in the
Dread and      process. It is a box of tools, and it gives off a soft "ping"
Utility        at regular intervals. A hammer is missing.
               It gives a +6 to mechanical, locks, traps rolls, 
               and +3 to McGyverism, find, dungeoneering rolls

The Shiny      The helmet really shines. Provides a 20 distance bright light.
Helmet         It will draw one mana at the start of every 30 rounds.
               It's a scary 5 object against monsters that dwell in darkness.

The One Ring   There is only this one!

Banors Gate    A large orb set in a stretcher-like carrying contraption.
               It requires a total of 15 str to carry around.
               Up to four people can help carry the device.
               The Orb is magically linked to a secondary portal plate.
               Touching the orb activates a teleport to the plate.
               Stepping on the plate activates a teleport to the orb.
               The plate cannot be moved once linked with the orb. That would
               break the link, and the orb must be placed on the plate to
               create a new binding link.
               The orb is charged with a certain numbers of teleport activations
               when it is bound to the plate. Re-charging requires re-binding.
               Teleporting from the orb requires contact with it for five
               rounds. Teleporting from the plate to the orb takes one round.

Metaphysical   A harness with a faintly glowing red tendril sticking out of its
cord of        back, extending about a meter, dissolving into nothing.
Wunjee         Activating the harness takes one round and the wearer is flung
               through meta-space to a anchoring point, taking 1d4 damage
               penetrating. The harness is then physically bound to the
               anchoring point by a thick solid tether 3sq long,
               with abs15 and hp10.
               Climbing out of the harness takes 3r.
               Most are single use, but can be recharged at lower cost than
               a new item.

Metaphysical   A device with chains, hooks, locks, spikes and other interesting
anchor of      protrusions. It must be fastened to something very sturdy.
Wunjee         A red glowing tendril extends for about a meter, dissolving into
               nothing. Fastening the anchor takes 20r, unclasping takes 10r.
               It had better be really securely fastened.
               Most are single use, but can be recharged at lower cost than
               a new item.

Eileens Eye    A glass eye. It looks very realistic.
               The owner of the eye always sees what the eye sees, regardless
               of how far they are apart.
               vision: range 20 arc 180deg.

Wailing Doll   A small child's toy, a rag doll. It is very afraid of monsters.
               When it sees a monster it comes alive and starts screaming very
               loudly. It stops when it doesn't see any more monsters.
               1hp, movement 0.
               vision 5 arc 180deg.

Whaling Doll   A giant child's toy, a wooden construct giant. It requires str20
               to move the doll. Up to four people can help carry it.
               When it sees a monster of str>=10 it comes alive, stands,
               and attacks with a large harpoon spear.
               It makes one attack each 3 rounds (Throw, retract, ready).
               str 30, hp50, abs3, movement 0,
               will turn to face enemy when active
               harpoon 7 dam15, range15
               If hit, the target is bound by the harpoon wire to the doll,
               which will reel it in and try to tear the target apart limb
               from limb (str vs str).
               vision 20 arc 225deg.

Time Fly       A small crystal with a strange looking bug encased.
               Crush the crystal and the bug will turn back time for the user
               by one action. The user may then choose to do something
               else than he tried last time. Perhaps run away?
               The user acts first for the redo action, ignoring initiative.
               Crushing the bug is a 0ap interrupt action ignoring initiative.
               This assumes the bug is in a quickdraw slot, otherwise it must
               already be held or it is useless.
               Time Flies are very exotic and expensive items.
               Ad: A monster charges and attacks. You valiantly defend,
               but get slaughtered! No problem, just Bash the Bug!
               And you are right back at the monster just charging you.
               This time you run away, seeing that you can ignore initiative!
               The (movement +) attack-defend exchange or (movement +)
               attack(miss) - counter attack is considered to take
               place in one action time as long as the bug-crusher has taken
               no further actions.
               weighs 0.1 enc

Stunning       A beautiful white javelin with dark black runes along the shaft.
Sticker        When striking a target and inflicting at least 1hp damage after
               armour the target must roll 10 vs (highest psy or con),
               or be stunned for diff/3 (round up) rounds.
               Even if resisted it gives a stun 6 effect from the shock.
               weapon stats as normal javelin, mod+1 for ranged and melee.

Sentinel Base  A large disc, usually made of stone and metal. When dropped and
               the top seal is broken it will summon a sentinel ghost warrior
               (see the spell). Charge, skill, weapons, and appearance can
               vary between different versions and manufacturers.
               Quite expensive, but very useful sometimes.
               Weighs around around 3.0 - 8.0 enc

Ward Rune      A large tattoo, usually on the back. When the character is
Tattoo         attacked the tattoo flares up and protects him.
               Usually it absorbs 1 damage and draws 1 mana. Various quality
               and power absorbs and draws differently. Usually the wearer
               can choose not to activate the tattoo if he wants to save mana.

Glowrb         A crystal orb which when touched starts to glow with a bright
               white light. It stays alight for 10-1000r or until touched again.
               Single use, charge, or permanent, depending on price and quality.
               enc 3.0 - 5.0

PortaPortal    Two large metal discs, comes as matched pairs. When both are
               placed on the ground and both seals are broken it creates a
               short lived portal between the disks. The portal usually lasts
               for about 10r. A disk must not be moved once the seal has
               been broken and that end of the portal activated.
               Often used to bring in a party to areas which is otherwise
               difficult to get in to, such as a well guarded keep, a remote
               mountain cave, or the inner treasure chamber of the dragon's
               lair. Just have your stealthling sneak in one of the discs of the
               portaportal, then activate. Meanwhile the rest of the party are
               safe and sound back in the tavern with the other end already
               active. Can also be very good for escaping a tight spot. Just
               make sure the monsters don't follow you through.
               Usually activates in 1-3r depending on quality.
               Single use only.
               enc 3.0 - 10.0 x2, usually ~5.0

Amulet of      When the character is taking hp damage the amulet activates and
random         absorbs 1d3 damage, draining one mana in the process.
protection     Some versions protect more, or drain more, or less, mana.
               enc 0.1-0.5

Proportional   The proportional plate mail absorbs 1/3rd of the incoming damage
Plate Mail     (round to nearest). Thus for heavy opponents it can
               be a real life saver, while for smaller critters it is nearly
               useless.

TurBoots       Magical stuffed fish slippers. Spend a mana and increase your
               movement: m+1 w+2 r+4 d+8. Must roll dex at end of movement 
               or fall. Take mod+3 to the roll if going off balance. Or just do 
               a faceplant as a controlled fall.
               Or get the maneuver "TurBoot Shuffle" (5xp) to skip the dex roll.

Feedback Football  angry glowing metal ball contraption
               glow 2sq light, destructuve feedback triggered range 6sq
               6 dam drain equal mana
               triggers on touch (pickup, but not kick)
               triggers on any casting in range, 8 charges
               -- TODO -- alt text --
               activates when thrown, deactivates when packed away, 8 charges
               attack on touch or detecting casting within 6sq: destructive feedback
               draws 1d6 mana for max 6 dam, penetrating : resistance 15 vs psy

Long Sword of  dam 8, abs 14, str 6 (max +2 str bonus)
Irdereng       on successful hit the attacker can choose to roll psy,
               if successful the attack does diff (psy roll) extra damage,
               penetrating, and draws one mana point from the user.
               finesse/poke/swing as usual

Angus' Shield  abs 20, parry+3, str 4
               when declaring parry, the user can choose to spend 1 mana
               and get another parry+3 mod. Only one mana per parry action.

The Very Evil  dam 2 pen 1 abs 5 (no str bonus), parry-3, todefend-3, finesse 9
sacrificial    Starting a ritual sacrifice takes 10r of mumbling mumbo jumbo.
dagger         Drains-converts 1hp to 1m per round (after 10r start).
               Cannot drain from dead/undead targets.

crystal ball   holds 20 mana, transfer 1m/r in/out
               Transfer is full round action, roll psy.
               Must stand still.

energy rod     holds 10 mana, transfer 3m/r out, transfer 1m/3r in
               Transfer is full round action:
               Roll psy, if fail: diff mana is lost in the transfer.

psycharmour    abs 2 +2/m must declare and roll for skill before damage is dealt
               (i.e. after hit, but before to damage rolls) -- not all versions.
               Some versions can declare and roll after damage is rolled.
               Some versions must declare and roll before attack is rolled.
               Defensive mana comes from armour store, not user.
               Defensive mana spending is not an action, it's a 0ap interrupt.
               Holds 15 mana, transfer 1m/2r in, full round action, roll psy.
               psycharmour skill is scf 0.33

powerstaff     dam 3/4 (1h/2h), abs 10, parry 0/+1, str 3 (max +1 str bonus)
               Holds 10m. Can only release 1 point per spell casting, and the
               released mana must be used to reduce the spell mana cost to the
               caster by 1 point. Release is a 0ap free action, included in the
               spell casting, no roll required.
               Transfer to staff 1m/r, full round action, roll psy.
               Cannot transfer power from staff to bearer.
\end{verbatim} \normalsize

\

There are of course also more normal boring magical items, weapons, and artefacts.

\

\small \begin{verbatim}
weapon +x     A weapon that is easier to use or does more damage. It can give
              the user a bonus to his weapon skill, or to his damage roll, or
              to both. Or have exceptional abs resiliance, or ...
              Weapons like this does not have to be magical. Master or Legendary
              craftsmen can make +1 or even +2 weapons.

healing potion
stamina potion
mana potion
[ability] potion
[skill boost] potion
[stat boost] potion
\end{verbatim} \normalsize

\

\clearpage
\TODO custom magical stuff is not finished yet... 

\

\textbf{Just throw out and rewrite!}

\

\emph{Custom magical items: the following are just idea sketches and should not be taken seriously. I need to work on this quite a lot before it is useful.}

Purchasing custom magical potions: A quick approximate cost for general low power magical potions: \\
cost in gold ~ 1 + (0.025 * (mana+2) * scf * psy * int) \\
The extra mana is to approximate the cost of trigger rune and embed magic spells.


\subsection*{Manufacturing custom magical potions}
%-------------------------------------------------
The wizard must know both "embed magic" and "trigger rune" to bind the spell into the potion. Then it requires a suitable potion base liquid and vessel. Those are not so cheap. The vessel and liquid is lost if either the spell, the binding or the trigger rune fails when casting.

The cost of a suitable liquid and vessel can be calculated as: \\
cost in gold ~ 0.25 + (0.01 * mana * scf * psy * int) \\
It is possible to use inferior liquids and vessels, but the potion will lose potency or stop working all together pretty quickly and unreliably.
Don't forget to include the mana cost of the trigger rune and embed magic spells when calculating the vessel cost.

Magical potion: \\
Get the right vessel and liquid, or other containment item.
Then cast the embedding spell and the actual spell. Since a potion is to be quaffed it requires no complex trigger rune, and the very simple triggers in the embed magic spell are enough.
E.g: a potion of fire storm can be created like this:
Fire storm spell is scf 0.33, int 6, psy 8, and the customer wants radius 2 and damage 6, so the spells requires 3 mana by itself. Then an "embed magic" spell is required to bind the fire storm into the potion. When binding into a "suitable material" the duration of the embedding is of no consequence, and thus only the capacity is of interest. The embed spell has a base capacity of 5 and that is enough for the fire storm spell.
The cost of the liquid and vessel is thus approximately:\\
base cost: 0.25 gold \\
fire storm requirements + (0.01 * 3 * 0.33 * 6 * 8) gold \\
embed magic requirements + (0.01 * 1 * 0.25 * 8 * 5 ) gold \\
which in total = 0.25 + 0.48 + 0.1 = 0.83 gold. \\
This is of course an approximation of the cost of the liquid and the flask. Actual prices will vary, and the suitable materials might be difficult to acquire.

Purchasing ready made magical potions is more expensive, with cost approximately \\ 1 + (0.25 * (mana+2) * scf * psy) gold.


\subsection*{Manufacturing custom magical equipment}
%---------------------------------------------------
Manufacturing magical items is more expensive than potions. There is a lot more magic involved and the materials to be enchanted need to be able to hold the magical energies and forces involved.

Single use or limited use magical items can be manufactured similar to the potions, except they usually need a proper trigger spell to focus the spells at the right target.
The approximate cost of materials can be estimated as: \\
1.0 + (0.25 * mana * scf * psy) gold \\
for simple single or multi use items such as wands, crystals, etc.

The interesting stuff, the permanently magical items require



Manufacturing custom magical equipment is similar, though often more expensive, but then the caster also needs to know the "permanence" spell to make the magic last, together with the "nexus", "transfer", or "drain" spells for powering the item. And if any of the spells fail the item will be warped, destroyed and useless.

The required magic to create a single use magical item is just like the magical potion, except it also needs a trigger rune, embedded with the spell.

To create a many use magical item is very similar as a one use, just cast the spell+trigger+embed multiple times, until the items magical capacity is full.

To create a permanent magical item things become more tricky. The "permanence" spell must be cast to encompass the payload spell, the trigger, mana source, and embedding. And of course they also require some form of mana source.

Mana sources come in different forms. The simplest is that the user of the item knows the "magic" skill to sufficient level to power the payload spell with his own mana. This is not so handy though, since most customers don't have the magic skill, and thus cannot empower magical spells.
The next step up is to embed a "mana transfer" spell together with the payload spell. This way the item can draw mana from a willing user.
The next step up is to embed a "storage" spell with the payload with sufficient mana to use the item once or a few times, before the storage needs to be recharged.
The next step up is to combine the storage with a "generate" spell which slowly replenishes the storage spell over time. This way the item will recharge by itself, and can still be manually recharged by a mana transfer spell if needed.
The coolest and ultimate mana supply is the "nexus" spell, which always can supply mana directly to the payload spell with high frequency ad infinitum.

Here follows an example throwing knife of slow motion \\
Payload: slow: mana 3 (+2m duration), scf 0.33, int 4, psy 3 \\
Trigger: trigger rune: mana 1 (contact with target), scf 0.25, int 9, psy 5 \\
Supply: storage: 12 mana (+9m capacity), scf 1.0, int 10, psy 10 \\
Supply: generate: 14 mana (+3m recharge), scf 1.0, int 12, psy 12 \\
Embed: embed magic: 7 mana (+6m capacity), scf 0.25, int 8, psy 5 \\
Permanence: permanence: 116 mana (+111m capacity), scf 1.0, int 10, psy 10 \\
Which creates a throwing knife of slow motion, which can be used three times on its own power, and recharges one use per day.

Now, the approximate cost of suitable material for the throwing knife would be something like this: \\
payload: 0.01 * 3 * 0.33 * 4 * 3 = 0.12 \\
trigger: 0.01 * 1 * 0.25 * 9 * 5 = 0.11 \\
supply: 0.01 * 12 * 1.00 * 10 * 10 = 12 \\
supply: 0.01 * 14 * 1.50 * 12 * 12 = 30 \\
embed: 0.01 * 7 * 0.25 * 8 * 5 = 0.70 \\
The permanence requirements does not fall on the target object, but rather the equipment of the wizard creating the item. He must have such an object available to build up the permanence spell in before sealing the spells of the throwing knife. \\
permanence: 0.01 * 116 * 1.0 * 10 * 10 = 116 \\
Thus the materials requirements for the knife will probably cost somewhere around 40-45 gold, at break even for the manufacturer! Expect the wizard to require additional funds for the work and whatnot, and of course the weapon smith will require quite a bit to custom make the knife itself, before it is handed over to the wizard to be enchanted.

The same knife, but drawing mana from the user is much cheaper to make: \\
Payload: slow: mana 3 (+2m duration), scf 0.33, int 4, psy 3 \\
Trigger: trigger rune: mana 1 (contact with target), scf 0.25, int 9, psy 5 \\
Supply: mana transfer: 1 mana (get 3 mana), scf 0.33, int 3, psy 3 \\
Embed: embed magic: 1 mana (5 capacity), scf 0.25, int 8, psy 5 \\
And the permanence requirement \\
Permanence: permanence: 35 mana (+30m capacity), scf 1.0, int 10, psy 10 \\
Which gives a total estimate of the knife material cost at: \\
payload: 0.01 * 3 * 0.33 * 4 * 3 = 0.12 \\
trigger: 0.01 * 1 * 0.25 * 9 * 5 = 0.11 \\
supply: 0.01 * 1 * 0.33 * 3 * 3 = 0.03 \\
embed: 0.01 * 1 * 0.25 * 8 * 5 = 0.10 \\
permanence: 0.01 * 35 * 1.0 * 10 * 10 = 35 \\



Magical capacity of weapons and items:\\
capacity factor of material * abs * encumbrance / 10.

steel:       5
star iron:  20
mage steel: 10
mithril:    50
copper:     15, brass 10
silver:     20, alloy 15
gold:       30, alloy 20
crystal:    50
gems:       1/gold value



\

\TODO: snipped moved from prices.tex :

Permanently magical items are always custom made, and very exotic equipment.
First of all the item to be enchanted needs to have enough magical capacity to withstand the forces of the magic to be embedded within it. This means that normal steel will almost never suffice, and more exotic and expensive materials will have to be used in the making of the item itself.

The magical capacity of an item can be calculated to approximately: \\
capacity = cf * abs \\
where cf is capacity factor of material and the skill of the maker, the abs is the absorption or hitpoints of the item.
E.g: a steel sword of regular making has a magical capacity of 10 = (1 * 10).

Here follows the magical capacity of some materials:

\

\small \begin{verbatim}
material          capacity      abs  enc  dam   price

wood              1.0    1      1.0  1.0  1.0    1.0   (for common use)
gnarlwood         5.0    5      1.0  1.0  1.0   10.0   (instead of wood)
steel             1.0    1      1.0  1.0  1.0    1.0   (for common use)
mage steel        5.0    5      1.0  1.0  1.0   10.0   (instead of steel)

copper alloy      9.0   15      0.5  1.2  0.8   30.0
silver alloy            25      0.4  1.5  0.7   50.0
gold alloy              40      0.3  2.0  0.6  100.0
strong crystal          50      0.5  1.0  1.2
mithril                         2.0  0.7  1.2
star iron

copper                     ~3/enc
silver                     ~5/enc
gold                      ~10/enc
jewels            1*cost ~100/enc

\end{verbatim} \normalsize

The capacity requirements of a permanently magical item are:
payload spell: cap = mana * psy (E.g: black bolt 3m = 21 capacity)
trigger spell: cap = 5 per payload spell for normal complexity triggers
no mana supply: cap = 0 (empower by user "magic" spell)
mana transfer: cap = 1 per mana (payload requirement)
mana storage: cap = 30 + 10 per mana
mana generation: cap = 50 + 30 per mana
mana nexus: cap = 100 + 100 per mana
embedding: cap = 1 per mana (total mana)

Permanence doesn't require cap of the target object, but from the ritual equipment of the wizard.






%-------------------------------------------------------------------------------
%F A M I L I A R S ,   M A G I C A L   C O M P A N I O N S
%---------------------------------------------------------

\phantomsection\addcontentsline{toc}{section}{familiars}
\section*{Magical familiars and companions}

Familiars and other magical companions are more intelligent than trained animals and often share a mind link with the owner.





%--------|---------|---------|---------|---------|---------|---------|---------|
%       10        20        30        40        50        60        70        80

%--------|---------|---------|---------|---------|---------|---------|---------|
%       10        20        30        40        50        60        70        80


\cleardoublepage

\phantomsection\addcontentsline{toc}{chapter}{Campaign}
\chapter*{Campaign}
\chaptermark{campaign}


The campaign aspects of the hactac project were the most fun, so we'll keep going with that. The death toll was lower than I expected through all the main campaign threads we've played so far. There is a surprisingly strong effect of Heroes being able to retreat and regroup when things were getting dire. Not intentionally, but good, and excellent for campaign gaming. Surviving Heroes generally make for better long form stories. 

For campaign gaming you'll need a group of players that is somewhat consistent and that can schedule sessions. The players should probably build their own heroes, perhaps starting from scratch with 100xp newbies. And you'll need a campaign arc spanning several adventures. There are already released campaigns available, and more is on the way. Ping the author if you want the notes from the as of yet unreleased adventures and campaigns. We have hundreds of hours worth of adventures available spanning several campaigns

\

The game is primarily designed for short sessions with fast gaming, and we play mainly by remote with a virtual tabletop (\emph{as of late 2019, still maptools and mumble, still working well even for the sometimes enormous battles of the Goblin Destiny campaign.}). We'll continue with the usual disconnected reality clichés, either by rules, circumstance, or magic, so that it is trivial to inject and extract characters through the sessions. I expect people to be connecting/disconnecting due to family, scheduling, net/bugs, etc. The campaign style and encounter design must accommodate for issues like that.




%===============================================================================
%                    C R E A T I N G   C H A R A C T E R S
%                    -------------------------------------

\phantomsection\addcontentsline{toc}{section}{create a character}
\section*{Creating characters}
%-----------------------------
\begin{enumerate}
\item Roll/select/build your brand new Hero-to-become character. Decide with your group and GM how the characters should be rolled up. Suggestions below.
\item Buy skills that makes sense for the style and role you want your character to have. Tailor to the game style your group is looking for: more role play, more fun nonsense, more tactical challenges?
\item Select some suitable shitty noob equipment together with the GM. Spend some of your hard earned and long saved coin to upgrade some critical equipment perhaps?
\item \emph{If} this is your first time, play through the \texttt{Dungeon of Testing} solo with your GM to see if your character has enough survival potential to leave his childhood homestead and venture into the World Outside for Danger and Adventure in Far Away Lands, or the village next by.
\end{enumerate}
% edit below (200205): hmm, fun, but gives the wrong impression actually, mortality has been shown to be lower than originally expected. Most characters survive well past 0.3k maturity. Approximated mortality is <20% up to 0.5k
%Don't spend too much time creating your first character, it will probably die soon anyway, and you'll have to make another, and another, and yet another one again. 
%It might even be a good idea to create a bunch of them in the first place so that you have the queue already waiting by the mouth of the dungeon, ready to charge into the darkness in search of a quick and painful death.

Now, all depending on game style: For combat oriented gaming there are a few things you should consider. It might be worth it to focus the character on a single role when creating a 100xp newbie. He will not have enough points to get good at several different tasks in the very beginning. Skills you expect to use regularly should have level 5 or better, and the main combat skill at lvl 7 or more is a good start. If possible, try to spread out some basic support or campaign skills over the Heroes in the party.

Early on it's good to have a strong defence. Shields and two handed staffs are good defensive options since their bonuses makes them more reliable in the early adventures. E.g: The cost to get an effective 8 in parrying defence action varies greatly. For sword it's 64xp, for 2h staff it's 28xp, and for a normal shield it's 17xp. The strongest defensive action is generally \emph{avoid}, but that costs 51xp for lvl 8.

A standard fighter should probably have a main weapon and a shield. An alternative if he has good mobility is a two handed high damage weapon and good avoid capacity. A ranged fighter can be fast to stay out of trouble by positioning, or also carry a defensive melee weapon such as a staff or shield. The same goes for spell casters. Close combat tends to come for everyone quite often. Archers should probably also be quick at changing between the bow and the melee weapon.

You can find some example Heroes at various experience levels in the \hyperref[cpt:characters]{characters section}, page \pageref{cpt:characters}.


\subsection*{Rolling character stats}
%------------------------------------
Further below are lists of different races and their specific stat rolls. The GM together with the player group should decide how to roll up characters. Straight up one roll? Rerolls? How many? Best of 3? Roll then distribute? Points scheme? How many points?

I recommend using the \verb|rollchars.py| script to roll up a set of characters that the player can choose from. Easy to tweak how many characters of the different races should be in a fresh rolled noob set.

%Note that you may use the "hero" version of the rolls if you want, and that you may roll movement instead of choosing the race standard if so desired.

%The GM should decide how many character sets a player is allowed to re-roll when creating characters. However, I suggest allowing the players to roll more humans than dwarves, elves, and orcs. For example, allow the player to roll three humans or goblins, but only two dwarves, elves, orcs, halflings. Then select the one he wants to play. The humans and goblins are less powerful than some of the other races. By allowing the player to roll more alternatives this power deficiency is balanced a bit.

%To make things simple I suggest using the \verb|rollchars.py| script. Just change the number of total characters to choose from and run the script. It will roll a selection of characters from all races.


\subsection*{Buying skills}
%--------------------------
The cost of buying skills is the "skill cost factor" (scf) times the square of the skill level, round down. \verb|cost = scf * sl^2|. However, a skill will always cost at least 1xp. \\
E.g: skill cost factor 1.3 to level 4 costs 1.3*16=20.8 round down ~20. \\
E.g: raise same skill from 4 to 6: 1.3*(36-16)=26. \\
Note that it can be a little bit cheaper to raise skills a little bit at a time instead of a lot in one go. However, if you want to do this you have to go adventuring between every skill purchase.

Sometimes characters start with some free skills rolled up with the character. Those are free above the rolled up starting xp value. Save your original pristine rolled up noob version of your character, as is, before spending any xp.


%\subsubsection*{some skills to consider}
%%---------------------------------------
%There are some basic skills that are worth considering in the early stage of your character development.
%gossip, find, track, \\
%avoid, jump, climb, swim, balance, ride, travel, \\
%literate, counting, (language), lock&traps, sneak, \\
%histography, monsterology, dungeoneering, first aid, \\




%NOTE: childhoods and professions disabled, this also disbles ability:munchkin
%
%\subsubsection*{Buy a childhood}
%%----------------------------
%\subsubsection*{Buy a profession}
%%--------------------------------
%\subsubsection*{Buy a childhood / Buy a profession}
%%--------------------------------------------------
%Childhoods and professions have been removed.
%They became obvious easy choices and made it
%too simplistic to opt for specific targets.
%Hence the character creation became less interesting.
%That was of course not the intent. So, gone for now, until fixed...
%
%%Still available in \emph{childhood-and-profession.tex} file.


\subsection*{The character's background}
%---------------------------------------
Once the character has a full set of traits, you go ahead and make up a suitably funny background. Also, get some starting gear that makes sense for the character, nothing too flashy and useful. He is a newbie after all. If it is expensive, it should be generally useless for combat and hard to sell.


%\subsection*{The inbred rule}
%----------------------------
%The shittier the character, the richer he may be. Nobility is rich, and inbred. Therefore, crappy stats \emph{might} mean that he is a dim witted offspring of some nobleman or other. It might also mean that he is a total looser, and poor on top of that.
%
%A general rule could be that the character may have non-combat equipment to a total value of 1 gold per total point under average in his primary stats (str-cha).


\subsection*{Selecting equipment}
%.-------------------------------
Any aspiring adventurer can get a "starter kit" suitable for his background, all for free! However, the starter kit equipment is generally of really poor quality. Not 2nd hand, but probably 13th hand. It is generally unsellable stuff. The shitty start equipment will have strange quirks and failure modes, and worse stats and penalties than the standard equipment.

See the "pregen newbies" in the character file for some examples. Some of them have spent money to improve their free starter kit gear.

%The inbred rule can be applied here to give a more expensive, (but generally non combat) gear to chars with lousy stats.

Improving the starer kit can be done by spending money, or by purchasing a newbie starter kit (see below) Look through the equipment list and see what is available. Newbies probably don't have access to exotic equipment, and perhaps only 50\% chance to special equipment.

Some interesting equipment to purchase can be: \\
light source: torch, lamp, candles, etc. \\
fire tools: flint and steel, coal box, etc. \\
food and wine, and a waterskin or two. \\
sacks or bags for loot. \\
backpack for easy carrying of equipment. \\
rope to help people climb up out of trap pits. \\
grappling hook if no-one is left outside the pit. \\
pick axe for getting out of the collapsed tunnel...

Some character backgrounds would allow them to start out rich, and with some expensive equipment, but then it should come with a serious down side. Perhaps the princelings manservant is an evil backstabber, plotting with the local highway robbers to steal everything and kill the little snot in the process?


\subsection*{Equipment newbie starter kits}
%------------------------------------------
Starting out in the Hero Adventuring business can be difficult, and it's so useful to have a small knapsack of stuff with you when you leave the old homestead. Character creation is probably also the only time when your Hero can buy equipment for xp instead of silver.

This equipment is of decent quality and not the kind of hand me down rusty old fail-at-the-worst-possible-moment standard starter fare that the less prepared newbie adventurers tote around.

\

\goodbreak \small \begin{samepage} \begin{verbatim}
Woodsman's backpack (cost 5xp):
    backpack 5enc, knife, spear or bow with 10 arrows,
    water bottle, food for 3+1d3 days, fishing gear, snares,
    flint & steel, tinder, 1d3 torches,
    water proof hooded cape, 5+D10 copper.
\end{verbatim}\goodbreak\begin{verbatim}
Uncle Smith's well wishes (cost 8xp):
    sword, shield, warm cape, water bottle, 20+D10 copper.
\end{verbatim}\goodbreak\begin{verbatim}
Farmer Johns goodbye (cost 4xp):
    knife, staff / axe, shoulder sack, water bottle, food for 5+D10 days,
    flint \& steel, warm blanket, 10+D10 copper.
\end{verbatim}\goodbreak\begin{verbatim}
Auntie's present (cost 5xp):
    3x healing salve, first aid kit, warm sweater, 5+D10 copper.
\end{verbatim} \end{samepage} \normalsize

\

\noindent Make up more suitable starting kits to suit your adventures and campaign setting.


\subsection*{rolling a character, or selecting one}
%--------------------------------------------------
For most campaigns I'll keep using the method of pre-rolling a set of potential Heroes for the player to choose from. The simple \verb|rollchars.py| script rolls up sets of basic characters. It's easy to tweak the selection set of races for a specific campaign or adventure setting.
For one-off or later added Heroes the player can instead roll a small set and choose a character from there.


\subsection*{Alternative movement rolls}
%---------------------------------------
Instead of taking the base line race movements you can roll the character's movement stats. You can either roll one d10 and apply the result to calculate all the movements, or you can roll for each movement stat.


\subsection*{Race traits}
%------------------------
Consider, what's the main focus of the game for yourself and your group? Is it tactical tabletop battles, more role playing and character identity, or just silly fun? Tailor the character creation process to support the goal of the game.

\

If ignoring combat power balance for the moment, different races have typical personality characteristics associated with them. Not that all dwarves are the same, just that there are clear cultural and racial tendencies. Change to suit your flavour of game world of course.

\begin{description}
\item[Humans] are the normal folk of most of the land. Found just about anywhere and in all occupations. Individually and culturally highly varied.
\item[Dwarves] are rough, stern, determined, disciplined, with a lovely sense of humour that few outsiders understand. They have strong collective responsibility.
\item[Elves] few and far between, artistic, aloof, or austere, all in all quite a bit strange. 
\item[Halflings] are usually happy with a great apetite for life, merriment, good food and drink.
\item[Orcs] have grit, nothing is too heavy or too dangerous. But don't get them angry.
\item[Goblins] have difficulties: lazy, impulsive, short sighted, wildly swingin between agressive and meek, cowardly and blood thirsty, usually hungry, and with a nasty inkling towards schadenfreude and dangerous ill thought through pranks. Loads of fun to play!
\end{description}

\noindent All races have their own general strengths and weaknesses. Some make better fighters, others better mages, yet others make great cannon fodder and compost. Some are just for fun.
There is no strict balance between how \emph{good} or \emph{powerful} the various character races are when making a new Hero. Instead I've chosen to apply some balance based on the number of characters that are rolled for selection. I recommend the following spread:

\

When choosing race before rolling, roll a small set of one race only:\\
3 human, 2 dwarf, 2 elf, 3 halfling, 2 orc, 4 goblin.\\
The characters created this way will be less powerful than the ones rolled as a larger set of any race.

\

When choosing one character regardless of race, roll:\\
2 human, 1 dwarf, 1 elf, 2 halfling, 1 orc, 3 goblin.\\
Odds are high that this will yield some overpowered newbie characters, but it will give a fun mix of races.

\

For my campaigns I've started using the second option and simply use \verb|rollchars.py| to roll small set of characters, of all races, for each player to choose a character from.

\

\emph{Please note:} that the values below are for rolling Heroes and important NPCs. Every Tom, Dick and Harry don't have cool rolled stats. For most NPCs and general uninteresting cannon fodder and back drop characters use the standard race values, and perhaps +/- a few points here and there for variation. The npc standard race averages are listed further below, and in the \verb|races| summary compilation.




%                 d10/10 1-9                 10     0,1  9,1
%                 d10/9  1-8               9-10     0,1  8,2
%                 d10/8  1-7               8-10     0,1  7,3
%                 d10/7  1-6               7-10     0,1  6,4
%                 d10/5  1-4      5-9        10     0-2  4,5,1
% below are the symmetric splits:
%                 d10/6  1-5               6-10     0,1  5-5
%                 d10/4  1-3      4-7      8-10     0-2  3-4-3
%                 d10/3  1-2   3-5   6-8   9-10     0-3  2-3-3-2
%                 d10/2  1  2-3 4-5 6-7 8-9  10     0-5  1-2-2-2-2-1




\

Here is the basic character sheet layout. It has already been explained above, in the \hyperref[sec:charsheet]{character sheet section}, page \pageref{sec:charsheet}.

\

\goodbreak
\begin{samepage} \begin{verbatim}
===================================
name                        (token)
-----------------------------------
str          hp abs
dex          m w r d
con          stamina
int          vision arc
psy          mana
per          ap
cha          xp
----------
skills
----------
spells
----------
equipment
money ( g s c )
===================================
\end{verbatim} \end{samepage}

\

\goodbreak 
\begin{samepage}
\noindent The basic stat acronyms:
\begin{verbatim}
basic character traits
----------------------
(long)        (eng) (old Swedish)
strength       str   sty
dexterity      dex   smi
constitution   con   fys
intelligence   int   int
psyche         psy   psy
perception     per   itf (attention att)
charisma       cha   kar/per

hitpoints      hp    kp
absorption     abs   abs
movement (maneuver walk run dash)   m/w/r/d
stamina        sta
vision         vis (distance) arc (angle)
magic power    mana
action points  ap
experience     xp

money          gold silver copper (g s c)
\end{verbatim} \end{samepage}

\


% =============================================================================
\TODO The race Hero roll listings below are not updated yet! Refer to, and use \verb|rollchars.py| as up to date. And see \verb|races| listing for summary.
% -----------------------------------------------------------------------------

\


\small

\goodbreak \begin{samepage} \begin{verbatim}
===================================
human                       (token)
-----------------------------------
str  2d5     hp 2d10 abs 0         (alt hp 2d6+8 for heroes [10-20])
dex  2d5     m1 w3 r6 d9           (alt sta 1d7+3 for heroes [4-10])
con  2d5     stamina 2d5           (alt movements (all round down))
int  2d5     vision 15+1d10         m = 1+d10/10
psy  2d5     mana 2d10              w = 2+d10/4
per  2d5     cap 7+1d6              r = 5+d10/4
cha  2d5     xp 100+1d20            d = 8+d10/3
vision arc 180+1d90 deg
yield bonus is 2+d10/4 round down
money 1d4 gold 1d8 silver 1d20 copper
\end{verbatim} \end{samepage}

\

\goodbreak \begin{samepage} \begin{verbatim}
===================================
dwarf                       (token)
-----------------------------------
str  2d5+2   hp 2d10+2 abs 0       (alt hp 2d6+10 for heroes [12-22])
dex  2d5-1   m1 w2 r4 d6           (alt movement (all round down))
con  2d5+2   stamina 2d5+5          m = 1+d10/8
int  2d5     vision 15+1d5 infra*   w = 2+d10/6
psy  2d5     mana 2d10              r = 3+d10/4
per  2d5     cap 8+1d4              d = 5+d10/4
cha  2d5-1   xp 110+1d20
vision arc 120+1d90 deg
yield bonus is 2+d10/4 round down
con+3 against poisons
money 1d10 gold 1d20 silver 1d20 copper + gem stone (5+1d6 gold)
* infra vision is dusk vision until good implementation exist in maptools
haggle bonus +3
dwarves without any gems, or with less than 5 gold total coin
suffers psy-1 mod until wealthy again
dwarves with 50+ gold in coin and gems gains psy+1 mod while wealthy.
\end{verbatim} \end{samepage}

\

\goodbreak \begin{samepage} \begin{verbatim}
===================================
elf                         (token)
-----------------------------------
str  2d5-2   hp 2d9 abs 0          (alt hp 2d6+6 for heroes [8-18])
dex  2d5+1   m2 w4 r8 d12          (alt movement (all round down))
con  2d5     stamina 1d5+7          m = 2+d10/8
int  2d5+1   vision 20+1d10 night   w = 3+d10/4
psy  2d5     mana 2d10+4            r = 7+d10/4
per  2d5+2   cap 9+1d4              d = 11+d10/4
cha  2d5+2   xp 120+1d20
vision arc 220+1d90 deg
yield bonus is 2+d10/3 round down
immune to poisons
money, what for?, Well I have 1d4 silver and 1d10 copper somewhere.
haggle bonus -3
elves who are staying in a city or cave without access to nature
suffer psy-1 mod per week to max -3. Immediately restored to mod-0
when returning to nature.
\end{verbatim} \end{samepage}

\

\goodbreak \begin{samepage} \begin{verbatim}
===================================
halfling                    (token)
-----------------------------------
str  2d5-3   hp 2d8 abs 0          (alt hp 2d5+6 for heroes [8-16])
dex  2d5+2   m2 w3 r6 d8           (alt sta 1d6+2 for heroes [3-8])
con  2d5-1   stamina 1d5+2         (alt movement (all round down))
int  2d5     vision 15+1d10         m = 2+d10/10
psy  2d5     mana 2d10              w = 2+d10/4
per  2d5+2   cap 8+1d6              r = 4+d10/3
cha  2d5+1   xp 100+1d20            d = 6+d10/3
vision arc 270+1d90 deg
yield bonus is 2+d10/3 round down
sneak bonus +3
find bonus +3
money 1d4 gold 1d8 silver 1d20 copper
\end{verbatim} \end{samepage}

\

\goodbreak \begin{samepage} \begin{verbatim}
===================================
orc                         (token)
-----------------------------------
str  2d5+3   hp 2d10+5 abs 0       (alt 2d8+9 for heroes [11-25])
dex  2d5     m2 w4 r6 d8           (alt movements (all round down))
con  2d5+3   stamina 2d5+5          m = 1+d10/4
int  2d5-2   vision 10+1d10 dusk    w = 3+d10/4
psy  2d5-2   mana 2d10-5            r = 5+d10/4
per  2d5     cap 7+1d5              d = 6+d10/3
cha  2d5-3   xp 80+d20
vision arc 120+1d60 deg
yield bonus is 2+d10/4 round down
veteran bonus +3
brawling bonus +3
attack: bite 1d4 dam 2+d2 4ap scf 0.75
double con against poisons
orcs without any war trophies suffer psy-1 mod.
money 1d6 silver 1d10 copper 1d4 large teeth/claws
\end{verbatim} \end{samepage}

\

\goodbreak \begin{samepage} \begin{verbatim}
===================================
goblin                      (token)
-----------------------------------
str  2d4-2   hp 2d6 abs 0          (alt 2d5+4 for heroes [6-14])
dex  2d5     m1 w3 r5 d7           (alt movement (all round down))
con  2d3     stamina 2d5-2          m = 1+d10/10
int  2d4     vision 15+1d10 dusk    w = 2+d10/4
psy  2d4     mana 2d8-4             r = 3+d10/3
per  2d5     cap 6+1d6              d = 5+d10/3
cha  2d4-2   xp 60+1d20
vision arc 180+1d90 deg
yield bonus is 2+d10/3 round down
sneak bonus +3
hide bonus +3
disengage bonus +3
attack: bite 1d4 dam 1d3 3ap (2ap if both hands free) scf 0.75
attack: claw 1d4 dam 1 2ap (1ap if both hands free) scf 0.66
money 1 silver 1d6 copper 1d4 teeth 1d3 stones 1d2 feathers 1d3 glass beads
\end{verbatim} \end{samepage}

\

\goodbreak \begin{samepage} \begin{verbatim}
Goblin runts should be crated from regular goblin rolls:
str/2, dex+1, con/2,
hp/2, ap+1, xp-10,
yield+1, avoid+2, sneak+2,
common-2, svartlingo-1
round up for all half values.

As the runt grows up to be a grown goblin he will snap into the full 
rolled values and loose his runt bonuses.
\end{verbatim} \end{samepage}

\normalsize





%===============================================================================
%E X A M P L E   N E W B I E   C H A R A C T E R S
%-------------------------------------------------

\subsection*{Example characters}
%-------------------------------

A set of example newbie characters can be found in the 
\hyperref[cpt:characters]{characters section}, 
page \pageref{cpt:characters}, 
amongst those: \\
 \\
Morten Flaff, an all round fighter neophyte. \\
MistaMuhda and GrabbaKill, two violent young goblins. \\
Parry Hotter the Fizzler, a young wizard's apprentice. \\
KrijgRauch EckStein, a young dwarf fighter fresh out of the mountains. \\
Ein von Dääken, a noble fop who would probably do better writing sonnets. \\
Pyttelina, a halfling thief ready to take (on) the world. \\
Lars LongShot, a young hunter who is nowadays hunting for gold instead of geese. \\
Newton the Happy Farmboy, a poor farm hand turned wannabe fighter. \\

\

\todo update and clean up the various remaining example characters

\

\noindent
Manny the Mage Trainee, another spellcaster in the making. \\
 \\
A few examples of more experienced characters can also be found after the newbie characters.










%===============================================================================
% A D V E N T U R E S   A N D   C A M P A I G N   G A M I N G
%------------------------------------------------------------

\phantomsection\addcontentsline{toc}{section}{adventures and campaigns}
\section*{Adventures and campaign gaming}
%----------------------------------------
This is of course the main purpose of the exercise, to be able to play interesting adventures and string them together into longer running campaigns. Several adventures were available for download from the rptools / maptool gallery when that service was still up. I'll see if I can put them together again and push them somewhere. In the meantime all adventures (in whatever state of completion or decrepitude) are available from the author.

This project is designed mainly for short evening sessions, two to three hours of gaming on weekday evenings. The guidelines below reflect our experience from this kind of game scheduling.

Balancing character progression has been shown to be difficult for many game designers and is often poorly done in many well known adventure and table top games. By now we have over 150 game sessions under the belt, playing around 25 different characters varying between 100-1000xp. The general quadratic cost of skill progression has been shown to work well for giving balanced mid and late game play. We have not yet played much with characters over 1000xp, so there is not enough data for very experienced heroes.

A good general idea is to try to design adventures and encounters to give a reward of around 5-10xp per character per session with another 10-30xp bonus for mission completion or for specific side tasks. This has been shown to give a suitable progression curve as well as the occasional nice payday feeling for the players after special accomplishments.
Most adventures are designed to take three to six sessions, i.e. 10-20h to play. The longer adventures should give partial completion rewards in between.

Differentiate between "opposition xp" and "mission xp". Give opposition xp after each session, but only mission xp when specific tasks have been completed or milestones reached.
Opposition xp should be divided between the characters and henchmen. If they are many, then they get less per person. Mission xp on the other hand should be handed out at the full amount to each participant.
Consider giving full or partial opposition xp also for monsters that were avoided, scared away, tricked, negotiated with, or otherwise managed even if they were not killed.

It could also be a good idea to set a minimum base line of ca 5xp or so per session regardless of accomplishments. That way there is always a slow progression ticking away, but nothing game breaking. I generally reduce the mission xp by the already handed out session xp.

XP is of course not the only reward. Money and equipment is also common. It is usually a good idea to keep the heroes poor most of the time with short bursts where they can roll in gold. A Quaff or mana potion should always be an expensive investment, and not just stuff you buy to pad the equipment list to the next encumbrance limit.

Silver coins close to XP reward is a decent balance for everyday adventuring, with more lavish pay-outs from some adventures. A lot of "civilised" monsters carry some coin, and their equipment is usually worth a little bit even if it is in really poor shape or of orcish origin. This should be taken into account when calculating rewards if the players are known to scavenge.

Be very careful when handing out magical or master crafted items. Those are worth a lot of money, and they are very powerful in game. Even a well crafted sword with mod+1 is a significant boost for a normal character, and that bonus is worth a lot of xp in the hands of an experienced character. A high abs sword or shield is in practice close to unbreakable and will never need replacing. Saves a lot of money.
I once did the \emph{huge} mistake of placing a (fast+1) 2ap 2h sword in a campaign. I'll never do that again :(


\subsection*{Death, Replacement, Inheritance}
%--------------------------------------------
Heroes will die occasionally, it's an occupational hazard. When that happens let the player roll a new one and have it inherit some of the XP from it's predecessor. E.g: First Fred dies after gaining 50xp from the first few adventures. Second Sam can then be built with +30xp inherited? Discuss it with the players to balance the loss against keeping pace with the rest of the party. Whatever works with the campaign difficulty scaling and keeps the game fun.

Generally, keep the party somewhat in power parity. Weak newbie characters cannot do much if surrounded by 1k ultra murder machines, and it quickly gets boring for the players.


\subsection*{Available campaings and adventures}
%-----------------------------------------------
%200204:
There are a bunch of adventures and campaigns already available, in various states of polish. Just grab and play. Contact the author if you want the notes from the yet not released adventures.

\begin{description}
    \item[00 Dungeon of Testing] is the first little mini adventure every new player should go through with their first character, solo.
    \item[00 Return of Uchly Namen] is the first campaign in the game and world setting. Intended for newbie characters and containing a bunch of diverse interconnected adventures through a cohesive campaign arc with a few off-shoots.
    \item[01 Dark Clan, Deep Cave, Rescue Nurensachs] are three linked adventures forming a mini campaign of it's own, intended for more experienced characters.
    \item[02a Overlord Orvar] is a stand alone adventure, somewhat classical in nature. Easy to insert as a significant side quest in an existing campaign.
    \item[02b Edwin the Chromophobe] is a stand alone adventure, a large dungeon hack with impact on the regional world map. 
    \item[03 Ottokar's Test Dungeon lvl 1] is for experienced Heroes who want to get a proper license for taking Hero Adventure jobs.
    \item[04 Destroy the Altar] is a Hero Adventurers' job for experienced characters and players. It small but has some challenging oddities.
    \item[05 Turn off the Light] is another Hero Adventurers' job for very experienced and tough Heroes and skilled players. The starting challenge is tactically very different from the previous experiences.
    \item[06 Goblin Destiny] is a long stand alone campaign for a large bunch of newbie goblins Heroes. More fun than serious, with unique tactical and role playing challenges. A large set of smaller adventures and events, spread over time, around the life of a goblin bandit clan.
\end{description}




% Here the basic chapters are finished,
% the rest are appendices, switch fill layout to raggedbottom,
% since the appendices have mostly large unbreakable blocks.
\appendix
\raggedbottom

% Add all appendices
%--------|---------|---------|---------|---------|---------|---------|---------|
%       10        20        30        40        50        60        70        80


\cleardoublepage

\phantomsection\addcontentsline{toc}{chapter}{equipment}
\chapter*{Equipment}
\chaptermark{equipment}



%-------------------------------------------------------------------------------
% W E A P O N S  &  A R M O U R
%------------------------------

\phantomsection\addcontentsline{toc}{section}{weapons and armour}
\section*{Weapons and armour}

Weapon damage stats are supposed to be read as 1dX. E.g: the sword does 1d6 damage not always 6. Weapon penetration is however always the max value and is not rolled. E.g: dam 5 pen 2 always does 1D5 damage but ignores up to 2 armour absorption.

The weapons listed below are the common weapon types and sizes. Custom made weapons are treated further down.

%TODO: explain the weapon stats:
%dam
%abs
%reach
%finesse
%mod,attack,defend,parry,todefend,toparry,toavoid
%str, str bonus cap
%maneuvers, poke, swing, etc
%stamina free attacks

% finesse is the limit of how extra difficult the attacks can be made for the target to defend agaist, using the fancy attacks skill or tricky attack maneuvers. E.g: Fancy Fred has fancy attacks 5 but swinging an axe which has finesse-3 and toparry-1 toavoid+1. FF cannot make the attacks more difficult than toparry-4 and toavoid-2, since the weapon finesse is the limiting factor. With a proper sword the FF could use his whole fancy attacks 5 giving his attacks a todefend-5 difficulty.

% typical: relative base, 1h/2h  dam , extras
% sword    +2   +2/+4                             finesse 6
% axe      +3   +3/+6                             parry-3 toparry-1 toavoid+1
% spear    +1   +1/+2 pen 0/1 parry -1/+1         reach
% staff    -1   -1/+1 parry +1/+2                 fast, reach
% braw     -3                                     (at str 6)
%   fist   -4                                     fast, deflect
%   kick   -2                                     deflect
%
% bow      =0   --/=0                             range, pen, 1/r
% crossbow +3   --/++ pen 1-3                     range, pen, 1/2r


\subsection*{Melee weapons}
%melee weapons
%-------------
\small
\begin{verbatim}
Regular Swords
attack maneuver poke costs 5xp
attack maneuver swing costs 5xp

small sword         dam 4, abs 6,
                    str 2 (max +1 str bonus), finesse-6
                    poke: mod-1 dam-1 pen+1
                    swing: slow-1 mod-1 dam+1 todefend+2

sword               dam 6, abs 10,
                    str 4 (max +2 str bonus), finesse-6
                    poke: mod-1 dam-1 pen+1
                    swing: slow-1 mod-1 dam+2 todefend+2

large sword         dam 8, abs 14,
                    str 6 (max +3 str bonus), finesse-6
                    poke: mod-1 dam-1 pen+1
                    swing: slow-1 mod-1 dam+2 todefend+2

small 2h sword      dam 8, abs 12,
                    str 4 (max +2 str bonus), finesse-6
                    reach 1 mod-7
                    poke: mod-1 dam-1 pen+1
                    swing: slow-1 mod-1 dam+2 todefend+2

2h sword            dam 10, abs 15,
                    str 6 (max +3 str bonus), finesse-6
                    reach 1 mod-6
                    poke: mod-1 dam-1 pen+1
                    swing: slow-1 mod-1 dam+2 todefend+2

large 2h sword      dam 12, abs 18,
                    str 8 (max +4 str bonus), finesse-6
                    reach 1 mod-5
                    poke: mod-1 dam-1 pen+1
                    swing: slow-1 mod-1 dam+3 todefend+2


\end{verbatim} \pagebreak[1] \begin{verbatim}
Infantry Swords

short sword         dam 5, abs 10,
                    str 3 (max +1 str bonus), finesse-4
                    poke: mod-1 dam-1 pen+1
                    swing: slow-1 mod-1 dam+1 todefend+2

broad sword         dam 7, abs 14,
                    str 5 (max +2 str bonus), finesse-4
                    poke: mod-1 dam-1 pen+1
                    swing: slow-1 mod-1 dam+2 todefend+2

long sword          dam 9, abs 18,
                    str 7 (max +3 str bonus), finesse-4
                    poke: mod-1 dam-1 pen+1
                    swing: slow-1 mod-1 dam+3 todefend+2


\end{verbatim} \pagebreak[1] \begin{verbatim}
Knives and fast blades

knife               dam 2, abs 3, parry-2, toparry-2, toavoid-1, finesse-9
                    str 1 (no str bonus)
                    fast+1 if str 2 and dex 4
                    first two attacks don't require stamina
                    poke: mod-1 dam-1 pen+1

dagger              dam 3, abs 4, parry-1, toparry-2, toavoid-1, finesse-9
                    str 2 (no str bonus)
                    fast+1 if str 3 and dex 5
                    first two attacks don't require stamina
                    poke: mod-1 dam-1 pen+1

rapier              dam 4, abs 4, toparry-2, toavoid-1, finesse-9
                    str 3 (no str bonus)
                    fast+1 if str 4 and dex 6
                    first two attacks don't require stamina
                    poke: mod-1 dam-1 pen+1

large rapier        dam 5, abs 5, toparry-2, toavoid-1, finesse-9
                    str 4 (no str bonus)
                    fast+1 if str 5 and dex 7
                    first two attacks don't require stamina
                    poke: mod-1 dam-1 pen+1


\end{verbatim} \pagebreak[1] \begin{verbatim}
Speciality Blades   costs 5xp to learn speciality, or suffer mod-1

duelling sword      dam 6, abs 6, toparry-2, toavoid-1, finesse-9
                    str 5, (no str bonus)
                    fast+1 if str 6 and dex 8
                    first two attacks don't require stamina
                    poke: mod-1 dam-1 pen+1
                    swing: slow-1 mod-1 dam+1 todefend+2

large duell. sword  dam 7, abs 7, toparry-2, toavoid-1, finesse-9
                    str 6, (no str bonus)
                    fast+1 if str 7 and dex 9
                    first two attacks don't require stamina
                    poke: mod-1 dam-1 pen+1
                    swing: slow-1 mod-1 dam+1 todefend+2

parrying dagger     dam 3, abs 8, attack-1, parry+1, finesse-6
                    str 2 (no str bonus)
                    fast+1 if str 3 and dex 5
                    poke: mod-1 dam-1 pen+1

baselard            dam 2, pen 1, abs 3, parry-3, toparry-1, finesse-6
                    str 2 (no str bonus)
                    fast+1 if str 3 and dex 5
                    first two attacks don't require stamina
                    poke: mod-1 dam=1 pen+2

bastard sword       dam 7/9, abs 12, finesse-6
(1h/2h)             str 5/5
                    poke: mod-1 dam-1 pen+1
                    swing: slow-1 mod-1 dam+2 todefend+2

great sword         dam 10, abs 20, toavoid+1, finesse-4
1h                  str 6 slow-1, str 9 normal speed
                    poke: mod-1 dam-1 pen+1
                    swing: slow-1 mod-1 dam+3 todefend+2

claymore            dam 14, abs 28, toavoid+1
2h                  str 8 slow-1, str 11 normal speed, finesse-4
                    reach 1 mod-6
                    poke: mod-1 dam-1 pen+1
                    swing: slow-1 mod-1 dam+3 todefend+2

estoc               dam 7, abs 8, toparry-1, toavoid-1, finesse-9
2h                  str 5 (no str damage bonus, max +1 penetrating bonus)
                    fast+1 if str 6 and dex 8
                    first two attacks don't require stamina
                    poke: mod-1 dam-2 pen+2
                    swing: slow-1 mod-1 dam+2 todefend+2

grande estoc        dam 9, abs 10, toparry-1, toavoid-1, finesse-9
2h                  str 7 (no str damage bonus, max +1 penetrating bonus)
                    fast+1 if str 8 and dex 10
                    first two attacks don't require stamina
                    poke: mod-1 dam-2 pen+2
                    swing: slow-1 mod-1 dam+2 todefend+2




\end{verbatim} \pagebreak[3] \begin{verbatim}
Regular Axes

small 1h axe        dam 5, abs 6, parry-3, toparry-1, toavoid+1, finesse-3
                    str 2 (max +2 str bonus)
                    swing: slow-1 mod-1 dam+2 toavoid+2

1h axe              dam 7, abs 8, parry-3, toparry-1, toavoid+1, finesse-3
                    str 4 (max +4 str bonus)
                    swing: slow-1 mod-1 dam+2 toavoid+2

large 1h axe        dam 9, abs 10, parry-3, toparry-1, toavoid+1, finesse-3
                    str 6 (max +6 str bonus)
                    swing: slow-1 mod-1 dam+3 toavoid+2

small 2h axe        dam 10, abs 9, parry-3, toparry-1, toavoid+1, finesse-3
                    str 4
                    swing: slow-1 mod-1 dam+3 toavoid+2

2h axe              dam 12, abs 11, parry-3, toparry-1, toavoid+1, finesse-3
                    str 6
                    swing: slow-1 mod-1 dam+3 toavoid+2

large 2h axe        dam 14, abs 13, parry-3, toparry-1, toavoid+1, finesse-3
                    str 8
                    swing: slow-1 mod-1 dam+4 toavoid+2


\end{verbatim} \pagebreak[1] \begin{verbatim}
Speciality Axes     costs 5xp to learn speciality, or suffer mod-1

battle axe          dam 8, abs 10, parry-2, toparry-2, finesse-4
1h                  str 5
                    swing: slow-1 mod-1 dam+2 toavoid+2
                    poke: mod-2 dam-3 pen+2

great axe           dam 12, abs 14, parry-4, toparry-3, toavoid+2, finesse-2
1h                  str 6 slow-1
                    str 9 normal speed
                    swing: slow-1 mod-1 dam+4 toavoid+2

lochaber            dam 16, abs 20, parry-4, toparry-3, toavoid+2, finesse-2
2h                  str 8 slow-1
                    str 11 normal speed
                    swing: slow-1 mod-1 dam+4 toavoid+2

pick axe            dam 4, pen 4, parry-3, toavoid+2, finesse-2
2h                  str 4, slow-1
                    double damage against stone, doors, etc
                    swing: slow-1 mod-1 dam+1, pen+1, toavoid+2

heavy pick axe      dam 6, pen 4, parry-3, toavoid+2, finesse-2
2h                  str 6, slow-1
                    double damage against stone, doors, etc
                    swing: slow-1 mod-1 dam+1, pen+1, toavoid+2



% staves plural of staff ?
\end{verbatim} \pagebreak[3] \begin{verbatim}
Staffs and Spears
spear slash maneuver costs 5xp and does slashing instead of piercing damage, 
but with dam-1 modifier.

Spears, pikes and halberds with enough reach can be used to attack over 
the shoulder or shield of a friend in a two layer formation. This requires 
phalanx of both characters. A three layer formation is possible if the 
first front line is kneeling or formed of shorter characters.

light staff         dam 2/3, abs 6, parry+1/+2, reach 1 mod-3, finesse-3/-5
1h/2h               str 3/2 (max +1/+2 str bonus)
                    2h: fast+1 if str 4 and dex 4

staff               dam 4/5, abs 8, parry+1/+2, reach 1 mod-3, finesse-3/-5
1h/2h               str 5/4 (max +1/+2 str bonus)
                    2h: fast+1 if str 6 and dex 6

heavy staff         dam 6/7, abs 10, parry+1/+2, reach 1 mod-3, finesse-3/-5
1h/2h               str 7/6 (max +1/+2 str bonus)
                    2h: fast+1 if str 8 and dex 8

light spear         dam 3/4, pen 0/1, abs 4, parry-1/+1, reach 1 mod-3
1h/2h               str 2/1 (str bonus max +1 dam then max +1 pen)
                    narrow tip spears are dam-1 pen+1 mod-1
                    narrow tip spears can have all str bonus as penetrating
                    broad tip spears are dam+1 mod-1
                    slash: mod-2, dam-1
                    finesse-4/-6

spear               dam 5/6, pen 0/1, abs 6, parry-1/+1, reach 1 mod-3
1h/2h               str 4/3 (str bonus max +1 dam then max +2 pen)
                    narrow tip spears are dam-1 pen+1 mod-1
                    narrow tip spears can have all str bonus as penetrating
                    broad tip spears are dam+1 mod-1
                    slash: mod-2, dam-1
                    finesse-4/-6

heavy spear         dam 7/8, pen 0/1, abs 8, parry-1/+1, reach 1 mod-3
1h/2h               str 6/5 (str bonus max +1 dam then max +3 pen)
                    narrow tip spears are dam-1 pen+1 mod-1
                    narrow tip spears can have all str bonus as penetrating
                    broad tip spears are dam+1 pen-1 mod-1
                    slash: mod-2, dam-1
                    finesse-4/-6

long spear          dam 5, pen 1, abs 7, parry-3, reach 1 mod-0
2h                  str 4  (str bonus max +1 dam then max +2 pen)
                    narrow tip spears are dam-1 pen+1 mod-1
                    narrow tip spears can have all str bonus as penetrating
                    broad tip spears are dam+1 pen-1 mod-1
                    finesse-3

heavy long spear    dam 7, pen 1, abs 9, parry-3, reach 1 mod-0
2h                  str 6 (str bonus max +1 dam then max +3 pen)
                    narrow tip spears are dam-1 pen+1 mod-1
                    narrow tip spears can have all str bonus as penetrating
                    broad tip spears are dam+1 pen-1 mod-1
                    finesse-3

infantry spear      dam 6/7, pen 0/1, abs 10, parry-1/+1, reach 1 mod-3
1h/2h               str 5 (str bonus max +1 dam then max +3 pen)
                    slash: mod-2, dam-1
                    finesse-3/-5


\end{verbatim} \pagebreak[1] \begin{verbatim}
Speciality Pole Arms  costs 5xp to learn speciality, or suffer mod-1

pike                dam 6, pen 1, abs 8, parry-6, slow-1
2h                  reach 0 mod-3, reach 1 mod-0, reach 2 mod-3
                    str 5 (max dam+2 str bonus, rest is penetrating)
                    narrow tip spears can have all str bonus as penetrating
                    broad tip spears are dam+1 mod-1
                    finesse-2

heavy pike          dam 8, pen 1, abs 10, parry-6, slow-1
2h                  reach 0 mod-3, reach 1 mod-0, reach 2 mod-3
                    str 7 (max dam+3 str bonus, rest is penetrating)
                    narrow tip spears can have all str bonus as penetrating
                    broad tip spears are dam+1 mod-1
                    finesse-2

halberd             dam 10, pen 3, abs 8, parry-6, slow-2, toavoid+2
2h                  reach 0 mod-2, reach 1 mod-0, reach 2 mod-5,
                    str 5
                    finesse-2
                    poke: mod-0 dam-3 pen+2
                    Halberds can be used either with axe or spear skill.

glave               dam 9, pen 2, abs 8, parry-3, slow-1, toavoid+1
2h                  reach 0 mod-1, reach 1 mod-0, reach 2 mod-6,
                    str 5
                    finesse-4
                    Glaves can be used either with sword or spear skill.



\end{verbatim} \pagebreak[3] \begin{verbatim}
Clubs, Hammers, Flails
attack maneuver swing costs 5xp

small club          dam 2/4, abs 6, parry-1, toavoid+1, finesse-2
1h/2h               str 2 (max +2/+3 str bonus)
                    1h: str vs str for knockback 1
                    2h: str+3 vs str for knockback 1
                    swing: slow-1 mod-1 dam+2 toavoid+2 knockback roll +3

club                dam 4/6, abs 10, parry-1, toavoid+1, finesse-2
1h/2h               str 4 (max +3/+4 str bonus)
                    1h: str vs str for knockback 1
                    2h: str+3 vs str for knockback 1
                    swing: slow-1 mod-1 dam+2 toavoid+2 knockback roll +3

large club          dam 6/8, abs 14, parry-1, toavoid+1, finesse-2
1h/2h               str 6 (max +3/+4 str bonus)
                    1h: str vs str for knockback 1
                    2h: str+3 vs str for knockback 1
                    swing: slow-1 mod-1 dam+2 toavoid+2 knockback roll +3

very large club     dam 8/10, abs 18, parry-1, toavoid+1, finesse-2
1h/2h               str 8  (max +3/+4 str bonus)
                    1h: str vs str for knockback 1
                    2h: str+3 vs str for knockback 1
                    swing: slow-1 mod-1 dam+3 toavoid+2 knockback roll +3

monstrous club      dam 10/12, abs 22, parry-1, toavoid+1, finesse-2
1h/2h               str 10  (max +3/+4 str bonus)
                    1h: str vs str for knockback 1
                    2h: str+3 vs str for knockback 1
                    swing: slow-1 mod-1 dam+3 toavoid+2 knockback roll +3

heavy clubs         dam+2, abs+50%, parry-3, toavoid+2, finesse-1, slow-1
                    knockback roll +3


\end{verbatim} \pagebreak[1] \begin{verbatim}
1h hammer           dam 6, abs 6, parry-3, toavoid+1, finesse-2, slow-1
                    str 4 (max +4 str bonus)
                    str+3 vs str for knockback 1
                    swing: slow-1 mod-1 dam+2 toavoid+2 knockback+1 (on success)

large 1h hammer     dam 8, abs 8, parry-3, toavoid+1, finesse-2, slow-1
                    str 6 (max +6 str bonus)
                    str+3 vs str for knockback 1
                    swing: slow-1 mod-1 dam+2 toavoid+2 knockback+1 (on success)

2h hammer           dam 8, abs 16, parry-3, toavoid+1, finesse-2, slow-1
                    str 6
                    knockback 1
                    swing: slow-1 mod-1 dam+3 toavoid+2 knockback+1

large 2h hammer     dam 10, abs 20, parry-3, toavoid+1, finesse-2, slow-1
                    str 8
                    knockback 1
                    swing: slow-1 mod-1 dam+3 toavoid+2 knockback+1


\end{verbatim} \pagebreak[1] \begin{verbatim}
1h flail            dam 4, abs 8, parry-6, toparry-3, finesse-6
                    str 4 (max +2 str bonus), slow-1
                    reach 1 mod-3, snag-3
                    swing: slow-1 mod-1 dam+1 toavoid+2

large 1h flail      dam 6, abs 12, parry-6, toparry-3, finesse-6
                    str 6 (max +4 str bonus), slow-1
                    reach 1 mod-3, snag-3
                    swing: slow-1 mod-1 dam+2 toavoid+2

2h flail            dam 8, abs 16, parry-6, toparry-3, finesse-6
                    str 6, slow-1
                    reach 1 mod-3, snag-3
                    swing: slow-1 mod-1 dam+3 toavoid+2

large 2h flail      dam 10, abs 20, parry-6, toparry-3, finesse-6
                    str 8, slow-1
                    reach 1 mod-3, snag-3
                    swing: slow-1 mod-1 dam+3 toavoid+2


\end{verbatim} \pagebreak[1] \begin{verbatim}
Speciality Bludgeoning Weapons   costs 5xp to learn speciality, or suffer mod-1

heavy hammer        dam 10, abs 20, parry-6, toavoid+2
1h                  str 7, slow-2
                    knockback 2
                    swing: slow-1 mod-1 dam+2 toavoid+2 knockback+1

maul                dam 14, abs 25, parry-6, toavoid+3
2h                  str 10, slow-2
                    knock back 2
                    slow-3 (6ap): knockback 3 costs one extra stamina
                    swing: slow-1 mod-1 dam+3 toavoid+2 knockback+1

morning star        dam 6, pen 1, abs 6, parry-3, toavoid+1, finesse-3
                    str 5
                    str vs str for knock back 1
                    swing: slow-1 mod-1 dam+2 toavoid+2 knockback roll +3

2h morning star     dam 9, pen 2, abs 9, parry-3, toavoid+1, finesse-3
                    str 7
                    str+3 vs str for knock back 1
                    swing: slow-1 mod-1 dam+3 toavoid+2 knockback roll +3

long chain flail    dam 7, abs 7, parry-6, toparry-6, toavoid-3, finesse-9
                    str 5, slow-2
                    reach 1 mod-0, snag-0
                    swing: slow-1 mod-1 dam+2 toavoid+2

2h long chain flail dam 10, abs 10, parry-6, toparry-6, toavoid-3, finesse-9
                    str 7, slow-2
                    reach 1 mod-0, snag-0
                    swing: slow-1 mod-1 dam+3 toavoid+2

\end{verbatim}
\normalsize






\subsection*{Shields and armour}
%shields and armour
%------------------
\small
\begin{verbatim}
Shields can be carried on one arm, on the back, or on a backpack.

Heavy version of shields have abs+50% but are either
slow-1 or require str+3 to retain normal speed

Plated heavy shields have double abs but are either
slow-2 or require str+3 to be slow-1 or str+6 to be normal speed

buckler             abs 6, parry+1
                    str 0
                    fast+1 if str 4 and dex 4
                    Ranged attacks mod-0 when in the way.
                    Hiding behind it (3ap) ranged mod-1

small shield        abs 8, parry+2,
                    str 2
                    Ranged attacks mod-1 when in the way.
                    Hiding behind it (3ap) ranged mod-2

shield              abs 10, parry+3,
                    str 4
                    Ranged attacks mod-2 when in the way.
                    Hiding behind it (3ap) ranged mod-4
                    tackle mod+1

large shield        abs 12, parry+4,
                    str 6, or slow-1 str 3
                    Ranged attacks mod-3 when in the way.
                    Hiding behind it (3ap) ranged mod-6
                    tackle mod+2

tower shield        abs 14, parry+5,
                    str 8, or slow-1 str 5
                    Ranged attacks mod-4 when in the way.
                    Hiding behind it (3ap) ranged mod-8
                    tackle mod+3

infantry shield     abs 13, parry+3,
                    str 5
                    Ranged attacks mod-2 when in the way.
                    Hiding behind it (3ap) ranged mod-4
                    tackle mod+1




\end{verbatim} \pagebreak[3] \begin{verbatim}
leather armour abs 1
thick cloth    acrobatics mod-1
               takes 2 rounds to put on or take off

chain mail     abs 2 ring, scale, brigantine, etc
scale mail     str 3 (str penalties affect all actions)
brigandine     dex-1,
               dash-1,
               per-1
               acrobatics mod-3
               martial arts mod-1
               spellcasting mod-1
               sneak mod-3
               takes 3 rounds to put on or take off

plate or       abs 3 plate armour of various sorts, incl helmet
heavy scale    str 5 (str penalties affect all actions)
h brigandine   dex-2, yield bonus -1
               run-1, dash-2,
               per-2, vision-20%, cone vision -20% max 270 deg
               acrobatics mod-6, climb-1, jump-1
               martial arts mod-3
               spellcasting mod-2
               sneak mod-4
               avoid mod-1
               max stamina mod-1
               Turning as separate actions cost more ap:
               45deg=0ap, 90deg=1ap, 135deg=2ap, 180deg=3ap
               (hex: 60deg=1ap, 120deg=2ap, 180deg=3ap)
               takes 4 rounds to put on or take off

full plate     abs 4 full plate armour with full helmet
               str 7 (str penalties affect all actions)
               dex-3, yield bonus -2
               run-2, dash-4,
               per-3, vision-30%, cone vision -30% max 180 deg
               acrobatics mod-9, climb-3, jump-3
               martial arts mod-6
               tackle mod+1
               spellcasting mod-3
               sneak mod-5
               avoid mod-2
               max stamina mod-2
               Turning as separate action cost more ap:
               45deg=1ap, 90deg=2ap, 135deg=3ap, 180deg=3ap
               (hex: 60deg=1ap, 120deg=3ap, 180deg=3ap)
               takes 7 rounds to put on or take off

heavy plate    abs 5 very heavy full plate armour with all the trimmings.
               str 9 (str penalties affect all actions)
               dex-4, yield bonus -3
               walk-1, run-3, dash-5,
               per-4, vision-40%, cone vision -40% max 120 deg
               acrobatics mod-12, climb-6, jump-6
               martial arts mod-9
               tackle mod+2
               spellcasting mod-4
               sneak mod-6
               avoid mod-3
               max stamina mod-3
               Turning as separate action cost more ap:
               45deg=1ap, 90deg=2ap, 135deg=3ap, 180deg=4ap
               (hex: 60deg=1ap, 120deg=3ap, 180deg=4ap)
               takes 10 rounds to put on or take off

\end{verbatim}
\normalsize



\subsection*{Ranged weapons}

Are the monsters a little too far away for your sword? Are you just a bit too lazy to walk up and say hi with your huge axe, or perhaps a tad shy? Then why not send a friendly murder greeting with a crossbow or throw them a heart felt javelin?

\

Thrown weapons include anything from knives and javelins to improvised missiles like beer bottles, rocks, and mistuned musical instruments.

Thrown weapons can also be used as poor versions of similar melee weapons in a pinch and with some mods. Regular melee weapons can often also be used as thrown weapons but often with mod-3 to mod-6 and 33\% to 66\% of the normal damage. When a thrown weapon is used as a temporary melee weapon it uses the melee weapon class skill, and when a melee weapon is used as a thrown weapon it uses the throw skill.

\small \begin{verbatim}
Throwing an already held weapon is a regular 3ap action. Drawing and throwing 
in one go is a 1r action. Faster throwing gives mods which can be mitigated by 
quick shot. Manually drawing with quickdraw then throwing as already held might 
be faster if the user is skilled in quickdraw.

Draw and throw:
normal  1r  mod=0
aimed   2r  mod+1
quick   6ap mod-3
snap    3ap mod-6
insane  2ap mod-9

Throwing already held weapon:
normal  3ap mod=0
aimed   6ap mod+1
quick   2ap mod-3
snap    1ap mod-6
insane  0ap mod-9

range for thrown weapons:
short    mod+1
long     mod-3
vlong    mod-6 dam-1
extreme  mod-9 dam-2

throwing knife    dam 2
                  range 6 + str/2
                  str 1 (no str bonus)
                  fast+1 if str 2 and dex 4
                  first two attacks do not require stamina
                  melee: knife mod-1, dam 2, abs 2, fancy-2
                  fast+1 if str 2 and dex 4

throwing axe      dam 4
                  range 4 + str/2
                  str 3 (max +1 str bonus)
                  melee: axe mod-1, dam 5, abs 7, parry-3, toparry-1, fancy-2
                  str 3 (max +1 str bonus)

javelin           dam 3, penetrating 1
                  range 4 + str
                  str 4 (max +1 dam str bonus, max +1 penetrating str bonus)
                  melee: spear mod-1, dam 3, abs 3, no reach or parry mods, fancy-3
                  str 4 (max +1 dam str bonus, max +1 penetrating str bonus)

heavy javelin     dam 5, penetrating 1,
                  range 2 + str
                  str 6 (max +1 dam str bonus, max +2 penetrating str bonus)
                  melee: spear mod-1, dam 5, abs 5, no reach or parry mods, fancy-3
                  str 6 (max +1 dam str bonus, max +2 penetrating str bonus)

dart              dam 0, penetrating 2
                  range 4 + str
                  str 1 (max +1 pen str bonus)

\end{verbatim} \pagebreak[3] \normalsize


Bows, crossbows, and ballistas all use arrows or bolts as ammunition. All rate of fire info below assumes that arrows are available from a readied easy to reach quiver, stuck in the ground, or similar.

If you choose the optional rule that characters need to keep track of their ammunution instead of just assuming "enough arrows" it might be worth investing some xp in arrow recovery.

Bows and crossbows have one less damage reduction on long ranges: Long dam-0, vlong dam-1, extreme dam-2.

\small \begin{verbatim}
normal rate of fire for bows:
this includes reload from accessible quiver.
normal mod=0 2r
quick  mod-3 1r / 9ap
aimed  mod+1 3r

rapid fire for bows:
Quick fire is 1r/9ap at mod-3. For every ap reduced below quick fire the
archer takes another mod-1. E.g: rapid fire bow at 4ap thus takes mod-8. 
Since quick is mod-3 and another mod-5 to reduce 9ap to 4ap.
An archer cannot bring the rapid fire ap cost below 3ap, mod-9.
Rapid fire from bows include reload from accessible quiver, arrows stuck 
in the ground, or similar.

short bow         dam 3,
                  range 10
                  str 3 (no str bonus)

bow               dam 4, pen 1,
                  range 14
                  str 5 (no str bonus)

heavy bow         dam 5, pen 2,
                  range 17
                  str 7 (no str bonus)

longbow           dam 5, pen 1,
                  range 20
                  str 6 (no str bonus)

heavy longbow     dam 6, pen 2,
                  range 25
                  str 8 (no str bonus)

Bows allow the use of different kinds of arrows:
piercing arrows   mod-1 dam-1 pen+1
broadhead arrows  mod-1 dam+1 pen-1
barbed arrows     dam+1
sharp arrows      pen+1
fine arrows       mod+1
fire arrows       mod-2 dam-1 range-25%
                  oil splash dam 3/r for 5r or until doused
                  light source as torch, burns 5r
                  light 1a with fire source
torch arrows      mod-2 dam-2 range-25%
                  light source as torch, burns 10r
                  light 1a with fire source
candle arrows     mod-2 dam-2 range-25%
                  light source as candle, burns 20r
                  light 1a with fire source


\end{verbatim} \pagebreak[1] \begin{verbatim}
Rate of fire for crossbow, heavy crossbows and arbalests:
Split between reload and fire actions.
The first shot with a loaded crossbow is reasonably fast.
Successive shots require reload actions between each shot.
The reload times assume readily available bolts in a 
quiver, stuck in ground, or similar.

first shot  mod=0 1r
first aimed mod+1 2r
first quick mod-3 3ap
first snap  mod-6 2ap
first wtf   mod-9 1ap

crossbow          dam 5, penetrating 1,
                  range 12
                  str 3 (no str bonus)
                  reload 2r (str 9 1r)

heavy crossbow    dam 6, penetrating 2,
                  range 15
                  str 5 (no str bonus)
                  reload 3r (str 12 2r)

arbalest          dam 7, penetrating 3,
                  range 18
                  str 7 (no str bonus)
                  slow-1 (only affects snap first shots)
                  reload 4r (str 15 3r)

exotic            as the regular crossbows except that they have two first
double type       shots, which can be used at the same time or separately.
crossbows         They also require str+2, and take 1r longer to reload.
                  If both shots are fired at the same time (same action)
                  you should still roll each bolt separately.

Some crossbows allow the use of speciality bolts:
fine bolts        mod+1
barbed bolts      dam+1
sharp bolts       pen+1
piercing bolts    mod-1 dam-1 pen+1
broadhead bolts   mod-1 dam+1 pen-1


\end{verbatim} \pagebreak[1] \begin{verbatim}
rate of fire for ballistas:
Ballistas take very long to reload and require at least two people.
snap  reload+1r mod-6
quick reload+2r mod-3
shot  reload+3r mod=0
aimed reload+4r mod+1

light ballista    dam 12, penetrating 4, knockback 1
                  range 24
                  carried: str 20 (no str bonus), mounted: str 3 (no bonus)
                  reload 5r, tot str 10, max 3ppl (only 1 person +1r)

ballista          dam 15, penetrating 5, knockback 2
                  range 30
                  carried: str 30 (no str bonus), mounted: str 5 (no bonus)
                  reload 10r, tot str 15, max 4ppl (only 1 person +3r)

heavy ballista    dam 20, penetrating 6, knockback 3
                  range 40
                  carried: str 40 (no str bonus), mounted: str 7 (no bonus)
                  reload 15r, tot str 20, max 5ppl (only 1 person +5r)

\end{verbatim}
\normalsize








\subsection*{custom weapons}
%---------------------------
It's always possible to have a craftsman build customised weapons that are suited specifically to the user's strength, dexterity, abs requirements, etc. These weapons cost a lot more, usually several times the normal list price, and require a skilled craftsman and sometimes some special materials.

The general guidelines for weapon classes look something like this:\\
\small \begin{verbatim}
1h blade:
    dam str+2, abs 2*dam-2
2h blade:
    dam str+4, abs 2*dam-2
fast blade:
    (3ap:) dam str+1, abs dam+0
    (2ap:) fast+1 if str>=dam+0 dex>=dam+2

1h axe:
    dam str+3, abs dam+1
2h axe:
    dam str+5, abs dam

staff (1h/2h):
    dam str-1/str+1, abs 2h dam+2
spear (1h/2h):
    dam str+0/str+1 and pen 1/2, abs 2h dam

club (1h/2h):
    dam str+0/+2, abs 2*str+2, knockback 1 str-vs-str
\end{verbatim} \normalsize








\subsection*{monster weapons}
%----------------------------
Some races, as NPCs, will sometimes use special versions of weapons which are different than those human craftsmen produce.

Orcs produce simple and heavy weapons. They have higher abs, lower finesse, higher str requirements. The market price is -50\% of human made weapons and their encumbrance is +33\%. When taking damage to abs they loose double abs.
\small \begin{verbatim}
orc war axe         dam 9, abs 12, parry-3, toparry-1, toavoid+1, finesse-2
1h                  str 7 (max +6 str bonus)
                    swing: slow-1 mod-1 dam+3 toavoid+2

orc war spear       dam 7/8, pen 1/2, abs 10, parry-1/+1, reach 1 mod-3
1h/2h               str 7 (max +3 str damage bonus, max+3 penetrating bonus)
                    finesse-3/-5

orc war shield      abs 14, parry+3,
                    str 7, or slow-1 str 4
                    Ranged attacks mod-2 when in the way.
                    Hiding behind it (3ap) ranged mod-5
                    tackle mod+3
\end{verbatim} \normalsize
\pagebreak[1]

Elves produce slender weapons of very high quality. They will last longer and when rolling for weapon breakage they have half the chance of breaking (round down) and their permanent abs damage should be rounded down instead of rounded up. Elven made weapons are very rare and difficult to find outside of elven cities. Their encumbrance is -30\% of the human version, and the price is at least 10x higher than a human made. Start at 5+1d5g + 10x base price.

The elven illspets is used as a fast light defensive weapon which can also be used to attack heavily armoured targets.
\small \begin{verbatim}
elven slender sword dam 8, abs 16,
1h                  str 6 (max +3 str bonus), finesse-9
                    poke: mod-1 dam-1 pen+1
                    swing: slow-1 mod-1 dam+2 todefend+2
                    deflect is mod-1 instead of mod-3
                    str 8 dex 10 is fast+1

elven illspets      dam 4, pen 4, abs 16, parry+1
1h                  str 6 (no str dam bonus, max +2 pen str bonus)
                    poke: mod-1 dam-1 pen+2
                    deflect is mod-1 instead of mod-3
                    str 6 dex 8 is fast+1

elven longbow       dam 5, penetrating 2,
                    range 24, short 12 mod+1, long 36 mod-3,
                    vlong 48 mod-6 dam-1, extreme 70 mod-9 dam-2,
                    str 6 (no str bonus)
\end{verbatim} \normalsize









%-------------------------------------------------------------------------------
% E Q U I P M E N T
%------------------

\phantomsection\addcontentsline{toc}{section}{other equipment}
\section*{Other equipment}


\small
\begin{verbatim}
antidote       counters the effects of a poison, weaker than the antidote's str.
               Very general antidotes you have here...

pain killer    small white pills, red pills, liquids, etc, with different
               strengths and side effects.
               eliminates some pain points equal to strength
               usually takes a few rounds to have effect.

alcohol        5r to take effect.
               reduces pain by -1 per dose,
               adds an inebriation mod-1 per dose above con/3 (round down)
               to future rounds ams, OR counters one dose of coffee.
               Effect remains until next day, or until countered.

coffee         5r to take effect, 2r if hot (use a thermos perhaps?)
               increases current and max stamina by one for each dose,
               OR if inebriated, then cancels all effects of one dose of
               alcohol per dose of coffee.
               After con/3 (round down) doses the character is jittery and must
               pass a psy roll each round he tries to rest.
               Effect remains until next day, or until countered.

health elixir  heals 3hp, 1hp/3r, and draws 3mana and 3stamina immediately
               when quaffed. It has no effect if the mana or stamina draw rolls
               fail.

health wafer   heals 3hp, 1hp/10r, and modifies max stam-1 until the Hero has
               slept.

grappling      mod+6 to climb when attached to a rope and successfully flung
hook           over a wall to get stuck. Roll dex + throw to hook it, 1r action.
               A careful 3r action gives a mod+3 to hook it

rope           mod+6 to climb

lock picks     mod+X to "pick lock" skill.
               X depends on the quality of the picks.

tool box       mod+X to "traps" and "McGyverism" skills.
               X depends on the quantity and quality of the tools.

\end{verbatim} \pagebreak[3]


%-------------------------------------------------------------------------------
% S P E C I A L   E Q U I P M E N T
%----------------------------------

%Special items are not so easy to come by and their prices vary a bit.

\begin{verbatim}
cave cart      A small (1sqr) cart that can be dragged (str 5 walk) into the
               narrow spaces of the caves where monsters and treasure hide.
               It can load 100 enc. Rummaging on the cart is same as in a normal
               container, except effective item number is items/10 round up.

Rucksack with  Regular backpack but with a few compartments, which makes it
compartments   faster when rummaging through it for the required item.
               Divide the amount of items by 2-4 when calculating rummage time.

tunnel plug    A small (1sqr) cart with a very large and sturdy shield that can
               be flipped into position to block up to three square wide so
               pursuing monsters cannot pass or attack through it.
               The cart can be dragged (str 3 walk) into the narrow spaces of
               caves, then anchored to the ground when flipping up.
               Takes 3r to flip up and anchor.
               plug shield: abs 5, 50hp.

tunnel plug    A medium (2x2) cart with a very large and sturdy shield that can
large          be flipped into position to block up to four squares wide so
               pursuing monsters cannot pass or attack through it.
               The cart can be dragged (str 5 walk) into the narrow spaces of
               caves, then anchored to the ground when flipping up.
               Takes 5r to flip up and anchor.
               plug shield abs 6, 100hp.

portadoor      A portable (enc 5.0) door that can be set up in 3r (full round
               actions). It can block one square so that pursuing monsters
               cannot follow or attack through it.
               abs 4, hp 40.

portagate      A portable (3x enc 5.0) gate that can be set up in 5r (full
               round actions). It can block one or two squares so that pursuing
               monsters cannot follow or attack through it.
               abs 5, hp 50

fightlight     A large brazier with a polished metal mirror behind it, creating
               a very bright 90deg cone of light, 40sq radius.
               Just put it on the ground and light it. The expensive models even
               have a cinder box or a spark lever for fast lighting.
               enc 10.0

fightlight     As the larger fightlight, but weaker. enc 5.0
small          90deg cone, 20sq radius

magnifying     Gives mod+3 to find, but takes one round extra for each find
lens

\end{verbatim}
\normalsize





%-------------------------------------------------------------------------------
% E X O T I C   E Q U I P M E N T
%--------------------------------

%Exotic items are difficult to find and their prices vary greatly.





%-------------------------------------------------------------------------------
% T R A P S
%----------


\phantomsection\addcontentsline{toc}{section}{traps}
\section*{Traps}

Traps are usually the kind of things that you fall into, put spikes into your belly, poke your eyes out, or that prick your finger injecting horribly painful poison. But they can also be your friends. If you have the time and money a few well placed traps can slow down or decimate your enemies as they advance to kill you dead.

All traps have a few different properties such as price and encumbrance when you buy and transport them. Then size, trigger type, damage, time to set up and arm, difficulty to hide, etc. Some traps require that you dig a hole, or hack out some empty space in a wall.

\small \begin{verbatim}

small snare: cost 5c, enc 0.5, size 1sq 50\% chance of triggering.
    Caught target can break free in one action on str vs 5 roll
    or disentangle in two rounds after successful int vs 2 roll
    or loop cut in one round if carrying small blade or similar.
    Takes 3r to set and must be anchored to something within 5sq.
    They are easy to hide at mod+3.

large snare: cost 1s, enc 1.0, size 1sq 100\% chance of triggering.
    Caught target can break free in one action on str vs 8 roll
    or disentangle in two rounds after successful int vs 2 roll
    or loop cut in one round if carrying small blade or similar.
    Takes 5r to set and must be anchored to something within 8sq.
    They are easy to hide at mod+3.

The common way to anchor snares is around tree trunks or other protruding solid
objects. Other ways is to use dirt wedges to anchor it in the ground. A dirt
wedge can be hammered down using heavy blunt object in 2r and can be retrieved
in 1d5 rounds using similar objects. They have a chance of 3+str to hold when
loaded. Dirt wedges cost 1s and weigh 0.5enc.
Snares can also be anchored in some stone walls using rock wedges. A rock wedge
can be inserted in a crevice using a hammer, butt of an axe, stone, or similar,
in 2r. It can be retrieved using similar tools in 1d5r. They have a chance of
5+(best of int or per) to hold when loaded. Rock wedges cost 5c and weigh
0.25enc.

\end{verbatim} \pagebreak[1] \begin{verbatim}
spike arm:
        Pre-made: cost 5s in materials, enc 5+1/dam.
        Half foraged: cost 3s in materials, enc 3 + 1/2dam + foraging.
        Completely foraged: cost nothing but requires a knife. not portable.
        Spike arm traps are the things that usually put a slab of pointy objects
        into the stomach of people. Excellent to rig just behind a corner or a
        tree, or perhaps directly in the ground covered by old leaves or snow.
        A skilled woodsman can forage for all the materials, but that takes 100r
        per dam extra. A smart man brings some stuff with him, and forages only
        for the scaffold and such, which takes 5r/dam. When in dungeons etc it
        might be a good idea to bring all of the materials in a bundle.
        A pre-made bundle is set in 10+dam rounds.
        A half foraged trap is rigged in forage time + 15+2*dam rounds.
        A completely foraged trap is constructed in forage time + 20+4*dam time.
        Hiding a spike arm trap is mod-0.

pit trap: cost nothing, requires perhaps a canvas, and some shovels to dig with.
        Pit traps are a cheap and common trap for when you have enough time.
        1) Dig a pit. In dirt outdoors this takes 100r/sq if the digger has
        a spade or shovel. Otherwise it takes four times as long. As many
        people can help as there are connecting side squares of the trap.
        2) Put down spikes or other nastiness, takes a few rounds per sq.
        3) Cover with suitably weak material, another few rounds per sq.
        4) Camouflage the trap, another couple of rounds per sq.
        A regular pit does 3 dam, a double deep pit does 6 dam, a triple deep
        pit does 10 dam. With spikes the damage is double, and the spikes get
        one point of penetrating per depth of the pit, including the first.
        Deep pits take longer to dig of course.
        Hiding a pit trap is mod+3.


\end{verbatim} \normalsize




%-------------------------------------------------------------------------------
% T R A I N E D   A N I M A L S
%------------------------------

\phantomsection\addcontentsline{toc}{section}{trained animals}
\section*{Trained animals}

Companion animals can be trained to do simple tasks. Each animal has a set of specific commands it knows and can perform. Animals also have a command modifier, depending on how well they are trained, which affect how easy it is to command them. The skill "animal command" is used to handle animals.
The GM has ultimate control over the details of the animal's actions, but it will follow the given command as it interprets it, and according to situation. The practical movement on the map, rolls, etc is generally handled by the commanding player.

Some smarter trained animals can have skills and xp.
It can for example be useful for an archer to train their attack hawk or war dog as a "target pointer".



\openitemslist

\eqitem{Horse:}
An average horse can carry a rider plus a 20enc pack load in saddle bags without problems, at a daily 10sq distance. Without rider it can take 60enc in sacks. Horses vary greatly in training, movement, travel, carrying capacity, etc.
\small \begin{samepage} \begin{verbatim}
===================================
average horse               (token)  set token size large (2x2)
-----------------------------------  position rider token at top rear square
str 20    hp 40 abs 0
dex  5    m3 w6 r12 d24
per  8    initiative 8
rear kick   5 dam8 pen2 knockback 2
front kick  6 dam4 pen1 knockback 1
bite        4 dam2
-----------------------------------
\end{verbatim} \end{samepage} \normalsize


\eqitem{Donkey:}
Normal donkeys are not trained for riding, but to carry load or drag a small cart. It carries 50enc in pack sacks for 10sq per day.
\small \begin{samepage} \begin{verbatim}
===================================
donkey                      (token)  set token size normal
-----------------------------------
str 10    hp 25 abs 0
dex  5    m2 w4 r8 d16
per  8    initiative 8
rear kick   6 dam6 pen1 knockback 1
front kick  6 dam3
bite        7 dam2
-----------------------------------
\end{verbatim} \end{samepage} \normalsize


\eqitem{War Dog:}
Big and dangerous breed of dogs.
\small \begin{samepage} \begin{verbatim}
===================================
war dog                     (token)  set token size small
-----------------------------------
str  4    hp 5 abs 0
dex  8    m2 w4 r8 d16
per 10    initiative 12
charge 6
balance 3
avoid 4 yield+4 always yield when possible
bite 5 dam 5
claw 5 dam 2 fast+1
-----------------------------------
\end{verbatim} \end{samepage} \normalsize
Command: attack target pointed at: Will charge the target and bite and claw. \\
Command: return to owner \\
Command: guard spot or small area: Will attack intruders that get within 5 range from the area. \\
Command: guard person or group: Will attack intruders that get within 5 range from the person or group. This is the default behaviour in combat situations. \\
Command: follow: Will follow owner and keep calm unless obvious enemy comes within 1d4sq. This is the default non-combat behaviour. \\
Command: fetch arrow: Will run to a corpse and fetch an arrow. Better trained dogs might fetch more than one. This is not a standard command and costs extra.


\eqitem{Attack Hawk:}
Well trained hawks or other large fast bird of prey. \\
hp 1, to hit -6 (incl fighting speed), movement fly 20, \\
claw and beak: 5 damage 1 \\
avoid  6 (incl always yield) \\
Command: attack target pointed at: \\
Will fly to the target and distract it by clawing at it's face. This gives the target a mod-3 to all actions, and must pass an int roll to not try to target the hawk. \\
Command: return to owner. \\
Command: follow, will follow the owner or sit on his shoulder.


\eqitem{Darkwing:}
Large bats, not very clever. \\
hp 1, to hit -8 (incl speed), movement fly 15 \\
avoid 5 (incl always yield) \\
Command: scout \\
Will fly away in the indicated direction. \\
Return immediately and chatter if found anything that might be dangerous. \\
Return after a while calmly if it didn't find anything interesting. \\
Command: follow \\
Will follow the owner, or cling to his back, and stay out of trouble.


\eqitem{Fetching Ferret:}
hp 1, to hit -6 (incl speed), movement r4 d12 \\
encumbrance limit 1.0, cannot carry more than that. \\
Command: fetch \\
Pick up indicated object and bring it back to owner. \\
Might not always pick up and bring what you want. \\
Command: port \\
Pickup an object from the first indicated person and give to the second. \\
Might not be what you want unless actually given the item. \\
Command: return to owner. \\
Comes back and clings to the owners shoulders or such.


\eqitem{Tripping Traccoon:}
hp 2, to hit-3 (incl speed), movement r4 d12 \\
encumbrance limit 2.0, cannot carry more than that. \\
Command: fetch (see ferret) but only stuff that glitters or smells interesting \\
Command: port (see ferret) \\
Command: return \\
returns to owner and follows or clings. \\
gives track+1


\eqitem{Wolverine:}
Very persistent, inexhaustible endurance, never gives up. With con99 it's just silly, but make it die or limp away when taken suitable amounts of damage for the situation. The maptool implementation means stamina and con must both be set. Also fun that it will always seem "rested" under most normal situations.
\goodbreak \small \begin{samepage} \begin{verbatim}
===================================
wolverine                   (token)  set token size small
-----------------------------------
str  4    hp 5 abs 0
dex  8    m2 w4 r6 d12
con 99    stamina 99 (makes it near impossible to actually kill)
per 10    initiative 10
pain threshold 5
black knight 2
balance 6
charge 3
avoid  4 yield+3 always yields when possible \\
avoid 4 yield+4 always yield when possible
claw: 5 dam 3 fast+1
bite: 6 dam 4 pen 1
hold: 9 dam 2 pen 2 only after successful bite (prev round ok, no movement)
                    target has mod-3 to all actions, persists over rounds
-----------------------------------
\end{verbatim} \end{samepage} \normalsize
Command: attack: Attacks indicated target and follows until called back, or target is dead. \\
Command: return: Returns to owner and follows, does not start fights unless provoked or commanded.


\eqitem{Explorat:}
Small, although large for a rat, curious, can gnaw it's way through most anything given time. \\
Can be trained to search for one specific thing, like monsters, gold, food, water, etc.\\
hp 1, to-hit-6, movement 4 \\
Release and it will return to find you sooner or later, hopefully indicating that it has found what it is trained for. \\
A few reports have come in that some breeds of explorats have a tendency to spontaneously and violently combust, or explode. This has never been proven and should under no circumstances deter you from purchasing one of this small but very helpful pets.


\eqitem{Bright Beetle:}
Is actually a rather large beetle. It is barely trainable insect, but it's calm and unafraid, and will go where it is commanded. \\
Command: Go \\
The beetle will travel to the indicated location, then stay there. \\
Command: Return \\
The beetle will return to the owner, then follow. \\
Command: Target \\
The beetle will travel to a target then stay a few squares away, following it.\\
hp 1, to hit-3 (melee) to hit -6 (ranged) \\
movement walk 3, run 6, fly 9 \\
It is too small to take up a square and cannot block movement. When killed it explodes doing dam 3 to the square it is on and dam 1 to radius 1.


\eqitem{Empowering Brightwing:}
A medium size very rare and exotic bird. It glows with a yellow light (light source range 5), and empowers those in contact with it with extra energy. A brightwing restores 1 stamina each full round it is sitting on a character's shoulder. It will not land, or stay on a shoulder when the character is moving faster than walk, or when he is in melee combat. When not sitting on its owner's shoulder it tends to stay in the vicinity. \\
Immune to fire. \\
hp 3, to hit-6 (incl movement), movement fly 10 \\
avoid 5 (incl always yield) \\
Command: go to \\
Will fly to indicated character and land on his shoulder. \\
Command: return \\
Will fly back to the owner and land on his shoulder.


\eqitem{Medicinal Serpent:}
A very rare and exotic snake, about 1-2m long. It is green and oozes slightly of a glowing greenish aura (light source range 2). It rests on a staff or other long stick. For each consecutive two full rounds the snake is coiled around a target the target heals 1hp. The target cannot take any actions, and not move faster than maneuver. The owner of the snake may move at walk speed and take minor actions that do not require much movement. \\
hp 5, to hit-3 (incl movement), movement slither 4. \\
Command: engage \\
The snake will slither onto the indicated target and coil around its abdomen.
Command: disengage \\
The snake will uncoil from the target and return to its resting place


\eqitem{Glow Fly:}
Various small critters that glow in the dark. They don't give off much light, but it can be enough to scout some areas, especially for Heroes with dusk vision or night vision. The glow bugs are very small, hp 1, mod-6 to hit them with melee weapons, and mod-9 with ranged weapons. Various bugs have different speeds, from crawl 1 to fly 10. They are impossible to train, and have to be commanded by use of special food pastes. The owner throws a dollop of paste at a location or target and the bug goes there to eat. It will stay for a few rounds depending on the size of the dollop. Roll for animal command, and apply the fail diff as random +/- rounds to the intended time. After it has finished eating it will return to the owner if he is within 10-30 sq, depending on how keen eyes, ears, or nose the bug has. \\
It is recommended to have the gob dispenser as quick draw, then the "command" of throwing a dollop takes two actions, one for select size and one for throw. Otherwise it takes two rounds. Hitting the intended area with a dollop is a throw roll mod+3. Hitting an intended target is a regular throw roll. Dollop base range is dex+str, but minimum 3sq.


\eqitem{Jumping Spider:}
Aggressive spiders that can barely be trained. Unless directly controlled by animal command the spiders will always attack the nearest living non-spider it can find. This includes allies. They jump onto the target then stays attached,
making one "bite" attack each round until called off or the target dies.
\goodbreak \small \begin{verbatim}
jumping spider: hp 1, move 6, jump 2, all terrain (no water)
                Tiny target: melee mod-3 ranged mod-6
jump-attach attack 7 (once per round)
        The jump attack (2sq) does not count against movement.
        The spider stays attached until cleaned off or target dies.
        Each attached spider gives the target mod-1 to all actions.
bite 8 (mod-1 per abs) (once per round, after attached)
        dam 1 penetrating
command: attack (point and whistle, 3ap) max range 6
        The spider will attack the target until it's dead.
        Then it will move to attack the closest living non-spider
        it can find, including allies.
command: return (whistle, 3ap) max range 15
        Spiders will disengage and return to handler.
It takes an action (3ap) and a dex roll to clean off a spider.
The spider is placed on an adjacent square chosen by the target.
Others have dex+3 when helping the target.
Attacking a spider which is attached to a target require success+3
otherwise rest of the damage hits the target. Fail-3 or worse will
hit the target.
\end{verbatim} \normalsize
Spiders can be housed in a "spider box", or "hive box". A box normally holds max ca 3 spiders per enc. E.g: spider box enc 4 has "ammo" 10+1d4 spiders. A spider box can normally spawn a spider each round.


\closeitemslist













%-------------------------------------------------------------------------------
% C O M P A N I O N S
%--------------------

\phantomsection\addcontentsline{toc}{section}{companions}
\section*{Companions}

The GM has ultimate control over the details of the companion's actions, but it will follow the given command as interpreted according to situation. The practical movement on the map, rolls, etc is generally handled by the commanding player. The character that handles companions need the companion command skill.

In general, most companions will take or require their fair part of XP and loot.


\openitemslist

\eqitem{Henchman:}
A henchman can be anything from a goblin to an elf, he can be rented for the day or a trusted old friend. Henchmen have a mind of their own and full stats and skills according to a regular character sheet.
Hiring henchmen is expensive. They generally require both a payment up front and a cut of the treasure. Of course, if he cannot count... But most henchmen can both count and haggle. Annoying, isn't it.

The most expensive part though is that the henchman will also take part of the XP from the adventuring.


\eqitem{Demons, spirits, etc:}
Some heroes can summon strange beings to do their bidding. Some of these odd critters can stay around for a long time, and they might even be difficult to get rid of. Some will require some sort of payment or compensation, be it in gold, blood, xp, special equipment items, or left ears of the righteous.

\closeitemslist


For calculating division of plunder when one or both of the parties fail their counting rolls see the "counting" skill description. Some loot requires proper valuation. This is why henchmen often have some levels in the appraise skill.

Each henchman or companion critter have their own personality and goals. It may be nothing special, simple greed, or it may be something which has special meaning in the current adventure or campaign plot arc.

When playing henchmen, be stringent with actually letting them behave in a reasonable way. They don't have magical foresight, perfect sense of tactics, unlimited bravery, or other feats which would be required to act as the players often wish them to act. Henchmen require clear orders, and they will generally adhere to those orders as long as it makes sense and does not send them into unreasonable danger.

Situations often apply where the controlling player might want the henchman to take other actions than he has been ordered to because it is "clearly better". First, would it be better for the henchman? Is it evident for the henchman? Is he smart enough to figure it out (roll int)? Secondly, does the henchman often have enough free reign to ignore orders? Is he often strongly ordered to do dangerous stuff? Does he have the presence of mind to think clearly (roll psy)? And so on.

Most henchmen will try to save their own skin when they start to get hurt. They might disobey orders, hesitate when ordered to attack, be more defensive, etc.


%--------|---------|---------|---------|---------|---------|---------|---------|
%       10        20        30        40        50        60        70        80


\cleardoublepage

\phantomsection\addcontentsline{toc}{chapter}{prices}
\chapter*{Prices}
\chaptermark{prices}




\section*{Currency}
%------------------
money: 1 gold = 10 silver = 100 copper\\
Some folk use beads, feathers, shells, teeth, etc. But since those critters often haggle and negotiate with their fists, the trade procedures and "agreed upon" price is ... fluctuating?



%-------------------------------------------------------------------------------
% E Q U I P M E N T   P R I C E   L I S T
%----------------------------------------

\section*{Listed and actual prices}
%----------------------------------
List prices are supposed to be approximate averages of actual market prices. Some places are cheaper, others more expensive.

Old used equipment can be found at greatly reduced prices, 50\% or even 25\% of original cost, but usually have lower stats and odd quirks and problems. How about a Lochaber that has a 1 in 10 chance of loosing it's head, which takes 1d3 rounds to re-fit? Or how about an armour that only absorbs on a roll of 1-7, or one in which you cannot make left turns or defend against the left flank? A crossbow that jams on 9-10, and which takes 1r to clear? Or how about a lantern that goes out randomly after 10+1d10 turns unless you stop and shake it carefully, just right, for 1r as a precaution?


\section*{Quality of items}
%--------------------------
The list prices and equipment stats are for regular equipment. Well made equipment has better stats perhaps, or weighs 10-30\% less, but costs 3-5 times more than regular quality. Master craftsman quality costs 5-10 times more than regular quality, and at least a few of gold. They cost a lot but has better stats, maybe a point or two extra damage or a tohit bonus, and weighs less and/or has 30-50\% better absorption, lower str or dex requirements, etc.

%A master craftsman sword can have a +1 to damage, or +1 bonus to the weapon skill, or +5 higher abs or 33\% lower weight. Or some such bonus or beneficial property.

Not all equipment or properties can be boosted like this. An abs 2 leather armour will cost much more than 5 gold (10x leather armour base price), if it is even possible to make, and would probably require the hide from some very exotic and dangerous monster.




% testing table ...

%\begin{tabular}{lccclcll}
%item             &   g & s & c & ~ ~ & enc  & ~ ~ & notes \\
%\hline
%dagger           &   0 & 2 & 0 & & 0.5  & & \\
%rapier           &   0 & 8 & 0 & & 0.75 & & \\
%short sword      &   0 & 6 & 0 & & 0.75 & & blajblaj \\
%\hline
%\end{tabular}

% nah, I'll stick with verbatim small




%-------------------------------------------------------------------------------
% weapons and armour
%-------------------

\goodbreak
\phantomsection\addcontentsline{toc}{section}{weapons and armour}
\section*{Weapons and armour}




% try keeping the price list looking good.
% just insert the line blow to allow page break at a good place
%\end{verbatim} \end{samepage} \goodbreak \begin{samepage} \begin{verbatim}


\small
%\begin{samepage}
\begin{verbatim}
                     g s c    encumbrance
                     -----    -----------

small sword            8 0       0.8
sword                1 0 0       1.0
large sword          2 0 0       1.3
small 2h sword       3 0 0       1.5
2h sword             4 0 0       2.0
large 2h sword       8 0 0       2.7

short sword          1 0 0       0.75  military, much cheaper in bulk
broad sword          1 5 0       1.2   military, much cheaper in bulk
long sword           3 0 0       1.3   military, much cheaper in bulk

knife                  0 5       0.25  tools, cheaper in bulk
dagger                 2 0       0.5
rapier               1 0 0       0.7
large rapier         3 0 0       0.8

duelling sword      10 0 0       0.9
large duell. sword  15 0 0       1.1
parrying dagger        7 0       0.7
baselard               7 0       0.6
bastard sword       10 0 0       1.5
great sword         10 0 0       2.0
claymore            15 0 0       3.0
estoc               12 0 0       1.3
grande estoc        15 0 0       1.5


small 1h axe           3 0       0.8   tools, cheaper in bulk
1h axe                 4 0       1.0   tools, cheaper in bulk
large 1h axe           7 0       1.3   tools, cheaper in bulk
small 2h axe           8 0       1.5   tools, cheaper in bulk
2h axe               1 0 0       2.0   tools, cheaper in bulk
large 2h axe         1 5 0       3.0   tools, cheaper in bulk

battle axe           1 2 0       1.5   military, much cheaper in bulk
great axe            2 0 0       2.0
lochaber             6 0 0       4.0
pick axe               4 0       1.8   tools, cheaper in bulk
heavy pick axe         6 0       2.2   tools, cheaper in bulk


                     g s c    encumbrance
                     -----    -----------

small staff            0 8       0.8
staff                  1 0       1.0
large staff            1 5       1.5

small spear            2 0       1.5
spear                  3 0       2.0
large spear            6 0       2.7
long spear             5 0       3.0
large long spear       8 0       4.0

infantry spear         5 0       2.5   military, much cheaper in bulk

pike                   8 0       4.0   military, much cheaper in bulk
heavy pike           1 2 0       5.0
halberd              1 5 0       5.0   military, much cheaper in bulk
glave                1 5 0       4.0

small club               3       0.5
club                     5       0.75
large club             1 0       1.0
very large club        1 2       1.2
monstrous club         1 5       1.5
heavy clubs          +30%        +30%

1h hammer              1 0       0.8   tools, cheaper in bulk
large 1h hammer        3 0       1.2   tools, cheaper in bulk
2h hammer              4 0       1.4   tools, cheaper in bulk
large 2h hammer      1 5 0       2.2   tools, cheaper in bulk

1h flail             3 0 0       0.8
large 1h flail       5 0 0       1.2
2h flail             6 0 0       1.6
large 2h flail       9 0 0       2.2

heavy hammer         4 0 0       3.5
maul                 8 0 0       4.5
morning star         3 0 0       1.4
2h morning star      5 0 0       2.0
long chain flail     5 0 0       1.2
2h long chain flail  9 0 0       2.2


                     g s c    encumbrance
                     -----    -----------

buckler                2 0       0.5
small shield           3 0       1.0
shield                 5 0       2.0
large shield           8 0       3.0
tower shield         1 5 0       5.0
heavy version        +100%      +40%
plated version       +300%      +80%

infantry shield        6 0       2.3   military, much cheaper in bulk

leather armour         5 0       1.0 worn 2.0 packed
chain mail           4 0 0       2.0 worn 4.0 packed   military, cheaper in bulk
plate mail          10 0 0       3.0 worn 6.0 packed
full plate          25 0 0       4.0 worn 10.0 packed
heavy plate         50 0 0       5.0 worn 15.0 packed


                     g s c    encumbrance
                     -----    -----------

throwing knife         1 0       0.2
throwing axe           2 0       0.33
javelin                1 5       0.5
heavy javelin          2 5       0.66
dart                     5       0.1


short bow              3 0       0.75
bow                    6 0       1.0
heavy bow            1 0 0       1.5
longbow              1 2 0       2.0   military, much cheaper in bulk
heavy longbow        1 8 0       2.5

normal arrows          1 0       0.33  10 arrow bundle
cheap arrows             1       0.33  10 arrow bundle
for special arrows, see special equipment section


small crossbow       1 0 0       1.0   military, much cheaper in bulk
crossbow             2 0 0       1.5   military, much cheaper in bulk
heavy crossbow       3 0 0       2.0   military, much cheaper in bulk

normal bolts           1 0       0.25  10 bolt bundle
cheap bolts              1       0.25  10 bolt bundle
for special bolts, see special equipment section


arbalest             6 0 0       3.0   special military, cheaper in bulk
broadhead              3 0       0.5   10 bolt bundle, special
spikehead              5 0       0.5   10 bolt bundle, special
blunthead              5 0       0.5   10 bolt bundle, special
fine bolt versions   2 0 0       0.5   10 bolt bundle, special, mod+1


light ballista      20 0 0      50.0 mounted weapon
ballista            30 0 0     100.0 mounted weapon
heavy ballista      50 0 0     200.0 mounted weapon

ballista ammunition, ca 5-50 copper per shot


\end{verbatim} \normalsize

\

\TODO remove old price list, go through to carry things over to new list above


%\small \begin{samepage} \begin{verbatim}
\small \begin{verbatim}
                     g s c    encumbrance
                     -----    -----------


crossbow             1 0 0       1.0   military, much cheaper in bulk
heavy crossbow       2 0 0       1.5
arbalest             6 0 0       2.0
light ballista      20 0 0      10.0
ballista            30 0 0      20.0

bowstring            0 0 2       0.1
10 arrows            0 0 1       0.33
10 bolts             0 0 1       0.25
quiver 10            0 0 5       0.5  +0.5*load (arrows/bolts)
quiver 20            0 0 8       0.8  +0.5*load (arrows/bolts)
quiver 30            0 1 0       1.0  +0.5*load (arrows/bolts)
quiver 50            0 3 0       1.5  +0.5*load (arrows/bolts)

\end{verbatim} \normalsize
%\end{verbatim} \end{samepage} \normalsize
\goodbreak





%-------------------------------------------------------------------------------
%basic equipment
%---------------

%\begin{samepage}
\goodbreak
\phantomsection\addcontentsline{toc}{section}{basic equipment}
\section*{Basic equipment price list}

Most basic equipment can be found in any village or town. Prices can vary quite a bit.


% try keeping the price list looking good.
% just insert the line blow to page break at a good place
%\end{verbatim} \end{samepage} \goodbreak \begin{samepage} \begin{verbatim}


\small
%samepage started before the section heading
\goodbreak \begin{verbatim}
                     g s c    encumbrance
                     -----    -----------

knife                    5       0.25

first aid kit          5 0       1.5    has 10 uses, refill for 3s
medical kit          2 0 0       2.0    has 10 uses, refill for 10s

simple food 5 days       1       5.0    simple food weigh 1enc per day's ration
food & room 1 day        1       ---    common room sleeping, simple food
provisions 1 day         1       0.5
travel rations 1d        5       0.3    high quality travel food, last a while
1l beer                  1       0.5
10l barrel             1 0       5.0
bottle of wine           2       0.5
coffee                   1       0.1  enough for a one cup dose, must be brewed
spirits                  1       0.1  enough for a one swig dose.
health elixir          3 0       0.1  heal 3hp (1/3r) draws 3mana+3stamina
health wafer           2 5       0.1  heal 3hp (1/10r) max stam-1 until slept

rag clothing             5       1.0 (only when packed)
simple clothing        2 0       1.0 (only when packed)
good clothing          5 0       1.5 (only when packed)
fine clothing        1 0 0       2.0 (only when packed)
extravagant clothing 3 0 0       3.0 (only when packed)
warm clothing          5 0       2.0 (only when packed)

blanket                  5       2.0
bed roll               1 0       3.0
tarp, waterproof       3 0       4.0 enough for a nice lean-to

belt 1 item  2.0 enc     5       0.2 (only when packed)
belt 2 items 3.0 enc     8       0.3 (only when packed)
belt 3 items 3.5 enc   1 3       0.4 (only when packed)
belt 4 items 4.0 enc   2 0       0.5 (only when packed)

water skin (1 day)       3       1.0

sack 5 enc               1       1.0 + 0.75 * load (0.25 packed)
sack 10 enc              2       1.5 + 0.75 * load (0.5 packed)
leather bag 5 enc        3       1.0 + 0.66 * load (0.5 packed)
leather bag 10 enc       5       1.5 + 0.66 * load (1.0 packed)
backpack 5 enc         1 0       0.5 + 0.50 * load
backpack 10 enc        2 0       1.0 + 0.50 * load
backpack 15 enc        4 0       1.5 + 0.50 * load

saddle bag 5 enc       2 0       2.0 + 0.50 * load
saddle bag 10 enc      4 0       3.0 + 0.50 * load
pack bag 25 enc        6 0       5.0 + 0.50 * load      tot fully loaded = 17.5
saddle etc             5 0       7.0
donkey               1 0 0       10 leagues/day     carries 50enc, ride-3
pony                 2 0 0       10 leagues/day     carries 50enc
horse                3 0 0       10-15 leagues/day  carries 75enc
heavy horse          5 0 0       10-15 leagues/day  carries 100enc
fast horse          10 0 0       20-25 leagues/day  carries 60enc
war horse           25 0 0       ~15 leagues/day    carries 90enc
More expensive horses can travel further per day, and might give bonus to ride.
Cheaper horses get tired faster and might give negative mods to ride.

hand cart            1 0 0       load 10*(str-2) enc (max 50, str7) size 1x1
small cart           3 0 0       1 pony/donkey, load 100 enc, size 1x1 or 2x2
cart                 5 0 0       1 horse, load 200 enc, size 2x2
wagon               10 0 0       1-2 horses, 2-6 ppl, load 300 enc, size 2x2 / 2x3
large wagon         15 0 0       2-4 horses, 4-10 ppl, load 500 enc
carriage            30 0 0       2-4 horses, 1-2 drivers 4-6 ppl, load 300 enc

rope 10m             0 0 2       1.0 climb+3
grappling hook       0 2 0       1.0 fixed
rope ladder 10m      0 5 0       5.0 climb+6

tool box             0 5 0       2.0
twine 50m            0 0 2       0.25

torch 20r            0 0 1       0.5 bright light (10sq torch, 20r duration)
candle 50r           0 0 1       0.25 weak light (4sq candle, 50r duration)
lantern 100r         0 5 0       1.0 weak light (7sq lamp, 100r)
spot lamp 100r       0 8 0       1.0 weak light (10sq spotlamp. 90deg, 100r)
lamp oil 500r        0 0 1       0.5 5x doses of 100r for lantern
flint & steel        0 0 3       0.2 fire in 1d10 rounds
tinder box           0 1 0       0.5 fire in 1d3 rounds
smoulder box         0 5 0       2.0 fire in 1 round

small snare          0 0 5       0.5 or 0.1 when packed. trap: 1sq 33%
large snare          0 1 0       1.0 or 0.2 when packed. trap: 1sq 66%
trigger snare        0 5 0       2.0 or 0.5 when packed. trap: 1sq 90%
rock wedge / piton   0 0 5       0.25  anchoring str 20
dirt wedge / piton   0 1 0       0.5   anchoring str 10
\end{verbatim} %\end{samepage} \goodbreak
\normalsize






%-------------------------------------------------------------------------------
%special equipment
%-----------------

%\begin{samepage}
\goodbreak
\phantomsection\addcontentsline{toc}{section}{special equipment}
\section*{Special equipment price list}

Special equipment can generally be found in larger towns and cities.
The prices vary quite a bit, half or double price is nothing strange.

\small
\begin{verbatim}
                     g s c    encumbrance
                     -----    -----------

Speciality bolts and arrows tend to be single use, or possibly a few times.
They easily break or get damaged to the point where their extra properties
are lost to damage. At which point they are useless or work as regular arrows
with some negative mods: mod-1 / dam-1 / pen-1

bow arrows:
10 normal arrows       1 0       0.33  no mods
10 cheap arrows          1       0.33  random: mod-1 or dam-1, cannot be reused
10 fine arrows       1 0 0       0.33  mod+1
10 barbed arrows       5 0       0.33  dam+1
10 sharp arrows        5 0       0.33  pen+1
10 piercing arrows     2 0       0.33  mod-1 dam-1 pen+1
10 broadhead arrows    2 0       0.33  mod-1 dam+1 pen-1
10 fire arrows         3 0       0.5   mod-2 dam-1 range-25%
                                       oil splash dam 3/r for 5r or until doused
                                       light source as torch, burns 5r
                                       light 1a with fire source
10 torch arrows        3 0       0.5   mod-2 dam-2 range-25%
                                       light source as torch, burns 10r
                                       light 1a with fire source
10 candle arrows       3 0       0.5   mod-2 dam-2 range-25%
                                       light source as candle, burns 20r
                                       light 1a with fire source

crossbow bolts:
10 normal bolts        1 0       0.25  no mods
10 cheap bolts         0 1       0.25  random: mod-1 or dam-1, cannot be reused
10 fine bolts        1 0 0       0.25  mod+1
10 barbed bolts        5 0       0.25  dam+1
10 sharp bolts         5 0       0.25  pen+1
10 piercing bolts      2 0       0.25  mod-1 dam-1 pen+1
10 broadhead bolts     2 0       0.25  mod-1 dam+1 pen-1

arbalest bolts:
10 broadhead bolts     3 0       0.5   see arbalest
10 spikehead bolts     5 0       0.5   see arbalest
10 blunthead bolts     5 0       0.5   see arbalest
10 fine version      2 0 0       0.5   mod+1


\end{verbatim} \goodbreak \begin{verbatim}
antidote             1s/str      0.25  many different versions exist with
                                       different strengths, colours, etc.

cure                 1s/str      0.25  various concoctions that might cure
                                       diseases

Quaff!               1 0 0       0.25  (TM) "Get the Real Stuff!"  (4hp+4sta 1r)

healing potion         5 0       0.25  restores 3hp 1hp/r
healing potion         8 0       0.25  restores 5hp 1hp/r
healing potion       1 0 0       0.25  restores 8hp 1hp/r
healing potion       2 0 0       0.25  restores 6hp 2hp/r

stamina potion         5 0       0.25  restores 4sta 2sta/r
stamina potion         8 0       0.25  restores 6sta 2sta/r
stamina potion       1 0 0       0.25  restores 10sta 2sta/r
stamina potion       2 0 0       0.25  restores 8sta 4sta/r

mana potion            5 0       0.25  restores 5mana in 1r
mana potion          1 0 0       0.25  restores 10mana 5mana/r
mana potion          2 0 0       0.25  restores 10mana in 1r

pain killer            1 0       0.25  eliminates 1 pain after 3r
pain killer            3 0       0.25  eliminates 2 pain after 3r
pain killer            5 0       0.25  eliminates 3 pain after 3r damages 1hp

\end{verbatim} \goodbreak \begin{verbatim}
staff light paste    1 0 0       0.25  stafflight 1 for 100r, spread on item
                                       takes 2r to apply.
                                       light as lamp

thermos              1 0 0       0.5   for that hot cup of coffee

compartmentalised    backpack price +50% per compartment
back pack                              for faster rummaging

lock picks             5 0       0.2
fine lock pick       2 0 0       0.2   mod+3 to pick lock

grappling hook       1 0 0       0.3   foldable, takes 1r to fold/unfold.

tool box             2 0 0       2.0   mod+3 to handling trap

cave cart            2 0 0

tunnel plug          5 0 0
tunnel plug large    8 0 0

\end{verbatim} \goodbreak \begin{verbatim}
fightlight           5 0 0       2.0 2r setup, then light, 90deg 40sq
fightlight small     3 0 0       1.0 2r setup, then light, 90deg 20sq
fastfire addon      +3gold       0.5 takes only 1a to light
hibrite oil          1 0 0       0.5 50 rounds of bright burning in fightlight

portadoor            2 0 0       5.0
portagate            5 0 0    3x 5.0

attack hawk          5 0 0  training command mod-3 +1/g
war dog              5 0 0  training command mod-3 +1/g
fetching ferret      3 0 0  training command mod-3 +1/g
tripping traccoon    5 0 0  training command mod-3 +1/g
glow fly             0 2 0  training command mod-3 +1/s, glow strength +1/s


\end{verbatim}
%\end{samepage}
\normalsize

Master crafted weapons can be purchased as "special" equipment, and usually cost 5x-10x the list value of the weapon plus a few gold as baseline cost to make even staffs and short swords expensive when master crafted. These weapons usually have 30-50\% improved abs, and perhaps a mod+1 to attack and/or parry or dam+1 properties as well. Master crafted weapon should not cost less than 5g.

A master craftsman can also make customised weapons. E.g: an axe or a sword that is suited specifically for the properties of the customer, with the damage and strength requirements matched perfectly. Custom items usually cost 3-5x of base value plus an extra gold or two. A custom weapon should not cost less than 3g.

Then it's possible to make a custom master crafted weapon, combining custom requirements with master crafted bonuses, but at even higher prices, 10g minimum.

It probably takes a week for a master craftsman to make high quality or custom weapon, provided he has enough of the high quality materials available. Otherwise there are always heroes around who can take it upon themselves to go out and find that rare lump of sky iron, or barrel of basilisk blood.

\

Magical potions and single use or multi use items are mostly custom made special or exotic equipment. A few magical potions and items are made in quantity and thus much cheaper, like Quaff! But mostly it's just a few custom items, expensive, time consuming

The cost of a custom made magical potion can be estimated to:\\
\verb|1 + (0.25 * (mana+1) * scf * psy) gold| \\
E.g: a potion of firestorm with 3 mana would cost approximately:\\
\verb|1 + (0.25 * (3+1) * 0.33 * 8 ) = 3.6 gold|

The cost of a custom made magical single use or multi use magical item can be estimated to: \\
\verb|single use:  3 + (0.25 * (mana+2) * scf * psy) gold| \\
\verb|multi use : 5 + uses * (0.25 * (mana+2) * scf * psy) gold| \\
E.g: a single use wand and one with 4 charges of 3 mana shock bolts would cost approximately: \\
\verb|3 + ( 0.25 * (3+2) * 0.33 * 7 ) = 5.9 gold| \\            % 3 + 2.8875
\verb|5 + 4 * ( 0.25 * (3+2) * 0.33 * 7 ) = 16.55 gold| \\      % 5 + 4 * 2.8875

\noindent Permanent magical items are much more expensive.






%-------------------------------------------------------------------------------
%exotic equipment
%----------------

%\begin{samepage}
\goodbreak
\phantomsection\addcontentsline{toc}{section}{exotic equipment}
\section*{Exotic equipment price list}

Exotic equipment can sometimes be found in large cities, or just randomly here and there. Prices vary greatly, half price to five times listed is not that strange.

\small
\begin{verbatim}
                     g s c    encumbrance
                     -----    -----------

antidote             1g/str      0.25  many different versions exist with
                                       different strengths, colours, etc.
                                       for exotic poisons

cure                 1g/str      0.25  various concoctions that might cure
                                       strange and exotic diseases

healing potion       5 0 0       0.25  restores 8hp 4hp/r
stamina potion       5 0 0       0.25  restores 12sta 6sta/r
mana potion          5 0 0       0.25  restores 20mana in 1r
pain killer          5 0 0       0.25  relieves 5 pain after 1r

revival powder      15 0 0       1.0   resurrects a character in 10r incl prep
                                       must also heal the character
                                       to at least hp=1 (first aid/medic/pot).
                                       Large bag of white glittery powder.

blast off            2 0 0       0.25  restores 5+1d5hp and all stamina in 5r
                                       handle with care.
                                       Can detonate: stun 9-3/sq from target.

exploway             4 0 0       0.25  maginade: stun 15-3/sq from target.

time fly             5 0 0       0.1   instant rewinds time one action, 
                                       ignores initiative.

sentinel base        5 0 0       5.0   15 charge ghost warrior
                                       skill 8, dam 4

\end{verbatim} \goodbreak \begin{verbatim}
Banor's Gate        30 0 0       str   size 2x2, str 15
Banor's Gate        60 0 0       str   size 1x1, str 8

Wunjee kit          10 0 0       ---   anchor + cord
Wunjee recharge      3 0 0

PortaPortal         10 0 0    2x 5.0   10r, 1sq, activates in 2r


trained darkwing         8g  training command mod-3 +1/g
trained wolverine        8g  training command mod-3 +1/g
empowering brightwing   25g  training command mod-3 +1/g
medicinal serpent       50g  training command mod-3 +1/g

\end{verbatim}
%\end{samepage}
\normalsize







%-------------------------------------------------------------------------------
%exceptional and legendary
%-------------------------


\goodbreak
\phantomsection\addcontentsline{toc}{section}{exceptional and legendary}
\section*{Exceptional and Legendary}
%-----------------------------------
The very rare legendary sword smiths and weapon wrights can make weapons of exceptional quality and properties, at exuberant prices, hundreds of gold. While a master craftsman can imbue a significant abs bonus and perhaps +1 modifiers, the legendary makers can build weapons with double abs and mod+2 and or dam+2. Some can even remove a slow-1 rating or add a fast+1, or modify parry and toparry stats, increase reach, etc.

All such weapons are custom made with damage and strength requirements suited for the person who orders the work. It takes some time to prepare weapons like these, probably months, and the work cannot be rushed.

Be sure that the manufacturing of artefacts like this will require many odd and rare materials and ingredients, which can be difficult to come by, and might require some adventuring...

The total cost including work, materials, and ingredients will come to at least 30x the list value of the weapon plus a minimum of 10 gold base line cost. This can be reduced to 15-20x by adventuring to procure some of the stuff instead of forcing the craftsman to buy it, if it can even be found in the market or region.


\phantomsection\addcontentsline{toc}{section}{magical}
\section*{Magical}
%-----------------
Permanently magical items are always custom made, and very exotic equipment.
First of all the item to be enchanted needs to have enough magical capacity to withstand the forces of the magic to be embedded within it. This means exceptional workmanship and the highest quality materials, often exotic minerals, woods, and stones go into the making of the item itself.

This could mean small side adventures, hundreds or thousands of gold, special conditions, etc.


\goodbreak
\phantomsection\addcontentsline{toc}{section}{hirelings and henchmen}
\section*{Hirelings and Henchmen}
%--------------------------------
Hiring people to help slay monsters or guard the escape route is somewhat expensive, both in gold and experience share. Hired swords usually request somewhere around 1gold per 100xp up front for a regular cave hack adventure, and then half share of the loot. Longer or more dangerous adventures will cost more to hire people for, and simple safe duties will be cheaper. Some regions have higher salary standard than others as well. A poor village might have people at a quarter to half rate, while a rich town might cost double or more.

It is also possible to hire people on speculation, where they get full share of the loot but no up front payment, or on fixed pay where they get no part of the loot, but much higher up front salary.

Hirelings also get a full share of the XP for the adventure. This is sure to keep the players from hiring hordes of archers or double lines of soldiers to toss at the critters they meet.

\

\small \begin{verbatim}
Hired swordsman / archer, 150xp        15s for a few days' work, half loot share

Veteran swordsman / archer, 300xp      30s, half share

Hired knight with horse, 450xp         50s, half share
\end{verbatim} \normalsize


%--------|---------|---------|---------|---------|---------|---------|---------|
%       10        20        30        40        50        60        70        80


% don't add glue spacing between segments, allow ragged bottoms instead
% turn on \flushbottom in the end.
\raggedbottom




%-------------------------------------------------------------------------------
% N P C s
%--------


\phantomsection\addcontentsline{toc}{chapter}{npcs}
\chapter*{NPCs}
\chaptermark{NPCs}

NPCs (Non Player Characters) are the ususal common folk, the watchmen and soldiers, the farmers and merchants, as well as the unique individuals that the Heroes encounter on their travels.

Below follows a few empty stat sheets for different styles of monsters. The first and second are for "real" intelligent adversaries, full and simplified. The rest are for unintelligent monsters, and then just mindless monsters and animals.

\

\raggedbottom

\goodbreak \begin{samepage} \small \begin{verbatim}
===================================
name                        (token)
-----------------------------------
str       hp abs
dex       m w r d
con       stamina
int       vision
psy       mana
per       ap
cha       xp
----------
skills
----------
spells
----------
equipment
money
===================================
\end{verbatim} \normalsize \end{samepage}

\

\goodbreak \begin{samepage} \small \begin{verbatim}
===================================
name                        (token)
-----------------------------------
str       hp abs
dex       m w r d
int       mana
psy       initiative
per       ap
----------
skills
----------
spells
----------
equipment
===================================
\end{verbatim} \normalsize \end{samepage}

\

\goodbreak \begin{samepage} \small \begin{verbatim}
===================================
name                        (token)
-----------------------------------
str       hp abs
dex       m w r d
psy       initiative
per       ap
----------
attacks
skills
===================================
\end{verbatim} \normalsize \end{samepage}

\

\flushbottom




%--------|---------|---------|---------|---------|---------|---------|---------|
%       10        20        30        40        50        60        70        80
\subsubsection*{Note on maptool: token size}
Most standard size humanoids like humans, elves, dwarves, orcs should be set to  normal size tokens. Halflings and goblins should be set to small size, and runts should be set to tiny size.

\






%--------|---------|---------|---------|---------|---------|---------|---------|
%       10        20        30        40        50        60        70        80
\goodbreak
\raggedbottom
\phantomsection\addcontentsline{toc}{section}{city watch and soldiers}

\goodbreak \begin{samepage} \small \begin{verbatim}
================================================================================
W A T C H   &   S O L D I E R S
-------------------------------

The brave who protect all the common folk.


Patrol Soldier
================== human ===================
str  6                    hp 12(15) abs 2
dex  5                    m1 w4 r7 d10
con  7                    stamina 7
int  4                    vision 24 day 256  human good
psy  4                    mana 15
per  6                    action points 3
cha  4                    xp 3 (110)
--------------------------------------------
Common     7
avoid      3
throw      2
veteran    1    1.0  1
crossbow   5    0.8  20
spear      7    0.8  39
shield     6    0.7  25
travel     4    0.5  8
track      3    0.5  4
phalanx    x    spc  10
ride       5    0.5  12  << group leader only
--------------------------------------------
yield +4
off balance
--------------------------------------------
money: 0 gold, 2 silver, 12 copper

spear      7
1h/2h  dam 5/6, pen 1/2, abs 6, parry-1/+1, reach 1 mod-3
       str 4 (max +2 str damage bonus, max+2 penetrating bonus)
       finesse-4/-6

shield     8(5)
       abs 10, parry+3,
       str 4
       Ranged attacks mod-2 when in the way.
       Hiding behind it (3ap) ranged mod-4
       tackle mod+1

crossbow   5
       dam 5, penetrating 1,
       range 12, short +1, long -3, extreme -6,
       str 3 (no str bonus)
reload 6ap if reloading with 0ap successful quick draw roll for quiver
reload 1r if reloading from quick draw quiver
reload 2r if reloading from accessible quiver
first shot  mod=0 1r
first aimed mod+1 2r
first quick mod-3 3ap
first snap  mod-6 2ap
--------------------------------------------
\end{verbatim} \normalsize \end{samepage}

\

\goodbreak \begin{samepage} \small \begin{verbatim}
================== human ==================
str  5                    hp 10 abs 0
dex  5                    m1 w3 r6 d9
con  5                    stamina 5
int  5                    vision 20 day 180
psy  5                    mana 5
per  5                    action points 3
cha  5                    xp 112
--------------------------------------------
Common   7
throw    5 (3)	0.8  12
ride     5	0.5  12
avoid    5 (3)	0.8  12
brawl    3	0.6  5
knife    5	0.7  17
animal command  3	0.5  4
axe/spear/bow   7	ca 1.0 50
--------------------------------------------
yield +3
off balance
--------------------------------------------
money: 3 gold, 1 silver, 10 copper
Farmers knows knife and one weapon skill like:

knife               5
	dam 2, abs 3, parry-2, toparry-2, toavoid-1, finesse-9
	str 1 (no str bonus)
	fast+1 if str 2 and dex 4
	first two attacks don't require stamina
	poke: mod-1 dam-1 pen+1

1h axe             8
	dam 7, abs 8, parry-3, toparry-1, toavoid+1, finesse-3
	str 4 (max +4 str bonus)
	swing: slow-1 mod-1 dam+2 toavoid+2

spear              6 2h: parry 8
1h/2h	dam 5/6, pen 1/2, abs 6, parry-1/+1, reach 1 mod-3
	str 4 (max +2 str damage bonus, max+2 penetrating bonus)
	narrow tip spears can have all str bonus as penetrating
	broad tip spears are dam+1 mod-1
	finesse-4/-6

bow	7 1r-3, 2r=0, 3r+1
	dam 4, penetrating 1
	range 14, short 7 mod+3, long 24 mod-3,
	vlong 30 mod-6 dam-1, extreme 35 mod-9 dam-2,
	str 5 (no str bonus)
--------------------------------------------
\end{verbatim} \normalsize \end{samepage}

\







%--------|---------|---------|---------|---------|---------|---------|---------|
%       10        20        30        40        50        60        70        80
%\pagebreak[2]
\goodbreak
\phantomsection\addcontentsline{toc}{section}{TODO: other ppl}

\goodbreak \begin{samepage} \small \begin{verbatim}
================================================================================
-----------------------------------
\end{verbatim} \normalsize \end{samepage}

\

\goodbreak \begin{samepage} \small \begin{verbatim}
-----------------------------------
\end{verbatim} \normalsize \end{samepage}

\

\goodbreak \begin{samepage} \small \begin{verbatim}
-----------------------------------
\end{verbatim} \normalsize \end{samepage}

\

\goodbreak \begin{samepage} \small \begin{verbatim}
===================================
\end{verbatim} \normalsize \end{samepage}

\

\goodbreak \begin{samepage} \small \begin{verbatim}
===================================
\end{verbatim} \end{samepage} \goodbreak \begin{samepage} \begin{verbatim}
-----------------------------------
\end{verbatim} \normalsize \end{samepage}

\

\goodbreak \begin{samepage} \small \begin{verbatim}
===================================
\end{verbatim} \normalsize \end{samepage}

\

\goodbreak \begin{samepage} \small \begin{verbatim}
===================================
-----------------------------------
\end{verbatim} \normalsize \end{samepage}

\

\goodbreak \begin{samepage} \small \begin{verbatim}
===================================
\end{verbatim} \normalsize \end{samepage}

\

\goodbreak \begin{samepage} \small \begin{verbatim}
===================================
\end{verbatim} \normalsize \end{samepage}

\

\goodbreak \begin{samepage} \small \begin{verbatim}
===================================
\end{verbatim} \end{samepage} \goodbreak \begin{samepage} \begin{verbatim}
-----------------------------------
\end{verbatim} \normalsize \end{samepage}

\







%--------|---------|---------|---------|---------|---------|---------|---------|
%       10        20        30        40        50        60        70        80
%\pagebreak[1]
\goodbreak
%\phantomsection \addcontentsline{toc}{subsection}{lizard men}
\phantomsection \addcontentsline{toc}{section}{other ppl}

\goodbreak \begin{samepage} \small \begin{verbatim}
================================================================================
O T H E R   P P L
-----------------
bla bla bla


===================================
asfasdf
-----------------------------------
-----------------------------------
\end{verbatim} \normalsize \end{samepage}

\

\goodbreak \begin{samepage} \small \begin{verbatim}
===================================
===================================
\end{verbatim} \normalsize \end{samepage}

\

\goodbreak \begin{samepage} \small \begin{verbatim}
===================================
===================================
\end{verbatim} \normalsize \end{samepage}

\

\goodbreak \begin{samepage} \small \begin{verbatim}
===================================
===================================
\end{verbatim} \normalsize \end{samepage}

\

\goodbreak \begin{samepage} \small \begin{verbatim}
===================================
===================================
\end{verbatim} \normalsize \end{samepage}

\

\goodbreak \begin{samepage} \small \begin{verbatim}
===================================
-----------------------------------
\end{verbatim} \normalsize \end{samepage}






%--------|---------|---------|---------|---------|---------|---------|---------|
%       10        20        30        40        50        60        70        80
%\pagebreak[2]
\goodbreak
\phantomsection\addcontentsline{toc}{section}{other ppl}

\goodbreak \begin{samepage} \small \begin{verbatim}
================================================================================
O T H E R   P P L
-----------------
bla bla bla


===================================
asfasdf
-----------------------------------
-----------------------------------
\end{verbatim} \normalsize \end{samepage}

\

\goodbreak \begin{samepage} \small \begin{verbatim}
-----------------------------------
\end{verbatim} \normalsize \end{samepage}

\

\goodbreak \begin{samepage} \small \begin{verbatim}
-----------------------------------
\end{verbatim} \normalsize \end{samepage}

\

\goodbreak \begin{samepage} \small \begin{verbatim}
-----------------------------------
\end{verbatim} \normalsize \end{samepage}

\

\goodbreak \begin{samepage} \small \begin{verbatim}
-----------------------------------
\end{verbatim} \normalsize \end{samepage}

\

\goodbreak \begin{samepage} \small \begin{verbatim}
-----------------------------------
\end{verbatim} \normalsize \end{samepage}

\

\goodbreak \begin{samepage} \small \begin{verbatim}
-----------------------------------
\end{verbatim} \normalsize \end{samepage}

\

\goodbreak \begin{samepage} \small \begin{verbatim}
-----------------------------------
\end{verbatim} \normalsize \end{samepage}

\

\goodbreak \begin{samepage} \small \begin{verbatim}
-----------------------------------
\end{verbatim} \normalsize \end{samepage}

\

\goodbreak \begin{samepage} \small \begin{verbatim}
-----------------------------------
\end{verbatim} \normalsize \end{samepage}

\








%--------|---------|---------|---------|---------|---------|---------|---------|
%       10        20        30        40        50        60        70        80

\


















%-------------------------------------------------------------------------------
% N P C s
%--------









% change back to normal font size
%\normalsize

% \raggedbottom command at the top tells to :
% don't add glue spacing between segments, allow ragged bottoms instead
%
% now turn on \flushbottom in the end.
\flushbottom

%--------|---------|---------|---------|---------|---------|---------|---------|
%       10        20        30        40        50        60        70        80


\cleardoublepage

\phantomsection\addcontentsline{toc}{chapter}{monsters}
\chapter*{Monsters}
\chaptermark{monsters}

Monsters, critters, adversaries and other evil doers. The monster stats below should be seen as examples. I suggest creating personalised monsters for your encounters. Change a stat here and there to make things more interesting. A good idea is to still keep the critters at the approximate "standard" potency level though, so that your players don't kill their 300xp characters by attacking a goblin soldier. That would likely get them pissed off.

Below follows a few empty stat sheets for different styles of monsters. The first and second are for "real" intelligent adversaries, full and simplified. The rest are for unintelligent monsters, and then just mindless monsters and animals.



% Make a simple table:
%\begin{tabular}{|l|c|r|} % will define a table with lines | between and around
                         % three columns, aligned left, centre, right
                         % leave out the | characters to remove lines
%aaa & bbb & ccc \\   % add the texts aaa bbb ccc to the three columns in a row
%\hline               % add a horizontal line
%ddd & eee & fff \\   % add ddd eee fff to the next line in three columns
%ggg & hhh & iii \\   % add ggg hhh iii, this time without a horizontal line
%\hline               % another horiz line
%\end{tabular}        % close the table

% Things to notice:
%the column entries are separated by the ampersand '&'
% remember that '$' is a naughty character, and must escape
% you can get rid of the horizontal lines by deleting the \hline at the end
% you can get rid of the vertical lines by deleting the '|' symbols at the top of the tabular command



% change to smaller font size
%\small -- nope, go for \tiny

\

\goodbreak \begin{samepage} \small \begin{verbatim}
===================================
name                        (token)
-----------------------------------
str       hp abs
dex       m w r d
con       stamina
int       vision
psy       mana
per       ap
cha       xp
----------
skills
----------
spells
----------
equipment
money
===================================
\end{verbatim} \normalsize \end{samepage}

\

\goodbreak \begin{samepage} \small \begin{verbatim}
===================================
name                        (token)
-----------------------------------
str       hp abs
dex       m w r d
int       mana
psy       initiative
per       ap
----------
skills
----------
spells
----------
equipment
===================================
\end{verbatim} \normalsize \end{samepage}

\

\goodbreak \begin{samepage} \small \begin{verbatim}
===================================
name                        (token)
-----------------------------------
str       hp abs
dex       m w r d
psy       initiative
per       ap
----------
attacks
skills
===================================
\end{verbatim} \normalsize \end{samepage}

\

\goodbreak \begin{samepage} \small \begin{verbatim}
===================================
name                        (token)
-----------------------------------
str       hp abs
dex       m w r d
per       initiative, ap
attacks
-----------------------------------
\end{verbatim} \normalsize \end{samepage}

\

\goodbreak \begin{samepage} \small \begin{verbatim}
===================================
name                        (token)
-----------------------------------
hp abs
m w r d
initiative, ap
attacks
-----------------------------------
\end{verbatim} \normalsize \end{samepage}




%--------|---------|---------|---------|---------|---------|---------|---------|
%       10        20        30        40        50        60        70        80
\subsubsection*{Notes on maptool: token size}
Most standard size humanoids like humans, elves, dwarves, orcs should be set to  normal size tokens. Halflings and goblins should be set to small size, and runts should be set to tiny size.

In general:\\
1-7hp : tiny size \\
5-15hp : small size \\
10-30hp : normal size \\
20-60hp : large (2x2) \\
30-100hp : huge (3x3) \\
Haven't used anything larger yet.

Some examples:\\
All standard low hp goblin mobs are set to small size even if they only have a couple of hp, while all runt are set to tiny even if they have upwards of 10hp in some cases.\\
Wolf is set to normal, dog is set to small. Just to better represent the potential danger rather than any actual "realistic" sizes.\\
Donkeys and cave ponies are set to normal size, but horses are usually set to large 2x2.\\
Jumping spider spawns are set to tiny, while black bug spawns are set to small size. Jumping spiders will die when you hit them, but the black bugs are more difficult with their abs and behave differently with their cling mods, even if they only have 5hp.

\






%--------|---------|---------|---------|---------|---------|---------|---------|
%       10        20        30        40        50        60        70        80
\goodbreak
\phantomsection\addcontentsline{toc}{section}{animals}

\goodbreak \begin{samepage} \small \begin{verbatim}
================================================================================
A N I M A L S
-------------

The usual stuff you'll find in the forest. Wolves, bears, snakes, etc.


===================================
wolf                       (wolf02)
-----------------------------------
str  7    hp 10 abs 0
dex 10    m2 w6 r12 d20
psy  5    vision 15, dusk
per  8    initiative 12, ap 6
----------
balance 6
bite    7 dam 6
claw    5 dam 3
charge  6, mobile 1
avoid   3 yield+3
===================================
\end{verbatim} \normalsize \end{samepage}

\

\begin{samepage} \small \begin{verbatim}
===================================
large wolf                 (wolf??)
-----------------------------------
str  9    hp 14 abs 0
dex 12    m2 w7 r14 d21
psy  7    vision 15, dusk
per  9    initiative 14, ap 6
----------
balance 6
bite    9 dam 7, pen 1
claw    7 dam 3
charge  9, mobile 2
avoid   6 yield+3
consistent 1
veteran 2
===================================
\end{verbatim} \normalsize \end{samepage}

\

\goodbreak \begin{samepage} \small \begin{verbatim}
===================================
bear                        (token)
-----------------------------------
str 14    hp 25 abs 1 skin
dex  8    m1 w4 r10 d16
psy  8    vision 15
per  8    initiative 10
----------
balance 6
bite    8 dam 10, pen 1
claw    7 dam 5, knockback 1
charge  3
avoid   2 yield+2
pain threshold 4, veteran 1
===================================
\end{verbatim} \normalsize \end{samepage}

\

\goodbreak \begin{samepage} \small \begin{verbatim}
===================================
large bear                  (token)
-----------------------------------
str 19    hp 35 abs 1 skin
dex  9    m2 w5 r12 d18
psy  9    vision 15
per  8    initiative 12
----------
balance 6
bite   10 dam 12, pen 2
claw    7 dam 5, knockback 1
charge  6, mobile 1
avoid   2 yield+2
consistent 1
pain threshold 5, veteran 2
===================================
\end{verbatim} \normalsize \end{samepage}

\







%--------|---------|---------|---------|---------|---------|---------|---------|
%       10        20        30        40        50        60        70        80
%\pagebreak[2]
\goodbreak
\phantomsection\addcontentsline{toc}{section}{monsters}

\goodbreak \begin{samepage} \small \begin{verbatim}
================================================================================
M O N S T E R S
---------------

All the creepy not-quite-animals critters that spawn from the disturbed
minds of game masters and story tellers.


===================================
bug                      (beetle01)    set tiny token size
-----------------------------------
str  1    hp 2 abs 1    tohit-3 tiny target    cannot block movement
dex  6    m3 w5 fly10   tohit-6 when flying    cannot be blocked
per  2    initiative 10, ap 5
bite      5 dam 1, penetrating:
            armour affects the bug's bite skill instead of damage
            armour abs > 3 counts as armour 3 (full coverage)
no pain or low hp mods
-----------------------------------

They attack on "sight"(smell), ca 10sq. There is a 50\% chance that they
will become hostile and attack each round when a potential target is in 
their smell range. Mostly they just mill around slowly, independent 
and uncoordinated. They will however attack any non-bug who has bug 
blood/gore on them, since it signals an enemy.
Killing a bug risks spreading gore on the attacker if he's close. Use e.g: 
splatterchance : 3*overkill on a d10. Gore takes several rounds to clean 
off and requires water.
\end{verbatim} \normalsize \end{samepage}

\

\goodbreak \begin{samepage} \small \begin{verbatim}
===================================
black bug                (beetle02)    set small token size (slight reduction)
-----------------------------------
str  5    hp 5 abs 2 to hit -1 (small)
dex  6    m3 w5 r- fly15
per  5    initiative 8, smell range 6
charge      6 (fly attack gives mod-3 total)
bite cling  7 dam 4 once per round
              hold after bite (clinging to target)
              each holding (clinging) bug gives:
                  mod-3 to all actions
                  cost+1 movement per step taken
nibble      5-abs when held dam 1 penetrating
            once per round
immune to pain and low hp mods
-----------------------------------

Mostly they stand around or mill around slowly.
Normally they are independent and uncoordinated.
They will however attack any non-bug who has bug
blood/gore on them, since it signals an enemy.

The Big Black Bug can breed them to ignore certain types of non-bug creatures.

30%/r to become hostile to non-bug in smell range, depends on breeding.
Will immediately become hostile to any non-bug covered in bug blood/gore.

Will try to attack and latch on to slow down target for the others,
then nibble to death.

It takes an action (3ap) and a str-vs-str roll to remove a clinging black bug.
The bug is placed on an adjacent square chosen by the opponent.
Others have str+3 when helping the target.

Attacking a bug which is attached to a target runs the risk of also
hurting the target you're trying to assist. Any excess damage after
the bug is dead continues into the target instead.
A success+3 or better ensures that the target will not get hurt this
way even if there is excess damage left after the bug died.
Fail-3 or worse will hit the target instead of the bug.
-----------------------------------
\end{verbatim} \normalsize \end{samepage}

\

\goodbreak \begin{samepage} \small \begin{verbatim}
===================================
big black bug            (beetle03)    set large token size (2x2)
-----------------------------------
str  15    hp 30 abs 3 to hit ranged +3 (large)
dex  6     m4 w6 run8 fly20
per  8     initiative 10
charge     6 (fly attack gives mod-3 total)
big bite   8 dam 10 pen 1
             hold after bite (target held in place):
                 mod-6 to all actions, cannot move,
                 str vs str to break free, full round action,
                 friends can help lend strength and help out
nibble       8-abs when held dam 3 penetrating
             once per round
ichor spray  template area range 6 90 deg, everyone within the area:
             tohit: 8-abs
             damage: poison dam 10 1/3r until wiped clean
             cleaning it off takes 1d3+2r
             the ichor stinks and is corrosive.
             in 5r it's 50% to reduce cloth and leather abs by 1
             Max 2 spray attack per 100r, max 3 per day
immune to pain and hp/2 mods
-----------------------------------

The Big Black Bug is the hive mommy. It can can direct the normal
smaller black bugs to attack in a coordinated fashion. It can also
start spawning more black bug critters if attacked. Spawning one
black bug per round for 1d5 rounds, then one every 3 rounds.

Screech-Whistle (3ap action) will call all small black bugs to immediately 
come to the aid of the Big Black Bug. They will then attack everything in 
their path.

BBB can coordinate small black bugs in intelligent movement, timing,
and concentration of force.

BBB can breed the small black bugs to ignore certain
types of non-bug creatures.
-----------------------------------
\end{verbatim} \normalsize \end{samepage}

\

\goodbreak \begin{samepage} \small \begin{verbatim}
===================================
brown spider             (spider01)
-----------------------------------
str  3    hp 5 abs 1
dex  5    m2 w6 r10
per  4    initiative 7, 5ap
bite      5 dam 3, fast+1,
            poison 5 vs con, cumulative mod-1/2r & mp-1/2r until paralysed
            roll once when bitten instead of each round
            poison mod affects ams, dex, str, sta, movement
spin web  5 full round attack
            each successive spin attack gives a webbing:
            dex-2, move-2, mod-3 to all actions.
            break free is 2*spin vs str
            friend can clean: takes 5r to clean one web
            parrying web with weapon or shield gives mod-1 web accumulation
            on the item for every parry. Takes 1r to clear 1 web mod.
            A missed spin attack gives the opponent a mod+3 to hit the spider.
avoid     3 yield+2
-----------------------------------

Since the web attack opens the spider to counter attack they are cautious to
use them. They try mainly when there is low risk of counter attacks.
They instead use their fast bite attack to poison the target before webbing.
Spiders do not get mods from low hp and are immune to pain.
\end{verbatim} \normalsize \end{samepage}

\

\goodbreak \begin{samepage} \small \begin{verbatim}
===================================
huge spider              (spider02)  set large (2x2) token size
-----------------------------------
str 20    hp body 30 (+legs 12hp) abs 2 both (size 4sq) ranged to hit+3
dex  8    m3 w5 r8 d14
per  4    initiative 12 aggressive, will go first most of the time.
large target:
ranged to hit body mod+6, to hit legs mod+0
melee to hit body mod+3, to hit legs mod+0
-----------
double    6 (the two front legs)
avoid     6 yield+3
leg stab  5 dam 8, reach 1
            if it does 4+ damage after armour the leg is stuck,
            and the target is held. Successful hits with both legs in the
            same round also holds the target as long as both do at least
            1 damage after armour is applied.
            It takes 1a, (str|dex) vs 8 to get free from leg stab hold.
            Friends can help, spending 1a and adding str (2h) or str/2 (1h).
            While held all actions, other than freeing, are mod-3.
bite    3/8 dam 5, poison 8\\con, cumulative mod-1/2r until paralysed
            poison mod affects ams, dex, str
            attack has 8 if target is held by leg stab.
spin    3/8 full round attack
            each successive spin attack gives a webbing:
            dex-2, move-2, mod-3 to all actions.
            break free is 2*spin\\str
            friend can clean: takes 5r to clean one spin
            parrying spin with weapon or shield gives mod-1 we accumulation
            on the item for every parry. Takes 1r to clear 1 mod.
            attack has 8 if target is held by leg stab.
All attacks gain mod+1 per web on the target.
The huge spider has a thick carapace that absorbs 3 damage
Can be used to make a light weight plate armour.
\end{verbatim} \end{samepage} \goodbreak \begin{samepage} \begin{verbatim}
Tactics:
The spider will try to stab-hold a target, then web it fast.
It will not bite until it has been in battle a few rounds and not been able to
successfully web the opponent.

Legs or body, disabling:
In the beginning of the battle the spider will keep the legs wide. During this
phase melee attackers need reach 1 to attack the body of the spider, otherwise
they will only hit the legs. The only way to hit the body with a reach 0 melee
weapon during this phase is to attack after the spider has attempted a bite or
web attack earlier in the round.
Once the spider has taken 12 damage to the legs (losing 1 maneuver movement)
it will keep the legs closer, and the body can now be attacked like normal.
Melee attackers can choose to target either body or legs during this phase.
Ranged, but not reach, attacks have mod-3 to hit the legs during any phase.

Each 3 damage to the legs reduces the spider's mobility, giving 1 movement mod.
m-mod/4 w-mod/3 r-mod/2 d-mod/1.
However, the leg damage does not count to the total hp of the spider body.
-----------------------------------
\end{verbatim} \normalsize \end{samepage}

\

\goodbreak \begin{samepage} \small \begin{verbatim}
===================================
cave worm                  (worm01)   set large (2x2) token size
-----------------------------------
str 10    hp 20 abs 0 tohit+2(melee)/+4(missile) (large)
dex  3    m1
per  4    initiative 5     perception smell
bite   5 dam 8 slow-3, acid drool 20% to reduce abs on contact equipment by 1
avoid  3
-----------------------------------
size large (4sq)
\end{verbatim} \normalsize \end{samepage}

\

\goodbreak \begin{samepage} \small \begin{verbatim}
===================================
slime monster      (slimemonster01)   set large (2x2) token size
-----------------------------------
str 15    hp 40 abs 1 to hit +3 (large), tentacles hp 10 abs 1 reg 2/r
dex  3    m1 w2
per  2    initiative 5
4x tentacle grapple  5
4x tentacle crush    - damage as grapple
quadruple            3 (triple 6, double 9)
4x tentacle swat     7 dam 5
gobble up            - dam 10, need to be grappled for 3r
                       consecutive damage 1/r in stomach
-----------------------------------
\end{verbatim} \normalsize \end{samepage}

\

\goodbreak \begin{samepage} \small \begin{verbatim}
===================================
BiteRunners        (toothcritter03)
-----------------------------------
str  12   hp 12 abs 0
dex  12   m8 w12 r16 d28
psy   5   vision 20 night
per   6   initiative 16   ap 6
----------
bite         6 dam 6
avoid        4 yield+4
synchronise  5
balance      3
===================================

High maneuver instead of mobile skill.
They attack in small groups of 3-6 individuals usually.
They attack and circle in rapid movement, usually just 1 attack per round.
Keep at around 3-4 steps out, then move in, bite, move out.
Preferably they synchronise attacks from different sides.
Circling, spreading and synchronising.
\end{verbatim} \normalsize \end{samepage}

\

\goodbreak \begin{samepage} \small \begin{verbatim}
===================================
iffygriff adult       (demonbird01)             set token size to large 2x2
-----------------------------------
str 12    hp 20 abs 1 (hide & feathers)
dex  8    m 3 w 5 r 10 fly 20
psy  4    vision 25
per 10    initiative 10
----------
mobile    2
sneak     6
avoid     6
charge    4
balance   3
beak peck 8 dam 6
stench aura radius: (6-dist) vs con or target will loose 1 action to gag reflex.
                    target gets mod+1 per previous success, roll first in round.
----------
Will protect it's young to the death.
If no young around, then flee at hp$\le$6
or after gaining any hit > 8hp.
Pain threshold dam/4
===================================

One or a pair of adults often has 1d4 chicks around and 1d4 eggs in the nest.
The chicks are used to circle the targets while the adults charge.
\end{verbatim} \normalsize \end{samepage}

\

\goodbreak \begin{samepage} \small \begin{verbatim}
===================================
iffygriff chick       (demonbird01)             set token size normal 1x1
-----------------------------------
str  6    hp 10 abs 0 (hide & feathers)
dex  8    m 2 w 4 r 8 fly 15
psy  3    vision 25
per  8    initiative 12
----------
mobile    1
sneak     8
avoid     6
balance   3
beak peck 6 dam 4
stench aura radius: (4-dist) vs con or target will loose 1 action to gag reflex.
                    target gets mod+1 per previous success
----------
Will flee at hp=4
pain threshold dam/3
===================================

The nest contains 1d4 iffygriff eggs, enc 2.0 each.
They can be used to scare off ghosts, spirits etc due to the horrible stench,
which even permeates the etheric. Lasts only for ~1week,
then looses it's potency. Worth about 1 gold each in Trade Town.
\end{verbatim} \end{samepage} \goodbreak \begin{samepage} \begin{verbatim}
info about iffygriffs: roll monsterology
+0 : bird monster, evil, foul, dangerous
+1 : fast and can fly
+2 : they are very foul fowl, requires strong stomach to stand close.
+3 : will protect it's young
+3 : likes to flank prey using the young.
+4 : their eggs are quite valuable for wizards and the like
+6 : their eggs can keep ghosts and spirits away

When first encountering an iffygriff area the Heroes have some chance of
realising it's an iffygriff around:

entering the area:
per+3 : notices strange call and foul stench
monsterology : it's an iffygriff

seeing the monster:
monsterology+3 : it's an iffygriff
\end{verbatim} \normalsize \end{samepage}

\







%--------|---------|---------|---------|---------|---------|---------|---------|
%       10        20        30        40        50        60        70        80
%\pagebreak[1]
\goodbreak
%\phantomsection \addcontentsline{toc}{subsection}{lizard men}
\phantomsection \addcontentsline{toc}{section}{lizard men}

\goodbreak \begin{samepage} \small \begin{verbatim}
================================================================================
L I Z A R D   M E N
-------------------

A group of species and races that form their own sub society in the world.
Different species and races have different functions in their society.
They speak lizard tongue.


===================================
lizard hound               (lizard)  set small token size
-----------------------------------
str  6    hp 6 abs 0
dex 10    m3 w6 r12 d20
per  6    initiative 10
2x claw   8 dam 3 (2x claw when charging) (1 claw 1 bite in melee)
bite      6 dam 6 (only when already in melee)
charge   10
tackle    6 chase down then 1 claw tackle attack
-----------------------------------
\end{verbatim} \normalsize \end{samepage}

\

\goodbreak \begin{samepage} \small \begin{verbatim}
===================================
lizardman scout       (lizardman02)  set small token size
-----------------------------------
str  3    hp 4 abs 1
dex  7    m2 w4 r8 d16
int  3    vision 20
psy  3    mana 3
per  7    initiative 12 (quick to flee)
----------
mobile       3 (m-0 w-0 r-3 d-6)
claw         6 dam 1, 2x mod-2
short sword  4 dam 5, abs 8
short bow    5 dam 4, quick 1r mod-3, shot 2r mod-0, aimed 3r mod+3
               range 12, short 6 mod+3, long 18 mod-3,
               vlong 24 mod-6 dam-1, extreme 36 mod-9 dam-2
avoid        8 yield+4
sneak        8
find         6
===================================
\end{verbatim} \normalsize \end{samepage}

\

\goodbreak \begin{samepage} \small \begin{verbatim}
===================================
lizardman fighter     (lizardman01,10,13)  set small token size
-----------------------------------
str  4    hp 6 abs 1
dex  5    m1 w4 r8 d12
int  2    vision 15
psy  4    mana 2
per  5    initiative 10 (careful)
----------
quick         3 (-0 -0 -3 -6)
mobile        3 (m-0 w-0 r-3 d-6)
claw          4 dam 1, 2x mod-2
sword         5 dam 6, abs 10,
small shield  5 abs 10, parry+2
avoid         6 yield+4
sneak         6
===================================
\end{verbatim} \normalsize \end{samepage}

\

\goodbreak \begin{samepage} \small \begin{verbatim}
===================================
lizardman warrior      (lizardman04,11,12)
-----------------------------------
str  8    hp 12 abs 2
dex  6    m1 w3 r6 d10
int  3    vision 10
psy  5    mana 4
per  5    initiative 8 (aggressive)
----------
quick     2
mobile    2
claw      6 dam 2, 2x mod-2
sword     7 dam 6 abs 13, toparry-1, str 4  (str bonus is toparry-1, fast)
shield    6 abs 15, parry+3
avoid     4 yield+3
veteran   4
charge    6
===================================
\end{verbatim} \normalsize \end{samepage}

\

\goodbreak \begin{samepage} \small \begin{verbatim}
===================================
lizardman chieftain   (lizardman05,14,15)
-----------------------------------
str  12   hp 22 abs 3
dex   8   m1 w3 r6 d8
con  10   stamina 20
int   6   vision 10
psy   6   mana 10
per   6
cha   4   initiative 8+1d4
----------
quick      4 (-0, -0, -2, -5)
mobile     2
claw       8 dam 3, 2x mod-2
2h sword  10 dam 10, abs 18, toparry-2, fast+1, reach 1 mod-6
             str 6 (str bonus is toparry-2, fast+1)
accurate   2  (success+3 2 roll choose best, success+6 3 roll choose best)
avoid      6 yield+2
veteran    6
charge     6
scary      5
wrestle    4
sneak      5
===================================
\end{verbatim} \normalsize \end{samepage}

\

\goodbreak \begin{samepage} \small \begin{verbatim}
===================================
lizardman worker      (lizardman03)  set small token size
-----------------------------------
str  2    hp 3 abs 0
dex  5    m2 w4 r6 d8
per  4    initiative 9 (flee)
claw      4 dam 1, 2x mod-2
knife     3 dam 3, parry-2
spear     3 dam 4, abs 6
avoid     6 yield+4
sneak     8
-----------------------------------
\end{verbatim} \normalsize \end{samepage}






%--------|---------|---------|---------|---------|---------|---------|---------|
%       10        20        30        40        50        60        70        80
\goodbreak
\phantomsection\addcontentsline{toc}{section}{goblins}


\goodbreak \begin{samepage} \small \begin{verbatim}
================================================================================
G O B L I N S
-------------

It is rare, I admit, but not as rare as some seem to think, that boss goblins,
shamans, and old crones have the power to put curses on people with their last
dying words. The goblin tribes have an unfortunate knack for keeping the old
black witchcraft alive and flourishing.


===================================
goblin bandit           (goblin05)
-----------------------------------
str  3    hp 4 abs 0
dex  5    m1 w3 r5 d7
int  2    vision 12 dusk (set goblin average)
psy  2    ap 4
per  6    initiative 7, ap 3, chicken
----------
sneak         7
charge        3
back stab     3
avoid         4 yield+4
===================================
alternatives:
small club    6 dam 2/4, abs 6, parry-1, toavoid+1
small spear   5 dam 3/4, pen 0/1, abs 4, parry-1/+1, reach 1 mod-5
small shield  5(3) abs 8 parry+2 crappy quality
                Ranged attacks mod-1 when in the way.
                Hiding behind it (primary action) ranged mod-2
short bow     4 dam 3, reload bow: quick 1r mod-3, norm 2r mod-0, aim 3r mod+1
                range 12, short 6 mod+3, long 18 mod-3,
                str 2 (no str bonus)
knife         3 dam 2, abs 5, crappy quality
\end{verbatim} \normalsize \end{samepage}

\

\goodbreak \begin{samepage} \small \begin{verbatim}
===================================
goblin fighter           (goblin02,05)   set small token size
-----------------------------------
str  3    hp 5 abs 0
dex  6    m1 w3 r5 d7
int  2    vision 10 dusk
psy  3    mana 0
per  5    initiative 8, ap 4, chicken
----------
sneak         5
charge        4
avoid         5 yield+4
===================================
small club    7 dam 2/4, abs 6, parry-1, toavoid+1
small spear   6 dam 3/4, pen 0/1, abs 4, parry-1/+1, reach 1 mod-5
small sword   5 dam 4, abs 6
small shield  6(4) abs 8 parry+2
                Ranged attacks mod-1 when in the way.
                Hiding behind it (primary action) ranged mod-2
short bow     5 dam 3, reload bow: quick 1r mod-3, norm 2r mod-0, aim 3r mod+1
                range 12, short 6 mod+3, long 18 mod-3,
                str 2 (no str bonus)
knife         4 dam 2, abs 3, parry-2, toparry-2, toavoid-1
\end{verbatim} \normalsize \end{samepage}

\

\goodbreak \begin{samepage} \small \begin{verbatim}
===================================
goblin warrior           (goblin07)   set small token size
-----------------------------------
str  4    hp 6 abs 1
dex  7    m1 w3 r5 d7
int  3    vision 12 dusk
psy  3    mana 0
per  6    initiative 9, ap 5, chicken
----------
sneak         4
charge        6
axe           8 dam 7, abs 7, parry-3, toparry-1, toavoid+1
short sword   7 dam 5, abs 8
small shield  6 abs 8, parry+2
avoid         6 yield+4
leather armour abs 1
===================================
\end{verbatim} \normalsize \end{samepage}

\

\goodbreak \begin{samepage} \small \begin{verbatim}
===================================
goblin champion          (goblin09)   set small token size
-----------------------------------
str  6    hp 8 abs 1
dex  8    m2 w3 r5 d8
int  3    vision 12 dusk
psy  4    mana 4
per  6    initiative 11, ap 7, careful
----------
quick         2
mobile        2
axe           9 dam 7, abs 8
shield        8 abs 12, parry+3
avoid         6 yield+4
sneak         3
charge        6
leather armour abs 1
===================================
\end{verbatim} \normalsize \end{samepage}

\

\goodbreak \begin{samepage} \small \begin{verbatim}
===================================
goblin soldier           (goblin01)   set small token size
-----------------------------------
str  4    hp 6 abs 1 (leather)
dex  5    m1 w3 r5 d7
int  2    vision 10 dusk
psy  3    mana 0
per  4    initiative 7
----------
short sword   6 dam 5, abs 8
small shield  7 abs 8, parry+2
avoid         5 yield+3
veteran       1
sneak         3
charge        3
phalanx       x
leather armour abs 1
===================================
Alternative Weapons (spear+shield or bow+knife4)
small spear   6 dam 4/5, abs  6, parry+0/+1, reach 1 mod-3, fast+0/+1
(1h/2h)         str 3 (max +1 str bonus, extra str bonus are penetrating)
short bow     7 dam 4, reload bow: quick 1r mod-3, norm 2r mod-0, aim 3r mod+3
(no shield)     range 12, short 6 mod+3, long 18 mod-3,
                vlong 24 mod-6 dam-1, extreme 36 mod-9 dam-2
                str 2 (no str bonus)
\end{verbatim} \normalsize \end{samepage}

\

\goodbreak \begin{samepage} \small \begin{verbatim}
===================================
goblin elite soldier    (goblin01b)  set small token size
-----------------------------------
str  5    hp 8 abs 2 (thin plate)
dex  6    m1 w3 r5 d7
int  3    vision 10 dusk
psy  4    mana 0
per  5    initiative 9
----------
quick         1
mobile        1
sword         8 dam 6, abs 10
shield        8 abs 12, parry+3
avoid         5 yield+3
veteran       2
sneak         3
charge        5
phalanx       x
accurate      1
leather armour abs 1
===================================
Alternative Weapons (spear+shield or bow+knife4)
spear          8  dam  5/6, abs  6, parry+0/+1, reach 1 mod-3,
(1h/2h)           str 3 (max +2 str bonus, extra str bonus are penetrating)
                  str+6 can instead give fast+1 when 2h grip
                  finesse-4
bow            8  dam 5, quick shot 3, sniper 3
(no shield)       range 16, short 8 mod+3, long 24 mod-3,
                  vlong 32 mod-6 dam-1, extreme 48 mod-9 dam-2,
                  str 3 (no str bonus)
\end{verbatim} \normalsize \end{samepage}

\

\goodbreak \begin{samepage} \small \begin{verbatim}
===================================
goblin shaman            (goblin04)   set small token size
-----------------------------------
str  2    hp 10 abs 0
dex  2    m1 w2 r3 d4 (very fat)
con  4    stamina 3
int  6    vision 12 dusk
psy  5    mana 10
per  4
cha  2
----------
short sword   3(4) dam 4(5), abs 8,       (mod-1 (weak))
rot           6 cast 3r 1m, dam 4 +1/m 1hp/r, range 8 +2/m, penetrating
poison gas    4 cast 5r 2m, dam 1/r, size 3x3 +2x2/m, duration 10r +5r/m
powercasting  3
sneak         4
===================================
\end{verbatim} \normalsize \end{samepage}

\

\goodbreak \begin{samepage} \small \begin{verbatim}
===================================
goblin runt          (goblinrunt01)   set small token size
-----------------------------------
str 1     hp 2 abs 0
dex 4     m1 w2 r4 d6
int 1     vision 10 dusk
psy 1     mana 1
per 4     initiative 7
----------
claw      4 dam 1, 2x -3
bite      4 dam 1
avoid     5 yield+3
sneak     6
===================================
\end{verbatim} \normalsize \end{samepage}

\

\goodbreak \begin{samepage} \small \begin{verbatim}
===================================
goblin runt knifer   (goblinrunt02)   set small token size
-----------------------------------
str 2     hp 3 abs 0
dex 5     m2 w3 r4 d6
int 1     vision 10 dusk
psy 1     mana 1
per 6     initiative 9
----------
claw      4 dam 1, autodouble 2x -3
bite      4 dam 2
knife     4 dam 2, abs 3, fast+1
avoid     6 yield+3
sneak     8
===================================
\end{verbatim} \normalsize \end{samepage}

\








%--------|---------|---------|---------|---------|---------|---------|---------|
%       10        20        30        40        50        60        70        80
%\pagebreak[2]
\goodbreak
\phantomsection\addcontentsline{toc}{section}{orcs}

\goodbreak \begin{samepage} \small \begin{verbatim}
================================================================================
O R C S
-------


===================================
orc fighter                 (orc01,06)
-----------------------------------
str 10    hp 16 abs 0
dex  6    m2 w4 r6 d8
con  8    vision 12 dusk
int  4    mana 5
psy  4    initiative 8
per  5    ap 4
----------
h vl 2h club   6 dam 13(12), abs 18, str 7, slow-1 knockback 1 16vs
bite           4 dam 3
punch          6 dam 3
avoid          4 yield+3
charge         4
tackle         4
wrestle        4
veteran        4 (1,+3)
===================================
\end{verbatim} \normalsize \end{samepage}

\

\goodbreak \begin{samepage} \small \begin{verbatim}
===================================
orc warrior                  (orc02/09)
-----------------------------------
str 11    hp 20 abs 1
dex  7    m2 w4 r6 d8
con  9    vision 12 dusk
int  4    mana 5
psy  4    initiative 9
per  6    ap 5
----------
focus          2
veteran        6 (incl race bonus)
vlarge club    7 dam 10(8), abs 15, str 5, finesse-3
shield         5 abs 13, parry+3, str 4
bite           5 dam 3
punch          7 dam 3
avoid          4 yield+3
charge         5
balance        2
wrestle        5
tackle         5
leather armour abs 1
===================================
\end{verbatim} \normalsize \end{samepage}

\

\goodbreak \begin{samepage} \small \begin{verbatim}
===================================
orc soldier                  (orc04/07/08/11)
-----------------------------------
str  9    hp 16 abs 2
dex  6    m2 w4 r6 d8
int  3    vision 10 dusk
psy  3    mana 2
per  5    initiative 7
----------
focus          3
veteran        6 (incl race bonus)
axe/spear      8
shield         7 abs 13, parry+3, str 4
bite           4 dam 3
punch          4 dam 3
avoid          4 yield+3
charge         4
wrestle        5
tackle         5
crappy plate armour abs 2
===================================
alternatives: axe+shield / spear+shield
orc war axe         dam 9, abs 12, parry-3, toparry-1, toavoid+1, finesse-2
                    str 7 (max +6 str bonus)
                    swing: slow-1 mod-1 dam+3 toavoid+2
orc war spear       dam 7/8, pen 1/2, abs 10, parry-1/+1, reach 1 mod-3
1h/2h               str 7 (max +3 str damage bonus, max+3 penetrating bonus)
                    finesse-3/-5
orc war shield      abs 14, parry+3,
                    str 7, or slow-1 str 4
                    Ranged attacks mod-2 when in the way.
                    Hiding behind it (3ap) ranged mod-5
                    tackle mod+3
\end{verbatim} \normalsize \end{samepage}

\

\goodbreak \begin{samepage} \small \begin{verbatim}
===================================
orc champion                 (orc10/05)
-----------------------------------
str 12    hp 22 abs 2
dex  8    m2 w4 r6 d8
con 10    stamina 14
int  5    vision 15 dusk
psy  5    mana 8
per  7    (@ xp ~334)
cha  3    initiative 8+d4
----------
quick       3 (-0 -0 -3 -6)
mobile      3 (-0 -0 -3 -6)
veteran     9 (incl race bonus)
2h axe      9 dam 16(12), abs 12, parry-3, toparry-2, str 6, finesse-3
              fast+1 if dam12 with maneuver fast strength
avoid       7 yield+3
slugger     4
charge      6
tackle      6
wrestle     6
roar        6
scary       6
partial plate armour abs 2
===================================
alternatives:
2h flail    8 dam 11(10), abs 10, parry-5, toparry-3, snag-3, str 7, finesse-6
2h sword    8 dam 12(10), abs 16, toparry-1, str 6, finesse-6
\end{verbatim} \normalsize \end{samepage}

\

\goodbreak \begin{samepage} \small \begin{verbatim}
===================================
orc chieftain            (orc14)
-----------------------------------
str 14    hp 22 abs 5 (0 from direct ahead)
dex  8    m2 w4 r6 d8
con 10    stamina 14
int  7    vision 15 dusk
psy  6    mana 8
per  8    (@ xp ~334)
cha  6    initiative 8+d4
----------
quick         3 (-0 -0 -3 -6)
mobile        3 (-0 -0 -3 -6)
veteran       9 (incl race bonus)
charge        6
tackle        6
wrestle       6
roar          6
scary         6
slugger       3
command       5
avoid         7 yield+3

partial armour      abs5 except from direct front where abs0
                    behaves as plate armour with tank3

heavy spear         dam 10(7) pen 2, abs 8, parry-3, reach 1 mod-3,
1h                  fast+1 at dam10pen0 with "fast strength"
                    str 5 (max +3 str bonus, extra str bonus are penetrating)
                    finesse-3

tower shield        abs 20, parry +5, str 8, slow-1
                    fast+1 with "fast strength" (push-3) (since slow and fast)
                    Ranged attacks mod-4 when in the way.
                    Hiding behind it (primary action) ranged mod-5
                    tackle mod+3

===================================
alternatives:
2h flail    8 dam 11(10), abs 10, parry-5, toparry-3, snag-3, str 7, finesse-6
2h sword    8 dam 12(10), abs 16, toparry-1, str 6, finesse-6
\end{verbatim} \normalsize \end{samepage}

\

\goodbreak \begin{samepage} \small \begin{verbatim}
===================================
orc shaman                  (orc03)
-----------------------------------
str  8    hp 14 abs 2
dex  4    m w r d
con  8    stamina 12
int  6    vision  12 dusk
psy  6    mana 10
per  5    initiative 7
cha  3
----------
focus            3
veteran          6 (incl race bonus)
heavy long club  6 dam 9(8), abs 18, str 5, parry+1, reach-3, finesse-3
avoid            4 yield+2
scary            6
----------
magic        4
black bolt   7 cast 1r 1m, dam 5 +2/m, penetrating, range 10 +5/m, long -2dam
fire ball    4 cast 3r 2m, dam 5 +1/m, range 10 +5/m, area 3x3 +1x1/m
heal         5 cast 2r 1m, heal 1 +1/m, time 1r/hp, range contact
drain        4 cast 2r 0m, psy/psy +3/m, range contact, drains diff mana points
----------
very heavy leather armour abs 2
===================================
\end{verbatim} \normalsize \end{samepage}

\





%--------|---------|---------|---------|---------|---------|---------|---------|
%       10        20        30        40        50        60        70        80
%\pagebreak[2]
\goodbreak
\phantomsection\addcontentsline{toc}{section}{svartfolk}

\goodbreak \begin{samepage} \small \begin{verbatim}
================================================================================
S V A R T F O L K ,   o t h e r
-------------------------------


===================================
small troll               (troll01)    (set token size large (2x2))
-----------------------------------
str 15    hp 30 abs 1 (skin)           large target ranged to hit mod+3
dex  8    m3 w5 r7 d8
con 10    stamina 20
int  2    vision 15 dusk
psy  7    mana 4
per  5    initiative 10
cha  1    ap 4
pain threshold 5
regenerates    1hp/5r
----------
veteran     3
brawling    7
    punch   dam 6
    kick    dam 7
double      5
throw       5
avoid       3 yield+2
charge      6
tackle      5
wrestle     5
roar        - 15 vs (psy+str)
            success+0: target cannot attack melee 1d3 rounds & initiative-5
            success+3: target must flee, until successful psy roll (1/r)
===================================
alternatives:
huge club   5 dam 12/15, abs 25, parry-3, toparry-3,
1h/2h         slow-2, finesse-0, str 15/12, reach 1 mod-3, swipe
              knockback 1
large rock  5 dam 8,
thrown        range 6, short 3 mod+3, long 12 mod-3,
              str 12
\end{verbatim} \normalsize \end{samepage}

\

\goodbreak \begin{samepage} \small \begin{verbatim}
===================================
basic ogre                 (ogre01)  (set token size large (2x2))
-----------------------------------
str  20   hp 30 abs 1 (skin) large target  ranged tohit+3
dex   7   m3 w6 r9 d12
con  13   stamina 25
int   3   vision
psy   8   mana
per   6   initiative 10
cha   1   ap 5
----------
reach 1
veteran 6 pain threshold 5
punch
kick
claw sweep
wrestle

skills
----------
spells
----------
equipment
money
===================================
\end{verbatim} \normalsize \end{samepage}

\








%--------|---------|---------|---------|---------|---------|---------|---------|
%       10        20        30        40        50        60        70        80
%\pagebreak[2]
\goodbreak
\phantomsection\addcontentsline{toc}{section}{dead and undead}

\goodbreak \begin{samepage} \small \begin{verbatim}
================================================================================
U N D E A D ,   Z O M B I E S ,   E T C
---------------------------------------

Braaiiiiin...
\end{verbatim} \normalsize \end{samepage}

\

\goodbreak \begin{samepage} \small \begin{verbatim}
===================================
zombie (slow)            (zombie01)
-----------------------------------
str  5    hp 15 abs 0
dex  2    m1 w2 r- d-
per  3    initiative 1
grapple     5 dam 1, penetrating 1, mod+1/zombie in base contact with target
              break free on (dex|str) vs str, mod-1 / zombie in bc w/ target
bite        5 dam 3, when grappled
tear apart  5 dam 5, penetrating 3, when grappled
-----------------------------------
\end{verbatim} \normalsize \end{samepage}

\

\goodbreak \begin{samepage} \small \begin{verbatim}
===================================
skeleton grunt         (skeleton02)
-----------------------------------
str  6    hp 10 abs 0
dex  5    m1 w3 r- d-
psy  3    vision 15
per  3    initiative 6, ap 4
----------
immune to pain
no low hp mods (Black Knight 3)
takes half hp from thrust/pierce attacks (arrows, spears, etc)
----------
spear     5 dam 5/6, pen 1/2, abs 8, reach 1 mod-3
sword     5 dam 6, abs 10
shield    8 abs 10, parry+3
-----------------------------------
healthy: >5hp : 3ap M, NO parry, just attack
damaged: will parry
===================================
aka: skeleton grunt, skeleton soldier
\end{verbatim} \normalsize \end{samepage}

\

\goodbreak \begin{samepage} \small \begin{verbatim}
===================================
skeleton guard         (?)
-----------------------------------
str  9    hp 15 abs 2
dex  5    m2 w4 r6 d8
psy  3    vision 15
per  9   initiative 8, 8ap
----------
immune to pain
no low hp mods (Black Knight 3)
takes half hp from thrust/pierce attacks (arrows, spears, etc)
----------
glave      8 dam 9, pen 2, abs 8, parry-3, slow-1, toavoid+1
             reach 0 mod-1, reach 1 mod-0, reach 2 mod-6,
avoid      7 yield+3
-----------------------------------
healthy >5hp : 8ap M, NO parry, just attack  :  2x attack glave
damaged: 8ap will avoid  :  1/2x avoid / 1x attack glave
glave attack, avoid defence
===================================
\end{verbatim} \normalsize \end{samepage}

\

\goodbreak \begin{samepage} \small \begin{verbatim}
===================================
evil skeleton          (skeleton03)
-----------------------------------
str  7    hp 20 abs 0
dex  5    m1 w3 r5 d8
psy  5    vision 15
per  4    initiative 8, ap 5
----------
immune to pain
no low hp mods (Black Knight 3)
takes half hp from thrust/pierce attacks (arrows, spears, etc)
----------
double      7
2x sword    8 dam 7(6), abs 10
===================================
\end{verbatim} \normalsize \end{samepage}

\

\goodbreak \begin{samepage} \small \begin{verbatim}
===================================
mummy                     (mummy01)
-----------------------------------
str  8    hp 10 abs 1
dex  4    m1 w3 r- d-
psy  5    vision 15
per  4    initiative 5
----------
fist        6 dam 3
wrestle     5
crush       dam str\\str diff per round when grappled
life drain  psy\\psy drain diff mana/rnd 1try/rnd, when in base contact
===================================
\end{verbatim} \normalsize \end{samepage}

\

\goodbreak \begin{samepage} \small \begin{verbatim}
===================================
evil mummy                (mummy02)
-----------------------------------
str 10    hp 20 abs 3
dex  6    m1 w3 r5 d7
int  7    vision 20
psy 10    mana 10
per  6    initiative 8
----------
terror howl 8 cast 1a, 1 mana, psy\\psy, duration 2*diff (psy\\psy),
              success+0: cannot attack. success+3: flee
              target gets psy+3 cumulative on successive howls.
focus       3
double      5
long sword  8 dam 10(8), abs 12
shield      8 abs 20, parry+3
drain aura  - psy\\psy drain 1 mana/rnd
grapple     8
crush       - dam str\\str diff per round when grappled
life drain  - psy\\psy drain diff mana/rnd 1try/rnd, when in base contact
===================================
\end{verbatim} \normalsize \end{samepage}








%--------|---------|---------|---------|---------|---------|---------|---------|
%       10        20        30        40        50        60        70        80
%\pagebreak[4]
%\goodbreak
\clearpage
\phantomsection\addcontentsline{toc}{section}{demons}

\goodbreak \begin{samepage} \small \begin{verbatim}
================================================================================
D E M O N S
-----------

Horrible visitors from otherworlds and different dimensions, bla bla bla...
\end{verbatim} \normalsize \end{samepage}

\

\goodbreak \begin{samepage} \small \begin{verbatim}
Grey Stalkers:
Live in hive packs, nomadic. One big and 4-8 small ones is normal.
Larger packs exist, but are uncommon. Clever, can speak a bit of ancient.
Aggressive and sneaky. Send in a small one, scout, bring in the large,
then it brings in the rest of the hive pack. They also have hive vision,
so one small scout is enough for the large to cast explosion spells etc.

It is possible to talk and deal with the Grey Stalkers. They are interested in
exotic foods, in strange knowledge, in rare objects (for them). They don't care
about precious metals, jewels,etc.
\end{verbatim} \normalsize \end{samepage}

\

\goodbreak \begin{samepage} \small \begin{verbatim}
===================================
big grey stalker         (demon10a)
-----------------------------------
str 16    hp 30 abs 3, set size 2x2
dex  7    m2 w6 r12
int  6    vision 30 arc 360 night
psy 12    mana 12
per  8    initiative 10, ap 6
----------
balance        6
sneak          10
veteran        3
ancient        5
charge         6
blink          blink out 1a, blink in 1a, out max 5r
bite           6 dam 8 slow-1
horn charge    8 dam 8+dst/3, acts like tackle
leg stab       9 dam 6 fast+1
----------
summon         8 1r 3m +1stalker/3m  summons a grey stalker
                 duration is until the big grey stalker dies
explosion      8 1r 2m dam 5 +1/r penetrate 1
                 range 10 radius 1 +1/2m
portal home    6 10r 6m duration 10r   creates a one way portal home.
===================================
\end{verbatim} \normalsize \end{samepage}

\

\goodbreak \begin{samepage} \small \begin{verbatim}
===================================
grey stalker             (demon10b)
-----------------------------------
str  8    hp 12 abs 2
dex  9    m3 w8 r14
int  5    vision 30 arc 360 night
psy 10    mana 8
per  7    initiative 12, ap 6
----------
balance        3
sneak          12
ancient        3
veteran        2
charge         6
blink          blink out 1a, blink in 1a, out max 3r
bite           5 dam 6 slow-1
horn charge    7 dam 6+dst/3, acts like tackle
leg stab       8 dam 4 fast+1
----------
summon         4 100r 8m   summons one big grey stalker
                 duration is permanent
teleport       6 1r 2m teleport to the big grey stalker
===================================
\end{verbatim} \normalsize \end{samepage}








%--------|---------|---------|---------|---------|---------|---------|---------|
%       10        20        30        40        50        60        70        80
%\pagebreak[4]
%\goodbreak
\clearpage

\goodbreak \begin{samepage} \small \begin{verbatim}
Blackscale Antorgs:
The Blackscale Antorg critters usually live underground, and are rarely seen.
They live in social gangs, where they can metamorphose to suit their role.
It is not known if they are demons or not, but they have been found in more than
one world reality. This of course makes them more demon than monster...
They dig for copper, and keep slaves of various races to work the digs sites.
They travel through magical portals which the antorg masters create.
\end{verbatim} \normalsize \end{samepage}

\

\goodbreak \begin{samepage} \small \begin{verbatim}
===================================
Blackscale antorg master (demon14a)
-----------------------------------
str 15    hp 30  abs 5 (melee&ranged) tohit+3    set size large (2x2)
dex  5    m1 w2 r4 d6
int 12    vision 30 magical
psy 12    mana 30
per 10    initiative 5
---skills-------
bite             8 dam 7
thunder stick    7 dam 4 +2/charge parry-3  25charges
                   stun 6 +6/charge
                   charge effects add to all effects with each charge
lightning rod    6 dam 5 pen 2 range 10, 15 charges
                   short 5 mod+3 dam+2, long 15 mod-3 pen 0
                   normal shot 1a mod-0,
                   quick shot 1a fast+1 mod-3
                   aimed shot 1r mod+3
ancient          5
common           3
dwarvish         6
svartlingo       6
literate         8
counting         6
\end{verbatim} \end{samepage} \goodbreak \begin{samepage} \begin{verbatim}
---magic-------
magic 6
power casting 3
force bolt   7  cast 1a 1m, dam 5, range 10, long -2dam, penetrating 1
               cast time 1a means it can be cast multiple times per round
               with the normal -3 cumulative mod per action
               int 7, psy 5
fire storm   7  cast 1r 1m, dam 5 +2/m, range self, radius 1 +1/m,
                duration 3r +3/m, damage each round,
                caster is immune to fire for the duration
ward shield  8 cast 2r 2m, range touch, duration 5r +3r/m
               charge 5 +3/m (pay on cast)
               Absorbs incoming damage before it strikes the character
               or his equipment. Each absorbed damage point costs
               one charge from the spell.
\end{verbatim} \end{samepage} \goodbreak \begin{samepage} \begin{verbatim}
force wall   8 cast 1r 1m, size 3 +1/m, duration 5r +3/m, range 5sq +3/m
               psy vs str +3/m to stop anyone passing the blocked passage
               missiles and magic will pass through the force wall with a
               mod-3 to hit or to cast respectively.
ward wall    8 cast 1r 1m, size 1sq +1/m, duration 5r +3/m, range 5sq +3/m
               charge 3 +2/m
               psy vs str +3/m to stop anyone passing the blocked square
               absorbs incoming magic at 1 charges per mana of incoming spell.
               absorbs incoming missiles 3dam per charge. at least one charge.
               absorbs incoming melee strikes at 5dam per charge. 1 charge min.
sleep gas    7 cast 3r 2m, range 10sq +5/m, area radius 4 +1/m, duration 5r +3/m
               sleep gas. mods con-3 each round. fall asleep when con=0
               when gas gone recovers 1 con each round. roll below con
               each round to see if wakes up.
blind        6 cast 1r 1m, range 15 +5/m, duration 5r +2/m
               target gets vision=1 for the duration.
heal         8 cast 3r 1m, heal 5 +3/m, time 2hp/r, range contact
mana transfer 5 cast 2r, transfer 3m/r +3/m, range 5,
                willing target or object only, in/out
                Transfer is full round action, can only maneuver
                and do no actions for the duration.
                Max psy mana per cast.
drain        7 cast 2r 0m, psy vs psy +3/m, range contact,
               drains diff mana points
slow spells (10+r):
mass teleport, open gate vortex, close gate vortex
\end{verbatim} \end{samepage} \goodbreak \begin{samepage} \begin{verbatim}
--equipment--------
thunder rod
lightning staff

===================================
hp 30 abs5   : 30
mana 30 :30
no pain
\end{verbatim} \normalsize \end{samepage}

\

\goodbreak \begin{samepage} \small \begin{verbatim}
===================================
Blackscale antorg fighter (demon14b/c)
-----------------------------------
str 8     hp 15 abs 3
dex 8     m1 w2 r5 d8
int 8     vision 30 magical
psy 6     mana 10
per 8     initiative 7
---skills-------
---spells-------
magic          3
powercasting   3
heal           7 cast 1r 1m, heal 1 +1/m, time 3hp/r, range 5
ward flash     8 cast 1a 1m, range personal, duration 1 attack and 1r,
                 absorbs 5dam +3dam/m on the first attack that hits the caster.
                 It is only active during the round it has been cast.
                 Absorbs incoming damage before it strikes the character
                 or his equipment.
---equipment-------
bite             8 dam 6 toparry-3 toavoid-3
thunder stick    7 dam 4 +2/charge parry-3  25charges
                   stun 6 +6/charge
                   charge effects add to all effects with each charge
lightning rod    6 dam 5 pen 2 stun 3, range 10, 15 charges
                   short 5 mod+3 dam+2, long 15 mod-3 pen 0
                   normal shot 1a mod-0,
                   quick shot 1a fast+1 mod-3
                   aimed shot 1r mod+3
===================================
hp 15 abs3      : 15
mana 10 : 10
no pain
ammo ts : 25
ammo lr : 15
\end{verbatim} \normalsize \end{samepage}

\

\goodbreak \begin{samepage} \small \begin{verbatim}
===================================
Blackscale antorg slave hunter (demon14b/c)
-----------------------------------
str 12     hp 20 abs 4
dex 6      m2 w4 r6 d10
int 8      vision 30 magical
psy 8      mana 15
per 10     initiative 10
---skills-------
quick           2
mobile          2
quick draw	6	thunder stick and lightning rod in quick draw slots
claw slash	11 dam7 fast+1
claw grab	8 dam4
wrestle	        6
---spells-------
magic           3
powercasting    3
heal            8 cast 1r 1m, heal 1 +1/m, time 3hp/r, range 5
ward flash      11 cast 1a 1m, range personal, duration 1 attack and 1r,
                 absorbs 5dam +3dam/m on the first attack that hits the caster.
                 It is only active during the round it has been cast.
                 Absorbs incoming damage before it strikes the character
                 or his equipment.
---equipment-------
bite             8 dam 6 toparry-3 toavoid-3
thunder stick    9 dam 4 +2/charge, parry-3,  25charges
                   stun 6 +6/charge
                   parrying weapon removes abs stun points,
                   rest continue into holder
                   charge effects add to all effects with each charge
                   (both damage and stun)
lightning rod    8 dam 5 pen 2 stun 3, range 10, 15 charges
                   short 5 mod+3 dam+2, long 15 mod-3 pen 0
                   normal shot 1a mod-0,
                   quick shot 1a fast+1 mod-3
                   aimed shot 1r mod+3
===================================
hp20 abs4      : 20
mana 15 : 15
no pain
ammo ts : 25
ammo lr : 15
\end{verbatim} \normalsize \end{samepage}










% change back to normal font size
%\normalsize

% \raggedbottom command at the top tells to :
% don't add glue spacing between segments, allow ragged bottoms instead
%
% now turn on \flushbottom in the end.
%\flushbottom  -  moved all this to the main book.tex file

%--------|---------|---------|---------|---------|---------|---------|---------|
%       10        20        30        40        50        60        70        80


\cleardoublepage

\phantomsection\addcontentsline{toc}{chapter}{characters}
\chapter*{Characters}
\chaptermark{characters}
\label{cpt:characters}

Here are some example Hero characters. First a few newbie wannabecome heroes rolled up by the author, then some examples of player built Heroes rolled up for the \texttt{Return of Uchly Namen} campaign, then some more experienced heroes.

\

Since the xp cost of various skills are constantly tweaked, all of the characters listed below are probably not up to date at any given time. I don't intend to ever hard lock down the xp costs since it's more fun to adapt to make the game more interesting and never have "the one best way" for any given adventure or campaign.


\

\goodbreak
\phantomsection \addcontentsline{toc}{section}{example newbies}
\section*{Example newbie characters}
%-----------------------------------

Below are a few example newbie Hero wannabecomes, ready to take on the world, and hopefully not die a horrible death immediately.

%\vspace{40mm}
%
%%\todo some sort of diverse party character sketch goes here ?
%
%\begin{figure}[h!]
%  \centering
%  \includegraphics[width=0.50\textwidth]{figs/swordshield.png}
%  \caption*{all in a good party}                   %  * remove "figure X" prefix
%\end{figure}
%% swordshield.png file is from pixabay.com, free license
%% original filename: armoured-1296516_960_720.png






\clearpage
\begin{samepage}

\noindent Morten Flaff is a decent basic noob fighter but lacks all the useful support skills like gossip, haggle, counting, literate, etc. And he has a basic but limited travel pack.

\

\small \begin{verbatim}
Morten Flaff      (human fighter wannabe)
================= human =================
str  6                     hp 18 abs 0
dex  5                     m1 w4 r7 d8
con  4                     stamina 10
int  4                     vision 16 day 192
psy  7                     mana 14
per  6                     ap 6 (4)
cha  7                     xp 3 (117)
-----------------------------------------                       cost, start, sum
Common 5                                                              0.5 (4)  4
brawl  3                                                              0.5 (3)  0
throw  2                                                              0.7 (2)  0
avoid  2                                                              0.8 (2)  0
yield +3                                                              spc (3)  0
off balance                                                             5 (x)  0
spear  8                                                              0.9 (0) 57
shield 6                                                              0.7 (0) 25
quick  2                                                              4.0 (0) 16
mobile 2                                                              3.0 (0) 12
-----------------------------------------
spear     8, dam 5/5, pen 0/1, abs 6, parry-1/+1, reach 1 mod-3      3s, 1.50enc
1h/2h     str 4/3 (str bonus max +1 dam then max +2 pen)
          slash: mod-2, dam-1
          finesse-4/-6
shield    9(6), abs 10, parry+3,                                     5s, 2.00enc
          str 4, fast 1 if str 8 and dex 12
          Ranged attacks mod-2 when in the way.
          Hiding behind it (3ap) ranged mod-4
          tackle mod+1
money: 2 gold, 3 silver, 3 copper
simple threadbare clothing                                            --, 0 worn
knife                                                                5c, 0.25enc
travel sack 10enc: 1.5enc + 0.75 * load                              2c, 9.00enc
    bed roll, tarp,                            1+3s, 3+4enc
    water skin,                                3c, 1enc
    flint & steel,                             3c, 0.2enc
    travel rations 2x                          2x5c, 2x0.3 = 0.6enc
-----------------------------------------
spent 15s on equipment, dumped his crappy starter gear
overencumbered 13/6 when travelling : encumbrance 2: mod-2 dash-2 run-1, etc
dumping sack    4/6 when fighting  : enc 0
=========================================
\end{verbatim} \end{samepage} \normalsize





\clearpage
\begin{samepage}

\noindent MistaMuhda is a specialised slayer noob goblin. He starts with no significant money and simply walk around with his massive but horribly shitty 2h axe, scavenging from the land or killing things when possible. At least he can speak Common a little bit.

\

\small \begin{verbatim}
MistaMuhda               (goblin slayer)
================ goblin ================
str  6(5)                hp 13(11) abs 0
dex  6                   m2 w4 r5 d6
con  5                   stamina 6(4)
int  6                   vision 18 dusk 265
psy  3                   mana 2
per  6                   ap 5(4)
cha  3                   xp 3 (98)
----------------------------------------
Svartlingo 5(4), Common 3                                                   4, 4
brawl 5(3), throw 3(2)                                                      8, 3
axe 6                                                                         32
avoid 6                                                                       28
yield +3, disengage 2, off balance                                             -
sneak 5 (2+3)                                                                  2
quick 1, enduring 2, resilient 2, strong 1                            4, 4, 4, 2
----------------------------------------
bite     5, dam 2, 3ap (2ap if both hands free)
scratch  5, dam 1, 2ap (1ap if both hands free), stamina(-1,0,0)
fist     5, dam 2, 2ap, stamina(-1,0)
kick     5, dam 4, 3ap
2h axe   5(6), dam 10(12), abs 7(12), parry -3, toparry -2, toavoid +1,
         str 6, finesse-2
         rotten quality: mod-1, dam-2, abs-5, 10% risk to break when drawn,
         when broken takes a few hours to repair
money: 1 silver, 4 copper, 1 teeth, 2 stones, 1 feathers, 2 glass beads
\end{verbatim} \end{samepage} \normalsize

\



%\clearpage
\goodbreak
\begin{samepage}

\noindent GrabbaKill is a heavy brawler, making some money here and there from village show fighting? Quick attacks and counter, but defence is weak with brawl deflects taking mods and still giving partial damage.

\

\small \begin{verbatim}
GrabbaKill             (goblin pugilist)
================ goblin ================
str  8(6)                hp 12 abs 1
dex  8                   m1 w3 r5 d7
con  4                   stamina 5
int  3                   vision 22 dusk 265
psy  6                   mana 1
per  8                   ap 6(5)
cha  2                   xp 0 (95)
----------------------------------------
Svartlingo 4, Common 2, gossip 3, haggle 3                            -, 2, 4, 4
brawl 9(2), throw 3(1), slugger 2, counter attack                  38, 5, 20, 10
yield +4, off balance                                                       -, -
quick 1, strong 2                                                           4, 8
----------------------------------------
bite     9, dam 3, 3ap (2ap if both hands free)
scratch  9, dam 1, 2ap (1ap if both hands free), stamina(-1,0,0)
fist     9, dam 4, 2ap, stamina(-1,0)
kick     9, dam 6, 3ap
leather armour  abs 1, shitty quality: creaks and smells sneak-6
money: 2 silver, 7 copper, 2 teeth, 3 stones, 1 feathers, 1 glass beads
\end{verbatim} \end{samepage} \normalsize





\clearpage
\begin{samepage}

\noindent Parry Hotter is a reasonably well rounded upcoming spell slinger. He looks really snazzy with his stuffed owl on his shoulder. Though he had to trade his wand for a proper staff.

\

\small \begin{verbatim}
===================================
Parry Hotter the Fizzler    (boy03)
-----------------------------------
str  4      hp 12 abs 0
dex  7      m1 w3 r6 d9
con  7      stamina 4
int  7      vision 23 arc 210
psy  6      mana 16
per  7      ap 3
cha  5      xp 16 (110)
----------
Common 8, literate 3, counting 2
staff 5, brawl 2, throw 2
avoid 3, yield +3, off balance
veteran 2
find 3
luck 1
----------
spell caster, magic 2, power casting 2
force bolt  7  cast 1r 1m, dam 3, range 10, int 7, psy 4
heal        4  cast 2r 1m, heal 3 +2/m, time 2hp/r, range contact, int 5, psy 4
stafflight  3  cast 2r 1m, duration 50 +25r/m, int 3, psy 2, scf 0.2
----------
staff  5, dam 3/4, abs 7, parry+1/+2, reach 1 mod-4, finesse-3/-5   (trusty 5xp)
1h/2h  str 4/3 (max +1/+2 str bonus)
       2h: fast 1 if str 6 and dex 6
knife  1, dam 2, abs 3, parry-2, toparry-2, toavoid-1, finesse-9
       str 1 (no str bonus)
       fast 1 if str 3 and dex 4
       every second attack in a round costs no stamina: stamina(-1,0)
       poke: mod-1 dam-1 pen+1
money 2g 8s 1c
wizard's robe, stuffed owl, sewn to the shoulder, pointy hat
small sack 5enc:
    rope 10m, snares 2x, blanket, flint & steel
    water skin, provisions 4x
\end{verbatim} \end{samepage} \normalsize

\

\noindent Note that he has knife 1 from the melee weapons similarity modifier -4 and staff 5.






\clearpage
\begin{samepage}

\noindent KrijgRauch the wünder dwarf, with both good defensive melee and ranged fighting capability.

\

\small \begin{verbatim}
===================================
KrijgRauch EckStein (dwarf-young01)
-----------------------------------
str  8      hp 18 abs 2
dex  4(5)   m1 w3 r5 d6(7)
con 10      stamina 12
int  4      vision 17 infra arc 160
psy 10      mana 17
per  7(8)   ap 3
cha  4      xp 1 (105)
----------
Dwarvish 7, Common 3, literate 2, counting 2, haggle 5(2+3)
axe 6, crossbow 5, shield 7
brawl 1, throw 1
avoid 2, yield +2, off balance
find 4
----------                                                                  =104
1h axe     6, dam 7, abs 6(8), parry-4(-3), toparry-1, toavoid+1, finesse-3 0.8e
           str 4 (max +4 str bonus)
           swing: slow 1 mod-1 dam+2 toavoid+2
shield     10(7), abs 9(10), parry+3,                                       2.0e
           str 4, or slow 1 str 1
           Ranged attacks mod-2 when in the way.
           Hiding behind it (3ap) ranged mod-4
crossbow   5, dam 5, penetrating 1,        normal  mod=0 1r                 1.0e
           range 12                        aimed   mod+1 2r
           str 3 (no str bonus)            quick   mod-3 3ap
           reload 2r (str 9 1r)
           quiver with 15 bolts                                             0.8e
chain mail abs 2 ring, scale, brigantine, etc                               2.0e
           str 3 (str penalties affect all actions)
           dex-1, dash-1, per-1
           acrobatics mod-3, martial arts mod-1, spellcasting mod-1, sneak mod-3
           takes 3 rounds to put on or take off

rusty chainmail abs 2, polished shining spotless old axe, battered shield
the chain mail has 10% risk of breaking when putting it on
when broken requires 1d3 days to repair

money 4g 8s 12c, 1 ruby worth 6gold
old worn warm clothes
large flask of beer (2 days) rolled in blanket                              4.0e
----------
dumps the blanket roll when fighting, encumbered 1 when travelling
\end{verbatim} \end{samepage} \normalsize






\clearpage
\begin{samepage}

\noindent Young Master von Dääken is a prime example of a pimply wastrel nobleman. Even if he's survived tutors and serious festivities he doesn't have much useful fighting skills. However, as a VIP or peripheral party member he can be quite funny and useful.

\

\small \begin{verbatim}
===================================
Ein von Dääken, the fop  (human xx)
-----------------------------------
str  5(4) hp 10 abs 0
dex  7    m1 w3 r6 d8
con  3    stamina 5
int  4    vision 17
psy  2    mana 4
per  5    ap 3
cha  2    xp 3 (160)
----------
Common 9, literate 5, counting 4, poetry 4
sword 8, poke
avoid 5, yield+3
histography 4
strong 1
whiny bickering 4 - causes "annoying" for 3+1d6r, takes 1r        (unique scf 1)
----------
rapier  8, dam 4, abs 4, toparry-2, toavoid-1, finesse-9
        str 3 (no str bonus)
        fast 1 if str 5 and dex 6
        every second attack in a round costs no stamina
        poke: mod-1 dam-1 pen+1

silk shirt that can soak up about 10hp worth of blood.
healing potion, heal 5hp, 1hp/r    1x
money 12g 4s 13c, credit

magnificent horse: outside the dungeon
    2x saddlebag 5 items:
        comb, mirror, shaving kit, fancy shirt, perfume
        wine, wine, wine, cold cuts, fruit

Mingold Meek: servant, weak human 70xp, watching the horse
    worn shoes
    donkey
        tent, mattress, cooking gear, provisions
\end{verbatim} \end{samepage} \normalsize






\clearpage
\begin{samepage}

\noindent Pyttelina the rogue / thief. What's mine is mine, and what's yours is soon mine as well, thank you very much.

\

\small \begin{verbatim}
===================================
Pyttelina, halfling       (rogue02)
-----------------------------------
str  4       hp 12 abs 1
dex  9       m3 w4 r5 d8
con  6       stamina 6(4)
int  4       vision 17 arc 357
psy  7       mana 10
per  8       ap 4
cha  5       xp 2 (113)
----------
Common 7, gossip 4
knife 5, throw 5
avoid 7, yield +4
sneak 5, find 4
locks & traps 3
enduring 2, pack mule 2
----------
dagger  5, dam 3, abs 4, parry-1, toparry-2, toavoid-1, finesse-9
        str 2 (no str bonus)
        fast 1 if str 4 and dex 5
        every second attack in a round costs no stamina
        poke: mod-1 dam-1 pen+1
throwing knife  5, dam 2, range 8(6+str/2)
4x      str 1 (no str bonus)
        fast 1 if str 2 and dex 4
        first two attacks do not require stamina
        melee: knife mod-1, dam 2, abs 2, fancy-2
        fast 1 if str 2 and dex 4
leather armour   abs 1, acrobatics-1, dark matte leather
quaff 2x       belt, restores 4hp 4sta in 1r
belt           4items 4.0enc (dagger, 2x quaff)
ruck sack      0.5 * enc
    empty ... have to put the loot somewhere

money: 0g 7s 6c

-----------------------------------
encumbrance = 3.3 / 6(4) = mod-0
-----------------------------
dagger 0.5
throwing knives 0.2*4 = 0.8
ruck sack 0.5
quaff 0.25*2 = 0.5
leather armour 1.0
===================================
\end{verbatim} \end{samepage} \normalsize






\clearpage
\begin{samepage}

\noindent Lars LongShot used to hunt rabbits and deer, now he mostly hunts beer, fame, and gold.
When he saves up some more money he can switch up to a heavy longbow since he has very high strength, and consider a smaller staff for fast 2ap defence. Alternatively burn 18xp on agile to meet the dex requirements to get the large staff fast and 4xp on quick to get 4ap base, giving 2x fast defence actions without mods and 3x with mod-2. Very useful when things get hectic and the bugs are swarming in.

\

\small \begin{verbatim}
===================================
Lars LongShot, human        (token)
-----------------------------------
str 10       hp 21(19) abs 1(0)
dex  5       m1 w3 r5 d8(9)
con  7       stamina 6(4)
int  6       vision 18 arc 255
psy  8       mana 20
per  4       ap 3
cha  8       xp 3 (112)
----------
Common 7, gossip 4, find 2
bow 6, quick shot 3, staff 6, accurate,
avoid 2, yield+2
veteran 2, enduring 2, resilient 2
luck 2
----------
longbow   6, dam 5, pen 1, range 20
          str 6 (no str bonus)
          aimed 3r mod+1, normal 2r mod-0, quick 1r mod-0
          readied quickdraw left shoulder
          quiver 20 arrows, left shoulder
large staff  6, dam 5/6, abs 9, parry+1/+2, reach 1 mod-4, finesse-3/-5
1h/2h     str 6/5 (max +1/+2 str bonus)
          2h: fast 1 if str 8 and dex 8       ! 3ap, not enough dex !
          readied quickdraw right shoulder
leather armour abs 1, old and worn, smells bad,
          acrobatics mod-1, dash-1, sneak-3
simple old clothing
money ( 1g 6s 10c ) left: 4s 3c
-----------------------------
quickdraw 4:
    bow / staff : left shoulder
    quiver : right shoulder
    quickdraw bow/staff assumes one of them held, and switching.
encumbrance: 5 / 10 = ~0.5
===================================
\end{verbatim} \end{samepage} \normalsize





\clearpage
\begin{samepage}

\noindent Newton spent his whole childhood fighting imaginary monsters with his overly large and heavy wooden sword, whittled by his uncle Marrowood. On his 18th birthday he packed a light knapsack and broke his piggy bank. Then said goodbye to his father and mother, kissed his girlfriend goodbye, and went out into the world looking for monsters to slay. First stop was the smithy, where he spent almost all his considerable savings on a brand new heroic 2h sword.

Without armour and only the sword for defence he's quite vulnerable, but perhaps 20hp makes up for it?

\

\small \begin{verbatim}
===================================
Newton the Happy Farmboy    (boy01)
-----------------------------------
str  7    hp 20 abs 0
dex  8    m1 w3 r7 d10
con  4    stamina 8
int  5    vision 16 arc 225
psy  9    mana 6
per  5    ap 5(4)
cha  6    xp 3 (initial 116)
----------
Common 5, gossip 2, counting 1
sword 8, brawl 2, throw 2, accurate 1
avoid 3, yield 3
quick 1, mobile 1
jump 2, climb 2, swim 3
veteran 1, balance 1, find 2
luck 2, black cat 1
----------
money: 0g 6s 7c
2h sword  8, dam 10, abs 15,                                                   2
          str 6 (max +3 str bonus), finesse-6
          reach 1 mod-6
          poke: mod-1 dam-1 pen+1
          swing: slow 1 mod-1 dam+2 todefend+2
sack 5enc (1.0 + 0.75*load):                                                ca 4
    red apple, cheese sandwich, lock of girlfriend's hair, water skin    1.0
    knife, rope 10m, flint and steel, torch                          1.0+1.0
blanket                                                                        2
----------
encumbrance 9/7 : dumps the sack and blanket when fighting
\end{verbatim} \end{samepage} \normalsize














%-------------------------------------------------------------------------------
%                          N E E D   U P D A T E
%                          ---------------------





\clearpage
\phantomsection\addcontentsline{toc}{section}{experienced heroes}
\section*{Experienced Heroes}
%----------------------------
In practice, so far, we've tested hundreds of hours of game time with dozens of characters ranging around 100-1000xp. We haven't had any super high level adventures yet, so we haven't really tested 1000xp+ Heroes. The game scales very well up to 1k, and probably a bit beyond. We just don't have the data.


\subsection*{Tråk-Ture}
%----------------------
Tråk-Ture is a very typical, uninspired, example Hero at 250xp. Intended to show new players how to build a simple character for the \texttt{07 Eviction} adventure. Though the base roll for the character was unusually good, with high str, dex, hp, stamina, movement, vision, ap.

There is nothing wrong with Typical Heroes, but it can be more fun to build gubbins that have some non standard aspects and opportunities. For first time players it's easier to get a quick feel for the game with a standard Hero which is simpler to play reasonably well.

\

\small \begin{verbatim}
Tråk-Ture  -  a typical, very boring, 2h sword fighter
================== elf 0 ===================
str  8                   hp 18(15) abs 0
dex  8(9)                m2 w3 r8(9) d10(11)
con  5                   stamina 11
int  6                   vision 30 night 247
psy  4                   mana 9
per  6(7)                action points 6(4)
cha  3                   xp 0(253)
--------------------------------------------
quick 2, mobile 2, fast 1, resilient 3, enduring 2, rapid 2
sword 10, avoid 7, yield +3, fancy 3, poke, swing, off balance
veteran 3, balance 3, charge 3, quickdraw 2
swim 3, climb 3, jump 3, find 3
Elvish 8, Common 5, haggle -2(0-2)
--------------------------------------------
immune to poisons
city: psy-1/w (max-3), restore immediate when in nature
--------------------------------------------
money: 0g 4s 3c                                   cost    enc      4.7/8  ~ 0enc

large 2h sword  10                                8 0 0   2.7
   dam 12, abs 18, reach 1 mod-5, finesse-6
   str 8 (max +4 str bonus)
   poke: mod-1 dam-1 pen+1                 9 3ap dam11 pen1
   swing: slow 1 mod-1 dam+3 todefend+2    9 4ap dam15 todefend+2

chain mail  abs 2                                 4 0 0   2.0/4.0
   str 3 (str penalties affect all actions)
   dex-1, run-1, dash-2, per-1
   acrobatics mod-3, martial arts mod-2, spellcasting mod-2, sneak mod-3
   takes 3 rounds to put on or take off

! remember to get your dude some survival gear, food, etc…
\end{verbatim} \normalsize






\

\vfill


\todo update and return the old player characters and experienced heroes

% moved the old not updated characters stuff to old.tex 210116

%\todo would be nice to have a cool sketch of heroes here



\clearpage
\begin{samepage}
\subsection*{Old Maz}
%--------------------
Old Maz was one of the Heroes who survived \texttt{Goblin Destiny}. Here is an earlier snapshot of the character, when he is around 300xp.

\

\small \begin{verbatim}
================== Old Maz ==================
str  4(2)                 hp 18(14) abs 1
dex  10 (9)               m1 w3 r7(6) d9 (7)
con  6(5)                 stamina 4(3)
int  10 (8)               vision 18 dusk 253
psy  8(7)                 mana 15(12)
per  7                    action points 7(6)
cha  6 (5)                xp 4 (306)
--------------------------------------------
yield bonus 4, sneak bonus +1
mutation: lila prickar
--------------------------------------------
Common 2
Svartlingo 6 (3)
literate 2
Ancient 2
counting 2
avoid 6
bite 4
disengage 4
scratch 5
sneak 2 (1+1)
throw 3
agile 1
Powerful 3
mobile 2
smart 2
determined 1
Charming 1
Leader 4
Veteran 2
ride 5
Enduring 1
Resilient 4
Scary 7
Rally 5
fast 2
Tactician 2
quick 1
strong 2
tough 1
--------------------------------------------------
magic 3, spellcaster
force bolt 11
heal 6
drain 9
slow 8
--------------------------------------------
yield
off balance
-----------------------------------------------
\end{verbatim} \end{samepage} \clearpage \begin{samepage} \begin{verbatim}
Force bolt
cast 3ap 1m, dam 6 +1/m, penetrating 2,
range 10 +5/m, long -2dam,
Int 10, Psy 8

Heal
cast 2r 1m, heal 3 +2/m, time 2hp/r, range contact
int 5, psy 4

Drain
cast 2r 0m, psy vs psy +3/m, range contact,
drains diff mana points
must be in contact for the last round of casting.
int 5, psy 8

Slow
slow cast 1r 1m, +1target/2m, range 15 +5/m, duration 5r +2/m
Movement costs double movement points.
All actions are slow-2 (takes +2 extra ap)
Full/multi-round actions take +1r
Continue slow on psy vs psy +2/m each round (mana paid once
on cast) up to the max duration. First round auto success.
Once slow is lost on one target that target is free from
the spell.
int 4, psy 3
--------------------------------------------
money: 0 silver, 1 copper, 4 teeth, 1 stones, 1 feathers, 3 glass beads
goblins can live on half rations, and can eat spoiled food
bite 4 dam 3 3ap (2ap if both hands free) scf 0.5
scratch 5 dam 1 2ap (1ap if both hands free) scf 0.5

5x leather sacks 1x waterproof tarp 1x rope 1x cart 1x donkey
horse+ sadel 2sadelbag
\end{verbatim} \end{samepage} \normalsize








\clearpage
\begin{samepage}
\subsection*{Maxkeles Grapplebottom}
%-----------------------------------
\noindent Maxkeles Grapplebottom is another \texttt{Goblin Destiny} veteran, shown here at around 340xp.

\

\small \begin{verbatim}
================== Maxkeles Grapplebottom ==================
str  5(2)                    hp 14 abs 1
dex 11                   m 2 (1) w5 (4) r7 (5) d10 (6)
con  6 (4)                   stamina 7 (5)
int  5                    vision 24 dusk 204
psy  3                    mana 6
per  4                    action points 6 (4)
cha  1                    xp 41 (339)
--------------------------------------------
yield bonus 3
sneak bonus +2
mutation: Moustache! Cool!
--------------------------------------------
Common 4
Svartlingo 4
avoid 5
bite 3
disengage 3
scratch 5
sneak 9 (7+2)
throw 4
Quick 2
Mobile 3
Shield 2
Spear 10
Charge 2
Block 6
Enduring 2
Strong 3
Consistent 1
Veteran 2
Tank 1
Fast 4
Balance 3
Precise 2
Accurate 1
Tough 2
--------------------------------------------
yield
off balance
--------------------------------------------
\end{verbatim} \end{samepage} \clearpage \begin{samepage} \begin{verbatim}
goblins can live on half rations, and can eat spoiled food
bite 3 dam 2 3ap (2ap if both hands free) scf 0.5
scratch 5 dam 1 2ap (1ap if both hands free) scf 0.5

chain mail     abs 2 ring, scale, brigantine, etc
               str 3 (str penalties affect all actions)
               dex-1,
               dash-1,
               per-1
               acrobatics mod-3
               martial arts mod-1
               spellcasting mod-1
               sneak mod-3
               takes 3 rounds to put on or take off

Spear      dam 6, pen 1/2, abs 5, parry-1/+1, reach 1 mod-3
1h/2h      str 4 (max +2 str damage bonus, max+2 penetrating bonus)
           finesse-4/-6

1xbelt (2items)
2xQuaff!

Sack 5 enc
1xBlanket
1xshield
1xknife
10 days food
1x waterproof tarp

money: 2 silver, 1 copper, 3 teeth, 2 stones, 2 feathers, 2 glass beads
\end{verbatim} \end{samepage} \normalsize











\end{document}
